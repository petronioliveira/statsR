% Options for packages loaded elsewhere
% Options for packages loaded elsewhere
\PassOptionsToPackage{unicode}{hyperref}
\PassOptionsToPackage{hyphens}{url}
%
\documentclass[
  portuguese,
  letterpaper,
]{scrbook}
\usepackage{xcolor}
\usepackage{amsmath,amssymb}
\setcounter{secnumdepth}{5}
\usepackage{iftex}
\ifPDFTeX
  \usepackage[T1]{fontenc}
  \usepackage[utf8]{inputenc}
  \usepackage{textcomp} % provide euro and other symbols
\else % if luatex or xetex
  \usepackage{unicode-math} % this also loads fontspec
  \defaultfontfeatures{Scale=MatchLowercase}
  \defaultfontfeatures[\rmfamily]{Ligatures=TeX,Scale=1}
\fi
\usepackage{lmodern}
\ifPDFTeX\else
  % xetex/luatex font selection
\fi
% Use upquote if available, for straight quotes in verbatim environments
\IfFileExists{upquote.sty}{\usepackage{upquote}}{}
\IfFileExists{microtype.sty}{% use microtype if available
  \usepackage[]{microtype}
  \UseMicrotypeSet[protrusion]{basicmath} % disable protrusion for tt fonts
}{}
\makeatletter
\@ifundefined{KOMAClassName}{% if non-KOMA class
  \IfFileExists{parskip.sty}{%
    \usepackage{parskip}
  }{% else
    \setlength{\parindent}{0pt}
    \setlength{\parskip}{6pt plus 2pt minus 1pt}}
}{% if KOMA class
  \KOMAoptions{parskip=half}}
\makeatother
% Make \paragraph and \subparagraph free-standing
\makeatletter
\ifx\paragraph\undefined\else
  \let\oldparagraph\paragraph
  \renewcommand{\paragraph}{
    \@ifstar
      \xxxParagraphStar
      \xxxParagraphNoStar
  }
  \newcommand{\xxxParagraphStar}[1]{\oldparagraph*{#1}\mbox{}}
  \newcommand{\xxxParagraphNoStar}[1]{\oldparagraph{#1}\mbox{}}
\fi
\ifx\subparagraph\undefined\else
  \let\oldsubparagraph\subparagraph
  \renewcommand{\subparagraph}{
    \@ifstar
      \xxxSubParagraphStar
      \xxxSubParagraphNoStar
  }
  \newcommand{\xxxSubParagraphStar}[1]{\oldsubparagraph*{#1}\mbox{}}
  \newcommand{\xxxSubParagraphNoStar}[1]{\oldsubparagraph{#1}\mbox{}}
\fi
\makeatother

\usepackage{color}
\usepackage{fancyvrb}
\newcommand{\VerbBar}{|}
\newcommand{\VERB}{\Verb[commandchars=\\\{\}]}
\DefineVerbatimEnvironment{Highlighting}{Verbatim}{commandchars=\\\{\}}
% Add ',fontsize=\small' for more characters per line
\usepackage{framed}
\definecolor{shadecolor}{RGB}{241,243,245}
\newenvironment{Shaded}{\begin{snugshade}}{\end{snugshade}}
\newcommand{\AlertTok}[1]{\textcolor[rgb]{0.68,0.00,0.00}{#1}}
\newcommand{\AnnotationTok}[1]{\textcolor[rgb]{0.37,0.37,0.37}{#1}}
\newcommand{\AttributeTok}[1]{\textcolor[rgb]{0.40,0.45,0.13}{#1}}
\newcommand{\BaseNTok}[1]{\textcolor[rgb]{0.68,0.00,0.00}{#1}}
\newcommand{\BuiltInTok}[1]{\textcolor[rgb]{0.00,0.23,0.31}{#1}}
\newcommand{\CharTok}[1]{\textcolor[rgb]{0.13,0.47,0.30}{#1}}
\newcommand{\CommentTok}[1]{\textcolor[rgb]{0.37,0.37,0.37}{#1}}
\newcommand{\CommentVarTok}[1]{\textcolor[rgb]{0.37,0.37,0.37}{\textit{#1}}}
\newcommand{\ConstantTok}[1]{\textcolor[rgb]{0.56,0.35,0.01}{#1}}
\newcommand{\ControlFlowTok}[1]{\textcolor[rgb]{0.00,0.23,0.31}{\textbf{#1}}}
\newcommand{\DataTypeTok}[1]{\textcolor[rgb]{0.68,0.00,0.00}{#1}}
\newcommand{\DecValTok}[1]{\textcolor[rgb]{0.68,0.00,0.00}{#1}}
\newcommand{\DocumentationTok}[1]{\textcolor[rgb]{0.37,0.37,0.37}{\textit{#1}}}
\newcommand{\ErrorTok}[1]{\textcolor[rgb]{0.68,0.00,0.00}{#1}}
\newcommand{\ExtensionTok}[1]{\textcolor[rgb]{0.00,0.23,0.31}{#1}}
\newcommand{\FloatTok}[1]{\textcolor[rgb]{0.68,0.00,0.00}{#1}}
\newcommand{\FunctionTok}[1]{\textcolor[rgb]{0.28,0.35,0.67}{#1}}
\newcommand{\ImportTok}[1]{\textcolor[rgb]{0.00,0.46,0.62}{#1}}
\newcommand{\InformationTok}[1]{\textcolor[rgb]{0.37,0.37,0.37}{#1}}
\newcommand{\KeywordTok}[1]{\textcolor[rgb]{0.00,0.23,0.31}{\textbf{#1}}}
\newcommand{\NormalTok}[1]{\textcolor[rgb]{0.00,0.23,0.31}{#1}}
\newcommand{\OperatorTok}[1]{\textcolor[rgb]{0.37,0.37,0.37}{#1}}
\newcommand{\OtherTok}[1]{\textcolor[rgb]{0.00,0.23,0.31}{#1}}
\newcommand{\PreprocessorTok}[1]{\textcolor[rgb]{0.68,0.00,0.00}{#1}}
\newcommand{\RegionMarkerTok}[1]{\textcolor[rgb]{0.00,0.23,0.31}{#1}}
\newcommand{\SpecialCharTok}[1]{\textcolor[rgb]{0.37,0.37,0.37}{#1}}
\newcommand{\SpecialStringTok}[1]{\textcolor[rgb]{0.13,0.47,0.30}{#1}}
\newcommand{\StringTok}[1]{\textcolor[rgb]{0.13,0.47,0.30}{#1}}
\newcommand{\VariableTok}[1]{\textcolor[rgb]{0.07,0.07,0.07}{#1}}
\newcommand{\VerbatimStringTok}[1]{\textcolor[rgb]{0.13,0.47,0.30}{#1}}
\newcommand{\WarningTok}[1]{\textcolor[rgb]{0.37,0.37,0.37}{\textit{#1}}}

\usepackage{longtable,booktabs,array}
\usepackage{calc} % for calculating minipage widths
% Correct order of tables after \paragraph or \subparagraph
\usepackage{etoolbox}
\makeatletter
\patchcmd\longtable{\par}{\if@noskipsec\mbox{}\fi\par}{}{}
\makeatother
% Allow footnotes in longtable head/foot
\IfFileExists{footnotehyper.sty}{\usepackage{footnotehyper}}{\usepackage{footnote}}
\makesavenoteenv{longtable}
\usepackage{graphicx}
\makeatletter
\newsavebox\pandoc@box
\newcommand*\pandocbounded[1]{% scales image to fit in text height/width
  \sbox\pandoc@box{#1}%
  \Gscale@div\@tempa{\textheight}{\dimexpr\ht\pandoc@box+\dp\pandoc@box\relax}%
  \Gscale@div\@tempb{\linewidth}{\wd\pandoc@box}%
  \ifdim\@tempb\p@<\@tempa\p@\let\@tempa\@tempb\fi% select the smaller of both
  \ifdim\@tempa\p@<\p@\scalebox{\@tempa}{\usebox\pandoc@box}%
  \else\usebox{\pandoc@box}%
  \fi%
}
% Set default figure placement to htbp
\def\fps@figure{htbp}
\makeatother

\ifLuaTeX
  \usepackage{luacolor}
  \usepackage[soul]{lua-ul}
\else
  \usepackage{soul}
\fi

% definitions for citeproc citations
\NewDocumentCommand\citeproctext{}{}
\NewDocumentCommand\citeproc{mm}{%
  \begingroup\def\citeproctext{#2}\cite{#1}\endgroup}
\makeatletter
 % allow citations to break across lines
 \let\@cite@ofmt\@firstofone
 % avoid brackets around text for \cite:
 \def\@biblabel#1{}
 \def\@cite#1#2{{#1\if@tempswa , #2\fi}}
\makeatother
\newlength{\cslhangindent}
\setlength{\cslhangindent}{1.5em}
\newlength{\csllabelwidth}
\setlength{\csllabelwidth}{3em}
\newenvironment{CSLReferences}[2] % #1 hanging-indent, #2 entry-spacing
 {\begin{list}{}{%
  \setlength{\itemindent}{0pt}
  \setlength{\leftmargin}{0pt}
  \setlength{\parsep}{0pt}
  % turn on hanging indent if param 1 is 1
  \ifodd #1
   \setlength{\leftmargin}{\cslhangindent}
   \setlength{\itemindent}{-1\cslhangindent}
  \fi
  % set entry spacing
  \setlength{\itemsep}{#2\baselineskip}}}
 {\end{list}}
\usepackage{calc}
\newcommand{\CSLBlock}[1]{\hfill\break\parbox[t]{\linewidth}{\strut\ignorespaces#1\strut}}
\newcommand{\CSLLeftMargin}[1]{\parbox[t]{\csllabelwidth}{\strut#1\strut}}
\newcommand{\CSLRightInline}[1]{\parbox[t]{\linewidth - \csllabelwidth}{\strut#1\strut}}
\newcommand{\CSLIndent}[1]{\hspace{\cslhangindent}#1}

\ifLuaTeX
\usepackage[bidi=basic]{babel}
\else
\usepackage[bidi=default]{babel}
\fi
% get rid of language-specific shorthands (see #6817):
\let\LanguageShortHands\languageshorthands
\def\languageshorthands#1{}


\setlength{\emergencystretch}{3em} % prevent overfull lines

\providecommand{\tightlist}{%
  \setlength{\itemsep}{0pt}\setlength{\parskip}{0pt}}



 


% preambulo.tex

% Pacotes matemáticos e símbolos
\usepackage{amsmath, amssymb, amsfonts}

% Tabelas e gráficos
\usepackage{graphicx}
\usepackage{booktabs}
\usepackage{float}

% Estilo de fonte e layout
\usepackage{lmodern}       % Fonte moderna
\usepackage[utf8]{inputenc}
\usepackage[T1]{fontenc}
\usepackage[brazil]{babel} % Português do Brasil

% Margens e espaçamento
\usepackage[a4paper, margin=2.5cm]{geometry}
\usepackage{setspace}
\onehalfspacing             % Espaçamento 1.5

% Estilo de cabeçalhos
\usepackage{fancyhdr}
\pagestyle{fancy}
\fancyhf{}
\fancyhead[L]{Bioestatística Usando o R}
\fancyhead[R]{\leftmark}
\fancyfoot[C]{\thepage}

% Legendas de figuras e tabelas
\usepackage{caption}
\captionsetup{font=small, labelfont=bf}

% Hiperlinks
\usepackage{hyperref}
\hypersetup{
  colorlinks=true,
  linkcolor=blue,
  citecolor=blue,
  urlcolor=blue,
  pdftitle={Bioestatística Usando o R},
  pdfauthor={Petrônio Fagundes de Oliveira Filho}
}
\usepackage{fontspec}
\usepackage{multirow}
\usepackage{multicol}
\usepackage{colortbl}
\usepackage{hhline}
\newlength\Oldarrayrulewidth
\newlength\Oldtabcolsep
\usepackage{longtable}
\usepackage{array}
\usepackage{hyperref}
\usepackage{float}
\usepackage{wrapfig}
\usepackage{booktabs}
\usepackage{caption}
\usepackage{anyfontsize}
\makeatletter
\@ifpackageloaded{tcolorbox}{}{\usepackage[skins,breakable]{tcolorbox}}
\@ifpackageloaded{fontawesome5}{}{\usepackage{fontawesome5}}
\definecolor{quarto-callout-color}{HTML}{909090}
\definecolor{quarto-callout-note-color}{HTML}{0758E5}
\definecolor{quarto-callout-important-color}{HTML}{CC1914}
\definecolor{quarto-callout-warning-color}{HTML}{EB9113}
\definecolor{quarto-callout-tip-color}{HTML}{00A047}
\definecolor{quarto-callout-caution-color}{HTML}{FC5300}
\definecolor{quarto-callout-color-frame}{HTML}{acacac}
\definecolor{quarto-callout-note-color-frame}{HTML}{4582ec}
\definecolor{quarto-callout-important-color-frame}{HTML}{d9534f}
\definecolor{quarto-callout-warning-color-frame}{HTML}{f0ad4e}
\definecolor{quarto-callout-tip-color-frame}{HTML}{02b875}
\definecolor{quarto-callout-caution-color-frame}{HTML}{fd7e14}
\makeatother
\makeatletter
\@ifpackageloaded{bookmark}{}{\usepackage{bookmark}}
\makeatother
\makeatletter
\@ifpackageloaded{caption}{}{\usepackage{caption}}
\AtBeginDocument{%
\ifdefined\contentsname
  \renewcommand*\contentsname{Índice}
\else
  \newcommand\contentsname{Índice}
\fi
\ifdefined\listfigurename
  \renewcommand*\listfigurename{Lista de Figuras}
\else
  \newcommand\listfigurename{Lista de Figuras}
\fi
\ifdefined\listtablename
  \renewcommand*\listtablename{Lista de Tabelas}
\else
  \newcommand\listtablename{Lista de Tabelas}
\fi
\ifdefined\figurename
  \renewcommand*\figurename{Figura}
\else
  \newcommand\figurename{Figura}
\fi
\ifdefined\tablename
  \renewcommand*\tablename{Tabela}
\else
  \newcommand\tablename{Tabela}
\fi
}
\@ifpackageloaded{float}{}{\usepackage{float}}
\floatstyle{ruled}
\@ifundefined{c@chapter}{\newfloat{codelisting}{h}{lop}}{\newfloat{codelisting}{h}{lop}[chapter]}
\floatname{codelisting}{Listagem}
\newcommand*\listoflistings{\listof{codelisting}{Lista de Listagens}}
\makeatother
\makeatletter
\makeatother
\makeatletter
\@ifpackageloaded{caption}{}{\usepackage{caption}}
\@ifpackageloaded{subcaption}{}{\usepackage{subcaption}}
\makeatother
\usepackage{bookmark}
\IfFileExists{xurl.sty}{\usepackage{xurl}}{} % add URL line breaks if available
\urlstyle{same}
\hypersetup{
  pdftitle={Bioestatística Usando o R},
  pdfauthor={Petrônio Fagundes de Oliveira Filho},
  pdflang={pt},
  hidelinks,
  pdfcreator={LaTeX via pandoc}}


\title{Bioestatística Usando o R}
\author{Petrônio Fagundes de Oliveira Filho}
\date{21/09/2025}
\begin{document}
\frontmatter
\maketitle

\renewcommand*\contentsname{Índice}
{
\setcounter{tocdepth}{2}
\tableofcontents
}

\mainmatter
\bookmarksetup{startatroot}

\chapter*{Prefácio}\label{prefuxe1cio}
\addcontentsline{toc}{chapter}{Prefácio}

\markboth{Prefácio}{Prefácio}

\begin{center}
\includegraphics[width=3.125in,height=\textheight,keepaspectratio]{capa.png}
\end{center}

\section*{\texorpdfstring{Bem-vindos ao \textbf{Bioestatística usando o
R}}{Bem-vindos ao Bioestatística usando o R}}\label{bem-vindos-ao-bioestatuxedstica-usando-o-r}
\addcontentsline{toc}{section}{Bem-vindos ao \textbf{Bioestatística
usando o R}}

\markright{Bem-vindos ao \textbf{Bioestatística usando o R}}

\begin{tcolorbox}[enhanced jigsaw, bottomrule=.15mm, opacitybacktitle=0.6, colframe=quarto-callout-tip-color-frame, arc=.35mm, coltitle=black, toptitle=1mm, colback=white, colbacktitle=quarto-callout-tip-color!10!white, breakable, bottomtitle=1mm, rightrule=.15mm, titlerule=0mm, toprule=.15mm, opacityback=0, leftrule=.75mm, left=2mm, title=\textcolor{quarto-callout-tip-color}{\faLightbulb}\hspace{0.5em}{Dica}]

\textbf{Bioestatística Usando o R} é uma jornada completa pela
estatística aplicada à saúde, com foco prático no uso do R e RStudio.
Esta segunda edição foi revisada, reorganizada em partes temáticas e
enriquecida com novos recursos visuais e pedagógicos.

\end{tcolorbox}

Este livro foi escrito para estudantes, pesquisadores e profissionais da
área biomédica que desejam compreender e aplicar os conceitos
estatísticos com o suporte de uma das ferramentas mais poderosas e
acessíveis: o R. \textbf{Bioestatística Usando o R} tem a ambição de ser
um pouco mais tolerável e amistoso, a fim de estimular o estudo da
Bioestatística e, assim, facilitar a crítica da literatura científica e
, desta forma, ajudar a evitar a intoxicação causada pela pseudociência.

Você encontrará aqui:

\begin{itemize}
\tightlist
\item
  Fundamentos teóricos da bioestatística
\item
  Aplicações práticas com a linguagem R
\item
  Visualizações elegantes com \texttt{ggplot2}
\item
  Testes estatísticos, regressões, ANOVA e muito mais
\item
  Aplicações em epidemiologia e saúde pública
\end{itemize}

\begin{quote}
\section*{Sobre o Autor}\label{sobre-o-autor}
\addcontentsline{toc}{section}{Sobre o Autor}

\markright{Sobre o Autor}

\textbf{Petrônio Fagundes de Oliveira Filho} nasceu em 4 de outubro de
1947, em Porto Alegre, Rio Grande do Sul, Brasil. Cursou o Ensino Médio
no Colégio do Rosário, em Porto Alegre. Graduou-se em Medicina pela
Universidade de Caxias do Sul (UCS) em 1973, realizou residência em
Pediatria no Hospital da Criança Conceição (Porto Alegre) em 1975, e
concluiu o mestrado em Saúde Pública Materno-Infantil pela Universidade
de São Paulo em 1998.
\end{quote}

\begin{center}
\includegraphics[width=0.2\linewidth,height=\textheight,keepaspectratio]{index_files/mediabag/uukObDQ.png}
\end{center}

\begin{quote}
Em 1980, obteve o Título de Especialista em Pediatria (TEP) e, em 2009,
o título de especialista em Estatística Aplicada pela UCS. Atuou como
pediatra no INAMPS até 2002 e mantém consultório privado até hoje.
Aposentou-se como professor da Universidade de Caxias do Sul em 2019,
após uma trajetória iniciada em 1975. Na UCS, lecionou nas áreas de
Pediatria, Epidemiologia e Bioestatística, foi coordenador do Serviço e
da Residência Médica em Pediatria, chefe de Departamento, coordenador do
curso de Medicina e diretor de Ensino do Hospital Geral de Caxias do Sul
--- hospital de ensino da universidade. Também integrou o Comitê de
Ética em Pesquisa da UCS, vinculado à CONEP (Conselho Nacional de Ética
em Pesquisa).

Por mais de 20 anos, fez parte do Núcleo de Consultoria e Epidemiologia
do Centro de Ciências da Saúde da UCS. É autor de dois livros:
Epidemiologia e Bioestatística: Fundamentos para a leitura crítica
(Editora Rubio, 2015/2018; 2ª edição em 2022) e SPSS -- Análise de Dados
Biomédicos, em coautoria com Valter Motta (MedBook, 2009). Participou
ainda de dezenas de publicações científicas e capítulos de livros.

Desde 1976, é casado com Lena Maria Cantergiani Fagundes de Oliveira.
Tem duas filhas, Nathalia e Andressa, e dois netos encantadores e
inteligentes: Gabriel e Felix. Ah, e havia também Floquinho --- um cão
da raça Shih Tzu, branco com manchas marrom claro e pretas --- que
acompanhava seus estudos e análises estatísticas. Floquinho latia toda
vez que ouvia o nome de Ronald Fisher. Faleceu em 2024, aos 17 anos,
deixando um grande vazio e muitas lembranças.
\end{quote}

email: petronioliveira@gmail.com\\
WhatsApp: +55 54 9997 15499

\part{Parte I - Fundamentos}

\chapter{Introdução}\label{introduuxe7uxe3o}

\section{Importância da
Bioestatística}\label{importuxe2ncia-da-bioestatuxedstica}

Os indivíduos variam em relação as suas características biológicas,
psicológicas e sociais na saúde e na doença. Esta variabilidade gera uma
grande quantidade de incertezas.

A Bioestatística, estatística aplicada às ciências biológicas e da
saúde, é a ferramenta utilizada pelos pesquisadores para trabalhar com
essas incertezas advindas da variabilidade. Várias definições foram
escritas para a estatística, uma delas é a seguinte (1):

\begin{quote}
Estatística é a disciplina interessada com o tratamento dos dados
numéricos obtidos a partir de grupos de indivíduos
\end{quote}

A Bioestatística lida com a variabilidade humana utilizando técnicas
estatísticas quantitativas (2) que ajudam a diminuir a ignorância em
relação a esta diversidade. A compreensão da variabilidade humana torna
a medicina mais ciência, diminuindo as incertezas, na tentativa de
verificar se os resultados encontrados de fato existem ou são apenas
obra do acaso.

Na década de 1990, houve um acesso maior aos computadores. Os
profissionais da saúde não estatísticos passaram a ter mais interesse no
campo da bioestatística. Isto gerou uma onda que facilitou o
aparecimento de novas ferramentas estatísticas de ponta. Apesar disso, o
conhecimento da Bioestatística permanece restrito aos especialistas na
área.

Nos últimos anos, os pacotes de softwares foram aprimorados, tornando-se
mais amigáveis e diminuindo significativamente o pânico ao se defrontar
com uma série de números uma vez que a maioria deles exige apenas
conhecimento básico de matemática.

Para a tomada de decisão em saúde é fundamental o acúmulo de
conhecimento adquirido através da prática clínica, geradora da
experiência do profissional, do intercâmbio com os pares e da análise
adequada das evidências científicas publicadas em periódicos de
qualidade. Para atingir este objetivo, é fundamental o conhecimento de
bioestatística, incluindo aqui que o pensamento que deve nortear os
profissionais da saúde ao lidar com o ser humano é o \emph{pensamento
probabilístico}.

\section{Pílulas históricas da Estatística}\label{sec-historia}

\begin{quote}
A história deve começar em algum lugar, mas a história não tem começo
(3)
\end{quote}

Entretanto, é natural, que se trace as raízes voltando ao passado, tanto
quanto possível. Alguns referem-se à curiosidade em relação ao registro
de dados à dinastia Shank, na China, possivelmente no século XIII a.c,
com a realização de censos populacionais. Há relatos bíblicos de
possíveis censos realizados por Moisés (1491 a.C.) e por Davi (1017
a.C.).

Os romanos e os gregos já realizavam censos por volta do século VIII a
IV a.C. Em 578-534 a.C., o imperador \emph{Servo Túlio} mandou realizar
um censo de população masculina adulta e suas propriedades que serviu
para estabelecer o recrutamento para o exército, para o exercício dos
direitos políticos e para o pagamento de impostos. Os romanos fizeram 72
censos entre 555 a.C. e 72 d.C. A punição para quem não respondia,
geralmente era a morte! Na Idade Média, na Europa, existem registros de
diversos censos: durante o domínio muçulmano, na Península Ibérica, nos
séculos VII a XV; no reinado de Carlos Magno (712-814) e ainda o maior
registro estatístico feito na época, o \emph{Domesday Book}
(Figura~\ref{fig-domesday}), realizado na Inglaterra, por Guilherme I
(3) , o Conquistador, onde registravam nascimentos, mortes, batismos e
casamentos. Houve, também, recenseamentos nas repúblicas italianas no
século XII ao XIII (4).

\begin{figure}[h]

\centering{

\includegraphics[width=0.5\linewidth,height=\textheight,keepaspectratio]{index_files/mediabag/NYbAP67.png}

}

\caption{\label{fig-domesday}Domesday Book}

\end{figure}%

\emph{John Graunt} (24/04/1620 - 18/04/1674) foi um cientista britânico
a quem se deve vários estudos demográficos ingleses. Foi o precursor da
construção de Tábuas de Mortalidade. Realizou estudos com William Petty
(1623 - 1687), economista britânico que propôs a \emph{aritmética
política}.

Em 1791, \emph{Sir John Sinclair} (1754 - 1835) concebeu um plano de uma
pesquisa empírica na Escócia para fornecer informações estatísticas. Foi
a primeira vez que o termo estatística foi usado em inglês.

\emph{Girolamo Cardano} (24/09/1501 - 21/09/1576) foi um médico,
matemático, físico e filósofo italiano. É tido como o primeiro a
introduzir ideias gerais da teoria das equações algébricas e as
primeiras regras da probabilidade, descritas no livro \emph{Liber de
Ludo Aleae}, publicado em 1663. Descreveu pela primeira vez a clínica da
febre tifoide. Foi amigo de Leonardo da Vinci.

\emph{Pierre-Simon Laplace}, Marquês Laplace (23/03/1749 - 05/03/1927)
foi um matemático, astrônomo e físico francês. Embora conduzisse
pesquisas substanciais sobre física, outro tema principal dos esforços
de sua vida foi a teoria das probabilidades. Em seu \emph{Essai
philosophique sur les probabilités}, Laplace projetou um sistema
matemático de raciocínio indutivo baseado em probabilidades, que hoje
coincidem com as ideias bayesianas.

\emph{Antoine Gombaud}, conhecido como Chevalier de Méré (1607 - 1684)
foi um nobre e jogador. Como não tinha mais sucesso nos jogos de azar,
buscou ajuda de Blaise Pascal (19/06/1623 -- 19/08/1662), matemático,
físico francês, que se correspondeu com Pierre Fermat (matemático e
cientista francês), nascendo desta colaboração a teoria matemática das
probabilidades (1812). Blaise Pascal foi mais tarde chamado de o Pai da
Teoria das Probabilidades.

A moderna teoria das probabilidades foi atribuída a \emph{Abraham De
Moivre} (25/05/1667 -- 27/11/1754), matemático francês, que adquiriu
fama por seus estudos na trigonometria, teoria das probabilidades e pela
equação da curva normal. Em 1742, Thomas Bayes (1701 -- 07/04/1761,
matemático e pastor presbiteriano, inglês, desenvolveu o Teorema de
Bayes que descreve a probabilidade de um evento ocorrer, baseado em um
conhecimento \emph{a priori}.

\emph{Adrien-Marie Legendre} (18/09/1752 - 10/01/1833) foi um matemático
francês. Em 1783, tornou-se membro adjunto da \emph{Academie des
Sciences}, instituição que esteve na vanguarda dos desenvolvimentos
científicos dos séculos XVII e XVIII. Fez importantes contribuições à
estatística, à teoria dos números e à álgebra abstrata.

\emph{Johann Carl Friedrich Gauss} (30/04/1777 - 23/02/1855) foi um
matemático, astrônomo e físico alemão (Figura~\ref{fig-gauss}) que
contribuiu em diversas áreas das ciências como teoria dos números,
estatística, geometria diferencial, eletrostática, astronomia e ótica.
Muitos referem-se a ele como o Príncipe da Matemática, o mais notável
dos matemáticos. Descobriu o método dos mínimos quadrados e a lei de
Gauss da distribuição normal de erros e sua curva em formato de sino,
hoje tão familiar para todos que trabalham com estatística.

\begin{figure}[h]

\centering{

\includegraphics[width=0.3\linewidth,height=\textheight,keepaspectratio]{index_files/mediabag/Pk9Il89.png}

}

\caption{\label{fig-gauss}Johann Carl Friedrich Gauss}

\end{figure}%

\emph{Lambert Adolphe Jacques Quételet} (22/02/1796 - 17/02/1874) foi um
astrônomo, matemático, demógrafo e estatístico francês. Seu trabalho se
concentrou em estatística social, criando regras de determinação de
propensão ao crime

\emph{Francis Galton} (16/02/1822 -- 17/01/1911) foi um antropólogo,
matemático e estatístico inglês. Entre muitos artigos e livros, criou o
conceito estatístico de correlação e da regressão à média. Ele foi o
primeiro a aplicar métodos estatísticos para o estudo das diferenças e
herança humanas de inteligência. Criou o conceito de eugenia e afirmava
que era possível a melhoria da espécie por seleção artificial.
Acreditava que a raça humana poderia ser melhorada caso fossem evitados
relacionamentos indesejáveis. Isto acompanhava o pensamento burguês
europeu da época. Criou a psicometria, onde desenvolveu testes de
inteligência para selecionar homens e mulheres brilhantes. Esta teoria
teve papel importante na formação do fascismo e nazismo (5).

\emph{William Farr} (30/11/1807 - 14/04/1883) foi um médico sanitarista
e estatístico inglês, nascido na vila de Kenley, Shropshire. Foi o
primeiro investigador a examinar séries temporais de morbimortalidade
para longos períodos e, assim, considerado o criador da Estatística da
Saúde Pública Moderna. Seus relatórios foram fundamentais para o
desencadeamento das reformas sanitárias britânicas, em meados e final do
século XIX (6).

\emph{Florence Nightingale} (12/05/1820 -- 13/08/1910) foi uma
enfermeira (Figura~\ref{fig-florence}) que ficou famosa por ser pioneira
no tratamento de feridos, durante a Guerra da Criméia (7). Ficou
conhecida na história pelo apelido de ``A dama da lâmpada'', pelo fato
de servir-se de uma lamparina para auxiliar no cuidado aos feridos
durante a noite. Também contribuiu no campo da Estatística, sendo
pioneira na utilização de métodos de representação visual de
informações, como por exemplo gráfico de setores (habitualmente
conhecido como gráfico do tipo ``pizza'')

\begin{figure}[h]

\centering{

\includegraphics[width=0.3\linewidth,height=\textheight,keepaspectratio]{index_files/mediabag/nHsLBsU.png}

}

\caption{\label{fig-florence}Florence Nightingale}

\end{figure}%

\emph{John Snow} (York, 15/03/1813 - Londres, 15/03/1858) foi um médico
inglês (Figura~\ref{fig-snow})), considerado pai da Epidemiologia
Moderna. Recebeu, em 1853, o título de Sir após ter anestesiado a rainha
Vitória no parto sem dor de seu oitavo filho, Leopoldo de Albany. Este
fato ajudou a divulgar a técnica entre os médicos da época. Demonstrou
que a cólera era causada pelo consumo de águas contaminadas com matérias
fecais, ao comprovar que os casos dessa doença se agrupavam em
determinados locais da cidade de Londres, em 1854, onde havia fontes
dessas águas (6).

\begin{figure}[h]

\centering{

\includegraphics[width=0.3\linewidth,height=\textheight,keepaspectratio]{index_files/mediabag/jh61uxb.png}

}

\caption{\label{fig-snow}John Snow}

\end{figure}%

\emph{Karl Pearson} (27/03/1857 - 27/04/1936) foi um importante
estatístico inglês, fundador do Departamento de Estatística Aplicada da
\emph{University College London} em 1911. Juntamente com Weldon e Galton
fundou, em 1901, a revista \emph{Biometrika} com o objetivo era
desenvolver as teorias estatísticas, editada até os dias de hoje. O
trabalho de Pearson como estatístico fundamentou muitos métodos
estatísticos de uso comum, nos dias atuais: regressão linear e o
coeficiente de correlação, teste do qui-quadrado de Pearson,
classificação das distribuições (8).

\emph{Charles Edward Spearman} (10/09/1863 - 17/09/1945) foi um
psicólogo inglês conhecido pelo seu trabalho na área da estatística,
como um pioneiro da análise fatorial e pelo coeficiente de correlação de
postos de Spearman. Ele também fez bons trabalhos de modelos da
inteligência humana.

\emph{William Sealy Gosset} (13/07/1876 - 16/10/1937) foi um químico e
estatístico inglês (Figura~\ref{fig-student})). Em 1907, enquanto
trabalhava químico da cervejaria experimental de Arthur Guinness \& Son,
criou a distribuição t que usou para identificar a melhor variedade de
cevada, trabalhando com pequenas amostras. A cervejaria Guinness tinha
uma política que proibia que seus empregados publicassem suas
descobertas em seu próprio nome. Ele, então, usou o pseudônimo
``Student'' e o teste é chamado ``t de Student'' em sua homenagem (9).

\begin{figure}[h]

\centering{

\includegraphics[width=0.3\linewidth,height=\textheight,keepaspectratio]{index_files/mediabag/Zyul6nM.png}

}

\caption{\label{fig-student}William Sealy Gosset}

\end{figure}%

\emph{Ronald Aylmer Fisher} (17/02/1890 - 29/07/1962) foi um
estatístico, biólogo e geneticista inglês. Em 1919, Fisher se envolveu
com pesquisa agrícola no centro de experimentos de \emph{Rothamsted
Research}, em Harpenden, Inglaterra, e desenvolveu novas metodologias e
teoria no ramo de experimentos (10). Durante sua vida, Fisher
(Figura~\ref{fig-fisher}) escreveu 7 livros e publicou cerca de 400
artigos acadêmicos em estatística e genética . Em um dos seus livros,
\emph{The design of Experiments} (1935), Fisher relata um experimento
que surgiu de uma pergunta curiosa: o gosto do chá muda de acordo com a
ordem em que as ervas e o leite são colocados? Essa simples questão
resultou em um estudo pioneiro na área e serviu de sustentação para
análise da aleatorização de dados experimentais (9). Ronald A. Fisher
foi descrito (11) como ``um gênio que criou praticamente sozinho os
fundamentos para o moderno pensamento estatístico''. Era muito
temperamental. Seus atritos com outros estatísticos ficaram famosos,
entre eles encontra-se ninguém menos do que Karl Pearson, outro notável
estatístico.

\begin{figure}[h]

\centering{

\includegraphics[width=0.3\linewidth,height=\textheight,keepaspectratio]{index_files/mediabag/6EVp4eC.png}

}

\caption{\label{fig-fisher}Ronald A. Fisher}

\end{figure}%

\emph{Austin Bradford Hill} (08/07/1897 - 18 /04/1991) foi um
epidemiologista e estatístico inglês (Figura~\ref{fig-hill})), pioneiro
no estudo do acaso nos ensaios clínicos e, juntamente com Richard Doll,
foi o primeiro a demonstrar a ligação entre o uso do cigarro e o câncer
de pulmão. Hill é amplamente conhecido pelos Critérios de Hill, conjunto
de critérios para a determinação de uma associação causal (12).

\begin{figure}[h]

\centering{

\includegraphics[width=0.3\linewidth,height=\textheight,keepaspectratio]{index_files/mediabag/MbHvZ7c.png}

}

\caption{\label{fig-hill}Bradford Hill}

\end{figure}%

\emph{John Wilder Tukey} (16/06/1915 - 26/07/2000) foi um estatístico
norte-americano. Desenvolveu uma filosofia para a análise de dados que
mudou a maneira de pensar dos estatísticos, sugerindo que se faça uma
visualização dos dados, interpretando o formato, centro, dispersão,
presença de valores atípicos, sumarizar numericamente e por fim escolher
um modelo matemático. Foi o criador do boxplot e introduziu a palavra
``bit'' como uma contração do termo \emph{binary digit}.

\emph{Douglas G. Altman} (12 /07/1948 - 03/06/2018) foi um estatístico
inglês (Figura~\ref{fig-altman})), conhecido por seu trabalho em
melhorar a confiabilidade dos artigos de pesquisa médica (13) e por
artigos altamente citados sobre metodologia estatística. Ele foi
professor de estatística em medicina na Universidade de Oxford. Há
praticamente 30 anos, Altman (14) escreveu um artigo sobre problema da
qualidade da pesquisa em medicina que causou um grande impacto e
permanece válido até hoje. Nesta publicação ele afirma:

\begin{quote}
A má qualidade de muitas pesquisas médicas é amplamente reconhecida,
mas, de forma perturbadora, os líderes da profissão médica parecem
apenas minimamente preocupados com o problema e não fazem nenhum esforço
aparente para encontrar uma solução.
\end{quote}

\begin{figure}[h]

\centering{

\includegraphics[width=0.3\linewidth,height=\textheight,keepaspectratio]{index_files/mediabag/X4wigwm.png}

}

\caption{\label{fig-altman}Douglas G. Altman}

\end{figure}%

\section{História resumida do R}\label{histuxf3ria-resumida-do-r}

O R é uma linguagem e um ambiente de desenvolvimento voltado
fundamentalmente para a computação estatística. Foi inspirado em duas
linguagens: S (John Chambers, do Bell Labs) que forneceu a sintaxe e
Scheme (Hal Abelson e Gerald Sussman) implementou e forneceu a
semântica.

O nome R provém em parte das iniciais dos criadores, \emph{George Ross
Ihaka} e \emph{Robert Gentleman} (Figura~\ref{fig-ross}), e também de um
jogo figurado com a linguagem S. Em 29 de Fevereiro de 2000, o software
foi considerado com funcionalidades e estável o suficiente para a versão
1.0.

O R é um projeto GNU \footnote{Esta sigla está associada ao animal gnu
  africano, símbolo de software de distribuição livre, quer dizer is Not
  Unix, sigla recursiva muito comum entre nerds!}. Software Livre
significa que os usuários têm liberdade para executar, copiar,
distribuir, estudar, alterar e melhorar o software. Foi desenvolvido em
um esforço colaborativo de pessoas em vários locais do mundo (15).

O projeto R fornece uma grande variedade de técnicas estatísticas e
gráficas. É uma linguagem e um ambiente similar ao S. A linguagem do S
que também é uma linguagem de computador voltada para cálculos
estatísticos. Um dos pontos fortes de R é a facilidade com que produções
gráficas de qualidade podem ser produzidas. O R é também altamente
expansível com o uso dos pacotes, que são bibliotecas para sub-rotinas
específicas ou áreas de estudo específicas. Um conjunto de pacotes é
incluído com a instalação de R e muito outros estão disponíveis na rede
de distribuição do R - \emph{Comprehensive R Archive Network} (CRAN)
(16).

\begin{figure}[h]

\centering{

\includegraphics[width=0.4\linewidth,height=\textheight,keepaspectratio]{index_files/mediabag/gjp0Rkc.png}

}

\caption{\label{fig-ross}Robert Gentlemen (E) e George Ross (D)}

\end{figure}%

A linguagem R é largamente usada entre estatísticos e analistas de dados
para desenvolver softwares de estatística e análise de dados. Pesquisas
e levantamentos com profissionais da área da saúde mostram que a
popularidade do R aumentou substancialmente nos últimos anos (17).

\chapter{Natureza dos Dados}\label{natureza-dos-dados}

\section{Variáveis e Dados}\label{variuxe1veis-e-dados}

As pesquisas manuseiam dados referentes às variáveis que estão sendo
estudadas. \emph{Variável} é toda característica ou condição de
interesse que pode de ser mensurada ou observada em cada elemento de uma
amostra ou população. Como o próprio nome diz, seus valores são
passíveis de variar de um indivíduo a outro ou no mesmo indivíduo. Em
contraste com a variável, o valor de uma constante é fixo. As variáveis
podem ter valores numéricos ou não numéricos. O resultado da mensuração
ou observação de uma variável é denominado \emph{dado}.

A Tabela~\ref{tbl-var} mostra um conjunto de variáveis e suas medidas
(dados) de um grupo de pacientes internados em uma determinada UTI. O
termo medida deve ser entendido num sentido amplo, pois não é possível
``medir'' o sexo (observação) ou o estado geral (critérios) de alguém,
ao contrário do peso e da pressão arterial que podem ser mensurados com
instrumentos.

\global\setlength{\Oldarrayrulewidth}{\arrayrulewidth}

\global\setlength{\Oldtabcolsep}{\tabcolsep}

\setlength{\tabcolsep}{2pt}

\renewcommand*{\arraystretch}{1.5}



\providecommand{\ascline}[3]{\noalign{\global\arrayrulewidth #1}\arrayrulecolor[HTML]{#2}\cline{#3}}

\begin{longtable}[c]{|p{0.41in}|p{0.76in}|p{0.67in}|p{0.97in}|p{0.59in}|p{0.60in}|p{1.18in}}

\caption{\label{tbl-var}Variáveis e dados}

\tabularnewline

\ascline{1.5pt}{666666}{1-7}

\multicolumn{1}{>{\centering}m{\dimexpr 0.41in+0\tabcolsep}}{\textcolor[HTML]{000000}{\fontsize{11}{11}\selectfont{\global\setmainfont{Arial}{\textbf{Id}}}}} & \multicolumn{1}{>{\centering}m{\dimexpr 0.76in+0\tabcolsep}}{\textcolor[HTML]{000000}{\fontsize{11}{11}\selectfont{\global\setmainfont{Arial}{\textbf{Nome}}}}} & \multicolumn{1}{>{\centering}m{\dimexpr 0.67in+0\tabcolsep}}{\textcolor[HTML]{000000}{\fontsize{11}{11}\selectfont{\global\setmainfont{Arial}{\textbf{Idade}}}}} & \multicolumn{1}{>{\centering}m{\dimexpr 0.97in+0\tabcolsep}}{\textcolor[HTML]{000000}{\fontsize{11}{11}\selectfont{\global\setmainfont{Arial}{\textbf{Sexo}}}}} & \multicolumn{1}{>{\centering}m{\dimexpr 0.59in+0\tabcolsep}}{\textcolor[HTML]{000000}{\fontsize{11}{11}\selectfont{\global\setmainfont{Arial}{\textbf{PAS}}}}} & \multicolumn{1}{>{\centering}m{\dimexpr 0.6in+0\tabcolsep}}{\textcolor[HTML]{000000}{\fontsize{11}{11}\selectfont{\global\setmainfont{Arial}{\textbf{PAD}}}}} & \multicolumn{1}{>{\centering}m{\dimexpr 1.18in+0\tabcolsep}}{\textcolor[HTML]{000000}{\fontsize{11}{11}\selectfont{\global\setmainfont{Arial}{\textbf{Estado\ Geral}}}}} \\

\ascline{1.5pt}{666666}{1-7}\endfirsthead 

\ascline{1.5pt}{666666}{1-7}

\multicolumn{1}{>{\centering}m{\dimexpr 0.41in+0\tabcolsep}}{\textcolor[HTML]{000000}{\fontsize{11}{11}\selectfont{\global\setmainfont{Arial}{\textbf{Id}}}}} & \multicolumn{1}{>{\centering}m{\dimexpr 0.76in+0\tabcolsep}}{\textcolor[HTML]{000000}{\fontsize{11}{11}\selectfont{\global\setmainfont{Arial}{\textbf{Nome}}}}} & \multicolumn{1}{>{\centering}m{\dimexpr 0.67in+0\tabcolsep}}{\textcolor[HTML]{000000}{\fontsize{11}{11}\selectfont{\global\setmainfont{Arial}{\textbf{Idade}}}}} & \multicolumn{1}{>{\centering}m{\dimexpr 0.97in+0\tabcolsep}}{\textcolor[HTML]{000000}{\fontsize{11}{11}\selectfont{\global\setmainfont{Arial}{\textbf{Sexo}}}}} & \multicolumn{1}{>{\centering}m{\dimexpr 0.59in+0\tabcolsep}}{\textcolor[HTML]{000000}{\fontsize{11}{11}\selectfont{\global\setmainfont{Arial}{\textbf{PAS}}}}} & \multicolumn{1}{>{\centering}m{\dimexpr 0.6in+0\tabcolsep}}{\textcolor[HTML]{000000}{\fontsize{11}{11}\selectfont{\global\setmainfont{Arial}{\textbf{PAD}}}}} & \multicolumn{1}{>{\centering}m{\dimexpr 1.18in+0\tabcolsep}}{\textcolor[HTML]{000000}{\fontsize{11}{11}\selectfont{\global\setmainfont{Arial}{\textbf{Estado\ Geral}}}}} \\

\ascline{1.5pt}{666666}{1-7}\endhead



\multicolumn{1}{>{\centering}m{\dimexpr 0.41in+0\tabcolsep}}{\textcolor[HTML]{000000}{\fontsize{11}{11}\selectfont{\global\setmainfont{Arial}{1}}}} & \multicolumn{1}{>{\centering}m{\dimexpr 0.76in+0\tabcolsep}}{\textcolor[HTML]{000000}{\fontsize{11}{11}\selectfont{\global\setmainfont{Arial}{João}}}} & \multicolumn{1}{>{\centering}m{\dimexpr 0.67in+0\tabcolsep}}{\textcolor[HTML]{000000}{\fontsize{11}{11}\selectfont{\global\setmainfont{Arial}{45}}}} & \multicolumn{1}{>{\centering}m{\dimexpr 0.97in+0\tabcolsep}}{\textcolor[HTML]{000000}{\fontsize{11}{11}\selectfont{\global\setmainfont{Arial}{masculino}}}} & \multicolumn{1}{>{\centering}m{\dimexpr 0.59in+0\tabcolsep}}{\textcolor[HTML]{000000}{\fontsize{11}{11}\selectfont{\global\setmainfont{Arial}{140}}}} & \multicolumn{1}{>{\centering}m{\dimexpr 0.6in+0\tabcolsep}}{\textcolor[HTML]{000000}{\fontsize{11}{11}\selectfont{\global\setmainfont{Arial}{90}}}} & \multicolumn{1}{>{\centering}m{\dimexpr 1.18in+0\tabcolsep}}{\textcolor[HTML]{000000}{\fontsize{11}{11}\selectfont{\global\setmainfont{Arial}{bom}}}} \\





\multicolumn{1}{>{\centering}m{\dimexpr 0.41in+0\tabcolsep}}{\textcolor[HTML]{000000}{\fontsize{11}{11}\selectfont{\global\setmainfont{Arial}{2}}}} & \multicolumn{1}{>{\centering}m{\dimexpr 0.76in+0\tabcolsep}}{\textcolor[HTML]{000000}{\fontsize{11}{11}\selectfont{\global\setmainfont{Arial}{Maria}}}} & \multicolumn{1}{>{\centering}m{\dimexpr 0.67in+0\tabcolsep}}{\textcolor[HTML]{000000}{\fontsize{11}{11}\selectfont{\global\setmainfont{Arial}{32}}}} & \multicolumn{1}{>{\centering}m{\dimexpr 0.97in+0\tabcolsep}}{\textcolor[HTML]{000000}{\fontsize{11}{11}\selectfont{\global\setmainfont{Arial}{feminino}}}} & \multicolumn{1}{>{\centering}m{\dimexpr 0.59in+0\tabcolsep}}{\textcolor[HTML]{000000}{\fontsize{11}{11}\selectfont{\global\setmainfont{Arial}{110}}}} & \multicolumn{1}{>{\centering}m{\dimexpr 0.6in+0\tabcolsep}}{\textcolor[HTML]{000000}{\fontsize{11}{11}\selectfont{\global\setmainfont{Arial}{70}}}} & \multicolumn{1}{>{\centering}m{\dimexpr 1.18in+0\tabcolsep}}{\textcolor[HTML]{000000}{\fontsize{11}{11}\selectfont{\global\setmainfont{Arial}{regular}}}} \\





\multicolumn{1}{>{\centering}m{\dimexpr 0.41in+0\tabcolsep}}{\textcolor[HTML]{000000}{\fontsize{11}{11}\selectfont{\global\setmainfont{Arial}{3}}}} & \multicolumn{1}{>{\centering}m{\dimexpr 0.76in+0\tabcolsep}}{\textcolor[HTML]{000000}{\fontsize{11}{11}\selectfont{\global\setmainfont{Arial}{Pedro}}}} & \multicolumn{1}{>{\centering}m{\dimexpr 0.67in+0\tabcolsep}}{\textcolor[HTML]{000000}{\fontsize{11}{11}\selectfont{\global\setmainfont{Arial}{27}}}} & \multicolumn{1}{>{\centering}m{\dimexpr 0.97in+0\tabcolsep}}{\textcolor[HTML]{000000}{\fontsize{11}{11}\selectfont{\global\setmainfont{Arial}{masculino}}}} & \multicolumn{1}{>{\centering}m{\dimexpr 0.59in+0\tabcolsep}}{\textcolor[HTML]{000000}{\fontsize{11}{11}\selectfont{\global\setmainfont{Arial}{120}}}} & \multicolumn{1}{>{\centering}m{\dimexpr 0.6in+0\tabcolsep}}{\textcolor[HTML]{000000}{\fontsize{11}{11}\selectfont{\global\setmainfont{Arial}{80}}}} & \multicolumn{1}{>{\centering}m{\dimexpr 1.18in+0\tabcolsep}}{\textcolor[HTML]{000000}{\fontsize{11}{11}\selectfont{\global\setmainfont{Arial}{grave}}}} \\





\multicolumn{1}{>{\centering}m{\dimexpr 0.41in+0\tabcolsep}}{\textcolor[HTML]{000000}{\fontsize{11}{11}\selectfont{\global\setmainfont{Arial}{4}}}} & \multicolumn{1}{>{\centering}m{\dimexpr 0.76in+0\tabcolsep}}{\textcolor[HTML]{000000}{\fontsize{11}{11}\selectfont{\global\setmainfont{Arial}{Teresa}}}} & \multicolumn{1}{>{\centering}m{\dimexpr 0.67in+0\tabcolsep}}{\textcolor[HTML]{000000}{\fontsize{11}{11}\selectfont{\global\setmainfont{Arial}{18}}}} & \multicolumn{1}{>{\centering}m{\dimexpr 0.97in+0\tabcolsep}}{\textcolor[HTML]{000000}{\fontsize{11}{11}\selectfont{\global\setmainfont{Arial}{feminino}}}} & \multicolumn{1}{>{\centering}m{\dimexpr 0.59in+0\tabcolsep}}{\textcolor[HTML]{000000}{\fontsize{11}{11}\selectfont{\global\setmainfont{Arial}{100}}}} & \multicolumn{1}{>{\centering}m{\dimexpr 0.6in+0\tabcolsep}}{\textcolor[HTML]{000000}{\fontsize{11}{11}\selectfont{\global\setmainfont{Arial}{60}}}} & \multicolumn{1}{>{\centering}m{\dimexpr 1.18in+0\tabcolsep}}{\textcolor[HTML]{000000}{\fontsize{11}{11}\selectfont{\global\setmainfont{Arial}{bom}}}} \\

\ascline{1.5pt}{666666}{1-7}


\end{longtable}

\arrayrulecolor[HTML]{000000}

\global\setlength{\arrayrulewidth}{\Oldarrayrulewidth}

\global\setlength{\tabcolsep}{\Oldtabcolsep}

\renewcommand*{\arraystretch}{1}

\section{População e Amostra}\label{populauxe7uxe3o-e-amostra}

Na pesquisa em saúde, a não ser quando se realiza um censo, coleta-se
dados de um subconjunto de indivíduos denominado de \emph{amostra},
pertencente a um grupo maior, conhecido como \emph{população}. A
população de interesse é, geralmente, chamada de população-alvo. A
amostra, para ser representativa da população, deve ter as mesmas
características desta. A partir da análise dos dados encontrados na
amostra, deduz-se sobre a população. Este processo é denominado de
\textbf{inferência estatística}. O interesse na amostra não está
propriamente nela, mas na informação que ela fornece ao investigador
sobre a população de onde ela provém. A amostra fornece estimativas
(estatísticas) da população (Figura~\ref{fig-pop}).

\begin{quote}
\textbf{População} ou \textbf{população-alvo} consiste em todos os
elementos (indivíduos, itens, objetos) cujas características estão sendo
estudadas.\\
\textbf{Amostra} é a parte, subconjunto, da população selecionada para
estudo.
\end{quote}

Em decorrência do acaso, diferentes amostras de uma mesma população
fornecem resultados diferentes. Este fato deve ser levado em
consideração ao usar uma amostra para fazer inferência sobre uma
população. Este fenômeno é denominado de \textbf{variação amostral} ou
\textbf{erro amostral} (Consulte também o \textbf{?@sec-cap9}) e é a
essência da estatística. O grau de certeza na inferência estatística
depende da representatividade da amostra.

O processo de obtenção da amostra é chamado de \textbf{amostragem}.
Mesmo que este processo seja adequado, a amostra nunca será uma cópia
perfeita da população de onde ela foi extraída. Desta forma, em qualquer
conclusão baseada em dados de uma amostra, sempre haverá o erro
amostral. Este erro deve ser tratado estatisticamente tendo em mente a
teoria da amostragem, baseada em probabilidades.

\begin{figure}

\centering{

\pandocbounded{\includegraphics[keepaspectratio]{index_files/mediabag/kEsGwnk.png}}

}

\caption{\label{fig-pop}População, amostra e inferência estatística}

\end{figure}%

\section{Estatística e Parâmetro}\label{estatuxedstica-e-paruxe2metro}

\textbf{Estatística} é uma característica que resume os dados de uma
amostra e o \textbf{parâmetro} é uma característica estabelecida da
população. Os valores dos parâmetros são normalmente desconhecidos,
porque, na maioria das vezes, é inviável medir uma população inteira. A
estatística é um valor aproximado, uma estimativa, do parâmetro. As
estatísticas são representadas por letras romanas\footnote{Também podem
  ser representadas pela letra grega correspondente ao respectivo
  parâmetro com um acento circunflexo, por exemplo, a média amostral é
  \(\hat{\mu}\), dita (mü chapéu).} e os parâmetros por letras gregas.
Por exemplo, a media da população é representada por \(\mu\) e a média
da amostra por \(\bar{x}\); o desvio padrão da população é denotado
\(\sigma\) e o desvio padrão da amostra por \emph{s}.

\section{Escalas de medição}\label{escalas-de-mediuxe7uxe3o}

Em um estudo científico, há necessidade de registrar os dados para que
eles representem acuradamente as variáveis observadas. Este registro de
valores necessita de escalas de medição. \textbf{Mensuração} ou
\textbf{medição} é o processo de atribuir números ou rótulos a objetos,
pessoas, estados ou eventos de acordo com regras específicas para
representar quantidades ou qualidades dos dados. Para a mensuração das
variáveis são usadas as escalas nominal, ordinal, intervalar e de razão
(18).

\subsection{Escala Nominal}\label{escala-nominal}

As escalas nominais são meramente classificativas, permitindo descrever
as variáveis ou designar os sujeitos, sem recurso à quantificação. É o
nível mais elementar de representação. São usados nomes, números ou
outros símbolos para designar a variável. Os números, quando usados,
representam códigos e como tal não permitem operações matemáticas. As
variáveis nominais não podem ser ordenadas. Podem apenas ser comparadas
utilizando as relações de igualdade ou de diferença, através de
\textbf{contagens}. Os números atribuídos às variáveis servem como
identificação, ou para associá-la a uma dada categoria. As categorias de
uma escala nominal são exaustivas e mutuamente exclusivas. Quando
existem duas categorias, a variável é dita \textbf{dicotômica} e com
três ou mais categorias, \textbf{politômicas}.

Os nomes e símbolos que designam as categorias podem ser intercambiáveis
sem alterar a informação essencial.

Exemplos: Tipos sanguíneos: A, B, AB, O; variáveis dicotômicas:
morto/vivo, homem/mulher, sim/não; cor dos olhos, etc.

\subsection{Escala Ordinal}\label{escala-ordinal}

As variáveis são medidas em uma escala ordinal quando ocorre uma ordem,
crescente ou decrescente, inerente entre as categorias, estabelecida sob
determinado critério. A diferença entre as categorias não é
necessariamente igual e nem sempre mensuráveis. Geralmente, designam-se
os valores de uma escala ordinal em termos de numerais ou postos
(\emph{ranks}), sendo estes apenas modos diferentes de expressar o mesmo
tipo de dados. Também não faz sentido realizar operações matemática com
variáveis ordinais. Pode-se continuar a usar contagem.

Exemplos: classe social (baixa, média, alta); estado geral do paciente:
bom, regular, mau; estágios do câncer: 0, 1, 2, 3 e 4; escore de Apgar:
0, 1, 2\ldots{} 10.

\subsection{Escala Intervalar}\label{escala-intervalar}

Uma escala intervalar contém todas as características das escalas
ordinais com a diferença de que se conhece as distâncias entre quaisquer
números. Em outras palavras, existe um espectro ordenado com intervalos
quantificáveis. Este tipo de escala permite que se verifique a ordem e a
diferença entre as variáveis, porém não tem um zero verdadeiro, o zero é
arbitrário.

O exemplo clássico é a mensuração da temperatura, usando as escalas de:
Celsius ou Fahrenheit. Aqui é legítimo ordenar, fazer soma ou médias. No
entanto, 0ºC não significa ausência de temperatura, portanto a operação
divisão não é possível. Uma temperatura de 40ºC não é o dobro de 20ºC.
Se 40ºC e 20ºC forem transformados para a escala Fahrenheit, passarão,
respectivamente, para 104ºF e 68ºF e, sem dúvida, 104 não é o dobro de
68!

\subsection{Escala de Razão}\label{escala-de-razuxe3o}

Há um espectro ordenado com intervalos quantificáveis como na escala
intervalar. Entretanto, as medidas iniciam a partir de um zero
verdadeiro e a escala tem intervalos iguais, permitindo as comparações
de magnitude entre os valores. Refletem a quantidade real de uma
variável, permitindo qualquer operação matemática.

Os dados tanto na escala intervalar como na de razão, podem ser
contínuos ou discretos. Dados contínuos necessitam de instrumentos para
a sua mensuração e assumem qualquer valor em um certo intervalo. Por
exemplo, o tempo para terminar qualquer tarefa pode assumir qualquer
valor, 10 min, 20 min, 35 min, etc., de acordo com o tipo de tarefa.
Outros exemplos: peso, dosagem de colesterol, glicemia.

Dados discretos possuem valores iguais a números inteiros, não existindo
valores intermediários. A mensuração é feita através da contagem. Por
exemplo: número de filhos, número de fraturas, número de pessoas.

\section{Tipos de Variáveis}\label{sec-tipovariavel}

A primeira etapa na descrição e análise dos dados é classificar as
variáveis, pois a apresentação dos dados e os métodos estatísticos
variam de acordo com os seus tipos. As variáveis, primariamente, podem
ser divididas em dois tipos: numéricas ou quantitativas e categóricas ou
qualitativas (19).

\subsection{Variáveis Numéricas}\label{variuxe1veis-numuxe9ricas}

As variáveis numéricas são classificadas em dois tipos de acordo com a
escala de mensuração: continuas e discretas.

As \textbf{variáveis contínuas} são aquelas cujos dados foram mensurados
em uma escala intervalar ou de razão, podendo assumir, como visto,
qualquer valor dentro de um intervalo de números reais, dependendo da
precisão do instrumento de medição. O tratamento estatístico tanto para
variável intervalar como de a razão é o mesmo. A diferença entre elas
está na presença do zero absoluto. As variáveis numéricas contínuas têm
unidade de medida. Por exemplo, um menino de 4 anos tem 104 cm.

Uma variável numérica é considerada \textbf{discreta} quando é apenas
possível quantificar os resultados possíveis através do processo de
contagem. Também têm unidade de medida -- \emph{número de elementos}.
Por exemplo, o número de fraturas, o número de acidentes, etc.

\subsection{Variáveis Categóricas}\label{variuxe1veis-categuxf3ricas}

As variáveis categóricas ou qualitativas são de dois tipos: nominal e
ordinal, de acordo com a escala de mensuração. Um tipo particularmente
comum é uma variável binária (ou variável dicotômica), que tem apenas
dois valores possíveis. Por exemplo, o sexo é masculino ou feminino.
Este tipo de variável é bastante utilizado na área da saúde, em
Epidemiologia. As variáveis nominais não têm quaisquer unidades de
medida e a nominação das categorias é completamente arbitrária e
pertencer a uma categoria não significa ter maior importância do que
pertencer à outra. Uma variável ordinal tem uma ordem inerente ou
hierarquia entre as categorias. Do mesmo modo que as variáveis nominais,
as variáveis ordinais não têm unidades de medida. Entretanto, a
ordenação das categorias não é arbitrária. Assim, é possível ordená-las
de modo lógico. Um exemplo comum de uma variável categórica ordinal é a
classe social, que tem um ordenamento natural da maioria dos mais
desfavorecidos para os mais ricos. As escalas, como a escore de Apgar e
a escala de coma de Glasgow (20), também são variáveis ordinais. Mesmo
que pareçam numéricas, elas apenas mostram uma ordem no estado dos
pacientes. O escore de Apgar (21) é uma escala, desenvolvida para a
avaliação clínica do recém-nascido imediatamente após o nascimento.
Originalmente, a escala foi usada para avaliar a adaptação imediata do
recém-nascido à vida extrauterina. A pontuação pode variar de zero a 10.
Uma pontuação igual ou maior do que oito, indica um recém-nascido
normal. Uma pontuação de sete ou menos pode significar depressão do
sistema nervoso e abaixo de quatro, depressão grave.

As variáveis ordinais, da mesma forma que as nominais, não são números
reais e não convém aplicar as regras da aritmética básica para estes
tipos de dados. Este fato gera uma limitação na análise dos dados.

\subsection{Como identificar o tipo da
variável?}\label{como-identificar-o-tipo-da-variuxe1vel}

A maneira mais fácil de dizer se os dados são numéricos é verificar se
eles têm unidades ligadas a eles, tais como: g, mm, ºC, ml, número de
úlceras de pressão, número de mortes e assim por diante. Se não, podem
ser ordinais ou nominais -- ordinais se os valores podem ser colocados
em ordem. A Figura~\ref{fig-caminho} é uma ajuda para o reconhecimento
do tipo de variável (22).

\begin{figure}

\centering{

\pandocbounded{\includegraphics[keepaspectratio]{index_files/mediabag/4s9Ln2w.png}}

}

\caption{\label{fig-caminho}Caminho para identificar o tipo de variável}

\end{figure}%

\subsection{Variáveis Dependentes e
Independentes}\label{variuxe1veis-dependentes-e-independentes}

De um modo geral as pesquisas são realizadas para testar as hipóteses
dos pesquisadores e, para isso, eles medem variáveis com a finalidade de
compará-las. A maioria das hipóteses podem ser expressas por duas
variáveis: uma variável explicativa ou preditora e uma variável desfecho
(19).

A \textbf{variável preditora} ou explanatória é a que se acredita ser a
causa e também é conhecida como variável independente, porque o seu
valor não depende de outras variáveis. Em Epidemiologia, é com
frequência referida como exposição ou fator de risco.

A \textbf{variável desfecho} é aquela que é o efeito, consequência ou
resultado da ação de outra variável, por isso, também chamada de
variável dependente. Em um estudo que tenta verificar se o tabagismo,
durante a gestação, pode interferir no peso do recém-nascido, tem o fumo
(variável categórica) como variável preditora (exposição ou fator de
risco) e o peso do recém-nascido (variável numérica contínua) como
variável desfecho

Na maioria dos estudos, são utilizadas amostras que fornecem estimativas
que, para serem representativas da população, devem ser probabilísticas.
Ou seja, a amostra deve ser recrutada de forma aleatória, permitindo que
cada um dos membros da população tenha a mesma probabilidade de ser
incluído na amostra. Além disso, uma amostra deve ter um tamanho
adequado para permitir inferências válidas.

\chapter{Produção dos Dados}\label{sec-producao}

\section{Processo de Pesquisa}\label{processo-de-pesquisa}

A pesquisa é um processo de construção do conhecimento. O objetivo deste
processo é gerar um novo conhecimento e/ou confirmar ou refutar algum
conhecimento prévio. A pesquisa é um processo de aprendizagem tanto do
pesquisador quanto da sociedade que se beneficiará deste novo
conhecimento. Para ser chamada de científica, a pesquisa deve obedecer
aos princípios consagrados pela ciência (23).

A pesquisa nasce de uma dúvida do pesquisador, de algum questionamento
que ele considerou interessante sobre o mundo, ou seja, de algo que se
costuma chamar de pergunta ou questão da pesquisa. Existem vários
motivos que geram questões de pesquisa:

\begin{itemize}
\tightlist
\item
  Avaliação crítica de pesquisas realizadas por outros pesquisadores.
\item
  Condução de uma pesquisa primária com a finalidade de responder uma
  questão (ou questões), gerando um novo conhecimento ou ampliação do
  conhecimento existente.
\item
  Para obter habilidades de pesquisa ou experiência, com frequência como
  parte de um programa educacional.
\item
  Testar a viabilidade de um projeto ou técnica de pesquisa.
\end{itemize}

\subsection{Questão de Pesquisa}\label{questuxe3o-de-pesquisa}

A pesquisa visa estabelecer novos conhecimentos em torno de um tema
específico. O tema de pesquisa pode surgir do próprio interesse ou
experiência do pesquisador, ou partir da encomenda de alguma instituição
financiadora. Algumas vezes, a pesquisa se origina de outros estudos
realizados pelo próprio pesquisador ou outros pesquisadores.

À medida que a ideia da pesquisa cresce, o pesquisador estabelece uma
pergunta de pesquisa específica ou um conjunto de questões que ele
deseja responder. Algumas vezes, o tema da pesquisa é tão amplo que o
pesquisador tem que ter cuidado para não se perder do seu objetivo. Este
objetivo é que vai guiá-lo no estabelecimento da pergunta ou perguntas a
serem respondidas no estudo. Estes questionamentos são conhecidos como
\textbf{questão de pesquisa} ou \textbf{pergunta de partida}.

O foco da questão de pesquisa pode ser na descrição de um fenômeno
clínico. Neste caso a pergunta é dita \textbf{descritiva}, por exemplo,
pesquisa de prevalência de uma enfermidade, proporção de utilização de
um serviço de saúde, características de um teste, etc. Quando a pergunta
busca a explicação para um fenômeno, ela é dita \textbf{analítica}, por
exemplo, comparação entre dois fenômenos. Em geral, perguntas analíticas
são mais interessantes. Entretanto, as perguntas descritivas são
fundamentais no início de um estudo analítico.

Uma boa pergunta de pesquisa deve ter as seguintes características (24):

\begin{itemize}
\tightlist
\item
  \textbf{Factível}: o pesquisador deve conhecer desde o início os
  limites e problemas práticos que podem interferir na pesquisa. A
  viabilidade está relacionada com o tamanho amostral, com o domínio
  técnico adequado, com o tempo e custos envolvidos e com um foco
  dirigido estritamente aos objetivos mais importantes.
\item
  \textbf{Interessante}: a questão de pesquisa deve despertar o
  interesse não apenas do pesquisador, mas também de seus pares e
  agentes financiadores.
\item
  \textbf{Nova}: a pesquisa deve ser inovadora, original, em algum
  sentido, para que o estudo seja uma contribuição ao conhecimento ou
  amplie um conhecimento existente;
\item
  \textbf{Ética}: se o estudo impõe riscos físicos ou invasão de
  privacidade ou não traz nenhuma informação nova, o pesquisador deve
  suspendê-lo. É importante discutir previamente com pesquisadores mais
  experientes ou com algum representante do Comitê de Ética em Pesquisa
  da instituição.
\item
  \textbf{Relevante}: nenhuma das características da questão de pesquisa
  é mais importante do que a sua relevância. Para isto basta pensar nos
  benefícios que os resultados da pesquisa trarão à Medicina atual.
\end{itemize}

Ou seja, antes de dedicar tempo e esforço para escrever um projeto de
pesquisa deve-se avaliar se a questão de pesquisa é FINER (Factível,
Interessante, Nova, Ética e Relevante).

\subsection{Hipótese de Pesquisa}\label{hipuxf3tese-de-pesquisa}

Uma vez estabelecida a(s) pergunta(s) de pesquisa adequada(s), os
pesquisadores formulam hipóteses para serem testadas. Enquanto a
pergunta de pesquisa possa ser um pouco vaga em sua natureza como:
``existe uma relação entre o tipo psicológico e a capacidade de parar de
usar drogas?'' Uma hipótese de pesquisa, necessita ser precisa. Há
necessidade de especificar qual o tipo psicológico está relacionado à
habilidade de parar de usar drogas.

A precisão da hipótese é fundamental em um projeto de pesquisa, pois ela
determinará o delineamento de pesquisa a ser seguido pelo pesquisador e
as técnicas estatísticas apropriadas para a análise dos dados. A fonte e
o tipo de dados são determinados pela característica do delineamento
recomendado pela hipótese de pesquisa.

O objetivo da pesquisa, usando o método científico, é refutar ou não as
hipóteses de pesquisa. Se a hipótese do pesquisador não for rejeitada,
houve a geração de um novo conhecimento.

\section{Processo de Amostragem}\label{processo-de-amostragem}

Após o estabelecimento das hipóteses a serem testadas, há necessidade de
coletar os dados. Uma vez que é praticamente impossível analisar toda a
população que constitui a \textbf{população-alvo}, extrai-se uma
\textbf{amostra} desta população. Este processo é denominado de
\textbf{amostragem} (25).

Uma amostra deve ser representativa da população, ou seja, deve ter
características semelhantes às da população e ser fidedigna. A
fidedignidade está relacionada à precisão dos dados que sofrem
influência dos instrumentos de aferição, questionários não validados e
falhas humanas. Uma amostra inadequada ameaça a validade da pesquisa. Os
dados coletados de maneira não aleatória são chamados de
\textbf{evidência anedótica}. O nível de confiança nos resultados de uma
pesquisa está diretamente relacionado à qualidade da amostra. A amostra
deve ser representativa.

Uma amostra deve conter apenas dados úteis que permitam a resposta da
pergunta de pesquisa, evitando desperdício e fuga dos objetivos
traçados. A aleatoriedade provoca uma diferença entre o resultado da
amostra e o verdadeiro valor da população que é denominada \textbf{erro
amostral}. Não importa quão bem a amostra seja coletada, os erros
amostrais irão sempre ocorrer. Entretanto, não existe técnica
estatística que salve amostras coletadas incorretamente, tendenciosas!

\subsection{Amostras probabilística}\label{amostras-probabiluxedstica}

Para evitar vieses, erros sistemáticos, que favorecem determinados
desfechos, o ideal é coletar uma amostra probabilística. A amostra
probabilística adota o princípio da \textbf{equiprobabilidade}, isto é,
``todos os sujeitos da população têm a mesma probabilidade de fazerem
parte da amostra''. Esta probabilidade é conhecida e diferente de zero.
As amostras probabilísticas têm o potencial de ser possível a
generalização para a população; ser imparcial e com menor erro amostral.

\textbf{Amostra aleatória simples}: é a mais utilizada pois garante
representatividade da amostra junto à população. A amostra aleatória
simples não emprega nenhum critério particular para a definição da
amostra. O mecanismo mais comum de obter este tipo de amostra é por um
simples sorteio, em geral, usando programas de computador.

\textbf{Amostra aleatória estratificada}: quando a população é
constituída por subpopulações ou estratos e é razoável supor que a
variável de interesse apresenta comportamento diferente nos diferentes
estratos, pode-se usar este tipo de amostragem. Neste caso, a amostra
deve ter a mesma estratificação da população para ser representativa. Um
exemplo comum de estratificação é o nível socioeconômico. A partir do
momento que os estratos estão definidos se procede uma amostra aleatória
simples de cada estrato.

\textbf{Amostra aleatória sistemática}: as unidades amostrais são
selecionadas a partir de um esquema rígido preestabelecido de
sistematização que tem o propósito de abranger toda a população-alvo.
Para isso, ordena-se os indivíduos da população (por exemplo, um grande
arquivo com 20000 fichas) e calcula-se uma constante conveniente,
\(c = N/n\), onde \(N\) é tamanho da população e \(n\) é o tamanho da
amostra. Se \(n = 500\), a constante será \(40\), ou seja, será
selecionado aleatoriamente o primeiro membro da amostra (\(k\)), de
maneira que \(k\) seja menor do que a constante e maior do que \(1\). A
partir daí os sucessivos membros serão: \(k + c\) ; \(k + 2c\) ;
\(k + 3c\) ; \ldots{} até atingir \(n\).

\textbf{Amostra aleatória por conglomerados (\emph{clusters})}: este
tipo de amostra é utilizada quando dentro da população são identificados
agrupamentos (\emph{clusters}) naturais, por exemplo, espaços, vilas,
etc. Neste tipo de amostragem o elemento focal não é o sujeito, mas o
\emph{cluster}. Identificados estes, sorteiam-se os conglomerados e se
analisa todos os indivíduos dos conglomerados sorteados.

\subsection{Amostras não
probabilísticas}\label{amostras-nuxe3o-probabiluxedsticas}

Na amostragem não aleatória ou intencionada há uma escolha deliberada da
amostra, subordinada a objetivos específicos do pesquisador. Não há
garantia de representatividade da população. É importante averiguar,
neste tipo de amostragem, a presença de \emph{conflitos de interesse}.

\textbf{Amostra de conveniência}: é uma técnica comum onde é selecionada
uma mostra que esteja acessível. Em outras palavras, os indivíduos são
recrutados porque eles estão prontamente disponíveis. Neste tipo de
amostra há incapacidade de fazer afirmações gerais com rigor estatístico
sobre a população.

\textbf{Amostra por cotas}: é uma versão não probabilística da amostra
estratificada. Tem três etapas:

\begin{enumerate}
\def\labelenumi{\arabic{enumi})}
\tightlist
\item
  Segmentação, onde se divide em grupos, por exemplo, sexo, classe
  social, região, etc.;
\item
  Definição do tamanho das cotas;
\item
  Seleção por meio de amostras de conveniência.
\end{enumerate}

\textbf{Amostra de resposta voluntária}: o pesquisador solicita aos
membros de uma população-alvo para que eles participem da amostra e as
pessoas decidem se entram ou não. Esses tipos de amostras são enviesados
porque as pessoas podem ter interesses particulares ou opiniões
negativas e tendem a querer participar.

\subsection{Tamanho amostral}\label{tamanho-amostral}

A determinação do tamanho de uma amostra é de suma importância, pois
amostras desnecessariamente grandes acarretam desperdício de tempo e de
dinheiro e amostras muito pequenas podem levar a resultados não
confiáveis, ameaçando a validade da pesquisa.

Não existe um número estabelecido para o tamanho da amostra. Há uma
solução para cada caso. O tamanho da amostra depende (26):

\begin{itemize}
\tightlist
\item
  do tipo de problema;
\item
  do tipo de variável;
\item
  da magnitude do erro estatístico aceito pelo pesquisador;
\item
  da diferença minimamente importante entre os grupos;
\item
  da probabilidade de que a amostra identifique uma diferença
  verdadeira: Poder estatístico;
\item
  do tempo, dinheiro e pessoal disponível, bem como da dificuldade em se
  obterem dados e da complexidade da pesquisa.
\end{itemize}

O tamanho amostral mínimo é determinado por fórmulas estatísticas
complexas. Os cálculos são muito pesados, mas agora, felizmente, existem
programas de computador disponíveis que realizam este trabalho, por
exemplo o \textbf{G-Power3} (27). Além disso, é possível acessar um site
que fornece informações e ferramentas para o cálculo amostral em
pesquisas da área da saúde \footnote{\url{http://calculoamostral.bauru.usp.br/calculoamostral/index.php}}.
Existem tabelas extensas para calcular o número de participantes (28)
para um determinado nível de poder (e vice-versa).

\section{Principais Delineamentos de
Pesquisa}\label{principais-delineamentos-de-pesquisa}

Em geral, a pesquisa clínica, é dividida em dois tipos de investigação.
O primeiro é aquele em que o observador apenas observa o doente, as
características da sua doença e sua evolução, sem atuar de modo a
modificar qualquer aspecto que esteja estudando. Trata-se de
\textbf{estudo observacional}.

O segundo corresponde aos \textbf{estudos experimentais}, onde o
pesquisador não se limita a observar, mas promove uma intervenção com o
objetivo de conhecer os efeitos dessa sobre os participantes da
pesquisa. A intervenção pode ser a prescrição de um medicamento, uma
dieta, atividade física ou repouso, ou simplesmente, o estabelecimento
de um programa de atenção à saúde.

Os estudos podem ser também classificados em primários ou secundários ou
integrativos (29). Estudos primários correspondem a pesquisas originais
que constituem a maioria das publicações encontradas nas revistas
médicas. Estudos secundários são aqueles que procuram sumarizar e
extrair conclusões de estudos primários

\begin{itemize}
\tightlist
\item
  \emph{Estudos Primários}

  \begin{itemize}
  \tightlist
  \item
    Estudos Observacionais

    \begin{itemize}
    \tightlist
    \item
      Relato de Caso e Série de Casos
    \item
      Estudo Transversal
    \item
      Estudo Caso-controle
    \item
      Estudo de Coorte
    \end{itemize}
  \item
    Estudos Experimentais

    \begin{itemize}
    \tightlist
    \item
      Experimento laboratorial
    \item
      Ensaio Clínico
    \end{itemize}
  \end{itemize}
\item
  \emph{Estudos Secundários}

  \begin{itemize}
  \tightlist
  \item
    Revisões não sistemáticas
  \item
    Revisões Sistemáticas
  \item
    Direrizes (\emph{Guidelines})
  \item
    Análise de decisão
  \item
    Análise Econômica
  \end{itemize}
\end{itemize}

\subsection{Elementos básicos de um delineamento de
pesquisa}\label{elementos-buxe1sicos-de-um-delineamento-de-pesquisa}

Os estudos contêm três elementos básicos:

\begin{enumerate}
\def\labelenumi{\arabic{enumi}.}
\tightlist
\item
  \ul{Variáveis componentes}: Nas investigações das relações entre as
  variáveis identificam-se pelo menos duas variáveis nos estudos
  epidemiológicos.

  \begin{enumerate}
  \def\labelenumii{\alph{enumii}.}
  \tightlist
  \item
    \emph{Desfecho}: Aquilo que vai acontecer durante uma investigação
    na mensuração da condição de saúde-doença. Sinônimo: variável
    dependente.
  \item
    \emph{Exposição}: O fator que precede o desfecho. Sinônimos: fator
    em estudo, variável preditora, variável independente.
  \end{enumerate}
\item
  \ul{Temporalidade}: Quanto ao tempo os estudos podem ser
  contemporâneos, retrospectivos e prospectivos, de acordo como os dados
  são obtidos em relação ao momento atual.
\item
  \ul{Enfoque}: Um estudo pode ter vários enfoques. Na maioria deles, na
  área médica, eles relacionam-se à prevenção, ao diagnóstico, à
  terapêutica e ao prognóstico.
\end{enumerate}

\section{Estudos Observacionais}\label{estudos-observacionais}

\subsection{Relato de Caso ou Série de
casos}\label{relato-de-caso-ou-suxe9rie-de-casos}

No relato de caso, descrevem-se casos raros, eventos não comuns ou
inesperados, doenças desconhecidas ou raras. Um evento notável deve ser
identificado. Um relato de caso tem a descrição de até dez casos. Acima
deste número tem-se uma série de casos (30).

Metodologicamente, faz-se um relato descritivo simples de
características interessantes observadas em um paciente ou grupo de
pacientes. Os indivíduos são acompanhados em um espaço de tempo curto e
não possuem participantes-controles. A coleta dos dados é, na maioria
das vezes, retrospectiva.

Uma série de casos não é planejada e não envolve quaisquer hipóteses
investigativas. Pode ser empregada como precursor de outros estudos.

\subsection{Estudos Transversais ou
Seccionais}\label{estudos-transversais-ou-seccionais}

Os estudos transversais são também conhecidos como estudos seccionais.
Este tipo de estudo fornece a informação sobre a prevalência, ou seja, a
proporção dos indivíduos que tem a doença ou condição clínica em um
determinado momento. Por este motivo são também conhecidos como estudos
de prevalência (31).

Observam dados coletados em um grupo de indivíduos em um único momento,
sem um período de seguimento. O desfecho e exposição são avaliados no
mesmo momento no tempo. Os dados são coletados apenas uma vez para cada
indivíduo, podendo ser em dias diferentes em diferentes sujeitos. As
informações são, em geral, obtidas em um curto espaço de tempo.

É um estudo estático, representa a ``fotografia'' de um momento.
Entretanto, se as variáveis preditora e de desfecho são definidas apenas
com base nas hipóteses causa-efeito do investigador e não no
delineamento do estudo, é possível também examinar associações.

Os estudos de corte transversal, de um modo geral, são desenhados para
determinar \emph{``O que está acontecendo?''}. São usados para:

\begin{itemize}
\tightlist
\item
  Determinar a prevalência de uma doença, como a prevalência de HIV em
  gestantes.
\item
  Pesquisar atitudes ou opiniões em relação a um determinado assunto
  (pesquisa de satisfação)
\item
  Verificar interrelações entre variáveis, como observação das
  características de fumantes pesados em relação ao sexo, idade, etc.
\item
  Enquetes
\end{itemize}

\subsubsection{Cuidados na interpretação de dados de estudos
transversais}\label{cuidados-na-interpretauxe7uxe3o-de-dados-de-estudos-transversais}

\begin{enumerate}
\def\labelenumi{\arabic{enumi})}
\tightlist
\item
  \emph{Efeito temporal}
\end{enumerate}

Como os dados (exposição e desfecho) são coletados no mesmo momento,
fica difícil estabelecer qualquer relação temporal entre eles (dilema
ovo/galinha). Por exemplo, não é possível estabelecer uma relação de
causalidade entre hipertensão e doença cardíaca se os dados são
coletados de forma a ficar impossível saber que surgiu em primeiro
lugar.

\begin{enumerate}
\def\labelenumi{\arabic{enumi})}
\setcounter{enumi}{1}
\tightlist
\item
  \emph{Estudos transversais repetidos}
\end{enumerate}

Os estudos transversais, algumas vezes, são repetidos em outro momento
ou em outros locais com a finalidade de verificar variabilidade nos
achados. Por exemplo, medir a prevalência de uma doença em momentos
diferentes ou em diferentes locais. Os indivíduos serão um pouco
diferentes, devendo-se interpretar as diferenças destes resultados com
cautela.

\begin{enumerate}
\def\labelenumi{\arabic{enumi})}
\setcounter{enumi}{2}
\tightlist
\item
  \emph{Estudos transversais que parecem longitudinais}
\end{enumerate}

Uma armadilha comum é confundir um estudo seccional com um longitudinal
porque os dados foram coletados através do tempo até completar o tamanho
amostral previsto. O importante é que os dados (variável preditora e
desfecho) foram coletados somente uma vez para cada indivíduo e no mesmo
momento. Isto gera uma interpretação errônea se analisarmos como um
estudo longitudinal.

\subsubsection{Análise dos Estudos
Transversais}\label{anuxe1lise-dos-estudos-transversais}

Quando se compara a prevalência de doença em expostos e não expostos, a
medida de associação usada é a \emph{Razão de Prevalência Pontual}
(RPP).

\subsection{Estudos Caso-Controle}\label{estudos-caso-controle}

Para examinar a possível associação de uma exposição a uma determinada
doença, identifica-se um grupo de doentes (casos) e, com a finalidade de
comparação, um grupo de pessoas sem a doença (controles) e determina-se
a chance (\emph{odds}) de exposição e não exposição entre casos e entre
controles.

Os estudos caso-controle, portanto, partem da presença ou ausência de um
desfecho e após olham para trás no tempo (retrospectivamente) para
detectar possíveis fatores de risco (Figura~\ref{fig-cc})) (32).
Analisam o que aconteceu e são usados para investigar fatores de risco
de doenças raras onde um estudo prospectivo seria muito longo para
identificar uma quantidade suficiente de casos.

É útil também para investigar surtos agudos (infecção alimentar) para
identificar se existe ou não associação entre a exposição e o desfecho
investigado. Com frequência, os estudos caso-controle são o primeiro
passo na busca de uma etiologia quando há suspeita de que alguma de
várias exposições esteja associada a uma determinada doença.

\begin{figure}[H]

\centering{

\includegraphics[width=0.8\linewidth,height=\textheight,keepaspectratio]{index_files/mediabag/LX0yUuP.png}

}

\caption{\label{fig-cc}Desenho de um estudo caso controle.}

\end{figure}%

\subsubsection{Seleção dos casos}\label{seleuxe7uxe3o-dos-casos}

Os casos podem ser selecionados de várias fontes, incluindo indivíduos
hospitalizados, de consultórios ou clínicas, principalmente quando
registros adequados são mantidos.

Muitos problemas podem ocorrer na seleção de casos, neste tipo de
estudo. Se os casos forem selecionados de um único hospital, quaisquer
fatores de risco identificados podem ser apenas daquele hospital, em
decorrência do padrão de referência e nível de atendimento (um hospital
terciário que apenas atende um determinado convênio, por exemplo, o
Sistema Único de Saúde). Por isso, devem ser utilizados casos
procedentes de vários hospitais da comunidade, pois aí os casos
pertenceriam a diferentes grupos sociais e diferentes graus de gravidade
da doença.

\ul{Casos incidentes ou prevalentes}

Os casos usados nos estudos caso-controle podem ser casos incidentes
(recém-diagnosticados) ou casos prevalentes da doença (pessoas que
apresentaram a doença em algum período).

O problema do uso de casos incidentes é que há necessidade de se esperar
que novos casos sejam diagnosticados e isto pode requerer muito tempo.
Enquanto os casos prevalentes já estão disponíveis havendo um maior
número disponível para o estudo. Em ambos os modelos existem problemas,
pois nos casos prevalentes algumas pessoas podem morrer logo após o
diagnóstico e estarem pouco representadas no estudo. Por outro lado, nos
casos incidentes, serão excluídos os pacientes que morreram antes do
diagnóstico ser feito. Não existe uma solução fácil para este problema,
mas é importante lembrar-se destas questões ao interpretar os resultados
e tirar conclusões do estudo.

\subsubsection{Seleção dos controles}\label{seleuxe7uxe3o-dos-controles}

Da mesma forma do que nos estudos experimentais, a escolha dos controles
afeta a comparação com os casos (33). A escolha dos controles inclui:

\begin{itemize}
\tightlist
\item
  Pacientes do mesmo hospital, mas com condições ou doenças não
  relacionadas;
\item
  Pacientes pareados um a um em relação a fatores prognósticos, tais
  como sexo e idade;
\item
  Uma amostra aleatória originária da mesma população de onde provêm os
  casos.
\end{itemize}

Sem dúvida, o melhor grupo controle é a terceira opção, mas esta é
raramente possível. Por este motivo, alguns estudos caso-controle
incluem mais de um grupo controle para tornar o estudo mais robusto

\ul{Controles pareados}

O emparelhamento é definido como processo de seleção dos controles para
que sejam semelhantes aos casos em algumas características como, por
exemplo, idade, gênero, raça, condição socioeconômica e ocupação.

Controles emparelhados são bastante comuns. O autor deve ter o cuidado
de especificar cuidadosamente o modo como houve o pareamento. Por
exemplo, ``emparelhado por idade dentro de dois anos'' mostra a
amplitude do pareamento. É difícil realizar o emparelhamento para muitos
fatores, pois um pareamento seguro não existe. Em um delineamento
pareado, a análise estatística deve levar em conta o emparelhamento e os
fatores usados por ele. Onde um indivíduo em um par tiver um dado
perdido, ambos devem ser omitidos da análise estatística.

\subsubsection{Estudos caso-controle
aninhados}\label{estudos-caso-controle-aninhados}

Um delineamento do tipo caso-controle aninhado é um estudo de
caso-controle '\,'aninhado'' em um estudo de coorte (34). É um excelente
desenho para variáveis preditoras que são caras para medir e que podem
ser avaliadas no final do estudo em indivíduos que desenvolvem o
resultado durante o estudo (casos) e em uma amostra daqueles que não o
fazem (controles).

O investigador começa com uma coorte adequada
(Figura~\ref{fig-aninhado}) (35) com casos suficientes ao final do
acompanhamento para fornecer poder adequado para responder à pergunta de
pesquisa. No final do estudo, aplica critérios que definem o resultado
de interesse para identificar todos aqueles que desenvolveram o
resultado (casos). Em seguida, seleciona uma amostra aleatória dos
indivíduos que não desenvolveram o resultado (controles).

A principal razão para usar delineamentos caso-controle aninhado é
reduzir o trabalho e o custo na coleta de dados. A principal desvantagem
desse projeto é que muitas questões e circunstâncias da pesquisa não são
passíveis de armazenamento para posterior análise.

\begin{figure}[H]

\centering{

\includegraphics[width=0.8\linewidth,height=\textheight,keepaspectratio]{index_files/mediabag/YvOf6Kb.png}

}

\caption{\label{fig-aninhado}Desenho de um estudo caso-controle
aninhado.}

\end{figure}%

\subsubsection{Estudo caso-controle de base
populacional}\label{estudo-caso-controle-de-base-populacional}

São os estudos caso-controle onde os casos e controles são uma amostra
completa ou probabilística de uma população definida.

\subsubsection{Limitações dos estudos
caso-controle}\label{limitauxe7uxf5es-dos-estudos-caso-controle}

Várias limitações podem afetar os estudos caso-controle:

\begin{itemize}
\tightlist
\item
  A escolha do grupo controle afeta as comparações entre casos e
  controles;
\item
  Os dados da exposição ao fator de risco são coletados
  retrospectivamente e dependem da memória dos participantes, registros
  médicos e, portanto, podem ser incompletos, sem acurácia ou enviesados
  (viés de memória);
\item
  Se o processo que conduz à identificação dos casos está relacionado a
  um possível fator de risco, a interpretação dos resultados será
  difícil (viés averiguação).

  \begin{itemize}
  \tightlist
  \item
    Por exemplo: suponha que os casos sejam mulheres jovens com
    hipertensão selecionadas de uma clínica de contracepção. Nesta
    situação, um possível fator de risco, o anticoncepcional oral (ACO),
    estará vinculado à seleção dos casos e, desta forma, o uso de ACO
    será mais comum entre os casos do que entre os controles
    populacionais.
  \end{itemize}
\end{itemize}

\subsubsection{Análise dos Estudos
Caso-controle}\label{anuxe1lise-dos-estudos-caso-controle}

A principal estratégia de análise é o cálculo da \emph{odds ratio}
(Razão de Chances), que pode ser interpretado como uma estimativa do
Risco Relativo.

O Risco Relativo somente pode ser calculado quando é possível o cálculo
da incidência (ver \textbf{?@sec-rr}). Nos estudos caso-controle, isso
não é possível, pois aqui o estudo começa com casos e controles em vez
de indivíduos expostos e não expostos ao fator de risco. Desta maneira,
se comparam as \emph{odds} (chance) de uma exposição passada a um fator
de risco suspeitado em indivíduos doentes e em controles não doentes.
Esta relação é denominada de \emph{odds ratio} (ver \textbf{?@sec-or}).

\subsection{Estudos de Coorte}\label{estudos-de-coorte}

Os estudos de coorte são considerados o padrão-ouro dos estudos
observacionais. Seu nome se originou das coortes dos soldados romanos,
cada uma delas constituída por 480 a 600 legionários. As coortes romanas
eram distintas entre si e tinham sua identidade determinada por, ao
menos, uma característica comum entre os indivíduos de cada grupo. Podia
ser por características estratégicas no campo de batalha, por uma cor
presente na indumentária, ou outras. Em Epidemiologia, o termo coorte
permaneceu com significado semelhante.

Em um estudo de coorte, um grupo de pacientes sadios (coorte), expostos
ou não a um suspeitado fator de risco, é seguido através do tempo para
determinar a incidência da doença em questão em cada um dos grupos (36).

Neste modelo de estudo, a característica comum aos dois grupos é a
exposição. Tem-se uma coorte de expostos e uma coorte de não expostos
que são acompanhadas por um período de tempo que permita o aparecimento
do desfecho. No final do estudo, compara-se a incidência do desfecho
(doença) entre os expostos com a incidência do desfecho entre os não
expostos. Se existe uma associação positiva entre a exposição e o
desfecho, se espera que a incidência do desfecho entre os expostos seja
maior do que a incidência de desfecho entre não expostos.

Um esquema simplificado de um estudo de coorte é mostrado na
Figura~\ref{fig-coorte} (37).

\begin{figure}[H]

\centering{

\includegraphics[width=0.8\linewidth,height=\textheight,keepaspectratio]{index_files/mediabag/SF9kiX7.png}

}

\caption{\label{fig-coorte}Desenho de um estudo de coorte sobre risco.}

\end{figure}%

Observar que como se identifica novos casos (incidência) à medida que
eles ocorrem, é possível determinar uma relação temporal entre a
exposição e a doença, isto é, se a exposição precedeu o início da
doença. Isto é \emph{fundamental para estabelecer uma relação causal
entre a exposição e a doença}.

Os estudos de coorte têm semelhança com os ensaios clínicos
randomizados. Ambos os estudos comparam grupos expostos a grupos não
expostos. Não havendo possibilidade de realizar a randomização, por
exemplo, por motivos éticos quando a exposição é sabidamente
prejudicial, é indicado um estudo de coorte. A diferença fundamental,
portanto, é a ausência de randomização nos estudos de coorte.

Existem duas maneiras básicas para formar os grupos:

\begin{enumerate}
\def\labelenumi{\arabic{enumi})}
\tightlist
\item
  Seleciona-se a população-alvo baseado no fato dos indivíduos estarem
  expostos ou não ao fator em estudo (Figura~\ref{fig-coorte});
\item
  Ou seleciona-se a população-alvo antes que qualquer um dos seus
  membros se torne exposto, ou antes, que a exposição seja identificada
  (Figura~\ref{fig-coorte2}). Um exemplo típico deste modelo é o
  clássico Estudo de Framingham (38).
\end{enumerate}

\begin{figure}[h]

\centering{

\includegraphics[width=0.8\linewidth,height=\textheight,keepaspectratio]{index_files/mediabag/1UgEuXR.png}

}

\caption{\label{fig-coorte2}Desenho de uma coorte com grupos expostos e
não expostos. (39).}

\end{figure}%

\subsubsection{Tipos de estudo de
coorte}\label{tipos-de-estudo-de-coorte}

De acordo com as características do seguimento, as coortes podem ser:

\begin{enumerate}
\def\labelenumi{\arabic{enumi})}
\item
  \textbf{Estudo de Coorte Prospectivo} (Coorte Concorrente ou
  Longitudinal), onde os grupos são montados no presente, coletados os
  dados basais deles e continua-se a coletar dados com o passar do tempo
  até a doença se desenvolver ou não.
\item
  \textbf{Estudo de Coorte Retrospectivo ou Histórico} (Coorte não
  concorrente), onde a exposição é avaliada em dados passados e o
  desfecho (doença ou não) é verificado no momento do início do estudo.
  O problema aqui é que a averiguação da exposição depende dos registros
  pregressos.
\item
  \textbf{Estudo de Coorte Misto} (Prospectivo e Retrospectivo), onde a
  exposição é verificada em registros objetivos no passado (como em uma
  coorte histórica) e o seguimento e a medida do desfecho se fazem no
  futuro.
\end{enumerate}

\subsubsection{Vieses em estudos de
coorte}\label{vieses-em-estudos-de-coorte}

Os potenciais vieses nos estudos de coorte são os seguintes:

\begin{enumerate}
\def\labelenumi{\arabic{enumi})}
\item
  \textbf{Viés de confusão} -- é a grande ameaça dos estudos
  observacionais. O confundimento causa um erro sistemático na
  inferência, podendo aumentar ou diminuir uma associação observada
  entre exposição e doença. Uma variável funciona como fator de confusão
  quando ela está associada com a exposição e ao mesmo tempo com a
  doença. Ela não deve fazer parte da cadeia causal da exposição à
  doença. Por exemplo, num estudo sobre fatores de risco, uma associação
  entre o hábito de beber café e a doença coronária é detectada. Porém,
  se não for considerado o fato de que os fumantes bebem mais café do
  que os não-fumantes, pode-se chegar à errônea conclusão de que o café
  é um fator de risco independente para doença coronária, o que não
  corresponde à realidade. Neste caso, o café é um fator de confusão e
  não um fator causal independente para a doença coronária (40).
\item
  \textbf{Viés na avaliação dos desfechos} -- este viés pode ocorrer
  quando o pesquisador que avalia o desfecho também sabe sobre o
  \emph{status} de exposição dos sujeitos da pesquisa. Evita-se este
  problema ``cegando'' a pessoa que faz a avaliação da doença.
\item
  \textbf{Viés de informação} -- ocorrem principalmente em estudos
  históricos onde as informações dependem de registros passados e podem
  ser diferentes entre as pessoas expostas e não expostas.
\item
  \textbf{Viés de não resposta e perdas de acompanhamento} -- a não
  participação e as perdas podem introduzir um grande viés, alterando o
  cálculo da incidência nos expostos e entre os não expostos.
\item
  \textbf{Viés de análise} -- se os estatísticos tiverem alguma hipótese
  em relação aos dados que estão analisando, eles podem introduzir
  vieses em suas análises.
\end{enumerate}

\subsubsection{Análise dos estudos de
coorte}\label{anuxe1lise-dos-estudos-de-coorte}

Para verificar se existe associação entre certo desfecho (doença) e uma
determinada exposição calcula-se o \textbf{Risco Relativo} (RR). Este é
definido como a razão entre a incidência (risco) em expostos e a
incidência (risco) em não expostos (ver \textbf{?@sec-rr}).

\subsubsection{Vantagens e desvantagens dos estudos de
coorte}\label{vantagens-e-desvantagens-dos-estudos-de-coorte}

\begin{enumerate}
\def\labelenumi{\arabic{enumi})}
\item
  Vantagens

  \begin{itemize}
  \tightlist
  \item
    Adequado para exposições raras
  \item
    Bom poder para testar hipóteses
  \item
    Importante em estudos etiológicos e prognósticos
  \item
    Salienta os múltiplos desfechos de uma exposição
  \end{itemize}
\item
  Desvantagens

  \begin{itemize}
  \tightlist
  \item
    Inadequado em desfechos raros
  \item
    Perdas no seguimento levam a viés de seleção
  \item
    Demorado/elevado custo
  \end{itemize}
\end{enumerate}

\section{Ensaios Clínicos}\label{sec-trials}

Experimentos são estudos nos quais o pesquisador \emph{manipula a
variável preditora} (intervenção) e observa o efeito no desfecho que
está sendo avaliado ao longo do tempo. A abordagem experimental,
especificamente, o ensaio clínico randomizado controlado é a ferramenta
de escolha para comparar terapêuticas ou intervenções.

Os estudos experimentais podem também comparar os cuidados prestados por
serviços de saúde, programas de educação em saúde e estratégias
administrativas. Os estudos experimentais realizados com seres humanos
são denominados de \textbf{ensaios clínicos}.

Nos ensaios clínicos não controlados os indivíduos servem como seus
próprios controles (antes-e-depois). Os resultados destes estudos estão
sujeitos vários problemas:

\begin{itemize}
\item
  \textbf{Melhora previsível}. Paciente melhora espontaneamente e não
  pelo tratamento.
\item
  \textbf{Flutuação na gravidade da doença}.
\item
  \textbf{Efeito Hawthorne}: o indivíduo melhora pela atenção e não pela
  terapêutica (41).
\item
  \textbf{Regressão à média}: uma limitação importante surge quando se
  quer avaliar a evolução de um grupo que tenha sido selecionado por
  estar no extremo de uma distribuição sem que haja um grupo controle.
  Empiricamente, observa-se que indivíduos que se encontrem num
  determinado momento, em um dos extremos de uma distribuição, tendem a
  estarem menos distantes da média em um momento posterior, sem que
  qualquer intervenção tenha sido desenvolvida. Este fenômeno é
  conhecido como efeito de \emph{regressão à média}. Por exemplo: uma
  pessoa com uma doença crônica tem dias piores e outros melhores. Se
  ela é medicada com gotas homeopáticas ou faz uso de florais nos dias
  em que se sente excepcionalmente mal vai notar que é frequente uma
  melhora, seguindo estes ``tratamentos''. Não que eles funcionem, mas
  pela regressão à média (42).
\end{itemize}

\subsection{Características de um ensaio
clínico}\label{caracteruxedsticas-de-um-ensaio-cluxednico}

Um ensaio clínico deve ter algumas características fundamentais
(Figura~\ref{fig-trial}) (43):

\begin{enumerate}
\def\labelenumi{\arabic{enumi})}
\item
  Os indivíduos devem ser designados por randomização para os grupos de
  comparação.

  \begin{itemize}
  \tightlist
  \item
    A randomização é a melhor abordagem no delineamento de um ensaio
    clínico (44).
  \item
    Randomizar significa sortear (por meio de computadores, tábua de
    números aleatórios) os indivíduos para decidir a alocação dos mesmos
    em um dos grupos de estudo. O elemento decisivo da randomização é a
    imprevisibilidade da próxima alocação.
  \end{itemize}
\item
  O pesquisador compara o grupo de estudo com um grupo controle
  apropriado.
\item
  O investigador manipula a variável independente (preditora).
\end{enumerate}

\begin{figure}[H]

\centering{

\includegraphics[width=0.8\linewidth,height=\textheight,keepaspectratio]{index_files/mediabag/kv9IYVe.png}

}

\caption{\label{fig-trial}Estrutura de um ensaio clínico randomizado.}

\end{figure}%

\subsection{Elementos básicos de um ensaio
clínico}\label{elementos-buxe1sicos-de-um-ensaio-cluxednico}

\begin{enumerate}
\def\labelenumi{\arabic{enumi})}
\tightlist
\item
  \ul{Seleção dos participantes}
\end{enumerate}

Os pesquisadores devem determinar e explicar detalhadamente os critérios
de inclusão e de exclusão:

\begin{itemize}
\tightlist
\item
  Objetivos dos critérios de inclusão e exclusão

  \begin{itemize}
  \tightlist
  \item
    Restringir a heterogeneidade da amostra
  \item
    Diminuir o número de variáveis independentes
  \item
    Fazer com que exista uma chance maior de que as diferenças nos
    desfechos estejam relacionadas aos tratamentos
  \item
    Melhorar a \emph{validade interna}, ou seja, o grau em que os
    resultados do estudo são consistentes para aquela amostra particular
    de indivíduos. Esta validade depende basicamente do rigor
    metodológico usado para delinear o ensaio clínico, podendo ser
    ameaçada por dois tipos de erros: sistemático ou aleatório.\\
  \item
    Tornar a generalização (validade externa) mais precisa. Entretanto
    deve-se ter cuidado com critérios de inclusão e exclusão muito
    rígidos, pois podem diminuir esta capacidade de generalização
  \end{itemize}
\end{itemize}

O grau de detalhamento deve ser suficientemente preciso para permitir
que outros reproduzam o estudo. O tamanho da amostra deve ser claramente
determinado pelo poder do teste estatístico. Poder é a habilidade de o
teste estatístico detectar diferenças entre os grupos, dado que tais
diferenças existam na população em estudo. Lembrar que resultados não
significativos podem ser apenas uma evidência para um inadequado tamanho
amostral.

O grupo controle deve ser selecionado utilizando-se os mesmos critérios
do grupo experimental. Prestar atenção em possíveis armadilhas que podem
gerar vieses:

\begin{itemize}
\tightlist
\item
  Uso de grupo controle histórico (não concorrente);
\item
  Grupo controle selecionado de outros locais (outras clínicas, outros
  hospitais).
\end{itemize}

O grupo controle adequado é um grupo controle concorrente, tratado no
mesmo momento e no mesmo local do grupo experimental. O característico é
o grupo controle não receber tratamento. Mais comumente recebem um
placebo, indistinguível do tratamento experimental, mas sem componente
ativo. Mesmo assim, pode haver melhora dos participantes do grupo
controle (Efeito Placebo ) (45). Quando não for ético suspender o
tratamento e administrar placebo, o grupo controle pode ser constituído
por indivíduos que recebem o tratamento padrão.

\begin{enumerate}
\def\labelenumi{\arabic{enumi})}
\setcounter{enumi}{1}
\tightlist
\item
  \ul{Alocação}
\end{enumerate}

A alocação deve ser aleatória. A randomização é a principal técnica para
reduzir o viés, criando grupos homogêneos. Como foi visto, é uma das
características fundamentais dos ensaios clínicos. O poder da
randomização depende da ocultação da sequência de alocação.

A randomização pode ser:

\begin{itemize}
\item
  \textbf{Completa}: os indivíduos que obedecem ao critério de inclusão
  e exclusão são randomizados de modo que todos têm a mesma
  probabilidade de pertencer a cada um dos grupos. Isto maximiza o
  poder. Pode ser feita por blocos para assegurar a igualdade numérica
  dos grupos (estudos multicêntricos).
\item
  \textbf{Estratificada}: os participantes são estratificados de acordo
  com possíveis variáveis de confusão (gravidade da doença, idade, sexo,
  etc.) e a randomização é realizada dentro de cada estrato.
\item
  \textbf{Randomização e alocação desigual}: os sujeitos têm uma maior
  probabilidade de ser randomizados em um grupo (em geral, grupo
  experimental) do que o outro (comparação). Este tipo de estudo tem
  menor poder.
\end{itemize}

\begin{enumerate}
\def\labelenumi{\arabic{enumi})}
\setcounter{enumi}{2}
\tightlist
\item
  \ul{Condução/Seguimento/Avaliação}
\end{enumerate}

Em um ensaio clínico deve estar assegurado de que o estudo tenha um
tempo de seguimento adequado, pois nem todos os indivíduos participam
conforme o plano original. Podem ocorrer perdas de alguns pacientes
durante o acompanhamento, seja porque com o tempo se constata que eles
não têm a doença em estudo ou porque não aderiram ao tratamento ou
intervenção e abandonaram o estudo. Quanto maior o número de pacientes
perdidos e menos informações sobre eles, menos confiança pode ser
colocada nos resultados do estudo. De um modo geral, não se deve tolerar
perdas que sejam maiores que a incidência do desfecho no estudo. Uma
regra simples é que perdas menores que 5\% produzem pouco viés e perdas
maiores que 20\% são uma ameaça importante à validade do estudo. As
perdas entre 5 e 20\% devem ser avaliadas com cuidado, se possível
utilizando-se uma análise de sensibilidade (pior cenário),
principalmente se as perdas forem diferentes nos grupos pelo maior risco
de viés.

Neste tipo de análise, nos estudos com resultado positivo, todos os
pacientes perdidos no grupo experimental, inicialmente, são considerados
como tendo o desfecho. Posteriormente, analisa-se como se nenhum dos
indivíduos perdidos no grupo controle atingiu o desfecho. Se o resultado
permanecer positivo, as perdas não afetaram a validade do estudo.
Estudos sem relato adequado ou nenhum relato de perdas ou exclusões
devem ser avaliados com muito cuidado.

Outro aspecto importante, no seguimento dos sujeitos da pesquisa, é o
tratamento igual de todos os grupos. Para garantir este princípio,
utiliza-se da técnica de \textbf{cegamento} ou \textbf{mascaramento}
(46). Esta técnica impede que os participantes da pesquisa
(pesquisadores, avaliadores e participantes) tomem conhecimento de qual
grupo de tratamento o participante se encontra. Este conhecimento
antecipado pode influenciar as expectativas, as opiniões e as crenças em
relação aos resultados do estudo. O cegamento tem como principal
finalidade a eliminação do viés de aferição, além de melhorar a adesão
ao tratamento, reduzir as perdas de seguimento e diminuir o viés causado
por co-intervenções (assistência suplementar maior para um dos grupos).

Quando o cegamento ocorre nos pacientes e nos pesquisadores, diz-se que
o estudo é \textbf{duplo-cego}. Se ele também incluir os avaliadores do
estudo, ele é \textbf{triplo cego}. Um ensaio clínico em que não há
cegamento é dito aberto (\emph{open label}, no caso de estudos com
fármacos).

A avaliação dos desfechos também pode afetar os resultados. É importante
garantir-se que aqueles que registram os desfechos estejam cegados em
relação a que grupo o sujeito da pesquisa pertence. Os autores devem
estabelecer regras cuidadosas para decidir se um desfecho ocorreu ou não
e despender esforços iguais para identificar desfechos para todos os
pacientes no estudo.

\begin{enumerate}
\def\labelenumi{\arabic{enumi}.}
\setcounter{enumi}{3}
\tightlist
\item
  \ul{Intenção de tratar}
\end{enumerate}

Os pesquisadores violam a randomização se omitirem da análise os
pacientes que não receberam a intervenção designada ou, pior ainda,
contarem eventos que ocorreram nos sujeitos não aderentes que foram
designados para a intervenção contra o grupo controle. Os sujeitos de
uma pesquisa, para evitar tal viés, devem ser analisados dentro do grupo
para o qual eles foram alocados pela randomização (47). Este princípio é
denominado \textbf{intenção de tratar}.

\begin{enumerate}
\def\labelenumi{\arabic{enumi})}
\setcounter{enumi}{4}
\tightlist
\item
  \ul{Análise da magnitude do efeito}
\end{enumerate}

Calcula-se uma série de estimativas quantitativas para analisar a
magnitude do efeito da intervenção em um ensaio clínico. Entre elas,
destacam-se o \textbf{Risco Relativo}, \textbf{Redução Relativa do
Risco}, \textbf{Número Necessário para Tratar} que serão estudados no
capítulo @ref(sec-cap18).

Outro método para avaliar resultados de um ensaio clínico para dados de
tempo até o evento é a \textbf{análise de sobrevida}. Esta fornece
informação sobre a rapidez com que os eventos ocorrem. A curva de
sobrevida pode utilizar dados de pacientes acompanhados por diferentes
períodos de tempo.

\subsection{Ensaios clínicos de equivalência e não
inferioridade}\label{ensaios-cluxednicos-de-equivaluxeancia-e-nuxe3o-inferioridade}

Ensaios clínicos controlados com placebo são ideais para avaliar a
eficácia de um tratamento. Eles permitem o controle do efeito placebo e
são mais eficientes, exigindo um menor número de pacientes para detectar
um efeito do tratamento. Um ensaio clínico placebo controlado é
eticamente justificado se não existe tratamento padrão, se o tratamento
padrão não se mostrou eficaz, não há riscos associados com o retardo no
tratamento e se a possiblidade de se retirar do estudo está incluída no
protocolo. Sempre que possível e justificado, os ensaios clínicos
placebo controlados devem ser a primeira escolha para avaliação de um
tratamento.

Dado que um grande número de tratamentos eficazes comprovados está
disponível, ensaios clínicos controlados por placebo são, muitas vezes,
antiéticos. Nestas situações, ensaios clínicos com controle ativo são
geralmente apropriados.

Se o objetivo do ensaio clínico é testar se um novo tratamento é similar
em eficácia a um tratamento já existente, ele é denominado de
\textbf{Estudo de Equivalência}. O Ensaio Clínico é delineado de maneira
que possa demonstrar que, dentro limites aceitáveis, os dois tratamentos
são igualmente eficazes. Existe equivalência quando a diferença
observada entre os dois tratamentos for menor que a máxima diferença
aceitável, determinada previamente. Estes limites devem ser clinicamente
apropriados. Se condição em investigação for muito grave, os limites
para a equivalência devem ser estreitados. Quanto menor forem os limites
de equivalência, maior o tamanho amostral. Este delineamento é útil se o
novo tratamento trouxer benefícios, tais como menores efeitos
colaterais, facilidade no uso e ser mais barato.

Em muitos estudos com controle ativo, os pesquisadores desejam comprovar
que o tratamento em estudo, no mínimo, não é substancialmente pior que o
tratamento controle. Estes estudos são chamados de \textbf{Estudos de
Não Inferioridade}. Um aspecto importante do delineamento e da
interpretação desses estudos é a determinação da margem de não
inferioridade. Os estudos de não inferioridade devem demonstrar, pelo
menos, que o tratamento em estudo tem alguma eficácia, não inferior ao
tratamento padrão. A análise dos estudos de não inferioridade é, por
natureza, unidirecional.

Quando um ensaio clínico busca evidenciar que um tratamento é melhor do
que outro ele é denominado \textbf{Estudos de Superioridade}. Quando o
ensaio clínico é delineado, ele deve ter uma hipótese bilateral e o
tamanho da amostra definido de maneira que haja alto poder estatístico
para detectar uma diferença clinicamente significativa entre os dois
tratamentos. Os ensaios clínicos clássicos têm esta característica.
Entretanto, nos dias atuais, este desenho de estudo pode não ser
eticamente possível, uma vez que é pouco provável que não exista um
tratamento com algum benefício comprovado. A comparação, portanto,
deverá ser feita com o tratamento já existente, provando que o
tratamento em estudo é similar ou, pelo menos, não seja inferior (48).

\subsection{Outros tipos de ensaios
clínicos}\label{outros-tipos-de-ensaios-cluxednicos}

\subsubsection{Ensaio clínico com delineamento
cruzado}\label{ensaio-cluxednico-com-delineamento-cruzado}

No delineamento cruzado (\emph{crossover design}), os sujeitos da
pesquisa são randomizados para um grupo e depois mudados para o outro
grupo (Figura~\ref{fig-cruzado}). Cada sujeito serve como seu próprio
controle, diminuindo a variabilidade intragrupo, aumentando o poder e
consequentemente, reduzindo o erro \(\beta\) (erro que ocorre quando a
análise estatística dos dados não consegue rejeitar uma hipótese, no
caso desta hipótese ser falsa). É um tipo de delineamento bastante
atrativo e útil (49).

A maior desvantagem é o efeito residual (\emph{carryover}), por isso os
estudos cruzados devem ter um período de \emph{washout}, período sem
nenhum tratamento. Este período de tempo deve ser suficiente para a
eliminação da droga para se ter certeza de que nenhum efeito da terapia
permaneceu. Também pode haver um viés de acordo com a ordem de
administração das terapias, pois os pacientes podem reagir de modo
diferente como resultado do entusiasmo no início do tratamento que pode
diminuir com o tempo.

\begin{figure}[H]

\centering{

\includegraphics[width=0.45\linewidth,height=\textheight,keepaspectratio]{index_files/mediabag/DDKu8iT.png}

}

\caption{\label{fig-cruzado}Ensaio clínico randomizado com delineamento
cruzado.}

\end{figure}%

\subsubsection{Delineamento Fatorial}\label{delineamento-fatorial}

Uma variação interessante de ensaio clínico é o \emph{delineamento
fatorial}. Este tipo de estudo permite que sejam testadas duas drogas em
apenas um estudo, assumindo que os desfechos antecipados para as duas
são diferentes e que seus modos de ação são independentes. Este desenho
de estudo gera economia.

Um exemplo de delineamento fatorial é observado no \emph{Physician's
Health Study} onde usando um delineamento fatorial 2 x 2 foi testada a
aspirina para a prevenção primária de doença cardiovascular (50), e
betacaroteno para a prevenção primária de câncer.

No estudo da prevenção primária do câncer, os autores concluíram, após
12 anos de suplementação de betacaroteno, que o mesmo não produziu nem
benefícios e nem prejuízos em termos de incidência de câncer (51).

\subsection{Fases de um ensaio
clínico}\label{fases-de-um-ensaio-cluxednico}

Para a realização de um ensaio clínico, a intervenção deve passar por
várias fases (52).

\subsubsection{Fase Não Clínica}\label{fase-nuxe3o-cluxednica}

Antes de começar a testar novos tratamentos em seres humanos, os
cientistas testam as substâncias em laboratórios (in vitro) e em animais
de experimentação. O objetivo principal desta fase é verificar como esta
substância se comporta em um organismo. Assim, após esta fase se pode
verificar se o medicamento é seguro para ser testado em seres humanos.
Todo este processo é regido por leis da bioética em pesquisa em animais.

\subsubsection{Fase Clínica}\label{fase-cluxednica}

A fase clínica é a fase de testes em seres humanos. Esta etapa é
constituída por quatro fases consecutivas e somente depois de
finalizadas todas as fases, a droga poderá ser autorizada para
comercialização e disponibilizada para uso em seres humanos. As
sucessivas fases dentro da fase clínica são:

\begin{itemize}
\item
  \textbf{Fase I} - Um estudo de fase I testa a droga pela primeira vez.
  O objetivo principal é avaliar a segurança do produto investigado.
  Nesta fase, o medicamento é testado em pequenos grupos (10 -- 30
  pessoas), geralmente, de voluntários sadios. Podemos ter exceções se
  estivermos avaliando medicamentos para câncer ou portadores de
  HIV-AIDS. Se a droga se mostrar segura, é possível ir para a Fase II.
\item
  \textbf{Fase II} - Nesta fase, o número de pacientes é maior (70 -
  100). O objetivo é avaliar a eficácia da medicação, isto é, se ela
  funciona para tratar determinada doença, e também conseguir
  informações mais detalhadas sobre a segurança (toxicidade). Somente se
  os resultados forem bons é que o medicamento será estudado como um
  estudo clínico fase III.
\item
  \textbf{Fase III} - Nesta fase, o novo tratamento é comparado com o
  tratamento padrão existente. São os ensaios clínicos. O número de
  pacientes aumenta e depende da hipótese (em geral, 100 a 1.000). Devem
  de preferência utilizar desfechos clínicos, grupo controle, além de
  serem randomizados e duplo-cegos.
\item
  \textbf{Fase IV} - Estes estudos são realizados para se confirmar que
  os resultados obtidos na fase III são aplicáveis a grande parte dos
  doentes. Nesta fase, o medicamento já foi aprovado para ser
  comercializado. A vantagem dos estudos fase IV é que eles permitem
  acompanhar os efeitos dos medicamentos em longo prazo. É uma fase de
  vigilância pós-comercialização.
\end{itemize}

\part{Parte II - Ambiente Computacional com R}

\chapter{Introdução ao uso do R}\label{introduuxe7uxe3o-ao-uso-do-r}

\section{Instalação do R básico}\label{instalauxe7uxe3o-do-r-buxe1sico}

Para usar o R, há necessidade de carregar o programa básico que contém a
sua linguagem de programação. O sistema é formado por um programa
básico, \emph{Graphical User Interface} (R-Gui) e muitos pacotes com
procedimentos adicionais.

O \href{https://www.r-project.org}{\textbf{site}} oficial do R fornece
as versões atualizadas do software e informações sobre este sofisticado
projeto de computação estatística.

Para baixar o R, usa-se um ``CRAN Mirror'', clicando em CRAN
(\emph{Comprehensive R Archive Network}) na margem esquerda, abaixo de
\emph{Download}. O CRAN é central no uso do R: é o local de onde se
carrega o software e todos os pacotes necessários para instalar e para
expandir o R.

Em vez de ter um único local, o CRAN é ``espelhado'' em diferentes
locais do mundo. ``Espelhado'' significa simplesmente que existem
versões idênticas do CRAN distribuídas por todo o mundo. É possível
baixar o R diretamente da
\href{https://cloud.r-project.org}{\textbf{nuvem}} ou escolher uma
origem mais próxima do seu local de atuação. No Brasil, encontram-se
várias opções, como a
\href{https://cran-r.c3sl.ufpr.br}{\textbf{Universidade Federal do
Paraná}}, \href{https://cran.fiocruz.br/}{\textbf{Fundação Oswaldo Cruz,
RJ}}, \href{https://vps.fmvz.usp.br/CRAN}{\textbf{Universidade de São
Paulo, São Paulo}} e
\href{https://brieger.esalq.usp.br/CRAN}{\textbf{Universidade de São
Paulo, Piracicaba}}

Após escolher uma das alternativas acima (pode ser qualquer uma delas)
surgirá a página \emph{The Comprehensive R Archive Network} com as
opções para escolher o sistema operacional. Escolha o sitema de acordo
com o seu computador (Windows, macOS ou Linux). Ao clicar em uma dessas
opções, se o sistema operacional escolhido é o Windows, aparecerá a
página \emph{R for Windows}. Nesta, deve-se clicar em \emph{base}. No
caso de outros sistemas operacionais, seguir as orientações mostradas no
site do R.

Clicando em \emph{base}, haverá um redirecionamento para a a página onde
aparece a versão do R para o Windows mais atual. Clique no link que diz
\emph{Download R-\ldots for Window} para baixar o instalador em um
diretório do computador, em geral \emph{Downloads}.

Para instalar o programa básico, basta executar o instalador
\emph{R-\ldots-win.exe} baixado no diretório. Ao fazer isso, aparece na
tela do computador,no canto esquerdo, em baixo, o arquivo salvo. Execute
este arquivo com um clique sobre ele. Aparecerá u,a janela perguntando
\emph{``Deseja permitir que este aplicativo faça alterações no seu
dispositivo?''}. Clique em \emph{Sim.} A seguir o instalador pedirá para
escolher o Idioma. Selecione Português Brasileiro.

Em sequência aparecerão informações sobre o diretório no qual o R será
instalado em seu computador. Recomenda-se aceitar a configuração padrão
sugerida pelo instalador do software.

A próxima janela pedirá para personalizar os componentes que serão
instalados. Recomenda-se usar as configurações sugeridas pelo instalador
que irá reconhecer automaticamente a arquitetura do seu sistema Windows
(32 e/ou 64 bits).

A partir daqui, siga as recomendações padrão propostas pelo instalador
até completar a instalação, clicando em \emph{Concluir}.

O R não precisa ser iniciado, pois o software que será usado, neste
livro, é o \emph{RStudio}. Este, para ser executado, necessita ter o R
instalado no computador. Ou seja, o R é o programa ``cérebro''
necessário para as análises de dados que serão realizadas. Ele precisa
estar instalado para permitir o funcionamento do \emph{RStudio}.

\section{Ambiente de desenvolvimento}\label{ambiente-de-desenvolvimento}

\subsection{RStudio}\label{rstudio}

Um ambiente de desenvolvimento integrado (IDE - \emph{Integrated
Development Environment}) é uma ferramenta que facilita a escrita,
execução e depuração de código.\\
O \emph{RStudio} é o ambiente de desenvolvimento integrado mais
utilizado com o R. Ele serve para facilitar a escrita, execução e
depuração de código R, bem como para gerenciar projetos, visualizar
dados e criar gráficos. O \emph{RStudio} é um membro ativo da comunidade
R. Foi fundado em 2009 por Joseph J. Allaire, engenheiro de software
americano. O \emph{RStudio}, inspirado pelas inovações dos usuários de R
em ciência, educação e indústria, desenvolveu ferramentas gratuitas e
abertas para facilitar o uso do R. O \emph{RStudio} é escrito em
linguagem C++ e foi inicialmente focado apenas na linguagem R. Com o
tempo o desenvolveu suporte para \emph{Python} e \emph{VSCode}. Em 2022,
para acompanhar essa mudança, foi anunciada a mudança do nome da empresa
que o desenvolve para \href{https://posit.co/}{Posit} e, recentemente,
introduzido um novo IDE, denominado \emph{Positron} (53), um projeto
inicial, com um ambiente semelhante ao \emph{RStudio} e que continua em
desenvolvimento. Talvez, no futuro, possa substituir o RStudio. Por
enquanto, isso será difícil , pois o \emph{Positron} não tem todas as
funcionalidades do \emph{RStudio} (54).

\subsubsection{\texorpdfstring{Instalação do \emph{R
Studio}}{Instalação do R Studio}}\label{instalauxe7uxe3o-do-r-studio}

Para instalar o \emph{RStudio} , acessar o
\href{https://www.rstudio.com/products/rstudio/download/}{\textbf{site}}
e clicar em \emph{Download} para obter a versão desejada. Recomenda-se a
versão \emph{RStudio Desktop} -- \emph{Open Source License} que é
gratuita. Esta versão entrega as ferramentas integradas para o R.

A seguir, aparecerão os instaladores disponíveis, conforme a plataforma
suportada pelo seu computador. As mais utilizadas são Windows e Mac OS
X. Neste livro, como base, serão mostrados os passos para a plataforma
Windows\footnote{A instalação para Mac OS X pode ser facilmente obtida
  em busca do \emph{Google}. Depois de instalado, o uso do
  \emph{RStudio} não difere do Windows} .

Em sequência, executar o instalador baixado
\emph{RStudio-2025.05.1-513.exe} \footnote{Disponível em 16/06/2025}e
seguir as suas instruções.

\subsubsection{\texorpdfstring{Iniciando o
\emph{RStudio}}{Iniciando o RStudio}}\label{iniciando-o-rstudio}

Para iniciar o \emph{RStudio} basta clicar no ícone indicativo
(Figura~\ref{fig-icone}) que se encontra no menu \emph{Iniciar} do
Windows.

\begin{figure}

\centering{

\includegraphics[width=0.15\linewidth,height=\textheight,keepaspectratio]{index_files/mediabag/vpHesDH.png}

}

\caption{\label{fig-icone}Ícone do RStudio}

\end{figure}%

O \emph{RStudio} abre como mostrado na Figura~\ref{fig-inicial}. O
\emph{RStudio} é uma interface mais funcional e amigável para o R.
Contém um conjunto de ferramentas integradas projetadas para ajudá-lo a
ser mais produtivo com o R.

\begin{figure}

\centering{

\includegraphics[width=0.75\linewidth,height=\textheight,keepaspectratio]{index_files/mediabag/oG2o4cl.png}

}

\caption{\label{fig-inicial}Tela inicial do RStudio}

\end{figure}%

Inclui o \emph{Console} , editor que suporta execução direta de códigos
e uma variedade de ferramentas robustas para plotagem, exibição de
histórico, depuração e gerenciamento de seu espaço de trabalho incluídos
em uma interface que está, inicialmente, dividida em 3 paineis:

\begin{enumerate}
\def\labelenumi{\arabic{enumi}.}
\tightlist
\item
  \emph{Console}
\item
  \emph{Environment, History, Connections, Tutorial}
\item
  \emph{Files, Plots, Packages, Help}
\end{enumerate}

\textbf{Console e R Script}

Do lado esquerdo fica o \textbf{Console} (Figura~\ref{fig-inicial}), em
vermelho), onde os comandos podem ser digitados e aparecem os resultados
da execução dos comandos. Ao abrir o \emph{RStudio} , vê-se no
\emph{Console} uma série de informações sobre o R, como versão em uso e,
por último, o diretório onde está armazenado o espaço de trabalho
(\emph{workspace}). Estas informações podem ser facilmente apagadas,
clicando na barra de ferramentas, no menu \emph{Edit}, e após em
\emph{Clear Console} ou, usando as teclas \emph{Ctrl+L}.

O \emph{Console} é a principal parte do R. Aqui é onde o R realmente
executa o comando. No início do \emph{Console}, existe um caractere
(\textgreater). Este é um \emph{prompt} que informa que o R está pronto
para receber um novo código. Pode-se digitar o código diretamente no
\emph{Console} após o \emph{prompt} e obter uma resposta imediata. Por
exemplo, se for digitado 1 + 1 e pressionado \emph{Enter}, o R
imediatamente gera uma saída de 2 (Figura~\ref{fig-console}).

\begin{figure}

\centering{

\includegraphics[width=0.7\linewidth,height=\textheight,keepaspectratio]{index_files/mediabag/3Fu5drg.png}

}

\caption{\label{fig-console}Console do RStudio}

\end{figure}%

Recomenda-se que a maior parte dos comandos sejam digitados no bloco de
notas do \emph{RStudio} , o \emph{R Script}. Reservar o \emph{Console}
apenas para depurar ou fazer análises e cálculos rápidos. A razão para
isso é simples: se o comando for digitado diretamente no \emph{Console},
ele não será salvo e se for cometido um erro na digitação, haverá
necessidade de digitar tudo novamente. Portanto, é melhor escrever os
comandos no \emph{R Script} e, quando estiver pronto para executar,
enviar para o \emph{Console}.

O \emph{R Script} é o quarto painel do \emph{RStudio} e seu bloco de
notas. Ele é criado através do menu \emph{File} \textgreater{} \emph{New
File} \textgreater{} \emph{R Script} ou clicando no botão verde com o
sinal (+), na barra de ferramentas de acesso rápido, na parte superior à
esquerda. Ao criar um novo \emph{R Script} será aberto o painel do bloco
de notas (Figura~\ref{fig-script}), em verde).

\begin{figure}

\centering{

\includegraphics[width=0.75\linewidth,height=\textheight,keepaspectratio]{index_files/mediabag/0URmX6w.png}

}

\caption{\label{fig-script}Bloco de Notas do RStudio: R Script}

\end{figure}%

Um diferencial do \emph{RStudio} é que os comandos são autocompletáveis.
Basta começar a escrever o comando, inserindo 3 ou mais caracteres, por
exemplo, \texttt{summ} referente a função \texttt{summary\ ()}, usada
para sumarizar um conjunto de dados, e surge um menu de opções,
facilitando a digitação (Figura~\ref{fig-auto}).

\begin{figure}

\centering{

\includegraphics[width=0.75\linewidth,height=\textheight,keepaspectratio]{index_files/mediabag/qNS4uYP.png}

}

\caption{\label{fig-auto}Menu autocompletável}

\end{figure}%

Após digitar no \emph{Console}, para que seja executado o comando há
necessidade de clicar na tecla \emph{Enter}; no \emph{RScript}, clicar
em \emph{Run}, acima, na barra, no lado direito, ou usar o atalho
\emph{Ctrl + Enter}. Textos podem ser copiados e colados no script e
linhas em branco podem ser inseridas. Além disso, no final da sua
sessão, é possível salvar o arquivo, que poderá ser recarregado no
futuro, se precisar refazer a análise.

Os \emph{scripts} do \emph{RStudio} são apenas arquivos de texto com a
extensão (.R). Quando se cria um \emph{R Script}, aparece como \emph{Sem
título} (\emph{Untitled}). Antes de começar a digitar um novo
\emph{script} no \emph{RStudio}, recomenda-se salvar o atual com um novo
nome de arquivo. Dessa forma, se algo no computador falhar durante o
trabalho, o código ficará protegido.

Ao digitar o código em um \emph{script}, o R não executa o código
enquanto se digita. Para que o R realmente avalie o código digitado, há
necessidade de primeiro enviar o código para o \emph{Console}, clicando
no botão \emph{Run} ou usando a tecla de atalho \emph{Crtl + Enter}.
Cada linha é marcada no início por um número em sequência.

Além da digitação de comandos, o \emph{R Script} permite fazer
comentários onde tudo que for escrito, após o símbolo \(\#\), não é
considerado, é apenas uma explicação, um esclarecimento. Os comentários
são literais, escritos diretamente para explicar o comando executado.
São repetidos na saída do \emph{Console} sem aparecer nos resultados.

\textbf{Ambiente, História, Conexão e Tutorial}

No lado superior direito há um painel com quatro abas
(Figura~\ref{fig-inicial}), em azul):

\begin{enumerate}
\def\labelenumi{\arabic{enumi})}
\item
  \textbf{Ambiente} (\emph{Environment}) - onde ficam armazenados os
  objetos criados, as bases de dados importadas, etc., na sessão ativa.
  É possível visualizar informações como o número de observações e
  linhas dos bancos de dados ativos. A guia também tem algumas ações
  clicáveis, como \emph{Import Dataset}, que permite importar arquivos
  csv, Excel, SPSS, etc.
\item
  \textbf{História} (\emph{History}) - onde fica o histórico dos
  comandos executados no \emph{Console}. Estes comandos podem ser
  pesquisados nesta guia. Os comandos são exibidos em ordem (mais
  recentes na parte inferior) e agrupados por bloco de tempo.
\item
  \textbf{Conexões} (\emph{Connections}) - mostra todas as conexões
  feitas com fontes de dados suportadas e permite saber quais conexões
  estão ativas no momento. O \emph{RStudio} suporta múltiplas conexões
  de banco de dados simultâneas.
\item
  \textbf{Tutorial} - a partir da versão 1.3, o \emph{R Script} ganhou
  um painel Tutorial dedicado, usado para executar tutoriais que
  ajudarão você a aprender e dominar a linguagem de programação R. Na
  primeira vez que se abre o programa, clicando nesta aba, o
  \emph{RStudio} solicita que seja instalado o pacote \texttt{learnr}
  (Figura~\ref{fig-tutorial})). Isto permite acesso a vários tutoriais
  úteis que merecem ser explorados
\end{enumerate}

\begin{figure}

\centering{

\includegraphics[width=0.5\linewidth,height=\textheight,keepaspectratio]{index_files/mediabag/wSgusUJ.png}

}

\caption{\label{fig-tutorial}Tutoriais do RStudio}

\end{figure}%

\textbf{Arquivos, Gráficos, Pacotes, Ajuda e Apresentação}

No lado direito, abaixo, existem outras abas muito úteis
(Figura~\ref{fig-inicial}), em amarelo):

\begin{enumerate}
\def\labelenumi{\arabic{enumi})}
\item
  \textbf{Arquivos (Files)} - esta guia dá acesso ao diretório onde se
  encontram os seus arquivos. Um bom recurso do painel \emph{Files} é
  que se pode usá-lo para definir seu diretório de trabalho. Para isso,
  clique em \emph{More} e depois em \emph{Set As Working Directory}.
\item
  \textbf{Gráficos (Plots)} - local onde ficam os gráficos gerados.
  Existem botões para abrir o gráfico em uma janela separada e exportar
  o gráfico como um \emph{.pdf} ou \emph{.jpeg}.
\item
  \textbf{Pacotes (Packages)} - mostra uma lista de todos os pacotes R
  instalados no seu computador e indica se eles estão atualmente
  carregados ou não. Pacotes que estão sendo executados na sessão atual,
  estão marcados, enquanto aqueles que estão instalados, mas ainda
  inativos, estão desmarcados.
\item
  \textbf{Ajuda (Help)} - menu de ajuda para as funções R. Você pode
  digitar o nome de uma função na janela de pesquisa (por exemplo,
  \texttt{histogram} ou usar o \texttt{?hist}), no \emph{Console} ou no
  \emph{R Script}, para procurar ajuda sobre uma função
  (Figura~\ref{fig-ajuda})). A Ajuda no \emph{R Studio} pode também ser
  acessada no menu \emph{Help} da barra de ferramentas onde existem
  várias opções. Para complementar, alguns livros são muito uteis, como
  o \emph{R Cookbook} (55) ou \emph{Using} R* for introductory
  statistics* (56). No entanto, na maioria das vezes a forma mais
  prática de conseguir ajuda com uma dúvida específica é a busca em
  fóruns na internet, como o \emph{Stack Overflow}:
  \url{https://stackoverflow.com/}.
\item
  \textbf{Apresentação (Presentation)} -- é visualizador de
  apresentações. Nas últimas versões do Rstudio, é possível com o
  Quarto, editar um código em R Markdown para construir uma
  apresentação. Não faz parte do objetivo deste livro desenvolver este
  assunto. É possível encontrar um tutorial em
  \url{https://quarto.org/docs/get-started/hello/rstudio.html}.
\end{enumerate}

\begin{figure}

\centering{

\includegraphics[width=0.85\linewidth,height=\textheight,keepaspectratio]{index_files/mediabag/XlGJgCt.png}

}

\caption{\label{fig-ajuda}Ajuda do RStudio}

\end{figure}%

\subsection{Pacotes}\label{pacotes}

Para que o R possa interagir com o usuário, realizar análises
estatísticas e gerar gráficos, a instalação de pacotes é essencial.\\
Um \textbf{pacote é um conjunto de funções, dados e documentação} que
amplia os recursos do R base. O uso de pacotes é fundamental para
explorar todo o potencial da ferramenta, sendo sua instalação orientada
pelas demandas específicas de cada projeto. Ao instalar o R básico,
diversos pacotes já vêm incluídos, permitindo uma ampla gama de
análises. No entanto, à medida que o uso do R se aprofunda, torna-se
necessário instalar pacotes adicionais desenvolvidos pela comunidade,
que oferecem funcionalidades extras por meio de novas funções e
comandos.

\subsubsection{Repositório de pacotes}\label{reposituxf3rio-de-pacotes}

Quando se identifica a necessidade de um novo pacote, é fundamental
saber onde ele se encontra. O principal repositório de pacotes é o CRAN
(\emph{Comprehensible R Archive Network}), já comentado anteriormente.
Para acessar este repositório, use o
\href{https://cran.r-project.org/mirrors.html}{\textbf{link}} e escolha
um espelho (\emph{0-Cloud} ou o mais próximo geograficamente). Depois
que o pacote for instalado, ele será mantido em sua biblioteca
(\emph{library}) R associada à sua versão principal atual do R. Haverá
necessidade de atualizar e reinstalar os pacotes sempre que atualizar
uma versão principal do R.

Estando na página do CRAN, no menu, à esquerda, clique em
\emph{Packages} . Isto o colocará na página dos \emph{Contributed
Packages}, onde a maioria dos pacotes podem ser encontrados em
\emph{Table of available packages, sorted by name} . Também é possível
clicar em CRAN \emph{Task Views} , onde estão os pacotes separados por
tópicos.

\subsubsection{Instalação de um novo
pacote}\label{instalauxe7uxe3o-de-um-novo-pacote}

Instalar um pacote significa simplesmente baixar o código do pacote em
um computador pessoal. Existem duas maneiras principais de instalar
novos pacotes. O método mais comum é baixá-los do CRAN, usando a função
\texttt{install.packages()}. Dentro dos parênteses, como argumento,
coloca-se entre aspas (duplas ou simples) o nome do pacote. Como visto,
deve-se, de preferência, digitar o comando no \emph{R Script}. Por
exemplo, para instalar o pacote \texttt{ggplot2}, usado para trabalhar
gráficos, se procede da seguite maneira:

\begin{Shaded}
\begin{Highlighting}[]
\FunctionTok{install.packages}\NormalTok{(}\StringTok{"ggplot2"}\NormalTok{)}
\FunctionTok{library}\NormalTok{(ggplot2)}
\end{Highlighting}
\end{Shaded}

Para carregar o pacote, isto é, para fazer com que suas funções se
tornem ativas para uso na na sessão, deve-se usar a função
\texttt{library()}, como mostrado no comando acima.\\
Se o \emph{RStudio} for fechado e reaberto, o pacote deverá ser
novamente ativado. Observe que a função \texttt{library()} não requer
que o nome do pacote seja digitado entre aspas. Isto acontece porque
antes de o pacote ser instalado o R não o reconhece , portanto, há
necessidade de indicar o nome (caracteres), para que o R procure na
internet o que deve deve baixar. Já, depois de instalado, o pacote é um
objeto conhecido pelo R, logo as aspas não são mais necessárias.

Uma outra maneira de instalar pacotes no R, é usar o botão
\textbf{Install}, localizado na aba \emph{Packages}, no painel inferior,
à direita. Clicando em \textbf{Install}, abre-se a caixa de diálogo da
Figura~\ref{fig-install}. Digitar em \emph{Packages} o nome do pacote
(\texttt{ggplot2}) e o \emph{RStudio} completará com opções para achar o
pacote. Clicar em \texttt{ggplot2} e verifique se \emph{Install
dependencies} foi selecionado. A seguir clicar em \emph{Install} e
aguardar aparecer no \emph{Console} a mensagem que o pacote foi
instalado com sucesso.

\begin{figure}

\centering{

\includegraphics[width=0.55\linewidth,height=\textheight,keepaspectratio]{index_files/mediabag/HBlM3FL.png}

}

\caption{\label{fig-install}Instalação do pacote `ggplot2' usando a
caixa de diálogo `Install Packages'}

\end{figure}%

\subsubsection{Atualização dos
pacotes}\label{atualizauxe7uxe3o-dos-pacotes}

Periodicamente, há necessidade de atualizar os pacotes instalados. Essa
necessidade advém do fato que, com o tempo, os autores de pacotes
lançarão novas versões com correções de defeitos e novos recursos e,
geralmente, é uma boa ideia manter-se atualizado. Para realizar a
atualização, use a função \texttt{update.packages()}, colocando o nome
do pacote entre aspas, por exemplo, \texttt{update.packages("ggplot2)}.

\subsubsection{Instalando e carregando mais de um
pacote}\label{sec-pacman}

Uma das funções, atualmente, mais usadas para executar essa ação é
fornecida pelo pacote \texttt{pacman} (57).

É interessante configurar o \texttt{pacman} para que ele esteja sempre
disponível assim que o R é aberto. Isso é feito editando o arquivo
chamado \texttt{.Rprofile}, que é carregado automaticamente toda vez que
você inicia uma sessão do R. Para configurar o \texttt{pacman} no
\texttt{.Rprofile}, abra-o no \emph{RStudio}:

\begin{Shaded}
\begin{Highlighting}[]
\FunctionTok{file.edit}\NormalTok{(}\StringTok{"\textasciitilde{}/.Rprofile"}\NormalTok{)}
\end{Highlighting}
\end{Shaded}

Adicione o seguinte código:

\begin{Shaded}
\begin{Highlighting}[]
\ControlFlowTok{if}\NormalTok{ (}\SpecialCharTok{!}\FunctionTok{requireNamespace}\NormalTok{(}\StringTok{"pacman"}\NormalTok{, }\AttributeTok{quietly =} \ConstantTok{TRUE}\NormalTok{)) \{}
  \FunctionTok{install.packages}\NormalTok{(}\StringTok{"pacman"}\NormalTok{)}
\NormalTok{\}}
\FunctionTok{library}\NormalTok{(pacman)}
\end{Highlighting}
\end{Shaded}

Esse trecho garante que:

• Se o \texttt{pacman} não estiver instalado, ele será instalado
automaticamente.\\
• Depois, ele será carregado para uso imediato.

Após esse procedimento, salve o \texttt{.Rprofile} e feche o arquivo.
Reinicie o R para que a configuração entre em vigor.

O pacman instala e carrega um ou mais pacotes, através da sua função
\texttt{p\_load()} da seguinte maneira, não havendo necessidade de
escrever o nome dos pacotes entre aspas:

\begin{Shaded}
\begin{Highlighting}[]
\NormalTok{pacman}\SpecialCharTok{::}\FunctionTok{p\_load}\NormalTok{(readxl, dplyr, ggplot2, car)}
\end{Highlighting}
\end{Shaded}

Além da função \texttt{p\_load()}, o pacote \texttt{pacman} tem outas
funções, entre elas a função \texttt{p\_update()} que atualiza o pacote
e , se usada sem especificar o pacote , atualiza todos. Para saber mais
sobre o pacote \texttt{pacman}, use a ajuda.

\subsubsection{Citação de pacotes em
publicações}\label{citauxe7uxe3o-de-pacotes-em-publicauxe7uxf5es}

No R existe um comando que mostra como citar o R ou um de seus pacotes.
Basta digitar a função \texttt{citation()} no \emph{Console} ou no
\emph{R Script} e observar a saída. Para um pacote específico, basta
colocar o nome do pacote entre aspas, na função.

\begin{Shaded}
\begin{Highlighting}[]
\FunctionTok{citation}\NormalTok{()}
\end{Highlighting}
\end{Shaded}

\begin{verbatim}
To cite R in publications use:

  R Core Team (2025). _R: A Language and Environment for Statistical
  Computing_. R Foundation for Statistical Computing, Vienna, Austria.
  <https://www.R-project.org/>.

Uma entrada BibTeX para usuários(as) de LaTeX é

  @Manual{,
    title = {R: A Language and Environment for Statistical Computing},
    author = {{R Core Team}},
    organization = {R Foundation for Statistical Computing},
    address = {Vienna, Austria},
    year = {2025},
    url = {https://www.R-project.org/},
  }

We have invested a lot of time and effort in creating R, please cite it
when using it for data analysis. See also 'citation("pkgname")' for
citing R packages.
\end{verbatim}

\begin{Shaded}
\begin{Highlighting}[]
\FunctionTok{citation}\NormalTok{ (}\StringTok{"ggplot2"}\NormalTok{)}
\end{Highlighting}
\end{Shaded}

\begin{verbatim}
To cite ggplot2 in publications, please use

  H. Wickham. ggplot2: Elegant Graphics for Data Analysis.
  Springer-Verlag New York, 2016.

Uma entrada BibTeX para usuários(as) de LaTeX é

  @Book{,
    author = {Hadley Wickham},
    title = {ggplot2: Elegant Graphics for Data Analysis},
    publisher = {Springer-Verlag New York},
    year = {2016},
    isbn = {978-3-319-24277-4},
    url = {https://ggplot2.tidyverse.org},
  }
\end{verbatim}

\subsection{Diretório de Trabalho}\label{diretuxf3rio-de-trabalho}

O diretório de trabalho (\textbf{Working Directory}) é uma pasta onde o
R lê e salva arquivos. Deve-se criar um diretório de trabalho para a
sessão . Para isso, no \emph{RStudio} siga o caminho: \emph{Session}
\textgreater{} \emph{Set Working Directory} \textgreater{} \emph{Choose
Directory} ou use o atalho \emph{Ctrl + Shift + H} e escolha o diretório
desejado ou crie um novo.

Ao finalizar, aparecerá no \emph{Console} (Figura~\ref{fig-wd}):

\begin{figure}

\centering{

\includegraphics[width=0.85\linewidth,height=\textheight,keepaspectratio]{index_files/mediabag/o27RYRI.png.png}

}

\caption{\label{fig-wd}Diretório de trabalho}

\end{figure}%

Note que o R usou a função \texttt{setwd()} que significa ``definir
diretório de trabalho''. Também é possível usar esta função diretamente
no \emph{R Script} ou no \emph{Console}, digitando conforme o caminho do
diretório.

Para saber qual é o diretório de trabalho que está sendo usado pelo R,
pode-se executar a função \texttt{getwd()}. A saída no \emph{Console}
mostrará o diretório de trabalho usado, portanto é recomendado que se
faça isso no início da sessão para verificar se há ou não necessidade de
modificar o diretório.

\subsection{Projeto}\label{projeto}

Uma funcionalidade importante do \emph{RStudio} é a possibilidade de se
criar projetos. Um projeto nada mais é do que uma pasta no seu
computador. Nessa pasta, estarão todos os arquivos que serão usados ou
criados na sua análise.

A principal razão de se utilizar projetos é simplesmente
\emph{organização}. Com eles, fica muito mais fácil importar conjunto de
dados para dentro do R, criar análises reprodutíveis e compartilhar o
trabalho realizado.

Ao se começar uma nova análise, é interessante criar um Novo Projeto.
Para isso, clicar \emph{File} \textgreater{} \emph{New Project} ou
clicar no menu que está na parte superior, à direita, \emph{Project
(none)} \textgreater{} \emph{New Project\ldots{}}. Abrirá a janela da
Figura~\ref{fig-projeto}).

\begin{figure}

\centering{

\includegraphics[width=0.55\linewidth,height=\textheight,keepaspectratio]{index_files/mediabag/6N4p3VY.png}

}

\caption{\label{fig-projeto}Assistente de novo projeto}

\end{figure}%

Clique em \emph{New Directory} para criar um novo diretório. Por
exemplo, para as aulas de Bioestatística, pode-se criar um diretório com
o nome de \emph{bioestatistica} ou qualquer outro nome\footnote{Evite
  acentos, maiúsculas ou caracteres especiais. Seja simples e objetivo,
  usando nomes que estejam relacionados com o assunto.}.

Quaisquer documentos Excel ou arquivos de texto associados podem ser
salvos nesta nova pasta e facilmente acessados, indo ao menu
\emph{Project (none)} \textgreater{} \emph{Open Project\ldots{}}. A
partir daí, é possível realizar análises de dados ou produzir
visualizações com seus dados importados.

Quando um projeto estiver aberto no \emph{RStudio}, o seu nome aparecerá
no canto superior direito da tela.\\
Na aba \emph{Files}, aparecerão todos os arquivos contidos no projeto.
Quando se clica no nome do projeto, abre um menu que torna muito fácil a
navegação pelos projetos existentes. Basta clicar em qualquer um deles
para trocar de projeto, isto é, deixar de trabalhar em uma análise e
começar a trabalhar em outra.

\section{Princípios básicos de uso do
R}\label{princuxedpios-buxe1sicos-de-uso-do-r}

\subsection{R como calculadora}\label{r-como-calculadora}

O R pode ser usado como uma calculadora desde as mais simples até as
mais complexas. Para isso, basta digitar as equações no \emph{Console}
ou no \emph{R Script}, usando os operadores:

\subsubsection{Operadores}\label{operadores}

Operadores são usados para realizar operações com variáveis e valores.

\ul{Operadores aritméticos}

No R, você pode usar operadores aritméticos para realizar operações
matemáticas comuns.

\begin{Shaded}
\begin{Highlighting}[]
 \DecValTok{10} \SpecialCharTok{+} \DecValTok{5}        \CommentTok{\# Adição}
\end{Highlighting}
\end{Shaded}

\begin{verbatim}
[1] 15
\end{verbatim}

\begin{Shaded}
\begin{Highlighting}[]
 \DecValTok{10} \SpecialCharTok{{-}} \DecValTok{5}        \CommentTok{\# Subtração}
\end{Highlighting}
\end{Shaded}

\begin{verbatim}
[1] 5
\end{verbatim}

\begin{Shaded}
\begin{Highlighting}[]
 \DecValTok{10} \SpecialCharTok{*} \DecValTok{5}        \CommentTok{\# Multiplicação}
\end{Highlighting}
\end{Shaded}

\begin{verbatim}
[1] 50
\end{verbatim}

\begin{Shaded}
\begin{Highlighting}[]
 \DecValTok{10} \SpecialCharTok{/} \DecValTok{5}        \CommentTok{\# Divisão}
\end{Highlighting}
\end{Shaded}

\begin{verbatim}
[1] 2
\end{verbatim}

\begin{Shaded}
\begin{Highlighting}[]
 \DecValTok{10} \SpecialCharTok{\^{}} \DecValTok{5}        \CommentTok{\# Potência}
\end{Highlighting}
\end{Shaded}

\begin{verbatim}
[1] 1e+05
\end{verbatim}

\begin{Shaded}
\begin{Highlighting}[]
 \DecValTok{10} \SpecialCharTok{\%\%} \DecValTok{3}       \CommentTok{\# Divisão modular (divisão com resto)}
\end{Highlighting}
\end{Shaded}

\begin{verbatim}
[1] 1
\end{verbatim}

\begin{Shaded}
\begin{Highlighting}[]
 \DecValTok{10} \SpecialCharTok{\%/\%} \DecValTok{3}      \CommentTok{\# Divisão inteiro}
\end{Highlighting}
\end{Shaded}

\begin{verbatim}
[1] 3
\end{verbatim}

Observe que o R repete a operação e coloca em baixo o resultado
precedido por {[}1{]}. O resultado da operação de exponenciação é
exibido como notação científica, onde \(e+05\) significa \(10^5\).

\ul{Operadores de atribuição}

Operadores de atribuição são usados para atribuir valores a variáveis,
como será visto na Seção~\ref{sec-objetos}, adiante.

\ul{Operadores de comparação}

São usados para comparar dois valores.

\begin{Shaded}
\begin{Highlighting}[]
\CommentTok{\# Igualdade}
\DecValTok{3} \SpecialCharTok{==} \DecValTok{3}
\end{Highlighting}
\end{Shaded}

\begin{verbatim}
[1] TRUE
\end{verbatim}

\begin{Shaded}
\begin{Highlighting}[]
\DecValTok{3} \SpecialCharTok{==} \DecValTok{4}
\end{Highlighting}
\end{Shaded}

\begin{verbatim}
[1] FALSE
\end{verbatim}

\begin{Shaded}
\begin{Highlighting}[]
\CommentTok{\# Não igual (diferente)}
\DecValTok{3} \SpecialCharTok{!=} \DecValTok{4}
\end{Highlighting}
\end{Shaded}

\begin{verbatim}
[1] TRUE
\end{verbatim}

\begin{Shaded}
\begin{Highlighting}[]
\CommentTok{\# Maior}
\DecValTok{6} \SpecialCharTok{\textgreater{}} \DecValTok{3}
\end{Highlighting}
\end{Shaded}

\begin{verbatim}
[1] TRUE
\end{verbatim}

\begin{Shaded}
\begin{Highlighting}[]
\CommentTok{\# Menor}
\DecValTok{3} \SpecialCharTok{\textless{}} \DecValTok{4}
\end{Highlighting}
\end{Shaded}

\begin{verbatim}
[1] TRUE
\end{verbatim}

\begin{Shaded}
\begin{Highlighting}[]
\CommentTok{\# Maior ou igual}
\DecValTok{5} \SpecialCharTok{\textgreater{}=} \DecValTok{3}
\end{Highlighting}
\end{Shaded}

\begin{verbatim}
[1] TRUE
\end{verbatim}

\begin{Shaded}
\begin{Highlighting}[]
\CommentTok{\# Menor ou igual  }
\DecValTok{3} \SpecialCharTok{\textless{}=} \DecValTok{4}
\end{Highlighting}
\end{Shaded}

\begin{verbatim}
[1] TRUE
\end{verbatim}

\begin{tcolorbox}[enhanced jigsaw, bottomrule=.15mm, opacitybacktitle=0.6, colframe=quarto-callout-note-color-frame, arc=.35mm, coltitle=black, toptitle=1mm, colback=white, colbacktitle=quarto-callout-note-color!10!white, breakable, bottomtitle=1mm, rightrule=.15mm, titlerule=0mm, toprule=.15mm, opacityback=0, leftrule=.75mm, left=2mm, title=\textcolor{quarto-callout-note-color}{\faInfo}\hspace{0.5em}{Atenção}]

Na linguagem R, o sinal de igualdade é escrito com duplo \(=\).

\end{tcolorbox}

\ul{Operadores lógicos}

Operadores lógicos são usados para combinar declarações condicionais:

\begin{Shaded}
\begin{Highlighting}[]
\CommentTok{\# Conjunção lógica E, retorna TRUE se ambos elementos são  verdadeiros }
\DecValTok{6} \SpecialCharTok{==} \DecValTok{6} \SpecialCharTok{\&} \DecValTok{7} \SpecialCharTok{==} \DecValTok{8}
\end{Highlighting}
\end{Shaded}

\begin{verbatim}
[1] FALSE
\end{verbatim}

\begin{Shaded}
\begin{Highlighting}[]
\CommentTok{\# Conjunção lógica E, retorna TRUE se ambos elementos são  verdadeiros}
\DecValTok{2} \SpecialCharTok{*} \DecValTok{3} \SpecialCharTok{\&\&} \DecValTok{1} \SpecialCharTok{*} \DecValTok{6}
\end{Highlighting}
\end{Shaded}

\begin{verbatim}
[1] TRUE
\end{verbatim}

\begin{Shaded}
\begin{Highlighting}[]
\CommentTok{\# Conjunção lógica OU, retorna TRUE se um dos elementos é verdadeiro}
\NormalTok{(}\DecValTok{2} \SpecialCharTok{*} \DecValTok{2}\NormalTok{) }\SpecialCharTok{|} \FunctionTok{sqrt}\NormalTok{(}\DecValTok{16}\NormalTok{)}
\end{Highlighting}
\end{Shaded}

\begin{verbatim}
[1] TRUE
\end{verbatim}

\begin{Shaded}
\begin{Highlighting}[]
\DecValTok{6} \SpecialCharTok{==} \DecValTok{6} \SpecialCharTok{|} \DecValTok{7} \SpecialCharTok{==} \DecValTok{8} 
\end{Highlighting}
\end{Shaded}

\begin{verbatim}
[1] TRUE
\end{verbatim}

\begin{Shaded}
\begin{Highlighting}[]
\CommentTok{\# Conjunção lógica NÃO, retorna FALSE se o  elemento é verdadeiro}
\SpecialCharTok{!}\DecValTok{6}\SpecialCharTok{==}\DecValTok{6}
\end{Highlighting}
\end{Shaded}

\begin{verbatim}
[1] FALSE
\end{verbatim}

\begin{Shaded}
\begin{Highlighting}[]
\SpecialCharTok{!}\DecValTok{2}\SpecialCharTok{==}\DecValTok{4}
\end{Highlighting}
\end{Shaded}

\begin{verbatim}
[1] TRUE
\end{verbatim}

\begin{Shaded}
\begin{Highlighting}[]
\CommentTok{\# Operador lógico que verifica se um elemento pertence a um conjunto (\%in\%)}

\NormalTok{ pares }\OtherTok{\textless{}{-}} \FunctionTok{c}\NormalTok{(}\DecValTok{0}\NormalTok{, }\DecValTok{2}\NormalTok{, }\DecValTok{4}\NormalTok{, }\DecValTok{6}\NormalTok{, }\DecValTok{8}\NormalTok{, }\DecValTok{10}\NormalTok{)}
 \DecValTok{5} \SpecialCharTok{\%in\%}\NormalTok{ pares}
\end{Highlighting}
\end{Shaded}

\begin{verbatim}
[1] FALSE
\end{verbatim}

\ul{Outros operadores}

\begin{Shaded}
\begin{Highlighting}[]
\CommentTok{\# Logarítmo natural (base e)}
\FunctionTok{log}\NormalTok{ (}\DecValTok{10}\NormalTok{) }
\end{Highlighting}
\end{Shaded}

\begin{verbatim}
[1] 2.302585
\end{verbatim}

\begin{Shaded}
\begin{Highlighting}[]
\CommentTok{\# Logarítmo base 10}
\FunctionTok{log10}\NormalTok{ (}\DecValTok{10}\NormalTok{)       }
\end{Highlighting}
\end{Shaded}

\begin{verbatim}
[1] 1
\end{verbatim}

\begin{Shaded}
\begin{Highlighting}[]
\CommentTok{\# Raiz quadrada}
\FunctionTok{sqrt}\NormalTok{ (}\DecValTok{81}\NormalTok{)}
\end{Highlighting}
\end{Shaded}

\begin{verbatim}
[1] 9
\end{verbatim}

\begin{Shaded}
\begin{Highlighting}[]
\CommentTok{\# Resultado absoluto}
\FunctionTok{abs}\NormalTok{ (}\DecValTok{3} \SpecialCharTok{{-}} \DecValTok{6}\NormalTok{)}
\end{Highlighting}
\end{Shaded}

\begin{verbatim}
[1] 3
\end{verbatim}

\subsection{Objetos}\label{sec-objetos}

O R permite salvar valores dentro de um \emph{objeto}. Os objetos são
criados utilizando o \emph{operador de atribuição} (\textless-). Para
digitar este operador, basta teclar o sinal \emph{menor que}
(\textless), seguido de \emph{hífen} (-) , sem espaços. Existe um atalho
que é pressionar (\texttt{Alt}) \(+\) (-). O símbolo \(=\) pode ser
usado no lugar de \textless-.

\textbf{Objeto} é um pequeno espaço na memória do computador onde o R
armazenará um valor ou o resultado de um comando, utilizando um nome
arbitrariamente definido. Tudo criado pelo R pode se constituir em um
objeto, por exemplo: uma variável, uma operação aritmética, um gráfico,
uma matriz ou um modelo estatístico. Através de um objeto torna-se
simples acessar os dados armazenados na memória. Ao criar um objeto, se
faz uma declaração. Isto significa que se está afirmando que uma
determinada operação aritmética irá, agora, tornar-se um objeto que irá
armazenar um determinado valor. As declarações são feitas uma em cada
linha do \emph{R Script}.

Os objetos devem receber um nome e é obrigatório que ele comece por uma
letra (ou um ponto) e não é permitido o uso do hífen. Pode-se usar o
ponto ou \emph{underlines} para separar palavras. Deve ser evitado o uso
de nomes que sejam de objetos do sistema, ou outros objetos já criados,
funções ou constantes. Por exemplo, não deve ser utilizado: \texttt{c},
\texttt{q}, \texttt{r}, \texttt{s}, \texttt{t}, \texttt{C}, \texttt{D},
\texttt{F}, \texttt{I}, \texttt{T}, \texttt{diff}, \texttt{exp},
\texttt{log}, \texttt{mean}, \texttt{pi}, \texttt{range}, \texttt{rank},
\texttt{var}, \texttt{NA}, \texttt{NaN}, \texttt{NULL}, \texttt{FALSE},
\texttt{TRUE}, \texttt{break}, \texttt{else}, \texttt{if},
\texttt{break}, \texttt{function}, \texttt{in}, \texttt{while} que devem
ser reservados, pois têm significados especiais.

Quando se usa um objeto com o nome \texttt{pi}, ele assumirá outro valor
diferente de 3,141593. Preservando este nome, toda vez que usarmos a
palavra \texttt{pi}, o R assume o valor pré-estabelecido. Além disso, o
R faz a diferença entre letras maiúsculas e minúsculas. Ou seja,
\texttt{soma} é um objeto diferente de \texttt{Soma} e ambos são
diferentes de \texttt{SOMA}.

Para exibir o conteúdo de um objeto, basta digitar seu nome no \emph{R
Script} ou no \emph{Console} e executar. Em análises mais extensas,
verificar se já há um objeto com o mesmo nome, pois seus valores serão
substituídos ao executar o novo objeto. Para saber se já existe um
objeto com o nome definido, digite as primeiras letras do objeto criado
e o \emph{R Studio} listará, usando a sua função de autocompletar, tudo
que começar com essas letras no arquivo. Assim ficará fácil verificar se
já existe um objeto com o nome desejado.

No comando abaixo, é criado um objeto que receberá a soma de dez
números, utilizando a função \texttt{sum()}. O objeto foi denominado de
\texttt{soma}. Para exibir o valor contido no objeto \texttt{soma}, é
necessário digitar \texttt{soma} no \emph{R Script} ou \emph{Console} e
executar:

\begin{Shaded}
\begin{Highlighting}[]
\NormalTok{soma }\OtherTok{\textless{}{-}} \FunctionTok{sum}\NormalTok{ (}\DecValTok{2}\NormalTok{, }\DecValTok{3}\NormalTok{, }\DecValTok{12}\NormalTok{, }\DecValTok{15}\NormalTok{, }\DecValTok{21}\NormalTok{, }\DecValTok{4}\NormalTok{, }\DecValTok{8}\NormalTok{, }\DecValTok{7}\NormalTok{, }\DecValTok{13}\NormalTok{, }\DecValTok{21}\NormalTok{)}
\NormalTok{soma}
\end{Highlighting}
\end{Shaded}

\begin{verbatim}
[1] 106
\end{verbatim}

\subsubsection{Atributos de dados inseridos nos
objetos}\label{atributos-de-dados-inseridos-nos-objetos}

Os dados inseridos nos objetos pertencem as seguintes classes
\footnote{Atributos de um objeto.}:

\begin{enumerate}
\def\labelenumi{\arabic{enumi}.}
\item
  \ul{Caracteres (character)}: são categóricos ou qualitativos. Em
  geral, são carcteres ou texto. São escritos entre aspas simples ou
  duplas e não são modificáveis.
\item
  \ul{Numérico (numeric)}: são números inteiros (\texttt{integer}) ou
  decimais ou ponto flutuante (\texttt{dbl} = \texttt{double}). A
  sintaxe para o número inteiro é 42L, 3L{[}\^{}04-introducaor-5{]},
  pois se fosse escrito como 42, o R automaticamente o trata como um
  número flutuante (\texttt{double}). {[}\^{}04-introducaor-5{]}: O
  \texttt{L} vem da linguagem C, onde \texttt{long} é um tipo inteiro. É
  uma convenção que o R herdou
\item
  \ul{Lógico (logical)}: são usados para combinar declarações
  condicionais. Retorna: verdadeiro (TRUE) ou falso (FALSE). São
  escritos obrigatoriamente com letras maiúsculas sem aspas. São
  utilizados para indicar opções onde há apenas duas opções ou como
  resultado de um teste lógico.
\end{enumerate}

É a partir do conhecimento do tipo de classe que as funções sabem o que
extamente fazer com um objeto. Por exemplo, não é possivel somar duas
letras e se for feita a tentativa de somar ``a'' e ``b'', o Rretorna um
erro: {Error in ``a'' + ``b'': non-numeric argument to binary operator}.

No R, os textos são escritos entre aspas simples ou duplas. As aspas
servem para diferenciar nomes (objetos, funções, pacotes) de textos
(letras e palavras). Os textos são muito comuns em variáveis categóricas
e são popularmente chamados de \emph{strings} ou \emph{character}. Além
desta classe, o R tem outras classes básicas que são a \emph{numeric} e
a \emph{logical}. Um objeto de qualquer uma dessas classes é chamado de
\emph{objeto atômico}. Esse nome se deve ao fato de essas classes não se
misturarem (58).

Para saber qual o tipo de classe que um objeto pertence, basta usar a
função \texttt{class()}.

\begin{Shaded}
\begin{Highlighting}[]
\NormalTok{idade }\OtherTok{\textless{}{-}} \FunctionTok{c}\NormalTok{(}\DecValTok{3}\NormalTok{, }\DecValTok{5}\NormalTok{, }\DecValTok{7}\NormalTok{, }\DecValTok{9}\NormalTok{, }\DecValTok{6}\NormalTok{, }\DecValTok{7}\NormalTok{)}
\FunctionTok{class}\NormalTok{ (idade)}
\end{Highlighting}
\end{Shaded}

\begin{verbatim}
[1] "numeric"
\end{verbatim}

\begin{Shaded}
\begin{Highlighting}[]
\NormalTok{nome }\OtherTok{\textless{}{-}} \FunctionTok{c}\NormalTok{(}\StringTok{"Pedro"}\NormalTok{, }\StringTok{"Maria"}\NormalTok{, }\StringTok{"Margarida"}\NormalTok{, }\StringTok{"Alice"}\NormalTok{, }\StringTok{"João"}\NormalTok{, }\StringTok{"Luís"}\NormalTok{)}
\FunctionTok{class}\NormalTok{(nome)}
\end{Highlighting}
\end{Shaded}

\begin{verbatim}
[1] "character"
\end{verbatim}

\section{Funções}\label{sec-funcoes}

A função é uma orientação ao R para que ele execute uma ação que é algum
procedimento específico. Em decorrência, em geral, uma função ttem um
nome sugestivo da ação que ela realiza. Por exemplo, a função
\texttt{mean\ ()} realiza a média aritmética de uma série de números. O
resultado, como regra geral, deve ser colocado em um objeto que será
armazenado na memória do computador.

Esta série de números, concatenados na função \texttt{c()}, é armazenada
por um objeto, nomeado \texttt{dados}e, posteriormente, se usa a função
\texttt{mean()}com este objeto \texttt{dados}. O resultado da função
\texttt{mean}, exibido no \emph{Console}, será recebido por outro objeto
\texttt{media\_dados} e colocado na memória do computador.

\begin{Shaded}
\begin{Highlighting}[]
\NormalTok{dados }\OtherTok{\textless{}{-}} \FunctionTok{c}\NormalTok{(}\DecValTok{3}\NormalTok{, }\DecValTok{5}\NormalTok{, }\DecValTok{7}\NormalTok{, }\DecValTok{9}\NormalTok{, }\DecValTok{6}\NormalTok{, }\DecValTok{7}\NormalTok{)}
\NormalTok{media\_dados }\OtherTok{\textless{}{-}} \FunctionTok{mean}\NormalTok{(dados)}
\NormalTok{media\_dados}
\end{Highlighting}
\end{Shaded}

\begin{verbatim}
[1] 6.166667
\end{verbatim}

De acordo com as necessidades pode-se criar funções pessoais,
customizadas. Entretanto, na maioria das vezes, elas são encontradas
prontas, fazendo parte de um pacote. Pacotes contêm muitas funções que
para serem executadas necessitam que estes estejam instalados e
carregados. As funções para exercerem a sua ação devem receber dentro
delas (entre parênteses) os \emph{argumentos} que elas exigem. Os
argumentos de uma função são sempre separados por vírgulas.

Para se saber quais argumentos necessários para uma determinada função
basta consultar a ajuda, onde se encontrará a documentação da mesma.
Para isso basta digitar no \emph{Console}, no caso da função
\texttt{mean()}, \texttt{help(mean)} ou \texttt{?mean}:

\begin{Shaded}
\begin{Highlighting}[]
\FunctionTok{help}\NormalTok{(mean)}
\end{Highlighting}
\end{Shaded}

O resultado deste comando aparecerá na aba \emph{Help}, na parte
inferior, à direita (Figura~\ref{fig-help}):

\begin{figure}

\centering{

\includegraphics[width=0.6\linewidth,height=\textheight,keepaspectratio]{index_files/mediabag/Dy3gA7R.png}

}

\caption{\label{fig-help}Ajuda para Média Aritmética}

\end{figure}%

Os principais argumentos da função \texttt{mean()} são:

\begin{itemize}
\tightlist
\item
  \textbf{x} \(\to\) vetor numérico
\item
  \textbf{trim} \(\to\) fração das observações (varia de 0 a 0,5)
  extraída de cada extremidade de x para calcular a média aparada
\item
  \textbf{na.rm} \(\to\) valor lógico (TRUE ou FALSE) que indicam se os
  valores ausentes (NA) devem ser removidos antes que o cálculo continue
\end{itemize}

Este último argumento é muito importante quando, na sequência de valores
existe algum não informado ou inexistente. No R, eles são denominados de
valores ausentes (\emph{missing values}) e denotados por \textbf{NA}
(\emph{Not Available}).

Por exemplo, em uma coleta de uma série de valores, correspondentes ao
peso de 15 recém-nascidos, havendo a ``falta'' de um dos registros, ao
calcular a média com a função \texttt{mean()}, ela retornará NA.

\begin{Shaded}
\begin{Highlighting}[]
\NormalTok{pesoRN }\OtherTok{\textless{}{-}} \FunctionTok{c}\NormalTok{ (}\DecValTok{3340}\NormalTok{,}\DecValTok{3345}\NormalTok{,}\DecValTok{3750}\NormalTok{,}\DecValTok{3650}\NormalTok{,}\DecValTok{3220}\NormalTok{,}\DecValTok{4070}\NormalTok{,}\ConstantTok{NA}\NormalTok{,}\DecValTok{3970}\NormalTok{,}\DecValTok{3060}\NormalTok{,}\DecValTok{3180}\NormalTok{,  }
             \DecValTok{2865}\NormalTok{,}\DecValTok{2815}\NormalTok{,}\DecValTok{3245}\NormalTok{,}\DecValTok{2051}\NormalTok{,}\DecValTok{2630}\NormalTok{)}
\FunctionTok{mean}\NormalTok{ (pesoRN)}
\end{Highlighting}
\end{Shaded}

\begin{verbatim}
[1] NA
\end{verbatim}

Colocando o argumento \texttt{na.rm\ =\ TRUE}, para remover os valores
faltantes, a função retornará a média aritmética sem este valor:

\begin{Shaded}
\begin{Highlighting}[]
\FunctionTok{mean}\NormalTok{ (pesoRN, }\AttributeTok{na.rm =} \ConstantTok{TRUE}\NormalTok{)}
\end{Highlighting}
\end{Shaded}

\begin{verbatim}
[1] 3227.929
\end{verbatim}

\subsection{Criando funções}\label{sec-funcpropria}

No R, é possível criar funções pessoais que podem simplificar um código
e, eventualmente, diminuir o tempo de execução das análises.

\subsubsection{Fórmula geral}\label{fuxf3rmula-geral}

As funções têm uma fórmula geral:

\begin{quote}
nome\_da\_funcao \textless- function (x)\{transformar x\}
\end{quote}

Por exemplo, a área de um circulo é igual a \(\pi\times raio^2\). Uma
função pode otimizar o cálculo da área:

\begin{Shaded}
\begin{Highlighting}[]
\NormalTok{area.circ }\OtherTok{\textless{}{-}} \ControlFlowTok{function}\NormalTok{(r)\{}
\NormalTok{  area }\OtherTok{\textless{}{-}}\NormalTok{ pi}\SpecialCharTok{*}\NormalTok{r}\SpecialCharTok{\^{}}\DecValTok{2}
  \FunctionTok{return}\NormalTok{(area)                }
\NormalTok{\}}
\end{Highlighting}
\end{Shaded}

Ou seja, foi usada a função \texttt{function()}, com o raio do círculo
como argumento. A seguir, entre chaves \texttt{\{\}}, coloca-se a ação
que a função realizará, no caso o cálculo da área do círculo. O
resultado deste cálculo (\texttt{pi*r\^{}2}) é recebido por um objeto
denominado \texttt{area}.\footnote{Foi usado o nome \texttt{area} sem
  acentuação, mas poderia ser qualquer nome.} A seguir, usou-se a função
\texttt{return\ ()} para retornar o resultado do cálculo realizado.\\
Ao executar essa função, é possível usá-la para calcular a área de um
círculo, cujo raio é igual a 5 cm:

\begin{Shaded}
\begin{Highlighting}[]
\FunctionTok{area.circ}\NormalTok{(}\DecValTok{5}\NormalTok{)}
\end{Highlighting}
\end{Shaded}

\begin{verbatim}
[1] 78.53982
\end{verbatim}

\subsubsection{Outros exemplos}\label{outros-exemplos}

O Indice de Massa Corporal é igual ao peso (kg) dividido pela
\(altura^2\), em metros. Uma função para fazer este cálculo é:

\begin{Shaded}
\begin{Highlighting}[]
\NormalTok{imc }\OtherTok{\textless{}{-}} \ControlFlowTok{function}\NormalTok{(peso, altura)\{}
\NormalTok{  res }\OtherTok{\textless{}{-}}\NormalTok{ peso}\SpecialCharTok{/}\NormalTok{altura}\SpecialCharTok{\^{}}\DecValTok{2}
  \FunctionTok{return}\NormalTok{(res)}
\NormalTok{\}}
\end{Highlighting}
\end{Shaded}

Logo, o IMC de um indivíduo que tenha 67 kg e 1,7 m é:

\begin{Shaded}
\begin{Highlighting}[]
\NormalTok{peso }\OtherTok{\textless{}{-}}  \DecValTok{67}
\NormalTok{altura }\OtherTok{\textless{}{-}}  \FloatTok{1.70}
\FunctionTok{imc}\NormalTok{(}\DecValTok{67}\NormalTok{, }\FloatTok{1.70}\NormalTok{)}
\end{Highlighting}
\end{Shaded}

\begin{verbatim}
[1] 23.18339
\end{verbatim}

Os exemplos mostrados são muito simples. Quase não haveria necessidade
de construir uma função. Entretanto, quando se tem uma ação mais
complexa, a função mostra a sua utilidade. Por exemplo, se for
necessário realizar a comparação entre duas médias, usando um teste
\emph{t} e apresentar o resultado junto com boxplots, a função fica mais
complexa. Sempre que for necessário cálculo semelhante, a função
automatiza a ação, sem necessidade de repetir os códigos:

\begin{Shaded}
\begin{Highlighting}[]
\NormalTok{plotBpT }\OtherTok{\textless{}{-}} \ControlFlowTok{function}\NormalTok{(df, var.x, var.y)\{}
  \FunctionTok{library}\NormalTok{(ggplot2)}
  \FunctionTok{library}\NormalTok{(ggpubr)}
  \FunctionTok{ggplot}\NormalTok{(df, }\FunctionTok{aes}\NormalTok{(}\AttributeTok{x =}\NormalTok{ \{\{var.x\}\}, }\AttributeTok{y =}\NormalTok{ \{\{var.y\}\}, }\AttributeTok{fill =}\NormalTok{ \{\{var.x\}\})) }\SpecialCharTok{+}
    \FunctionTok{geom\_errorbar}\NormalTok{(}\AttributeTok{stat =} \StringTok{"boxplot"}\NormalTok{, }\AttributeTok{width =} \FloatTok{0.1}\NormalTok{) }\SpecialCharTok{+}
    \FunctionTok{geom\_boxplot}\NormalTok{() }\SpecialCharTok{+}
    \FunctionTok{theme\_classic}\NormalTok{() }\SpecialCharTok{+}
    \FunctionTok{theme}\NormalTok{(}\AttributeTok{legend.position =} \StringTok{"none"}\NormalTok{) }\SpecialCharTok{+}
    \FunctionTok{stat\_compare\_means}\NormalTok{(}\AttributeTok{method =} \StringTok{"t.test"}\NormalTok{, }\AttributeTok{label.x =} \FloatTok{0.5}\NormalTok{)}
\NormalTok{\}}
\end{Highlighting}
\end{Shaded}

Neste momento, não serão discutidos os códigos da função. Ela será
utilizada como uma função qualquer com os dados do arquivo
\texttt{dadosPop.xlsx}\^{}{[}Para maiores detalhes, consulte a o
Capítulo~\ref{sec-testet}. Os argumentos da função são o dataframe (df =
dados), a variável x (var.x = pop) e a variável y (var.y = altura).
Dessa forma, está se comparando a altura de mulheres de duas populações
de duas regiões diferentes :

\begin{Shaded}
\begin{Highlighting}[]
\NormalTok{dados }\OtherTok{\textless{}{-}}\NormalTok{ readxl}\SpecialCharTok{::}\FunctionTok{read\_excel}\NormalTok{(}\StringTok{"dados/dadosPop.xlsx"}\NormalTok{)}
\NormalTok{dados}\SpecialCharTok{$}\NormalTok{pop }\OtherTok{\textless{}{-}} \FunctionTok{as.factor}\NormalTok{(dados}\SpecialCharTok{$}\NormalTok{pop)}
\FunctionTok{plotBpT}\NormalTok{ (}\AttributeTok{df =}\NormalTok{ dados, }\AttributeTok{var.x =}\NormalTok{ pop, }\AttributeTok{var.y =}\NormalTok{ altura)}
\end{Highlighting}
\end{Shaded}

\begin{figure}[H]

\centering{

\includegraphics[width=0.7\linewidth,height=\textheight,keepaspectratio]{04-introducaoR_files/figure-pdf/fig-twopop-1.pdf}

}

\caption{\label{fig-twopop}Comparação da altura de mulheres em duas
populações}

\end{figure}%

Observe na Figura~\ref{fig-twopop} o resultado do teste \emph{t}
(\(P = 2,2 \times 10^-16\)) e os boxplots que têm posições bem
diferentes (veja \textbf{?@sec-bxp}).

\subsubsection{Ativação de uma função
criada}\label{ativauxe7uxe3o-de-uma-funuxe7uxe3o-criada}

Para ativar uma função previamente criada, usa-se a função nativa
\texttt{source\ ()}. O argumento desta função é o caminho (no exemplo, é
o diretório do autor) onde se encontra a função buscada, por exemplo, a
função \texttt{imc()} criada acima:

\begin{Shaded}
\begin{Highlighting}[]
\FunctionTok{source}\NormalTok{(}\StringTok{\textquotesingle{}C:/Users/petro/Dropbox/Estatistica/Bioestatistica\_usando\_R/Funcoes/imc.R\textquotesingle{}}\NormalTok{)}
\end{Highlighting}
\end{Shaded}

\section{Vetores}\label{sec-vetores}

Os vetores (vector) são objetos que armazenam um ou mais elementos, do
tipo \texttt{numeric}, \texttt{character} ou \texttt{logical}. Um
\textbf{vetor} é uma variável com um ou mais valores do mesmo tipo. Por
exemplo, o número de filhos em 10 famílias foi 4, 5, 3, 2, 2, 1, 2, 1, 3
e 2. O vetor nomeado de \texttt{n.filhos} é um objeto numérico. Todo
vetor tem duas características: comprimento e nomes dos elementos que
podem ser encontradas com as funções length() e names, repectivamente. A
maneira mais fácil de criar um vetor é concatenar (ligar) os 10 valores,
usando a função concatenar \texttt{c()}:

\begin{Shaded}
\begin{Highlighting}[]
\NormalTok{n.filhos }\OtherTok{\textless{}{-}} \FunctionTok{c}\NormalTok{(}\DecValTok{4}\NormalTok{, }\DecValTok{5}\NormalTok{, }\DecValTok{3}\NormalTok{, }\DecValTok{2}\NormalTok{, }\DecValTok{2}\NormalTok{, }\DecValTok{1}\NormalTok{, }\DecValTok{2}\NormalTok{, }\DecValTok{1}\NormalTok{, }\DecValTok{3}\NormalTok{, }\DecValTok{2}\NormalTok{)}
\NormalTok{n.filhos}
\end{Highlighting}
\end{Shaded}

\begin{verbatim}
 [1] 4 5 3 2 2 1 2 1 3 2
\end{verbatim}

\begin{Shaded}
\begin{Highlighting}[]
\FunctionTok{length}\NormalTok{(n.filhos)}
\end{Highlighting}
\end{Shaded}

\begin{verbatim}
[1] 10
\end{verbatim}

\begin{Shaded}
\begin{Highlighting}[]
\FunctionTok{names}\NormalTok{(n.filhos)}
\end{Highlighting}
\end{Shaded}

\begin{verbatim}
NULL
\end{verbatim}

Como não há nomes no objeto \texttt{n.filhos}, a saída é \texttt{NULL}

\subsection{Indexação de vetores}\label{indexauxe7uxe3o-de-vetores}

Como os vetores são conjuntos \emph{indexados}, pode-se dizer que cada
valor dentro de um vetor tem uma \textbf{posição}. Essa posição é dada
pela ordem em que os elementos foram colocados no momento em que o vetor
foi criado. Isso nos permite acessar individualmente cada valor de um
vetor (58).\\
Para acessar um determinado valor, basta colocar a posição do mesmo
entre colchetes {[} {]}. Se há interesse em conhecer o número de filhos
da quinta família, procede-se da seguinte forma:

\begin{Shaded}
\begin{Highlighting}[]
\NormalTok{n.filhos[}\DecValTok{5}\NormalTok{]}
\end{Highlighting}
\end{Shaded}

\begin{verbatim}
[1] 2
\end{verbatim}

Se houver tentativa de acessar um valor inexixtente, o R retorna
\texttt{NA}.

\begin{Shaded}
\begin{Highlighting}[]
\NormalTok{n.filhos[}\DecValTok{11}\NormalTok{]}
\end{Highlighting}
\end{Shaded}

\begin{verbatim}
[1] NA
\end{verbatim}

Se houver necessidade de excluir um dos elementos, basta colocar entre
colchetes a posição do mesmo com sinal negativo. Por exemplo, para
excluir o valor correspondente a sexta família, usa-se:

\begin{Shaded}
\begin{Highlighting}[]
\NormalTok{n.filhos[}\SpecialCharTok{{-}}\DecValTok{6}\NormalTok{]}
\end{Highlighting}
\end{Shaded}

\begin{verbatim}
[1] 4 5 3 2 2 2 1 3 2
\end{verbatim}

Observa-se que o valor 1 foi excluído da série de elementos.

Quando são colocados elementos em um vetor que pertençam a classes
diferentes, o R promove o que se denomina de \textbf{coerção}, pois o
vetor pode ter apenas uma classe de objeto. Dessa forma, as classes mais
fortes reprimem as mais fracas. Por exemplo, sempre que for misturado
números e texto em um vetor, os números serão considerados como texto:

\begin{Shaded}
\begin{Highlighting}[]
\NormalTok{vetor }\OtherTok{\textless{}{-}} \FunctionTok{c}\NormalTok{(}\DecValTok{12}\NormalTok{, }\DecValTok{15}\NormalTok{, }\DecValTok{4}\NormalTok{, }\DecValTok{6}\NormalTok{, }\StringTok{"A"}\NormalTok{, }\StringTok{"D"}\NormalTok{)}
\NormalTok{vetor}
\end{Highlighting}
\end{Shaded}

\begin{verbatim}
[1] "12" "15" "4"  "6"  "A"  "D" 
\end{verbatim}

Veja que, agora, todos os elementos do vetor passaram a ser textos e,
por isso, estão entre aspas.

\subsection{Tipos de vetores}\label{tipos-de-vetores}

Dado um vetor, pode-se determinar seu tipo com \texttt{typeof()}, ou
verificar se é um tipo específico com uma das funções:
\texttt{is.character()},
'is.double()\texttt{,}is.integer()\texttt{,}is.logical( )`.

\begin{Shaded}
\begin{Highlighting}[]
\NormalTok{n.filhos }\OtherTok{\textless{}{-}} \FunctionTok{c}\NormalTok{(}\DecValTok{4}\NormalTok{, }\DecValTok{5}\NormalTok{, }\DecValTok{3}\NormalTok{, }\DecValTok{2}\NormalTok{, }\DecValTok{2}\NormalTok{, }\DecValTok{1}\NormalTok{, }\DecValTok{2}\NormalTok{, }\DecValTok{1}\NormalTok{, }\DecValTok{3}\NormalTok{, }\DecValTok{2}\NormalTok{)}
\FunctionTok{typeof}\NormalTok{(n.filhos)}
\end{Highlighting}
\end{Shaded}

\begin{verbatim}
[1] "double"
\end{verbatim}

\begin{Shaded}
\begin{Highlighting}[]
\FunctionTok{is.numeric}\NormalTok{(n.filhos)}
\end{Highlighting}
\end{Shaded}

\begin{verbatim}
[1] TRUE
\end{verbatim}

As expressões do tipo \emph{character} devem aparecer entre aspas duplas
ou simples. Os números no R são geralmente tratados como objetos
numéricos (números reais de dupla precisão). Mesmo números inteiros são
tratados como numéricos. Para fazer um número inteiro ser tratado como
objeto inteiro, deve-se utilizar a letra L após o número.

Os valores lógicos (ou booleanos) são TRUE ou FALSE. T ou F também são
aceitos.

\begin{Shaded}
\begin{Highlighting}[]
\NormalTok{n.filhos }\OtherTok{\textless{}{-}} \FunctionTok{c}\NormalTok{(}\DecValTok{4}\NormalTok{L, }\DecValTok{5}\NormalTok{L, }\DecValTok{3}\NormalTok{L, }\DecValTok{2}\NormalTok{L, }\DecValTok{2}\NormalTok{L, }\DecValTok{1}\NormalTok{L, }\DecValTok{2}\NormalTok{L, }\DecValTok{1}\NormalTok{L, }\DecValTok{3}\NormalTok{L, }\DecValTok{2}\NormalTok{L)}
\FunctionTok{typeof}\NormalTok{(n.filhos)}
\end{Highlighting}
\end{Shaded}

\begin{verbatim}
[1] "integer"
\end{verbatim}

\begin{Shaded}
\begin{Highlighting}[]
\FunctionTok{is.numeric}\NormalTok{(n.filhos)}
\end{Highlighting}
\end{Shaded}

\begin{verbatim}
[1] TRUE
\end{verbatim}

\begin{Shaded}
\begin{Highlighting}[]
\FunctionTok{is.double}\NormalTok{(n.filhos)}
\end{Highlighting}
\end{Shaded}

\begin{verbatim}
[1] FALSE
\end{verbatim}

\begin{Shaded}
\begin{Highlighting}[]
\NormalTok{nomes }\OtherTok{\textless{}{-}} \FunctionTok{c}\NormalTok{(}\StringTok{\textquotesingle{}Maria\textquotesingle{}}\NormalTok{, }\StringTok{\textquotesingle{}João\textquotesingle{}}\NormalTok{, }\StringTok{\textquotesingle{}Manuel\textquotesingle{}}\NormalTok{, }\StringTok{\textquotesingle{}Petronio\textquotesingle{}}\NormalTok{, }\StringTok{\textquotesingle{}José\textquotesingle{}}\NormalTok{)}
\FunctionTok{typeof}\NormalTok{(nomes)}
\end{Highlighting}
\end{Shaded}

\begin{verbatim}
[1] "character"
\end{verbatim}

\begin{Shaded}
\begin{Highlighting}[]
\FunctionTok{is.numeric}\NormalTok{(nomes)}
\end{Highlighting}
\end{Shaded}

\begin{verbatim}
[1] FALSE
\end{verbatim}

\begin{Shaded}
\begin{Highlighting}[]
\FunctionTok{is.double}\NormalTok{(nomes)}
\end{Highlighting}
\end{Shaded}

\begin{verbatim}
[1] FALSE
\end{verbatim}

\begin{Shaded}
\begin{Highlighting}[]
\NormalTok{altura }\OtherTok{\textless{}{-}} \FunctionTok{c}\NormalTok{(}\FloatTok{1.60}\NormalTok{, }\FloatTok{1.78}\NormalTok{, }\FloatTok{1.55}\NormalTok{, }\FloatTok{1.67}\NormalTok{, }\FloatTok{1.69}\NormalTok{)}
\FunctionTok{typeof}\NormalTok{(altura)}
\end{Highlighting}
\end{Shaded}

\begin{verbatim}
[1] "double"
\end{verbatim}

\begin{Shaded}
\begin{Highlighting}[]
\FunctionTok{is.numeric}\NormalTok{(altura)}
\end{Highlighting}
\end{Shaded}

\begin{verbatim}
[1] TRUE
\end{verbatim}

\begin{Shaded}
\begin{Highlighting}[]
\FunctionTok{is.double}\NormalTok{(altura)}
\end{Highlighting}
\end{Shaded}

\begin{verbatim}
[1] TRUE
\end{verbatim}

\section{Fatores}\label{sec-fatores}

A classe \texttt{factor} serve para designar categorias para um vetor.
Essa classe é semelhante a um vetor da classe \texttt{character}, mas
tem importancia maior na modelagem estatística e análise exploratóriade
dados (59). Tanto números quanto caracteres podem ser convertidos em
fatores usando a função \texttt{factor()}. Esses fatores podem não ser
ordenados representado apenas diferentes categorias ou podem representar
categorias ordenadas. A função \texttt{ordered()} gera fatores
ordenados, em ordem crescente ou descrescente.

O vetor a seguir representa o tipo de parto ocorrido em 15 anscimentos,
ond 1 = parto normal e 2 = parto cesáreo:

\begin{Shaded}
\begin{Highlighting}[]
\NormalTok{tipoParto }\OtherTok{\textless{}{-}} \FunctionTok{c}\NormalTok{ (}\DecValTok{1}\NormalTok{,}\DecValTok{1}\NormalTok{,}\DecValTok{2}\NormalTok{,}\DecValTok{1}\NormalTok{,}\DecValTok{2}\NormalTok{,}\DecValTok{2}\NormalTok{,}\DecValTok{1}\NormalTok{,}\DecValTok{2}\NormalTok{,}\DecValTok{1}\NormalTok{,}\DecValTok{1}\NormalTok{,}\DecValTok{1}\NormalTok{,}\DecValTok{2}\NormalTok{,}\DecValTok{1}\NormalTok{,}\DecValTok{1}\NormalTok{,}\DecValTok{1}\NormalTok{)}
\NormalTok{tipoParto}
\end{Highlighting}
\end{Shaded}

\begin{verbatim}
 [1] 1 1 2 1 2 2 1 2 1 1 1 2 1 1 1
\end{verbatim}

Verificando a classe desse objeto:

\begin{Shaded}
\begin{Highlighting}[]
\FunctionTok{class}\NormalTok{(tipoParto)}
\end{Highlighting}
\end{Shaded}

\begin{verbatim}
[1] "numeric"
\end{verbatim}

Como é númerico, teoricamente, seria possível realizar operações
matemáticas com ele. Entretanto, sabe-se que os números 1 e 2,
representam categorias e, consequentemente, devem ser transformados em
fatores, usando a função \texttt{factor()}:

\begin{Shaded}
\begin{Highlighting}[]
\NormalTok{tipoParto }\OtherTok{\textless{}{-}} \FunctionTok{factor}\NormalTok{(tipoParto, }
                    \AttributeTok{levels =} \FunctionTok{c}\NormalTok{(}\DecValTok{1}\NormalTok{,}\DecValTok{2}\NormalTok{),}
                    \AttributeTok{labels =} \FunctionTok{c}\NormalTok{(}\StringTok{"normal"}\NormalTok{,}\StringTok{"cesareo"}\NormalTok{))}
\NormalTok{tipoParto}
\end{Highlighting}
\end{Shaded}

\begin{verbatim}
 [1] normal  normal  cesareo normal  cesareo cesareo normal  cesareo normal 
[10] normal  normal  cesareo normal  normal  normal 
Levels: normal cesareo
\end{verbatim}

\begin{Shaded}
\begin{Highlighting}[]
\FunctionTok{class}\NormalTok{(tipoParto)}
\end{Highlighting}
\end{Shaded}

\begin{verbatim}
[1] "factor"
\end{verbatim}

Agora, a variável \texttt{tipoParto} é um fator. Se for utilizada a
função \texttt{table()} para ver o número de ocorrências de cada tipo de
parto, tem-se:

\begin{Shaded}
\begin{Highlighting}[]
\FunctionTok{table}\NormalTok{(tipoParto)}
\end{Highlighting}
\end{Shaded}

\begin{verbatim}
tipoParto
 normal cesareo 
     10       5 
\end{verbatim}

\section{Matrizes}\label{matrizes}

Uma \textbf{matriz} é uma estrutura bidimensional (linhas e colunas) que
\textbf{contém apenas um tipo de dado} --- geralmente números ou
caracteres. É ideal para operações matemática e álgebra linear.

Uma matriz pode ser criada com a função \texttt{matrix()}:

\begin{Shaded}
\begin{Highlighting}[]
\NormalTok{m }\OtherTok{\textless{}{-}} \FunctionTok{matrix}\NormalTok{(}\AttributeTok{data =} \DecValTok{1}\SpecialCharTok{:}\DecValTok{6}\NormalTok{, }\AttributeTok{nrow =} \DecValTok{2}\NormalTok{, }\AttributeTok{ncol =} \DecValTok{3}\NormalTok{)}
\NormalTok{m}
\end{Highlighting}
\end{Shaded}

\begin{verbatim}
     [,1] [,2] [,3]
[1,]    1    3    5
[2,]    2    4    6
\end{verbatim}

A indexação de uma matriz é feita com {[}linha, coluna{]}, por exemplo:

\begin{Shaded}
\begin{Highlighting}[]
\NormalTok{m[}\DecValTok{1}\NormalTok{,}\DecValTok{2}\NormalTok{]  }\CommentTok{\# retorna o valor da primeira linha, segunda coluna}
\end{Highlighting}
\end{Shaded}

\begin{verbatim}
[1] 3
\end{verbatim}

\section{Listas}\label{listas}

Uma \textbf{lista} é uma estrutura flexível e heterogênea que pode
conter elementos de tipos diferentes, como vetores, strings, funções,
até outras listas.\\

Cada elemento da lista pode ter um nome que pode ser acessado com
\texttt{\$} ou \texttt{{[}{[}{]}{]}}. É muito usado para estruturas
complexas, como resultados de modelos estatísticos.

Uma lista pode ser criada com a função `list()`` :

\begin{Shaded}
\begin{Highlighting}[]
\NormalTok{lista }\OtherTok{\textless{}{-}} \FunctionTok{list}\NormalTok{(}\AttributeTok{aluno =} \StringTok{"Gabriel"}\NormalTok{, }\AttributeTok{idade =} \DecValTok{14}\NormalTok{, }\AttributeTok{notas =} \FunctionTok{c}\NormalTok{(}\FloatTok{9.5}\NormalTok{, }\FloatTok{8.7}\NormalTok{, }\DecValTok{10}\NormalTok{, }\FloatTok{9.1}\NormalTok{))}
\NormalTok{lista}
\end{Highlighting}
\end{Shaded}

\begin{verbatim}
$aluno
[1] "Gabriel"

$idade
[1] 14

$notas
[1]  9.5  8.7 10.0  9.1
\end{verbatim}

Se o interesse é saber as notas:

\begin{Shaded}
\begin{Highlighting}[]
\NormalTok{lista}\SpecialCharTok{$}\NormalTok{notas}
\end{Highlighting}
\end{Shaded}

\begin{verbatim}
[1]  9.5  8.7 10.0  9.1
\end{verbatim}

É possível calcular a média das notas, usando a função \texttt{mean()}:

\begin{Shaded}
\begin{Highlighting}[]
\FunctionTok{mean}\NormalTok{(lista}\SpecialCharTok{$}\NormalTok{notas)}
\end{Highlighting}
\end{Shaded}

\begin{verbatim}
[1] 9.325
\end{verbatim}

\chapter{Manipulando dados no R}\label{manipulando-dados-no-r}

\section{Dataframes no R}\label{sec-dataframes}

DataFrames são objetos de dados genéricos em formato tabular, onde os
dados são organizados de maneira lógica em linha-e-coluna semelhante ao
de uma planilha do Excel. O dataframe é uma estrutura bidimensional. Os
DataFrames podem ser formados com objetos criados previamente, desde que
tenham o mesmo comprimento (60).

Uma das formas de criar um DataFrame no R é a partir de um conjunto de
vetores como, por exemplo, este relacionados a 15 nascimentos em uma
determinada maternidade:

\begin{Shaded}
\begin{Highlighting}[]
\NormalTok{id }\OtherTok{\textless{}{-}} \FunctionTok{c}\NormalTok{(}\DecValTok{1}\NormalTok{, }\DecValTok{2}\NormalTok{, }\DecValTok{3}\NormalTok{, }\DecValTok{4}\NormalTok{, }\DecValTok{5}\NormalTok{, }\DecValTok{6}\NormalTok{, }\DecValTok{7}\NormalTok{, }\DecValTok{8}\NormalTok{, }\DecValTok{9}\NormalTok{, }\DecValTok{10}\NormalTok{, }\DecValTok{11}\NormalTok{, }\DecValTok{12}\NormalTok{, }\DecValTok{13}\NormalTok{, }\DecValTok{14}\NormalTok{, }\DecValTok{15}\NormalTok{)}
\NormalTok{pesoRN }\OtherTok{\textless{}{-}} \FunctionTok{c}\NormalTok{ (}\DecValTok{3340}\NormalTok{,}\DecValTok{3345}\NormalTok{,}\DecValTok{3750}\NormalTok{,}\DecValTok{3650}\NormalTok{,}\DecValTok{3220}\NormalTok{,}\DecValTok{4070}\NormalTok{,}\DecValTok{3380}\NormalTok{,}\DecValTok{3970}\NormalTok{,}\DecValTok{3060}\NormalTok{,}\DecValTok{3180}\NormalTok{,  }
             \DecValTok{2865}\NormalTok{,}\DecValTok{2815}\NormalTok{,}\DecValTok{3245}\NormalTok{,}\DecValTok{2051}\NormalTok{,}\DecValTok{2630}\NormalTok{)  }
\NormalTok{compRN }\OtherTok{\textless{}{-}} \FunctionTok{c}\NormalTok{ (}\DecValTok{50}\NormalTok{,}\DecValTok{48}\NormalTok{,}\DecValTok{52}\NormalTok{,}\DecValTok{48}\NormalTok{,}\DecValTok{50}\NormalTok{,}\DecValTok{51}\NormalTok{,}\DecValTok{50}\NormalTok{,}\DecValTok{51}\NormalTok{,}\DecValTok{47}\NormalTok{,}\DecValTok{47}\NormalTok{,}\DecValTok{47}\NormalTok{,}\DecValTok{49}\NormalTok{,}\DecValTok{51}\NormalTok{,}\DecValTok{50}\NormalTok{,}\DecValTok{44}\NormalTok{)}
\NormalTok{sexo }\OtherTok{\textless{}{-}} \FunctionTok{c}\NormalTok{ (}\DecValTok{2}\NormalTok{,}\DecValTok{2}\NormalTok{,}\DecValTok{2}\NormalTok{,}\DecValTok{1}\NormalTok{,}\DecValTok{1}\NormalTok{,}\DecValTok{1}\NormalTok{,}\DecValTok{2}\NormalTok{,}\DecValTok{1}\NormalTok{,}\DecValTok{1}\NormalTok{,}\DecValTok{1}\NormalTok{,}\DecValTok{2}\NormalTok{,}\DecValTok{2}\NormalTok{,}\DecValTok{1}\NormalTok{,}\DecValTok{1}\NormalTok{,}\DecValTok{2}\NormalTok{)}
\NormalTok{tipoParto }\OtherTok{\textless{}{-}} \FunctionTok{c}\NormalTok{ (}\DecValTok{1}\NormalTok{,}\DecValTok{1}\NormalTok{,}\DecValTok{2}\NormalTok{,}\DecValTok{1}\NormalTok{,}\DecValTok{2}\NormalTok{,}\DecValTok{2}\NormalTok{,}\DecValTok{1}\NormalTok{,}\DecValTok{2}\NormalTok{,}\DecValTok{1}\NormalTok{,}\DecValTok{1}\NormalTok{,}\DecValTok{1}\NormalTok{,}\DecValTok{2}\NormalTok{,}\DecValTok{1}\NormalTok{,}\DecValTok{1}\NormalTok{,}\DecValTok{1}\NormalTok{)}
\NormalTok{idadeMae }\OtherTok{\textless{}{-}} \FunctionTok{c}\NormalTok{ (}\DecValTok{40}\NormalTok{,}\DecValTok{19}\NormalTok{,}\DecValTok{26}\NormalTok{,}\DecValTok{19}\NormalTok{,}\DecValTok{32}\NormalTok{,}\DecValTok{24}\NormalTok{,}\DecValTok{27}\NormalTok{,}\DecValTok{20}\NormalTok{,}\DecValTok{21}\NormalTok{,}\DecValTok{19}\NormalTok{,}\DecValTok{23}\NormalTok{,}\DecValTok{36}\NormalTok{,}\DecValTok{21}\NormalTok{,}\DecValTok{23}\NormalTok{,}\DecValTok{23}\NormalTok{) }
\end{Highlighting}
\end{Shaded}

Este grupo de vetores (variáveis) isolados fica difícil de manusear.
Portanto, seria útil reuni-los em um só objeto. Pode-se fazer isso,
usando a função \texttt{data.frame()}, do R base. Este DataFrame será
atribuído a um novo objeto de nome \texttt{dadosNeonatos}.

\begin{Shaded}
\begin{Highlighting}[]
\NormalTok{dadosNeonatos }\OtherTok{\textless{}{-}} \FunctionTok{data.frame}\NormalTok{ (id,}
\NormalTok{                             pesoRN, }
\NormalTok{                             compRN, }
\NormalTok{                             sexo, }
\NormalTok{                             tipoParto, }
\NormalTok{                             idadeMae)}
\end{Highlighting}
\end{Shaded}

Verificando a classe deste novo objeto, tem-se:

\begin{Shaded}
\begin{Highlighting}[]
\FunctionTok{class}\NormalTok{ (dadosNeonatos)}
\end{Highlighting}
\end{Shaded}

\begin{verbatim}
[1] "data.frame"
\end{verbatim}

Para observar a modificação realizada, pode-se usar a função
\texttt{str()} do R base, digitando no \emph{R Script}:

\begin{Shaded}
\begin{Highlighting}[]
\FunctionTok{str}\NormalTok{(dadosNeonatos)}
\end{Highlighting}
\end{Shaded}

\begin{verbatim}
'data.frame':   15 obs. of  6 variables:
 $ id       : num  1 2 3 4 5 6 7 8 9 10 ...
 $ pesoRN   : num  3340 3345 3750 3650 3220 ...
 $ compRN   : num  50 48 52 48 50 51 50 51 47 47 ...
 $ sexo     : num  2 2 2 1 1 1 2 1 1 1 ...
 $ tipoParto: num  1 1 2 1 2 2 1 2 1 1 ...
 $ idadeMae : num  40 19 26 19 32 24 27 20 21 19 ...
\end{verbatim}

Na saida da função, verifica-se que o dataframe contém 15 linhas e 6
colunas.

\subsection{Acrescentando variáveis a um
dataframe}\label{acrescentando-variuxe1veis-a-um-dataframe}

Será adicionada ao dataframe uma nova variável chamada \texttt{utiNeo},
que indica se cada recém-nascido foi encaminhado ou não para a UTI
neonatal logo após o nascimento. Essa variável será construída a partir
de um vetor contendo a situação de cada um dos 15 recém-nascidos, e será
incorporada como uma nova coluna no dataframe. A sintaxe utilizada para
essa operação segue o padrão , resultando em:
\texttt{nome-do-dataframe\$nome-da-variável}.

\begin{Shaded}
\begin{Highlighting}[]
\NormalTok{dadosNeonatos}\SpecialCharTok{$}\NormalTok{utiNeo }\OtherTok{\textless{}{-}} \FunctionTok{c}\NormalTok{ (}\StringTok{"não"}\NormalTok{,}\StringTok{"não"}\NormalTok{,}\StringTok{"não"}\NormalTok{,}\StringTok{"não"}\NormalTok{,}\StringTok{"sim"}\NormalTok{,}\StringTok{"não"}\NormalTok{,}\StringTok{"sim"}\NormalTok{,}\StringTok{"não"}\NormalTok{,}\StringTok{"não"}\NormalTok{,}\StringTok{"não"}\NormalTok{,}\StringTok{"não"}\NormalTok{,}\StringTok{"sim"}\NormalTok{,}\StringTok{"não"}\NormalTok{,}\StringTok{"não"}\NormalTok{,}\StringTok{"não"}\NormalTok{)}
\end{Highlighting}
\end{Shaded}

Com isso,
´utiNeo\texttt{passa\ a\ fazer\ parte\ do\ conjunto\ de\ dados,\ permitindo\ análises\ específicas\ sobre\ a\ necessidade\ de\ cuidados\ intensivos\ neonatais.\ A\ função}str()`,
pode ser usada, novamente, para observar a transformação:

\begin{Shaded}
\begin{Highlighting}[]
\FunctionTok{str}\NormalTok{(dadosNeonatos)}
\end{Highlighting}
\end{Shaded}

\begin{verbatim}
'data.frame':   15 obs. of  7 variables:
 $ id       : num  1 2 3 4 5 6 7 8 9 10 ...
 $ pesoRN   : num  3340 3345 3750 3650 3220 ...
 $ compRN   : num  50 48 52 48 50 51 50 51 47 47 ...
 $ sexo     : num  2 2 2 1 1 1 2 1 1 1 ...
 $ tipoParto: num  1 1 2 1 2 2 1 2 1 1 ...
 $ idadeMae : num  40 19 26 19 32 24 27 20 21 19 ...
 $ utiNeo   : chr  "não" "não" "não" "não" ...
\end{verbatim}

\subsection{Transformação de variáveis}\label{sec-transform}

Observa-se que todas as variáveis estão como variáveis numéricas
(\texttt{num}), exceto a variável
´utiNeo\texttt{questá\ como\ caractere\ (}chr\texttt{).\ Isto\ não\ está\ correto,\ pois\ as\ variáveis}sexo\texttt{,}tipoParto\texttt{são\ variáveis\ categóricas,\ bem\ como\ a\ \ a\ variável}utiNeo\texttt{,\ adicionada\ posteriormente.\ Elas\ necessitam\ ser\ transformadas\ para\ fatores,\ usando\ a\ função}factor()`.
Os principais argumentos dessa função são:

\begin{itemize}
\tightlist
\item
  \textbf{x} \(\to\) vetor numérico
\item
  \textbf{levels} \(\to\) vetor opcional dos valores que \emph{x} pode
  assumir
\item
  \textbf{labels} \(\to\) vetor de caracteres dos rótulos para os
  níveis, na mesma ordem
\item
  \textbf{ordered} \(\to\) vetor lógico (TRUE ou FALSE). Se TRUE, os
  níveis dos fatores são assumidos como ordenados
\end{itemize}

No exemplo, as variáveis não têm uma ordem lógica, então, o argumento
\texttt{ordered} não é necessário.

\begin{Shaded}
\begin{Highlighting}[]
\NormalTok{dadosNeonatos}\SpecialCharTok{$}\NormalTok{tipoParto }\OtherTok{\textless{}{-}} \FunctionTok{factor}\NormalTok{(dadosNeonatos}\SpecialCharTok{$}\NormalTok{tipoParto, }
                                  \AttributeTok{levels =} \FunctionTok{c}\NormalTok{(}\DecValTok{1}\NormalTok{,}\DecValTok{2}\NormalTok{),}
                                  \AttributeTok{labels =} \FunctionTok{c}\NormalTok{(}\StringTok{"normal"}\NormalTok{,}\StringTok{"cesareo"}\NormalTok{))}
\NormalTok{dadosNeonatos}\SpecialCharTok{$}\NormalTok{sexo }\OtherTok{\textless{}{-}} \FunctionTok{factor}\NormalTok{ (dadosNeonatos}\SpecialCharTok{$}\NormalTok{sexo, }
                               \AttributeTok{levels =} \FunctionTok{c}\NormalTok{(}\DecValTok{1}\NormalTok{,}\DecValTok{2}\NormalTok{), }
                               \AttributeTok{labels =} \FunctionTok{c}\NormalTok{(}\StringTok{"M"}\NormalTok{,}\StringTok{"F"}\NormalTok{)) }
\end{Highlighting}
\end{Shaded}

A variável \texttt{utiNeo} já foi inserida como \texttt{string} (texto)
e pertence a classe \texttt{character}, então, basta usar a
\texttt{as.factor()} sem necessidade de alterar os rótulos
(\texttt{labels}) nos níveis (\texttt{levels})

\begin{Shaded}
\begin{Highlighting}[]
\NormalTok{dadosNeonatos}\SpecialCharTok{$}\NormalTok{utiNeo }\OtherTok{\textless{}{-}} \FunctionTok{as.factor}\NormalTok{ (dadosNeonatos}\SpecialCharTok{$}\NormalTok{utiNeo)}
\end{Highlighting}
\end{Shaded}

Após a transformação, executa-se, novamente, a função \texttt{str()}
para ver como ficou a estrutura do dataframe:

\begin{Shaded}
\begin{Highlighting}[]
\FunctionTok{str}\NormalTok{(dadosNeonatos)}
\end{Highlighting}
\end{Shaded}

\begin{verbatim}
'data.frame':   15 obs. of  7 variables:
 $ id       : num  1 2 3 4 5 6 7 8 9 10 ...
 $ pesoRN   : num  3340 3345 3750 3650 3220 ...
 $ compRN   : num  50 48 52 48 50 51 50 51 47 47 ...
 $ sexo     : Factor w/ 2 levels "M","F": 2 2 2 1 1 1 2 1 1 1 ...
 $ tipoParto: Factor w/ 2 levels "normal","cesareo": 1 1 2 1 2 2 1 2 1 1 ...
 $ idadeMae : num  40 19 26 19 32 24 27 20 21 19 ...
 $ utiNeo   : Factor w/ 2 levels "não","sim": 1 1 1 1 2 1 2 1 1 1 ...
\end{verbatim}

Agora, as três varáveis passaram a ser fatores e as outras mantiveram-se
numéricas.

\subsection{Salvando um dataframe}\label{salvando-um-dataframe}

O dataframe, criado e modificado anteriormente, pode ser salvo para uso
posterior no diretório de trabalho.

A função \texttt{save()} realiza esta ação, usando como argumentos o
dataframe a ser salvo e o nome do arquivo (\texttt{file\ =}) entre
aspas. Por convenção, esta função salva com a extensão \texttt{.RData}
que deve ser digitada, pois o R não a adiciona automaticamente.

\begin{Shaded}
\begin{Highlighting}[]
\FunctionTok{save}\NormalTok{(dadosNeonatos, }\AttributeTok{file =} \StringTok{"dadosNeonatos.RData"}\NormalTok{)}
\end{Highlighting}
\end{Shaded}

Este comando colocará o arquivo no diretório de trabalho em uso.
Portanto, se o objetivo é salvar em outro local, deve ser informado qual
o novo diretório.

Para carregar o objeto salvo anteriormente com o comando
\texttt{save()}, usa-se a função \texttt{load()}. Se o arquivo a ser
lido não estiver no diretório de trabalho da sessão, há necessidade de
especificar o caminho até o arquivo:

\begin{Shaded}
\begin{Highlighting}[]
\FunctionTok{load}\NormalTok{(}\StringTok{"dadosNeonatos.RData"}\NormalTok{)}
\end{Highlighting}
\end{Shaded}

É possível salvar em outro tipo de extensão como Excel (\texttt{.xlsx}),
Valores Separados por Vírgula (\texttt{.csv}), etc. O procedimento é o
mesmo, mudando a função. Para salvar em uma extensão
\texttt{.xlsx},utiliza-se a função \texttt{write\_xlsx\ ()} do pacote
\texttt{writexl} (61):

\begin{Shaded}
\begin{Highlighting}[]
\NormalTok{writexl}\SpecialCharTok{::}\FunctionTok{write\_xlsx}\NormalTok{(dadosNeonatos, }\StringTok{"dados/dadosNeonatos.xlsx"}\NormalTok{)}
\end{Highlighting}
\end{Shaded}

Para salvar com a extensão \texttt{.csv}, usar a função
\texttt{write.csv()} ou \texttt{write.csv2()} que faz parte do pacote
\texttt{utils}, incluido no R base. A primeira função, usa \texttt{"."}
para a separação dos decimais e \texttt{","} para separar as variáveis;
a segunda função usa \texttt{","} para os decimais e \texttt{";"} para
separar as variáveis, convenção do Excel para algumas localidades, como
o Brasil (62). Portanto, uma maneira de salvar o arquivo é:

\begin{Shaded}
\begin{Highlighting}[]
\FunctionTok{write.csv2}\NormalTok{ (dadosNeonatos, }\StringTok{"dados/dadosNeonatos.csv"}\NormalTok{)}
\end{Highlighting}
\end{Shaded}

\section{\texorpdfstring{Importando dados de outros
\emph{softwares}}{Importando dados de outros softwares}}\label{importando-dados-de-outros-softwares}

É possível inserir dados diretamente no \emph{R Script}, como mostrado
na Seção~\ref{sec-dataframes}. Entretanto, se o conjunto de dados for
muito extenso, torna-se complicado. Desta forma, é melhor construir o
dataframe em outro software, como o Excel, SPSS, etc. e, após, quando
necessário, importar os dados para o R.

\subsection{Importando dados de um arquivo CSV}\label{sec-csv}

O formato CSV significa \emph{Comma Separated Values}, ou seja, é um
arquivo de valores separados por vírgula. Esse formato de armazenamento
é simples e agrupa informações de arquivos de texto em planilhas. É
possível gerar um arquivo \texttt{.csv}, a partir de uma planilha do
\emph{Excel}, usando o menu \texttt{salvar\ como} e escolher
\texttt{CSV}.

As funções \texttt{read.csv()} e \texttt{read.csv2()}, incluídas no R
base, podem ser utilizadas para importar arquivos CSV. Existe uma
pequena diferença entre elas. Dois argumentos dessas funções têm padrão
diferentes em cada uma. São eles: \texttt{sep} (separador de colunas) e
\texttt{dec} (separador de decimais). Na \texttt{read.csv()}, o padrão é
\texttt{sep\ =\ ”,”} e \texttt{dec\ =\ ”.”} e em \texttt{read.csv2()} o
padrão é \texttt{sep\ =\ “;”} e \texttt{dec\ =\ ”,”}. Portanto, quando
se importa um arquivo \texttt{.csv}, é importante saber qual a sua
estrutura. Verificar se os decimais estão separados por \emph{ponto} ou
por \emph{vírgula} e se as colunas (variáveis), por \emph{vírgula} ou
\emph{ponto e vírgula}. Para ver isso, basta abrir o arquivo em um bloco
de notas (por exemplo, \emph{Bloco de Notas do Windows}, \emph{Notepad
++}).

Quando se usa o \texttt{read.csv()} há necessidade de informar o
separador e o decimal, pois senão ele usará o padrão inglês e o arquivo
não será lido. Já com \texttt{read.csv2()}, que usa o padrão brasileiro,
não há necessidade de informar qual o separador de colunas e nem o
separador dos decimais.

Além disso, é necessário saber em que diretório do computador está o
arquivo para informar ao comando. Recomenda-se colocar o arquivo na
pasta do diretório de trabalho, pois assim basta apenas colocar o nome
do arquivo na função de leitura dos dados. Caso contrário, tem-se que se
usar todo o caminho (\emph{path}).

Como exemplo, será importado o arquivo \texttt{dadosNeonatos.csv} que se
encontra no diretório de trabalho do autor, salvo anteriormente. Para
obter o arquivo, siga os passos da Seção~\ref{sec-dataframes} ou clique
\href{https://github.com/petronioliveira/Arquivos/blob/main/dadosNeonatos.csv}{\textbf{aqui}}
e salve em seu diretório de trabalho.

A estrutura deste arquivo mostra que as colunas estão separadas por
ponto-e-virgula e, portanto, a leitura dos dados será feita com a função
\texttt{read.csv2()} e, como o arquivo está no diretório de trabalho,
não há necessidade de informar o diretório completo. Os dados serão
colocados em um objeto de nome \texttt{neonatos} \footnote{A mudança do
  nome do dataframe de \texttt{dadosNeonatos} para \texttt{neonatos} é
  desnecessária. Foi realizada apenas por questões didáticas.}:

\begin{Shaded}
\begin{Highlighting}[]
\NormalTok{neonatos }\OtherTok{\textless{}{-}} \FunctionTok{read.csv2}\NormalTok{(}\StringTok{"dados/dadosNeonatos.csv"}\NormalTok{)}
\end{Highlighting}
\end{Shaded}

Use a função \texttt{str()} para visualizar o conjunto de
dados:\footnote{Observe, na saída, que a variável \texttt{utiNeo}
  aparece palavras com acentuação (``não''). Às vezes, ao abrir o
  arquivo com a função \texttt{read.csv2()}, pode acontecer de esta
  palavra aparecer, por exemplo, como: ``n\xe3o''. Louco, não é? Se
  ocorrer isso, use, após o nome do arquivo e separado por vírgula, o
  argumento \texttt{fileEncoding\ =\ “latin1”}. Dessa forma, o erro será
  corrigido.}

\begin{Shaded}
\begin{Highlighting}[]
\FunctionTok{str}\NormalTok{(neonatos)}
\end{Highlighting}
\end{Shaded}

\begin{verbatim}
'data.frame':   15 obs. of  8 variables:
 $ X        : int  1 2 3 4 5 6 7 8 9 10 ...
 $ id       : int  1 2 3 4 5 6 7 8 9 10 ...
 $ pesoRN   : int  3340 3345 3750 3650 3220 4070 3380 3970 3060 3180 ...
 $ compRN   : int  50 48 52 48 50 51 50 51 47 47 ...
 $ sexo     : chr  "F" "F" "F" "M" ...
 $ tipoParto: chr  "normal" "normal" "cesareo" "normal" ...
 $ idadeMae : int  40 19 26 19 32 24 27 20 21 19 ...
 $ utiNeo   : chr  "não" "não" "não" "não" ...
\end{verbatim}

Como se observa na saída do comando, as variáveis foram importadas em
classes que vão necessitar transformações para serem usadas. Isto deve
ser feito como foi visto na Seção~\ref{sec-transform}.

\begin{tcolorbox}[enhanced jigsaw, bottomrule=.15mm, opacitybacktitle=0.6, colframe=quarto-callout-caution-color-frame, arc=.35mm, coltitle=black, toptitle=1mm, colback=white, colbacktitle=quarto-callout-caution-color!10!white, breakable, bottomtitle=1mm, rightrule=.15mm, titlerule=0mm, toprule=.15mm, opacityback=0, leftrule=.75mm, left=2mm, title=\textcolor{quarto-callout-caution-color}{\faFire}\hspace{0.5em}{Atenção}]

Toda vez que se importa um dataset, deve-se verificar atentamente a sua
estrutura antes de dar seguimento as análises

\end{tcolorbox}

Recentemente, foi desenvolvido o pacote \texttt{readr}, incluído no
conjunto de pacotes \texttt{tidyverse} (63), para lidar rapidamente com
a leitura de grandes arquivos. O pacote fornece substituições para
funções como \texttt{read.csv()}. As funções \texttt{read\_csv()} e
\texttt{read\_csv2()} oferecidas pelo \texttt{readr} são análogas às do
R base. Entretanto, são muito mais rápidas e fornecem mais recursos,
como um método compacto para especificar tipos de coluna. Além disso,
produzem \texttt{tibbles} (ver adiante, Seção~\ref{sec-tibble}) que são
mais reproduzíveis, pois as funções básicas do R herdam alguns
comportamentos do sistema operacional e das variáveis de ambiente,
portanto, o código de importação que funciona no seu computador pode não
funcionar no de outra pessoa. Para usar a função é necessário instalar e
ativar o pacote \texttt{readr}. A função \texttt{read\_csv2()} será
utilizada para criar um outro objeto de nome \texttt{recemNascidos}, mas
o conjunto de dados a ser ativado é o mesmo (\texttt{dadosNeonatos}):

\begin{Shaded}
\begin{Highlighting}[]
 \FunctionTok{library}\NormalTok{(readr)}
\NormalTok{ recemNascidos }\OtherTok{\textless{}{-}} \FunctionTok{read\_csv2}\NormalTok{(}\StringTok{"dados/dadosNeonatos.csv"}\NormalTok{)}
\end{Highlighting}
\end{Shaded}

Quando você executa \texttt{read\_csv2()}, ele imprime uma especificação
de coluna que fornece o nome e o tipo de cada coluna.

Novamente, a função \texttt{str()} mostrará a estrutura do arquivo,
incluindo mais detalhes \footnote{Da mesma maneira, como acontece com a
  função \texttt{read.csv2()}, a função equivalente do \texttt{readr}
  pode retornar erro na leitura de palavras com acento. Para corrigir
  isso, usa-se o argumento \texttt{locale\ (encoding\ =\ "latin1")}}:

\begin{Shaded}
\begin{Highlighting}[]
\FunctionTok{str}\NormalTok{(recemNascidos)}
\end{Highlighting}
\end{Shaded}

\begin{verbatim}
spc_tbl_ [15 x 8] (S3: spec_tbl_df/tbl_df/tbl/data.frame)
 $ ...1     : num [1:15] 1 2 3 4 5 6 7 8 9 10 ...
 $ id       : num [1:15] 1 2 3 4 5 6 7 8 9 10 ...
 $ pesoRN   : num [1:15] 3340 3345 3750 3650 3220 ...
 $ compRN   : num [1:15] 50 48 52 48 50 51 50 51 47 47 ...
 $ sexo     : chr [1:15] "F" "F" "F" "M" ...
 $ tipoParto: chr [1:15] "normal" "normal" "cesareo" "normal" ...
 $ idadeMae : num [1:15] 40 19 26 19 32 24 27 20 21 19 ...
 $ utiNeo   : chr [1:15] "não" "não" "não" "não" ...
 - attr(*, "spec")=
  .. cols(
  ..   ...1 = col_double(),
  ..   id = col_double(),
  ..   pesoRN = col_double(),
  ..   compRN = col_double(),
  ..   sexo = col_character(),
  ..   tipoParto = col_character(),
  ..   idadeMae = col_double(),
  ..   utiNeo = col_character()
  .. )
 - attr(*, "problems")=<externalptr> 
\end{verbatim}

\subsection{Importando um arquivo do Excel}\label{sec-xlsx}

O pacote \texttt{readxl}, pertencente ao conjunto de pacotes do
\texttt{tidyverse}, facilita a obtenção de dados do Excel para o R,
através da função \texttt{read\_excel()}. esta função tem o argumento
\texttt{sheet\ =} , que deve ser usado indicando o número ou o nome da
planilha, colocado entre aspas. Este argumento é importante se houver
mais de uma planilha, caso contrário, ele é opcional. Para saber os
outros argumentos da função, colque o cursor dentro da função e aperte a
tecla \texttt{Tab} (Figura~\ref{fig-sheet}). Isto abrirá um menu com os
argumentos:

\begin{figure}

\centering{

\includegraphics[width=0.8\linewidth,height=\textheight,keepaspectratio]{index_files/mediabag/5qY7z0C.png}

}

\caption{\label{fig-sheet}Argumentos da função para importar arquivos
.xlsx}

\end{figure}%

Será feita a leitura dos mesmos dados, usados na leitura de dados
\texttt{csv}, apenas o arquivo agora está no formato \texttt{.xlsx}.
Para obter o arquivo, siga os mesmos passos, usados anteriormente.
Clique
\href{https://github.com/petronioliveira/Arquivos/blob/main/dadosNeonatos.xlsx}{\textbf{aqui}}
e salve em seu diretório de trabalho.

Os dados serão atribuídos a um objeto com outro nome
(\texttt{recemNatos}):

\begin{Shaded}
\begin{Highlighting}[]
\NormalTok{recemNatos }\OtherTok{\textless{}{-}}\NormalTok{ readxl}\SpecialCharTok{::}\FunctionTok{read\_excel}\NormalTok{(}\StringTok{"dados/dadosNeonatos.xlsx"}\NormalTok{)}
\FunctionTok{str}\NormalTok{(recemNatos)}
\end{Highlighting}
\end{Shaded}

\begin{verbatim}
tibble [15 x 7] (S3: tbl_df/tbl/data.frame)
 $ id       : num [1:15] 1 2 3 4 5 6 7 8 9 10 ...
 $ pesoRN   : num [1:15] 3340 3345 3750 3650 3220 ...
 $ compRN   : num [1:15] 50 48 52 48 50 51 50 51 47 47 ...
 $ sexo     : chr [1:15] "F" "F" "F" "M" ...
 $ tipoParto: chr [1:15] "normal" "normal" "cesareo" "normal" ...
 $ idadeMae : num [1:15] 40 19 26 19 32 24 27 20 21 19 ...
 $ utiNeo   : chr [1:15] "não" "não" "não" "não" ...
\end{verbatim}

Como se vê ao analisar a estrutura, deve-se proceder transformações nas
variáveis, como visto na Seção~\ref{sec-transform}.

Na Figura~\ref{fig-sheet}, o duplo dois pontos (\texttt{::}) precedido
do nome do pacote, no caso \texttt{readxl}, especifica a procedência da
função usada. Nesta situação, não há necessidade de usar a função
\texttt{library()} para carregar o pacote já instalado em um diretório
(biblioteca) previamente.

\subsection{Importando arquivos com o
RStudio}\label{importando-arquivos-com-o-rstudio}

O \texttt{RStudio} permite importar arquivos sem a necessidade de
digitar comandos, que, para alguns podem ser tediosos.

Na tela inicial do \texttt{RStudio}, à direita, na parte superior,
clique na aba \emph{Environment} e em \texttt{Import\ Dataset}. Esta
ação abre um menu que permite importar arquivos .csv, Excel, SPSS, etc.

Por exemplo, para importar o arquivo \texttt{dadosNeonatos.xlsx}, clicar
em \texttt{From\ Excel...} Abre uma janela com uma caixa de diálogo.
Clicar no botão \texttt{Browse...}, localizado em cima à direita, para
buscar o arquivo \texttt{dadosNeonatos.xlsx}. Assim que o arquivo for
aberto, ele mostra uma \emph{preview} do arquivo e, em baixo, à direita
mostra uma \emph{preview} do código (Figura~\ref{fig-import})), igual ao
digitado anteriormente, que cria um objeto denominado
\texttt{dadosNeonatos}, nome do objeto escolhido pelo R, mas pode ser
modificado na janela, à esquerda, \texttt{Import\ Option} em
\texttt{Name}, onde pode-se digitar qualquer nome. Após encerrar as
escolhas, clicar em \texttt{Import}. É um caminho diferente para fazer o
mesmo. Este é um dos fascínios do R!

\begin{figure}

\centering{

\includegraphics[width=1\linewidth,height=\textheight,keepaspectratio]{index_files/mediabag/xjEpK6A.png}

}

\caption{\label{fig-import}Importando arquivos do excel com o RStudio}

\end{figure}%

\section{Tibble}\label{sec-tibble}

A maneira mais comum de armazenar dados no R é usar \texttt{data.frames}
ou \texttt{tibble}.

\emph{Tibble} é um novo tipo de dataframe. É como se fosse um dataframe
mais moderno. Ele mantém muitos recursos importantes do data frame
original, mas remove muitos dos recursos desatualizados.

A maioria dos pacotes do R usa dataframes tradicionais, entretanto é
possível transformá-los para \texttt{tibble}, usando a função
\texttt{as\_tibble()}, incluída no pacote \texttt{tidyr} (64). O único
propósito deste pacote é simplificar o processo de criação dados
arrumados organizados (\texttt{tidy\ data}). A transformação de um
dataframe tradicional em um \texttt{tibble}, é um procedimento
rescomendável, em função da maior flexibilidade destes.

Como exemplo deste procedimento, será usado o dataframes criado na
Seção~\ref{sec-dataframes}: \texttt{dadosNeonatos}.

Este é um conjunto de dados da classe \texttt{data.frame}, contendo 15
observações de 7 variáveis (colunas), pode convetrtido a um
\texttt{tibble}, usando a função \texttt{as.tibble()}:

\begin{Shaded}
\begin{Highlighting}[]
\FunctionTok{library}\NormalTok{(tidyr)}
\FunctionTok{as\_tibble}\NormalTok{(dadosNeonatos)}
\end{Highlighting}
\end{Shaded}

\begin{verbatim}
# A tibble: 15 x 7
      id pesoRN compRN sexo  tipoParto idadeMae utiNeo
   <dbl>  <dbl>  <dbl> <fct> <fct>        <dbl> <fct> 
 1     1   3340     50 F     normal          40 não   
 2     2   3345     48 F     normal          19 não   
 3     3   3750     52 F     cesareo         26 não   
 4     4   3650     48 M     normal          19 não   
 5     5   3220     50 M     cesareo         32 sim   
 6     6   4070     51 M     cesareo         24 não   
 7     7   3380     50 F     normal          27 sim   
 8     8   3970     51 M     cesareo         20 não   
 9     9   3060     47 M     normal          21 não   
10    10   3180     47 M     normal          19 não   
11    11   2865     47 F     normal          23 não   
12    12   2815     49 F     cesareo         36 sim   
13    13   3245     51 M     normal          21 não   
14    14   2051     50 M     normal          23 não   
15    15   2630     44 F     normal          23 não   
\end{verbatim}

Por padrão, quando o dataset é muito longo, apenas as primeiras dez
linhas são mostradas. Aqui, aparece o toda a estrutura dos dados. São
apresentadas a dimensão da tabela e as classes de cada coluna.
Verifica-se que não houve grandes mudanças, apenas o conjunto de dados
está estruturalmente mais organizado, mais flexível.

\section{Pacote tidyverse}\label{sec-tidyverse}

A denominada ciência de dados é difícil de definir, pois a definição
depende da formação específica de cada cientista de dados. Entretanto, é
possível mostrar como a ciência de dados pode ser realizada na prática,
constituindo aquilo que se costuma chamar de \emph{Ciclo da Ciência de
Dados} (Figura~\ref{fig-ciclo}).\\
Primeiramente, os dados brutos são coletados de diversas maneiras (veja
Capítulo~\ref{sec-producao}). Após, são armazenados, por exemplo, em
Excel, e, em seguida, são arrumados para reduzir problemas de
padronização, conceituais e erros ou exclusão de variáveis e casos que
não fazem parte do objetivo estabelecido no projeto de pesquisa. Isto
constituirá a base de dados analítica.\\
A base de dados analítica é então transformada, refinada, para produzir
medidas resumidoras, tabela e gráficos. Quando necessário, são
produzidos modelos estatíticos. O resultado final deve ser comunicado
através dos meios de divulgação científica (relatórios, periódicos,
livros, jornadas, congressos, GitHub, etc.).

\begin{figure}

\centering{

\includegraphics[width=0.8\linewidth,height=\textheight,keepaspectratio]{index_files/mediabag/UYe9AIO.png}

}

\caption{\label{fig-ciclo}Ciclo da Ciência de Dados}

\end{figure}%

O pacote \texttt{tidyverse} é uma coleção de pacotes para a linguagem de
programação R, pensada e desenvolvida para facilitar e otimizar o fluxo
de trabalho em ciência de dados (63).\\
Em vez de ser um pacote único, ele é um ``meta-pacote'', o que significa
que, ao instalá-lo, você instala vários pacotes menores que trabalham em
conjunto.

A filosofia principal por trás do \texttt{tidyverse} é a de dados
``tidy'' (arrumados ou organizados) (65), onde:

\begin{itemize}
\item
  Cada variável está em uma coluna.
\item
  Cada observação está em uma linha.
\item
  Cada valor está em uma célula.
\end{itemize}

Seguindo essa lógica, as funções e pacotes do \texttt{tidyverse} são
projetados para facilitar a manipulação dos dados dentro do Ciclo da
Ciência de Dados (@fig-ciclo), tornando a manipulação, a análise e a
visualização de dados mais consistentes e intuitivas.

\subsection{Principais Pacotes do
tidyverse}\label{principais-pacotes-do-tidyverse}

O \texttt{tidyverse} simplifica tarefas complexas, oferecendo
ferramentas específicas para cada etapa do Ciclo de Ciências de Dados.
Alguns dos pacotes mais importantes são:

\begin{itemize}
\item
  \textbf{dplyr}: Para manipulação de dados. Contém funções essenciais
  para filtrar linhas, selecionar colunas, criar novas variáveis, e
  resumir dados.
\item
  \textbf{ggplot2}: Para visualização de dados. É um dos pacotes mais
  populares do R para criar gráficos esteticamente agradáveis e
  informativos.
\item
  \textbf{tidyr}: Para organizar os dados. Ele transforma dados
  ``bagunçados'' (que não seguem o formato tidy) em um formato mais
  limpo e organizado.
\item
  \textbf{readr}: Para importar dados. Permite ler arquivos de texto
  (como CSVs) de forma rápida e robusta.
\item
  \textbf{tibble}: Uma versão aprimorada do data.frame do R base. O
  \texttt{tibble} é mais fácil de usar e interage melhor com os pacotes
  do \texttt{tidyverse}.
\item
  \textbf{stringr}: Para manipular strings (textos). Simplifica as
  tarefas de trabalhar com dados de texto.
\item
  \textbf{forcats}: Para lidar com fatores (variáveis categóricas).
\end{itemize}

\subsection{Princípios do tidyverse}\label{sec-printidyverse}

O \texttt{tidyverse} causou quase uma revolução na comunidade do R.
Aguns chegam a dizer que existe uma linguagem R antes e depois do
\texttt{tidyverse}. Pode parecer exagero, mas o certo é que o uso dos
princípios do \texttt{tidyverse} foi abraçado pela maioria dos usuários
de R e, em função disso, foram criados uma grande quantidade de pacotes
que conversam entre si, facilitando o manuseio dos dados.

Os princípios fundamentais são:

\begin{enumerate}
\def\labelenumi{\arabic{enumi}.}
\item
  Reutilizar estruturas de dados existentes.
\item
  Organizar funções simples usando o \texttt{pipe}.\footnote{O nome é
    uma referência ao famoso quadro do pintor belga René Magritte
    \href{https://decorem.com.br/blogs/news/significado-da-obra-a-traicao-das-imagens-de-rene-magritte?srsltid=AfmBOoqG3gSMhNdnwgmQ3piepQPRSkJlXkNQjQv5mDq72QYRwub3vJKz}{\emph{La
    Trahison des images}} \emph{(Ceci n'est pas une pipe)}.}. Esse
  operador, introduzido por Stefan Milton Bache no pacote
  \texttt{magrittr} (66), permite encadear múltiplas operações em uma
  sequência clara e legível, de forma que a saída de uma função se torna
  a entrada da próxima. Isso torna o código mais fácil de ler e
  entender, eliminando a necessidade de criar muitas variáveis
  intermediárias. O \texttt{pipe} pode ser acionado digitando
  \texttt{\%\textgreater{}\%} ou usando o atalho
  \texttt{ctrl\ +\ shift\ +\ M}. \footnote{Além do \texttt{pipe},
    indiretamente, embutido no \texttt{tidyverse}, existe o pipe nativo
    do R (\texttt{\textbar{}\textgreater{}}) que também pode ser usado
    com as mesmas funções do \texttt{\%\textgreater{}\%}.}
\item
  Usar uma sintaxe consistente: As funções dos pacotes tidyverse seguem
  uma lógica de nomeação e de argumentos parecida, facilitando a
  memorização e o uso.
\item
  Projetado com foco nos seres humanos, priorizando a eficiência do
  programador.
\end{enumerate}

Quando o \texttt{tidyverse} é carregado, todos os pacotes embutidos
nele, serão carregados.

\begin{Shaded}
\begin{Highlighting}[]
\FunctionTok{library}\NormalTok{(tidyverse)}
\end{Highlighting}
\end{Shaded}

\begin{verbatim}
-- Attaching core tidyverse packages ------------------------ tidyverse 2.0.0 --
v dplyr     1.1.4     v purrr     1.1.0
v forcats   1.0.0     v stringr   1.5.1
v ggplot2   3.5.2     v tibble    3.3.0
v lubridate 1.9.4     
-- Conflicts ------------------------------------------ tidyverse_conflicts() --
x dplyr::filter() masks stats::filter()
x dplyr::lag()    masks stats::lag()
i Use the conflicted package (<http://conflicted.r-lib.org/>) to force all conflicts to become errors
\end{verbatim}

\begin{tcolorbox}[enhanced jigsaw, bottomrule=.15mm, opacitybacktitle=0.6, colframe=quarto-callout-warning-color-frame, arc=.35mm, coltitle=black, toptitle=1mm, colback=white, colbacktitle=quarto-callout-warning-color!10!white, breakable, bottomtitle=1mm, rightrule=.15mm, titlerule=0mm, toprule=.15mm, opacityback=0, leftrule=.75mm, left=2mm, title=\textcolor{quarto-callout-warning-color}{\faExclamationTriangle}\hspace{0.5em}{Conflito de funções ao carregar tidyverse}]

Ao carregar o pacote \texttt{tidyverse} ou qualquer outro, podem surgir
mensagens de conflito indicando que funções previamente disponíveis
foram sobrescritas por versões de mesmo nome.

No exemplo apresentado, as funções \texttt{filter()} e \texttt{lag()} do
pacote \texttt{stats} foram substituídas pelas versões do pacote
\texttt{dplyr}.

Para utilizar as funções originais do pacote \texttt{stats} após esse
carregamento, é necessário especificar o namespace diretamente:\\
\texttt{stats::filter()} e \texttt{stats::lag()}.

\end{tcolorbox}

\section{Pacote dplyr}\label{sec-dplyr}

Um dos pacotes de maior utilidade abarcado pelo \texttt{tidyverse} é o
\texttt{dplyr}. Ele permite realizar transformação dos dados de uma
forma simples e eficiente. O uso dos verbos \texttt{dplyr}, aliado ao
operador \texttt{pipe}, tendem a tornar os scripts mais ``enxutos'' e
elegantes (67).\\
As principais funções do \texttt{dplyr} são:

\begin{itemize}
\tightlist
\item
  \texttt{select()} - seleciona colunas\\
\item
  \texttt{arrange()} - ordena uma variável em ordem crescente ou
  descrescente\\
\item
  \texttt{filter()} - filtra linhas\\
\item
  \texttt{mutate()} - cria/modifica colunas\\
\item
  \texttt{group\_by()} - agrupa por fatores\\
\item
  \texttt{summarise()} - sumariza a base
\end{itemize}

Todas as funções tem as mesmas características: o primeiro argumento é
um tibble e o que será a ação da função nos outrso argumentos.

Neste capítulo e em muitos outros deste livro, será outilizado o
dataframe \texttt{dadosMater.xlsx}.

\section{\texorpdfstring{Dataframe
\texttt{dadosMater.xlsx}}{Dataframe dadosMater.xlsx}}\label{sec-dadosMater}

O arquivo \texttt{dadosMater.xlsx} é um dataframe constituído por dados
de 1568 partos consecutivos do Hospital Geral de Caxias do Sul (HGCS)
\footnote{Hospital Escola da Universidade de Caxias do Sul, RS.} ,
durante um estudo sobre infecções congênitas (68). Para baixar esses
dados, clique
\href{https://github.com/petronioliveira/Arquivos/blob/main/dadosMater.xlsx}{aqui}
e faça o \emph{download} para o diretório de trabalho, para uso
posterior.

\subsection{Leitura dos dados}\label{leitura-dos-dados}

Para ler arquivos do Excel (\texttt{.xlsx}), o pacote ideal é o
\texttt{readxl}, que também faz parte do \texttt{tidyverse}. Ele é leve,
rápido e não depende do Excel instalado. A função a ser usada é
\texttt{read\_excel()}. O objeto \texttt{mater} será utilizado para
receber os dados:

\begin{Shaded}
\begin{Highlighting}[]
\NormalTok{mater }\OtherTok{\textless{}{-}}\NormalTok{ readxl}\SpecialCharTok{::}\FunctionTok{read\_excel}\NormalTok{(}\StringTok{"dados/dadosMater.xlsx"}\NormalTok{)}
\end{Highlighting}
\end{Shaded}

\begin{tcolorbox}[enhanced jigsaw, bottomrule=.15mm, opacitybacktitle=0.6, colframe=quarto-callout-important-color-frame, arc=.35mm, coltitle=black, toptitle=1mm, colback=white, colbacktitle=quarto-callout-important-color!10!white, breakable, bottomtitle=1mm, rightrule=.15mm, titlerule=0mm, toprule=.15mm, opacityback=0, leftrule=.75mm, left=2mm, title=\textcolor{quarto-callout-important-color}{\faExclamation}\hspace{0.5em}{Importante}]

O comando para carregar o conjunto de dados somente funciona ,sem
colocar o caminho completo, se tudo está sendo realizado no diretório de
trabalho

\end{tcolorbox}

Como rotina, em análise de dados, após a leitura é interessante explorar
a estrutura do dados. A função \texttt{as\_tibble()} é interessante para
ver a estrutura de um \texttt{tibble}:

\begin{Shaded}
\begin{Highlighting}[]
\FunctionTok{as\_tibble}\NormalTok{(mater)}
\end{Highlighting}
\end{Shaded}

\begin{verbatim}
# A tibble: 1,368 x 30
      id idadeMae altura  peso ganhoPeso anosEst   cor eCivil renda  fumo
   <dbl>    <dbl>  <dbl> <dbl>     <dbl>   <dbl> <dbl>  <dbl> <dbl> <dbl>
 1     1       42   1.65  69.9       3.9       3     2      1  1.45     2
 2     2       29   1.66  78        16.5      11     1      2  2.41     2
 3     3       19   1.72  81         5         9     2      1  1.93     2
 4     4       31   1.55  74        43         5     2      2  1.45     2
 5     5       34   1.6   60        15         7     2      2  0.48     2
 6     6       29   1.5   60        11.4       8     2      2  0.96     1
 7     7       30   1.54  75.5      10.5       4     1      2  1.2      1
 8     8       34   1.63  61         9         6     1      2  2.41     2
 9     9       17   1.68  57        15        10     1      2  2.17     2
10    10       32   1.5   70        11.4       1     2      2  0.72     2
# i 1,358 more rows
# i 20 more variables: quantFumo <dbl>, prenatal <dbl>, para <dbl>,
#   droga <dbl>, ig <dbl>, tipoParto <dbl>, pesoPla <dbl>, sexo <dbl>,
#   pesoRN <dbl>, compRN <dbl>, pcRN <dbl>, apgar1 <dbl>, apgar5 <dbl>,
#   utiNeo <dbl>, obito <dbl>, hiv <dbl>, sifilis <dbl>, rubeola <dbl>,
#   toxo <dbl>, infCong <dbl>
\end{verbatim}

Por padrão, a função retorna as dez primeiras linhas. Além disso,
colunas que não couberem na largura da tela serão omitidas. Também são
apresentadas a dimensão da tabela e as classes de cada coluna.
Observa-se que ele tem 1368 linhas (observações) e 30 colunas
(variáveis). Além disso, verifica-se que todas as variáveis estão como
numéricas (\texttt{dbl}) e, certamente, algumas, dependendo do objetivo
na análise, precisarão ser transformadas.

O significado de cada uma das variáveis do \texttt{tibble}
\texttt{mater} é o seguinte:

\begin{itemize}
\tightlist
\item
  \textbf{id} \(\to\) identificação do participante\\
\item
  \textbf{idadeMae} \(\to\) idade da parturiente em anos\\
\item
  \textbf{altura} \(\to\) altura da parturiente em metros\\
\item
  \textbf{peso} \(\to\) peso da parturiente em kg\\
\item
  \textbf{ganhoPeso} \(\to\) aumento de peso durante a gestação\\
\item
  \textbf{anosEst} \(\to\) anos de estudo completos\\
\item
  \textbf{cor} \(\to\) cor declarada pela parturiente: 1 = branca; 2 =
  não branca\\
\item
  \textbf{eCivil} \(\to\) estado civil: 1 = solteira; 2 = casada ou
  companheira\\
\item
  \textbf{renda} \(\to\) renda familiar em salários minimos\\
\item
  \textbf{fumo} \(\to\) tabagismo: 1 = sim; 2 = não\\
\item
  \textbf{quantFumo} \(\to\) quantidade de cigarros fumados
  diariamente\\
\item
  \textbf{prenatal} \(\to\) realizou pelo menos 6 consultas no
  pré-natal? 1 = sim; 2 = não
\item
  \textbf{para} \(\to\) número de filhos paridos\\
\item
  \textbf{droga} \(\to\) drogadição? 1 = sim; 2 = não\\
\item
  \textbf{ig} \(\to\) idade gestacional em semanas\\
\item
  \textbf{tipoParto} \(\to\) tipo de parto: 1 = normal; 2 = cesareana\\
\item
  \textbf{pesoPla} \(\to\) peso da placenta em gramas
\item
  \textbf{sexo} \(\to\) sexo do recém-nascido (RN): 1 = masc; 2 = fem\\
\item
  \textbf{pesoRN} \(\to\) peso do RN em gramas\\
\item
  \textbf{compRN} \(\to\) comprimento do RN em cm\\
\item
  \textbf{pcRN} \(\to\) perímetro cefálico dorecém-nascido em cm\\
\item
  \textbf{apgar1} \(\to\) escore de Apgar no primeiro minuto\\
\item
  \textbf{apgar5} \(\to\) escore de Apgar no quinto minuto\\
\item
  \textbf{utiNeo} \(\to\) RN necessitou de terapia intesiva? 1 = sim; 2
  = não\\
\item
  \textbf{obito} \(\to\) obito no período neonatal? 1 = sim; 2 = não\\
\item
  \textbf{hiv} \(\to\) parturiente portadora de HIV? 1 = sim; 2 = não\\
\item
  \textbf{sifilis} \(\to\) parturiente portadora de sífilis? 1 = sim; 2
  = não\\
\item
  \textbf{rubeola} \(\to\) parturiente portadora de rubéola? 1 = sim; 2
  = não\\
\item
  \textbf{toxo} \(\longrightarrow\) parturiente portadora de
  toxoplasmose? 1 = sim; 2 = não\\
\item
  \textbf{infCong} \(\to\) parturiente portadora de alguma infecção
  congênita? 1 = sim; 2 = não
\end{itemize}

O \emph{tibble} mater pode ser facilmente modificado com os verbos do
pacote \texttt{dplyr}.

\subsection{Selecionando colunas}\label{selecionando-colunas}

A função \texttt{select\ ()} pode ser usada para escolher quais colunas
(variáveis) entrarão na análise. Ela recebe como primeiro argumento o
conjunto de dados e os demais argumentos são os nomes das colunas. O
nome das colunas devem estar entre aspas.\\
O conjunto de dados \texttt{mater} contém 30 colunas e muitas podem ser
removidas, dependendo do objetivo da análise.

\begin{Shaded}
\begin{Highlighting}[]
\NormalTok{mater }\OtherTok{\textless{}{-}}\NormalTok{ dplyr}\SpecialCharTok{::}\FunctionTok{select}\NormalTok{(mater, }\SpecialCharTok{{-}}\NormalTok{ obito, }\SpecialCharTok{{-}}\NormalTok{hiv, }\SpecialCharTok{{-}}\NormalTok{sifilis, }\SpecialCharTok{{-}}\NormalTok{rubeola, }\SpecialCharTok{{-}}\NormalTok{toxo)}
\end{Highlighting}
\end{Shaded}

Note que foi usado o sinal de menos (-) antes das variáveis, porque elas
foram removidas. Também poderiam ser listadas as variáveia que
permanecem que, automaticamente, as não listadas serão removidas.

Outra maneira, pode ser colocando o número da coluna como abaixo, o
sinal de subtração antes da função concatenar c()com os números das
colunas a serem removidas (25 a 29):

\begin{Shaded}
\begin{Highlighting}[]
\NormalTok{mater }\OtherTok{\textless{}{-}}\NormalTok{ dplyr}\SpecialCharTok{::}\FunctionTok{select}\NormalTok{(mater, }\SpecialCharTok{{-}} \FunctionTok{c}\NormalTok{(}\DecValTok{25}\SpecialCharTok{:}\DecValTok{29}\NormalTok{))}
\end{Highlighting}
\end{Shaded}

A função \texttt{str()} permite visualizar a nova estrutura:

\begin{Shaded}
\begin{Highlighting}[]
\FunctionTok{str}\NormalTok{(mater)}
\end{Highlighting}
\end{Shaded}

\begin{verbatim}
tibble [1,368 x 25] (S3: tbl_df/tbl/data.frame)
 $ id       : num [1:1368] 1 2 3 4 5 6 7 8 9 10 ...
 $ idadeMae : num [1:1368] 42 29 19 31 34 29 30 34 17 32 ...
 $ altura   : num [1:1368] 1.65 1.66 1.72 1.55 1.6 1.5 1.54 1.63 1.68 1.5 ...
 $ peso     : num [1:1368] 69.9 78 81 74 60 60 75.5 61 57 70 ...
 $ ganhoPeso: num [1:1368] 3.9 16.5 5 43 15 11.4 10.5 9 15 11.4 ...
 $ anosEst  : num [1:1368] 3 11 9 5 7 8 4 6 10 1 ...
 $ cor      : num [1:1368] 2 1 2 2 2 2 1 1 1 2 ...
 $ eCivil   : num [1:1368] 1 2 1 2 2 2 2 2 2 2 ...
 $ renda    : num [1:1368] 1.45 2.41 1.93 1.45 0.48 0.96 1.2 2.41 2.17 0.72 ...
 $ fumo     : num [1:1368] 2 2 2 2 2 1 1 2 2 2 ...
 $ quantFumo: num [1:1368] 0 0 0 0 0 10 20 0 0 0 ...
 $ prenatal : num [1:1368] 2 1 2 2 2 1 1 2 2 1 ...
 $ para     : num [1:1368] 5 0 0 1 2 1 2 1 0 4 ...
 $ droga    : num [1:1368] 2 2 2 2 2 2 2 2 2 2 ...
 $ ig       : num [1:1368] 29 33 33 33 33 33 33 33 34 34 ...
 $ tipoParto: num [1:1368] 2 2 1 1 2 1 2 1 1 2 ...
 $ pesoPla  : num [1:1368] 224 1118 452 432 574 ...
 $ sexo     : num [1:1368] 2 2 2 2 2 2 2 2 2 2 ...
 $ pesoRN   : num [1:1368] 1035 2300 1580 1840 2475 ...
 $ compRN   : num [1:1368] 35.5 45 39 41 47 41 44 44 47 48 ...
 $ pcRN     : num [1:1368] 28 32 28 32 32 29 32 32 30 27 ...
 $ apgar1   : num [1:1368] NA NA NA NA NA NA NA NA NA NA ...
 $ apgar5   : num [1:1368] NA NA NA NA NA NA NA NA NA NA ...
 $ utiNeo   : num [1:1368] 1 2 1 1 1 1 2 2 1 1 ...
 $ infCong  : num [1:1368] 2 2 2 2 1 2 2 2 2 2 ...
\end{verbatim}

A função \texttt{select\ ()} pode ser combinada com outras funções, como
\texttt{filter\ ()}.

\subsection{Modificando e criando novas
colunas}\label{modificando-e-criando-novas-colunas}

Para esta ação, usa-se a função \texttt{mutate()}. Por exemplo, todas as
variáveis no \emph{tibble} \texttt{mater} foram lidas como numéricas.
Entretanto, as variáveis \texttt{cor}, \texttt{estCivil}, \texttt{fumo},
\texttt{prenatal}, \texttt{droga}, \texttt{tipoParto}, \texttt{sexo},
\texttt{utiNeo} e \texttt{infCong} são categóricas e devem ser
convertidas para fator, usando a função \texttt{factor()} associada a
\texttt{mutate()}:

\begin{Shaded}
\begin{Highlighting}[]
\NormalTok{mater }\OtherTok{\textless{}{-}}\NormalTok{ dplyr}\SpecialCharTok{::}\FunctionTok{mutate}\NormalTok{(mater,}
                        \AttributeTok{cor =} \FunctionTok{factor}\NormalTok{(cor, }
                                     \AttributeTok{levels =} \FunctionTok{c}\NormalTok{(}\DecValTok{1}\NormalTok{,}\DecValTok{2}\NormalTok{), }
                                     \AttributeTok{labels =} \FunctionTok{c}\NormalTok{(}\StringTok{"branca"}\NormalTok{, }\StringTok{"não branca"}\NormalTok{)), }
                        \AttributeTok{eCivil =} \FunctionTok{factor}\NormalTok{(eCivil, }
                                          \AttributeTok{levels =} \FunctionTok{c}\NormalTok{(}\DecValTok{1}\NormalTok{,}\DecValTok{2}\NormalTok{), }
                                          \AttributeTok{labels =} \FunctionTok{c}\NormalTok{(}\StringTok{"solteira"}\NormalTok{, }\StringTok{"não branca"}\NormalTok{)),}
                        \AttributeTok{fumo =} \FunctionTok{factor}\NormalTok{(fumo, }
                                      \AttributeTok{levels =} \FunctionTok{c}\NormalTok{(}\DecValTok{1}\NormalTok{,}\DecValTok{2}\NormalTok{), }
                                      \AttributeTok{labels =} \FunctionTok{c}\NormalTok{(}\StringTok{"sim"}\NormalTok{, }\StringTok{"não"}\NormalTok{)), , }
                        \AttributeTok{prenatal =} \FunctionTok{factor}\NormalTok{(prenatal, }
                                          \AttributeTok{levels =} \FunctionTok{c}\NormalTok{(}\DecValTok{1}\NormalTok{,}\DecValTok{2}\NormalTok{), }
                                          \AttributeTok{labels =} \FunctionTok{c}\NormalTok{(}\StringTok{"sim"}\NormalTok{, }\StringTok{"não"}\NormalTok{)), }
                        \AttributeTok{droga =} \FunctionTok{factor}\NormalTok{(droga, }
                                       \AttributeTok{levels =} \FunctionTok{c}\NormalTok{(}\DecValTok{1}\NormalTok{,}\DecValTok{2}\NormalTok{), }
                                       \AttributeTok{labels =} \FunctionTok{c}\NormalTok{(}\StringTok{"sim"}\NormalTok{, }\StringTok{"não"}\NormalTok{)), }
                        \AttributeTok{tipoParto =} \FunctionTok{factor}\NormalTok{(tipoParto, }
                                           \AttributeTok{levels =} \FunctionTok{c}\NormalTok{(}\DecValTok{1}\NormalTok{,}\DecValTok{2}\NormalTok{), }
                                           \AttributeTok{labels =} \FunctionTok{c}\NormalTok{(}\StringTok{"normal"}\NormalTok{, }\StringTok{"cesareo"}\NormalTok{)), }
                        \AttributeTok{sexo =} \FunctionTok{factor}\NormalTok{(sexo, }
                                      \AttributeTok{levels =} \FunctionTok{c}\NormalTok{(}\DecValTok{1}\NormalTok{,}\DecValTok{2}\NormalTok{), }
                                      \AttributeTok{labels =} \FunctionTok{c}\NormalTok{(}\StringTok{"masc"}\NormalTok{, }\StringTok{"fem"}\NormalTok{)), }
                        \AttributeTok{utiNeo =} \FunctionTok{factor}\NormalTok{(utiNeo, }
                                        \AttributeTok{levels =} \FunctionTok{c}\NormalTok{(}\DecValTok{1}\NormalTok{,}\DecValTok{2}\NormalTok{), }
                                        \AttributeTok{labels =} \FunctionTok{c}\NormalTok{(}\StringTok{"sim"}\NormalTok{, }\StringTok{"não"}\NormalTok{)),}
                        \AttributeTok{infCong =} \FunctionTok{factor}\NormalTok{(infCong, }
                                         \AttributeTok{levels =} \FunctionTok{c}\NormalTok{(}\DecValTok{1}\NormalTok{,}\DecValTok{2}\NormalTok{), }
                                         \AttributeTok{labels =} \FunctionTok{c}\NormalTok{(}\StringTok{"sim"}\NormalTok{, }\StringTok{"não"}\NormalTok{)))}
\end{Highlighting}
\end{Shaded}

O conjunto de dados está, agora, estruturado de forma correta.

\begin{Shaded}
\begin{Highlighting}[]
\FunctionTok{str}\NormalTok{(mater)}
\end{Highlighting}
\end{Shaded}

\begin{verbatim}
tibble [1,368 x 25] (S3: tbl_df/tbl/data.frame)
 $ id       : num [1:1368] 1 2 3 4 5 6 7 8 9 10 ...
 $ idadeMae : num [1:1368] 42 29 19 31 34 29 30 34 17 32 ...
 $ altura   : num [1:1368] 1.65 1.66 1.72 1.55 1.6 1.5 1.54 1.63 1.68 1.5 ...
 $ peso     : num [1:1368] 69.9 78 81 74 60 60 75.5 61 57 70 ...
 $ ganhoPeso: num [1:1368] 3.9 16.5 5 43 15 11.4 10.5 9 15 11.4 ...
 $ anosEst  : num [1:1368] 3 11 9 5 7 8 4 6 10 1 ...
 $ cor      : Factor w/ 2 levels "branca","não branca": 2 1 2 2 2 2 1 1 1 2 ...
 $ eCivil   : Factor w/ 2 levels "solteira","não branca": 1 2 1 2 2 2 2 2 2 2 ...
 $ renda    : num [1:1368] 1.45 2.41 1.93 1.45 0.48 0.96 1.2 2.41 2.17 0.72 ...
 $ fumo     : Factor w/ 2 levels "sim","não": 2 2 2 2 2 1 1 2 2 2 ...
 $ quantFumo: num [1:1368] 0 0 0 0 0 10 20 0 0 0 ...
 $ prenatal : Factor w/ 2 levels "sim","não": 2 1 2 2 2 1 1 2 2 1 ...
 $ para     : num [1:1368] 5 0 0 1 2 1 2 1 0 4 ...
 $ droga    : Factor w/ 2 levels "sim","não": 2 2 2 2 2 2 2 2 2 2 ...
 $ ig       : num [1:1368] 29 33 33 33 33 33 33 33 34 34 ...
 $ tipoParto: Factor w/ 2 levels "normal","cesareo": 2 2 1 1 2 1 2 1 1 2 ...
 $ pesoPla  : num [1:1368] 224 1118 452 432 574 ...
 $ sexo     : Factor w/ 2 levels "masc","fem": 2 2 2 2 2 2 2 2 2 2 ...
 $ pesoRN   : num [1:1368] 1035 2300 1580 1840 2475 ...
 $ compRN   : num [1:1368] 35.5 45 39 41 47 41 44 44 47 48 ...
 $ pcRN     : num [1:1368] 28 32 28 32 32 29 32 32 30 27 ...
 $ apgar1   : num [1:1368] NA NA NA NA NA NA NA NA NA NA ...
 $ apgar5   : num [1:1368] NA NA NA NA NA NA NA NA NA NA ...
 $ utiNeo   : Factor w/ 2 levels "sim","não": 1 2 1 1 1 1 2 2 1 1 ...
 $ infCong  : Factor w/ 2 levels "sim","não": 2 2 2 2 1 2 2 2 2 2 ...
\end{verbatim}

O Índice de Massa Corporal (IMC) é um cálculo que relaciona o peso e a
altura de uma pessoa para avaliar se ela está com o peso ideal, abaixo
do peso, acima do peso ou obesa. É uma ferramenta simples e amplamente
utilizada na área da saúde para triagem e acompanhamento do estado
nutricional. Para obter esse índice, serão utilizadas as variáveis peso
e altura da gestante no início da gravidez. Como visto na
Seção~\ref{sec-funcoes}, o cálculo do IMC é dado pela razão entre o peso
em kg e a altura em metros elevada ao quadrado. Para criar uma nova
coluna com a variável \texttt{imc}, pode-se também usar o
\texttt{mutate()}.

\begin{Shaded}
\begin{Highlighting}[]
\NormalTok{mater }\OtherTok{\textless{}{-}} \FunctionTok{mutate}\NormalTok{(mater,}
                \AttributeTok{imc =}\NormalTok{ peso}\SpecialCharTok{/}\NormalTok{altura}\SpecialCharTok{\^{}}\DecValTok{2}\NormalTok{)}
\end{Highlighting}
\end{Shaded}

\subsection{Filtrando linhas}\label{filtrando-linhas}

A função \texttt{filter()} é usada para criar um subconjunto de dados
que obedeçam determinadas condições lógicas: \& (e), \textbar{} (ou) e !
(não). Por exemplo:

\begin{itemize}
\tightlist
\item
  \textbf{y \& !x} \(\to\) seleciona \emph{y} e não \emph{x}
\item
  \textbf{x \& !y} \(\to\) seleciona \emph{x} e não \emph{y}
\item
  \textbf{x \textbar{} !x} \(\to\) seleciona \emph{x} ou \emph{y}
\item
  \textbf{x \& y} \(\to\) seleciona \emph{x} e \emph{y}
\end{itemize}

Um recém-nascido é dito a termo quando a duração da gestação é igual a
37 a 42 semanas incompletas. Para extrair do banco de dados
\texttt{mater} os recém-nascidos a termo (\texttt{dadosRNT}), pode-se
usar a função \texttt{filter()}:

\begin{Shaded}
\begin{Highlighting}[]
\NormalTok{dadosRNT }\OtherTok{\textless{}{-}}\NormalTok{ dplyr}\SpecialCharTok{::}\FunctionTok{filter}\NormalTok{ (mater, ig}\SpecialCharTok{\textgreater{}=}\DecValTok{37} \SpecialCharTok{\&}\NormalTok{ ig}\SpecialCharTok{\textless{}}\DecValTok{42}\NormalTok{)}
\end{Highlighting}
\end{Shaded}

Observe que, agora, o conjunto de dados \texttt{dadosRNT} tem 1085
linhas, número de recém-nascidos a termo do banco de dados original
\texttt{mater} (1368). Logo, os recém nascidos a termo correspondem a
79.3\% dos nascimentos, nesta maternidade.

\textbf{Outro exemplo}

Para selecionar apenas os meninos, nascidos a termo (\texttt{dadosRNT}),
codificados como \texttt{"masc"}, procede-se da seguinte maneira:

\begin{Shaded}
\begin{Highlighting}[]
\NormalTok{dadosRNT\_masc }\OtherTok{\textless{}{-}} \FunctionTok{filter}\NormalTok{ (dadosRNT, sexo }\SpecialCharTok{==} \StringTok{\textquotesingle{}masc\textquotesingle{}}\NormalTok{)}
\end{Highlighting}
\end{Shaded}

\begin{tcolorbox}[enhanced jigsaw, bottomrule=.15mm, opacitybacktitle=0.6, colframe=quarto-callout-tip-color-frame, arc=.35mm, coltitle=black, toptitle=1mm, colback=white, colbacktitle=quarto-callout-tip-color!10!white, breakable, bottomtitle=1mm, rightrule=.15mm, titlerule=0mm, toprule=.15mm, opacityback=0, leftrule=.75mm, left=2mm, title=\textcolor{quarto-callout-tip-color}{\faLightbulb}\hspace{0.5em}{Alerta}]

Não esquecer que o sinal de igualdade, no R, é representado por um ==
(duplo igual)

\end{tcolorbox}

\subsection{Sumarizando uma coluna}\label{sumarizando-uma-coluna}

Para resumir uma coluna, utilizando uma métrica de interesse, como
média, mediana, desvio padrão, etc. (Capítulo~\ref{sec-resumo}), pode-se
usar a função \texttt{summarize()}.

\begin{Shaded}
\begin{Highlighting}[]
\NormalTok{resumo }\OtherTok{\textless{}{-}}\NormalTok{ dplyr}\SpecialCharTok{::}\FunctionTok{summarize}\NormalTok{(dadosRNT,}
  \AttributeTok{n =} \FunctionTok{length}\NormalTok{ (id),}
  \AttributeTok{media =} \FunctionTok{mean}\NormalTok{(pesoRN, }\AttributeTok{na.rm =} \ConstantTok{TRUE}\NormalTok{),}
  \AttributeTok{desvpad =} \FunctionTok{sd}\NormalTok{(pesoRN, }\AttributeTok{na.rm =} \ConstantTok{TRUE}\NormalTok{)}
\NormalTok{)}
\NormalTok{resumo}
\end{Highlighting}
\end{Shaded}

\begin{verbatim}
# A tibble: 1 x 3
      n media desvpad
  <int> <dbl>   <dbl>
1  1085 3216.    462.
\end{verbatim}

Muitas vezes, há necessidade de sumarizar uma coluna agrupada pelas
categorias de uma segunda coluna. Por exemplo, peso dos recém-nascidos a
termo por sexo. Para isso, além do \texttt{summarize()}, utilizamos
também a função \texttt{group\_by()}.\\
Para facilitar o trabalho, será usado o operador pipe que pode ser
acionado digitando \texttt{\%\textgreater{}\%} ou usando o atalho
\texttt{ctrl\ +\ shift\ +\ M}\footnote{Para que o pipe
  (\texttt{\%\textgreater{}\%}) seja ativado é necessário carregar o
  pacote \texttt{dplyr}; para o pipe nativo
  (\texttt{\textbar{}\textgreater{}}) não há necessidade.}, como
observado na Seção~\ref{sec-printidyverse}. Em vez de passar o argumento
para a função separadamente, é possível escrever o valor ou objeto e, em
seguida, usar o \texttt{pipe} para convertê-lo como o argumento da
função na mesma linha. Funciona como se o \texttt{pipe} jogasse o objeto
dentro da função seguinte.

\begin{Shaded}
\begin{Highlighting}[]
\FunctionTok{library}\NormalTok{(dplyr)}
\NormalTok{resumo }\OtherTok{\textless{}{-}}\NormalTok{ dadosRNT }\SpecialCharTok{\%\textgreater{}\%} 
        \FunctionTok{group\_by}\NormalTok{(sexo) }\SpecialCharTok{\%\textgreater{}\%} 
        \FunctionTok{summarize}\NormalTok{(}
  \AttributeTok{n =} \FunctionTok{length}\NormalTok{ (id),}
  \AttributeTok{media =} \FunctionTok{mean}\NormalTok{(pesoRN, }\AttributeTok{na.rm =} \ConstantTok{TRUE}\NormalTok{),}
  \AttributeTok{desvpad =} \FunctionTok{sd}\NormalTok{(pesoRN, }\AttributeTok{na.rm =} \ConstantTok{TRUE}\NormalTok{)}
\NormalTok{)}
\NormalTok{resumo}
\end{Highlighting}
\end{Shaded}

\begin{verbatim}
# A tibble: 2 x 4
  sexo      n media desvpad
  <fct> <int> <dbl>   <dbl>
1 masc    592 3274.    458.
2 fem     493 3147.    458.
\end{verbatim}

\subsection{Selecionando linhas
específicas}\label{selecionando-linhas-especuxedficas}

A função \texttt{slice()} do pacote \texttt{dplyr} é usada para
selecionar linhas específicas de um dataframe (ou \emph{tibble}) com
base em suas posições. Ela é bastante útil quando se quer extrair
subconjuntos de dados sem usar condições lógicas, mas sim índices de
linha.

Diferente de \texttt{filter()}, que seleciona linhas baseado em
condições, \texttt{slice()} usa números de linhas. Por exemplo, para
visualizar as cinco primeiras linhas do conjunto de dados
\texttt{dadosRNT}, usa-se:

\begin{Shaded}
\begin{Highlighting}[]
\FunctionTok{slice}\NormalTok{(dadosRNT, }\DecValTok{1}\SpecialCharTok{:}\DecValTok{5}\NormalTok{)}
\end{Highlighting}
\end{Shaded}

\begin{verbatim}
# A tibble: 5 x 26
     id idadeMae altura  peso ganhoPeso anosEst cor        eCivil    renda fumo 
  <dbl>    <dbl>  <dbl> <dbl>     <dbl>   <dbl> <fct>      <fct>     <dbl> <fct>
1    20       28   1.5   48.5      11         6 não branca não bran~  3.13 não  
2    21       31   1.55  65        24         5 branca     não bran~  0.72 não  
3    22       27   1.6   60        15         8 não branca não bran~  2.41 sim  
4    23       28   1.58  47         9         8 branca     não bran~  1.69 não  
5    24       18   1.76  65.5       6.5       7 branca     solteira   1.93 sim  
# i 16 more variables: quantFumo <dbl>, prenatal <fct>, para <dbl>,
#   droga <fct>, ig <dbl>, tipoParto <fct>, pesoPla <dbl>, sexo <fct>,
#   pesoRN <dbl>, compRN <dbl>, pcRN <dbl>, apgar1 <dbl>, apgar5 <dbl>,
#   utiNeo <fct>, infCong <fct>, imc <dbl>
\end{verbatim}

A função \texttt{slice()} é compatível com agrupamentos, por exemplo,
para selecionar os cinco primeiros casos do tibble \texttt{dadosRNT} por
\texttt{sexo}:

\begin{Shaded}
\begin{Highlighting}[]
\NormalTok{dadosRNT }\SpecialCharTok{\%\textgreater{}\%}
        \FunctionTok{group\_by}\NormalTok{(sexo) }\SpecialCharTok{\%\textgreater{}\%} 
        \FunctionTok{slice}\NormalTok{(}\DecValTok{1}\SpecialCharTok{:}\DecValTok{5}\NormalTok{)}
\end{Highlighting}
\end{Shaded}

\begin{verbatim}
# A tibble: 10 x 26
# Groups:   sexo [2]
      id idadeMae altura  peso ganhoPeso anosEst cor        eCivil   renda fumo 
   <dbl>    <dbl>  <dbl> <dbl>     <dbl>   <dbl> <fct>      <fct>    <dbl> <fct>
 1    20       28   1.5   48.5      11         6 não branca não bra~  3.13 não  
 2    21       31   1.55  65        24         5 branca     não bra~  0.72 não  
 3    22       27   1.6   60        15         8 não branca não bra~  2.41 sim  
 4    23       28   1.58  47         9         8 branca     não bra~  1.69 não  
 5    24       18   1.76  65.5       6.5       7 branca     solteira  1.93 sim  
 6   751       17   1.65  60        11.4       7 não branca solteira  1.92 não  
 7   752       30   1.6   54        12         5 branca     não bra~  1.92 não  
 8   753       27   1.53  43.5      20.5      11 branca     não bra~  1.93 não  
 9   755       28   1.4   60        11.4       8 não branca não bra~  2.17 não  
10   756       17   1.55  78        20        10 branca     solteira  4.82 sim  
# i 16 more variables: quantFumo <dbl>, prenatal <fct>, para <dbl>,
#   droga <fct>, ig <dbl>, tipoParto <fct>, pesoPla <dbl>, sexo <fct>,
#   pesoRN <dbl>, compRN <dbl>, pcRN <dbl>, apgar1 <dbl>, apgar5 <dbl>,
#   utiNeo <fct>, infCong <fct>, imc <dbl>
\end{verbatim}

O \texttt{tidyverse} introduziu funções auxiliares como:

\begin{itemize}
\tightlist
\item
  \texttt{slice\_head()}: que seleciona as primeiras \texttt{n} linhas.
  A saída dessa função é a mesma anterior, se for solicitado n = 5:
\end{itemize}

\begin{verbatim}
# A tibble: 5 x 26
     id idadeMae altura  peso ganhoPeso anosEst cor        eCivil    renda fumo 
  <dbl>    <dbl>  <dbl> <dbl>     <dbl>   <dbl> <fct>      <fct>     <dbl> <fct>
1    20       28   1.5   48.5      11         6 não branca não bran~  3.13 não  
2    21       31   1.55  65        24         5 branca     não bran~  0.72 não  
3    22       27   1.6   60        15         8 não branca não bran~  2.41 sim  
4    23       28   1.58  47         9         8 branca     não bran~  1.69 não  
5    24       18   1.76  65.5       6.5       7 branca     solteira   1.93 sim  
# i 16 more variables: quantFumo <dbl>, prenatal <fct>, para <dbl>,
#   droga <fct>, ig <dbl>, tipoParto <fct>, pesoPla <dbl>, sexo <fct>,
#   pesoRN <dbl>, compRN <dbl>, pcRN <dbl>, apgar1 <dbl>, apgar5 <dbl>,
#   utiNeo <fct>, infCong <fct>, imc <dbl>
\end{verbatim}

\begin{itemize}
\tightlist
\item
  \texttt{slice\_tall()}: que seleciona as últimas \texttt{n} linhas.\\
\item
  \texttt{slice\_sample()}: seleciona linhas aleatórias. Uma amostra de
  n = 200 será extraída do \texttt{tibble} \texttt{dadosRNT}, como
  exemplo:
\end{itemize}

\begin{Shaded}
\begin{Highlighting}[]
\NormalTok{dadosRNT200 }\OtherTok{\textless{}{-}}\NormalTok{ dadosRNT }\SpecialCharTok{\%\textgreater{}\%} \FunctionTok{slice\_sample}\NormalTok{(}\AttributeTok{n =} \DecValTok{200}\NormalTok{)}
\end{Highlighting}
\end{Shaded}

\begin{itemize}
\tightlist
\item
  \texttt{slice\_min()}: selecioan as linhas com os menores valores em
  uma coluna específica. Não ordena todo o dataframe, mas sim identifica
  e extrai as linhas que têm os menores valores na coluna indicada com
  \texttt{order\_by}. Por exemplo, se o objetivo é extrarir do tibble
  dadosRNT so cinco menores pesos ao nascer por sexo:
\end{itemize}

\begin{Shaded}
\begin{Highlighting}[]
\NormalTok{dadosRNT }\SpecialCharTok{\%\textgreater{}\%} 
  \FunctionTok{select}\NormalTok{(pesoRN, sexo) }\SpecialCharTok{\%\textgreater{}\%} 
  \FunctionTok{slice\_min}\NormalTok{(}\AttributeTok{order\_by =}\NormalTok{ pesoRN, }
            \AttributeTok{n =} \DecValTok{5}\NormalTok{, }
            \AttributeTok{with\_ties =} \ConstantTok{TRUE}\NormalTok{, }
            \AttributeTok{by =}\NormalTok{ sexo)}
\end{Highlighting}
\end{Shaded}

\begin{verbatim}
# A tibble: 10 x 2
   pesoRN sexo 
    <dbl> <fct>
 1   1425 masc 
 2   1440 masc 
 3   1795 masc 
 4   1810 masc 
 5   1980 masc 
 6   1715 fem  
 7   1785 fem  
 8   1895 fem  
 9   2090 fem  
10   2095 fem  
\end{verbatim}

\begin{itemize}
\tightlist
\item
  \texttt{slice\_max()}: funcion a da mesma que o slice\_min(), apenas
  para os maiores valores.
\end{itemize}

\begin{tcolorbox}[enhanced jigsaw, bottomrule=.15mm, opacitybacktitle=0.6, colframe=quarto-callout-tip-color-frame, arc=.35mm, coltitle=black, toptitle=1mm, colback=white, colbacktitle=quarto-callout-tip-color!10!white, breakable, bottomtitle=1mm, rightrule=.15mm, titlerule=0mm, toprule=.15mm, opacityback=0, leftrule=.75mm, left=2mm, title=\textcolor{quarto-callout-tip-color}{\faLightbulb}\hspace{0.5em}{Exercício}]

Verificar a média e o desvio padrão dos pesos dos recém-nascidos a termo
de mães fumantes e não fumantes, por sexo.

\end{tcolorbox}

\subsection{Ordenando uma coluna}\label{ordenando-uma-coluna}

Para ordenar os dados de uma coluna, pode-se usar a função
\texttt{arrange()} do \texttt{dplyr}. O primeiro argumento é o conjunto
de base. Os demais argumentos são as colunas a serem ordenadas.\\
Por padrão, a função ´ coloca os dados em ordem crescente, mas é
possível alterar e organizar em ordem decrescente usando a função
\texttt{desc()}, que recebe o nome da coluna como argumento e ordena os
valores em ordem decrescente.

Por exemplo, a renda familiar das parturientes será ordenada de forma
ascendente. Em primeiro lugar, apenas como exercício, a renda familiar
em salários mínimos será convertida em reais, tomando como base o valor
de 2025 de 1518 reais. Após, a variável \texttt{renda} será colocada em
ordem crescente com a função \texttt{arrange()}. A seguir, usando as
funções \texttt{slice\_head()} e \texttt{slice\_tail()}, se verificará
as 5 menores e as 5 maiores rendas que serão atribuídos a dois objetos
(\texttt{menores} e \texttt{maiores}). Estes vão ser exibidos juntos,
usando a função \texttt{bind\_rows()}, também do \texttt{dplyr}:

\begin{Shaded}
\begin{Highlighting}[]
\CommentTok{\# Cinco menores salários em ordem crescente}
\NormalTok{menores }\OtherTok{\textless{}{-}}\NormalTok{ mater }\SpecialCharTok{\%\textgreater{}\%} \FunctionTok{select}\NormalTok{(renda) }\SpecialCharTok{\%\textgreater{}\%} 
    \FunctionTok{mutate}\NormalTok{(}\AttributeTok{renda =}\NormalTok{ renda }\SpecialCharTok{*}\FloatTok{1518.00}\NormalTok{) }\SpecialCharTok{\%\textgreater{}\%} 
    \FunctionTok{arrange}\NormalTok{(renda) }\SpecialCharTok{\%\textgreater{}\%} 
    \FunctionTok{slice\_head}\NormalTok{(}\AttributeTok{n=}\DecValTok{5}\NormalTok{)}
\NormalTok{maiores }\OtherTok{\textless{}{-}}\NormalTok{ mater }\SpecialCharTok{\%\textgreater{}\%} \FunctionTok{select}\NormalTok{(renda) }\SpecialCharTok{\%\textgreater{}\%} 
    \FunctionTok{mutate}\NormalTok{(}\AttributeTok{renda =}\NormalTok{ renda }\SpecialCharTok{*}\FloatTok{1518.00}\NormalTok{) }\SpecialCharTok{\%\textgreater{}\%} 
    \FunctionTok{arrange}\NormalTok{(renda) }\SpecialCharTok{\%\textgreater{}\%} 
    \FunctionTok{slice\_tail}\NormalTok{(}\AttributeTok{n=}\DecValTok{5}\NormalTok{)}

\CommentTok{\# Exibição}
\FunctionTok{bind\_rows}\NormalTok{(menores, maiores)}
\end{Highlighting}
\end{Shaded}

\begin{verbatim}
# A tibble: 10 x 1
    renda
    <dbl>
 1   288.
 2   364.
 3   622.
 4   653.
 5   729.
 6 14634.
 7 14634.
 8 14634.
 9 16227.
10 16455.
\end{verbatim}

\subsection{\texorpdfstring{Função
\texttt{count()}}{Função count()}}\label{funuxe7uxe3o-count}

Permite contar rapidamente os valores únicos de uma ou mais variáveis.
Esta função tem os seguintes argumentos.

\begin{itemize}
\tightlist
\item
  \textbf{x} \(\to\) dataframe
\item
  \textbf{wt} \(\to\) pode ser NULL (padrão) ou uma variável
\item
  \textbf{sort} \(\to\) padrão = FALSE; se TRUE, mostrará os maiores
  grupos no topo
\item
  \textbf{name} \(\to\) O nome da nova coluna na saída; padrão = NULL
\end{itemize}

Quando o argumento \texttt{name} é omitido, a função retorna \emph{n}
como nome padrão.

Usando o dataframe \texttt{mater}, a função \texttt{count()} irá contar
o número de parturientes fumantes, variável dicotômica \texttt{fumo}:

\begin{Shaded}
\begin{Highlighting}[]
\FunctionTok{count}\NormalTok{(mater, fumo)}
\end{Highlighting}
\end{Shaded}

\begin{verbatim}
# A tibble: 2 x 2
  fumo      n
  <fct> <int>
1 sim     301
2 não    1067
\end{verbatim}

\begin{tcolorbox}[enhanced jigsaw, bottomrule=.15mm, opacitybacktitle=0.6, colframe=quarto-callout-tip-color-frame, arc=.35mm, coltitle=black, toptitle=1mm, colback=white, colbacktitle=quarto-callout-tip-color!10!white, breakable, bottomtitle=1mm, rightrule=.15mm, titlerule=0mm, toprule=.15mm, opacityback=0, leftrule=.75mm, left=2mm, title=\textcolor{quarto-callout-tip-color}{\faLightbulb}\hspace{0.5em}{Exercício}]

Calcule o percentual de partos cesáreos no tibble mater.

\end{tcolorbox}

\section{Pacote forcats()}\label{pacote-forcats}

O pacote \texttt{forcats()} é uma das maravilhas do \texttt{tidyverse}
voltada exclusivamente para o tratamento de variáveis categóricas no R
--- ou seja, os fatores. O nome vem de ``\emph{for categorical
variables}'', e ele foi criado para resolver os desafios que surgem ao
lidar com fatores, especialmente em visualizações e modelagens (69).

A finalidade do pacote \texttt{forcats()}:

Oferece funções intuitivas e poderosas para:

\begin{itemize}
\item
  Reordenar os níveis de um fator;\\
\item
  Modificar, combinar e recodificar níveis;\\
\item
  Lidar com níveis raros ou ausentes;\\
\item
  Preparar fatores para gráficos com ggplot2 (Seção~\ref{sec-ggplot2})\\
  As principais funções doforcats() são:
\item
  \texttt{fct\_reorder()} - Reordena os níveis com base em outra
  variável (ex: média);\\
\item
  \texttt{fct\_infrequent()} - Reordena os níveis pela frequência (mais
  comum primeiro);\\
\item
  \texttt{fct\_rev()} - Inverte a ordem dos níveis;\\
\item
  \texttt{fct\_lump()} - Agrupa níveis menos frequentes em ``outros'':\\
\item
  \texttt{fct\_recode()} - Renomeia níveis manualmente;\\
\item
  \texttt{fct\_drop()} - Remove níveis não utilizados;\\
\item
  \texttt{fct\_expand()} - Adiciona novos níveis.
\end{itemize}

Como exemplo, será modificada a ordem de como a variável sexo será
apresentada. Para ver a ordem dos níveis, pode-se usar:

\begin{Shaded}
\begin{Highlighting}[]
\FunctionTok{levels}\NormalTok{(mater}\SpecialCharTok{$}\NormalTok{sexo)}
\end{Highlighting}
\end{Shaded}

\begin{verbatim}
[1] "masc" "fem" 
\end{verbatim}

Ou seja, o sexo masculino está colocado antes do feminino. Como é uma
variável dicotômica, basta inverter a ordem, usando a função
\texttt{fct\_rev()}:

\begin{Shaded}
\begin{Highlighting}[]
\NormalTok{mater}\SpecialCharTok{$}\NormalTok{sexo }\OtherTok{\textless{}{-}} \FunctionTok{fct\_rev}\NormalTok{(mater}\SpecialCharTok{$}\NormalTok{sexo)}

\FunctionTok{levels}\NormalTok{(mater}\SpecialCharTok{$}\NormalTok{sexo)}
\end{Highlighting}
\end{Shaded}

\begin{verbatim}
[1] "fem"  "masc"
\end{verbatim}

\section{Manipulação de datas}\label{manipulauxe7uxe3o-de-datas}

Originalmente, todos os que trabalham com o R queixavam-se de como era
frustrante trabalhar com datas. Era um processo que causava grande perda
de tempo nas análises. O pacote \texttt{lubridate} (70) foi criado para
simplificar ao máximo a leitura de datas e extração de informações das
mesmas.

\begin{Shaded}
\begin{Highlighting}[]
\FunctionTok{library}\NormalTok{(lubridate)}
\end{Highlighting}
\end{Shaded}

Quando o \texttt{lubridate} é carregado aparece uma mensagem, avisando
que alguns nomes de funções também estão contidas no pacote base do R.\\
Para evitar confusões e verificar que as funções corretas estão sendo
usadas, usa-se o duplo dois pontos (\texttt{::}) antes do nome da
função, precedido do nome do pacote, por exemplo:
\texttt{lubridate::date()}.

Para obter a data atual ou a data-hora, você pode usar as funções
\texttt{today()} ou \texttt{now()}:

\begin{Shaded}
\begin{Highlighting}[]
\FunctionTok{today}\NormalTok{()}
\end{Highlighting}
\end{Shaded}

\begin{verbatim}
[1] "2025-09-21"
\end{verbatim}

\begin{Shaded}
\begin{Highlighting}[]
\FunctionTok{now}\NormalTok{()}
\end{Highlighting}
\end{Shaded}

\begin{verbatim}
[1] "2025-09-21 19:36:24 -03"
\end{verbatim}

\subsection{\texorpdfstring{Convertendo \emph{strings} ou caractere para
data}{Convertendo strings ou caractere para data}}\label{convertendo-strings-ou-caractere-para-data}

Para converter \emph{string} ou caracteres em datas, basta executar
funções específicas adequadas aos dados. Elas determinam automaticamente
o formato quando você especifica a ordem do componente. Para usá-los,
identifique a ordem em que o ano, o mês e o dia aparecem em suas datas
e, em seguida, organize ``y'', ``m'' e ``d'' na mesma ordem. Isso lhe dá
o nome da função do \texttt{lubridate} que analisará a data. Por
exemplo, suponhamos a data de 25/12/2022:

\begin{Shaded}
\begin{Highlighting}[]
\NormalTok{natal }\OtherTok{\textless{}{-}} \StringTok{"25/12/2022"}
\NormalTok{natal}
\end{Highlighting}
\end{Shaded}

\begin{verbatim}
[1] "25/12/2022"
\end{verbatim}

Aparentemente, o \emph{R} aceitou a informação como uma data.
Entretanto, se for verificada a classe do objeto, tem-se:

\begin{Shaded}
\begin{Highlighting}[]
\FunctionTok{class}\NormalTok{(natal)}
\end{Highlighting}
\end{Shaded}

\begin{verbatim}
[1] "character"
\end{verbatim}

Estando como caractere, esta data não poderá ser usada em operações com
datas, pois necessitaria estar como uma classe \texttt{date}. Para
converte-la em data, usa-se a função \texttt{dmy()}:

\begin{Shaded}
\begin{Highlighting}[]
\NormalTok{natal }\OtherTok{\textless{}{-}} \FunctionTok{dmy}\NormalTok{(natal)}
\NormalTok{natal}
\end{Highlighting}
\end{Shaded}

\begin{verbatim}
[1] "2022-12-25"
\end{verbatim}

\begin{Shaded}
\begin{Highlighting}[]
\FunctionTok{class}\NormalTok{(natal)}
\end{Highlighting}
\end{Shaded}

\begin{verbatim}
[1] "Date"
\end{verbatim}

Dessa forma, a data, agora está sendo reconhecida pelo \emph{R} como
\texttt{date}. É sempre importante verificar a classe da data.

Às vezes, as datas escritas estão com o mês abreviado, como 25/dez/2022.
O procedimento é o mesmo

\begin{Shaded}
\begin{Highlighting}[]
\NormalTok{minha.data }\OtherTok{\textless{}{-}} \StringTok{"25/dez/2022"}
\FunctionTok{class}\NormalTok{ (minha.data)}
\end{Highlighting}
\end{Shaded}

\begin{verbatim}
[1] "character"
\end{verbatim}

\begin{Shaded}
\begin{Highlighting}[]
\NormalTok{minha.data }\OtherTok{\textless{}{-}} \FunctionTok{dmy}\NormalTok{(minha.data)}
\FunctionTok{class}\NormalTok{ (minha.data)}
\end{Highlighting}
\end{Shaded}

\begin{verbatim}
[1] "Date"
\end{verbatim}

Se além da data, houver necessidade de especificar o horário, basta usar
\texttt{dmy\_h()}, \texttt{dmy\_hm()} e \texttt{dmy\_hms()}. No padrão
americano, pode ser usado \texttt{ymd()}.

O \texttt{lubridate} traz diversas funções para extrair os componentes
de um objeto da classe date.

\begin{itemize}
\tightlist
\item
  \texttt{second()} \(\to\) extrai os segundos.
\item
  \texttt{minute()} \(\to\) extrai os minutos.
\item
  \texttt{hour()} \(\to\) extrai a hora.
\item
  \texttt{wday()} \(\to\) extrai o dia da semana.
\item
  \texttt{mday()} \(\to\) extrai o dia do mês.
\item
  \texttt{month()} \(\to\) extrai o mês.
\item
  \texttt{year()} \(\to\) extrai o ano.
\end{itemize}

Por exemplo, usando a data de nascimento (\texttt{dn}) de um dos netos
do autor:

\begin{Shaded}
\begin{Highlighting}[]
\NormalTok{dn }\OtherTok{\textless{}{-}} \FunctionTok{dmy}\NormalTok{(}\StringTok{"06/06/2018"}\NormalTok{)}
\FunctionTok{year}\NormalTok{(dn)}
\end{Highlighting}
\end{Shaded}

\begin{verbatim}
[1] 2018
\end{verbatim}

Para acrescentar um horário ao objeto data de nascimento
(\texttt{dn})\^{}{[}UTC = Coordinated Universal Time\}:

\begin{Shaded}
\begin{Highlighting}[]
\FunctionTok{hour}\NormalTok{(dn) }\OtherTok{\textless{}{-}} \DecValTok{18}
\NormalTok{dn}
\end{Highlighting}
\end{Shaded}

\begin{verbatim}
[1] "2018-06-06 18:00:00 UTC"
\end{verbatim}

\subsection{Juntando componentes de
datas}\label{juntando-componentes-de-datas}

Para juntar componentes de datas e horas, pode-se utilizar as funções
\texttt{make\_date()} e \texttt{make\_datetime()}. Em muitos arquivos,
os componentes da data estão em colunas diferentes e há necessidade de
juntá-los em uma única coluna para compor a data:

\begin{Shaded}
\begin{Highlighting}[]
\NormalTok{felix }\OtherTok{\textless{}{-}} \FunctionTok{make\_date}\NormalTok{(}\AttributeTok{year =} \DecValTok{2018}\NormalTok{, }\AttributeTok{month =} \DecValTok{06}\NormalTok{, }\AttributeTok{day =} \DecValTok{06}\NormalTok{)}
\NormalTok{felix}
\end{Highlighting}
\end{Shaded}

\begin{verbatim}
[1] "2018-06-06"
\end{verbatim}

Para juntar ano, mês, dia, hora e minuto:

\begin{Shaded}
\begin{Highlighting}[]
\NormalTok{minha.data }\OtherTok{\textless{}{-}} \FunctionTok{make\_datetime}\NormalTok{(}\AttributeTok{year =} \DecValTok{2018}\NormalTok{, }
                            \AttributeTok{month =} \DecValTok{06}\NormalTok{, }
                            \AttributeTok{day =} \DecValTok{06}\NormalTok{, }
                            \AttributeTok{hour =} \DecValTok{18}\NormalTok{ ,}
                            \AttributeTok{min =} \DecValTok{00}\NormalTok{, }
                            \AttributeTok{sec =} \DecValTok{15}\NormalTok{)}
\NormalTok{minha.data}
\end{Highlighting}
\end{Shaded}

\begin{verbatim}
[1] "2018-06-06 18:00:15 UTC"
\end{verbatim}

\subsection{Extraindo componentes de
datas}\label{extraindo-componentes-de-datas}

Quando temos objetos do tipo POSIXt\footnote{POSIXt é uma classe de
  objetos do R que representa datas e horas. POSIXt significa Portable
  Operating System Interface for Unix Time, que é um padrão para medir o
  tempo em segundos desde 1 de janeiro de 1970. Existem duas formas
  internas de implementar POSIXt: POSIXct e POSIXlt. POSIXct armazena os
  segundos desde a época UNIX e POSIXlt armazena uma lista de dia, mês,
  ano, hora, minuto, segundo, etc.} podemos extrair componentes ou
elementos deles. Para isso são usadas algumas funções específicas do
pacote \texttt{lubridate} como mostrado a seguir.

\begin{Shaded}
\begin{Highlighting}[]
\NormalTok{data }\OtherTok{\textless{}{-}} \FunctionTok{now}\NormalTok{()}

\FunctionTok{year}\NormalTok{(data)            }\CommentTok{\# Extrai o ano}
\end{Highlighting}
\end{Shaded}

\begin{verbatim}
[1] 2025
\end{verbatim}

\begin{Shaded}
\begin{Highlighting}[]
\FunctionTok{month}\NormalTok{(data)           }\CommentTok{\# Extrai o mês}
\end{Highlighting}
\end{Shaded}

\begin{verbatim}
[1] 9
\end{verbatim}

\begin{Shaded}
\begin{Highlighting}[]
\FunctionTok{week}\NormalTok{(data)            }\CommentTok{\# Extrai a semana}
\end{Highlighting}
\end{Shaded}

\begin{verbatim}
[1] 38
\end{verbatim}

\begin{Shaded}
\begin{Highlighting}[]
\FunctionTok{day}\NormalTok{(data)             }\CommentTok{\# Extrai o dia}
\end{Highlighting}
\end{Shaded}

\begin{verbatim}
[1] 21
\end{verbatim}

\begin{Shaded}
\begin{Highlighting}[]
\FunctionTok{minute}\NormalTok{(data)          }\CommentTok{\# Extrai o minuto}
\end{Highlighting}
\end{Shaded}

\begin{verbatim}
[1] 36
\end{verbatim}

\begin{Shaded}
\begin{Highlighting}[]
\FunctionTok{second}\NormalTok{(data)          }\CommentTok{\# Extrai o segundo}
\end{Highlighting}
\end{Shaded}

\begin{verbatim}
[1] 24.16666
\end{verbatim}

Para verificar o número de dias tem em um determinado mês, usa-se a
função \texttt{days\_in\_month()}:

\begin{Shaded}
\begin{Highlighting}[]
\NormalTok{ data1 }\OtherTok{\textless{}{-}} \FunctionTok{dmy}\NormalTok{(}\StringTok{"25/02/2000"}\NormalTok{)}
 \FunctionTok{days\_in\_month}\NormalTok{(data1)          }
\end{Highlighting}
\end{Shaded}

\begin{verbatim}
Feb 
 29 
\end{verbatim}

\subsection{Operações com datas}\label{operauxe7uxf5es-com-datas}

O pacote \texttt{lubridate} possui funções de duração e de período para
manipular as datas. As funções de duração calculam o número de segundos
em um determinado num determinado número de dias. As funções de duração
não levam em consideração anos bissextos e horário de verão, enquanto as
funções de período consideram esses fatores.

\begin{Shaded}
\begin{Highlighting}[]
\FunctionTok{ddays}\NormalTok{ (}\DecValTok{1}\NormalTok{)           }\CommentTok{\# Número de segundos em 1 dia}
\end{Highlighting}
\end{Shaded}

\begin{verbatim}
[1] "86400s (~1 days)"
\end{verbatim}

\begin{Shaded}
\begin{Highlighting}[]
\FunctionTok{dhours}\NormalTok{ (}\DecValTok{1}\NormalTok{)          }\CommentTok{\# Número de segundos em 1 hora}
\end{Highlighting}
\end{Shaded}

\begin{verbatim}
[1] "3600s (~1 hours)"
\end{verbatim}

\begin{Shaded}
\begin{Highlighting}[]
\FunctionTok{dminutes}\NormalTok{ (}\DecValTok{1}\NormalTok{)        }\CommentTok{\# Número de segundos em 1 minuto}
\end{Highlighting}
\end{Shaded}

\begin{verbatim}
[1] "60s (~1 minutes)"
\end{verbatim}

\begin{Shaded}
\begin{Highlighting}[]
\FunctionTok{days}\NormalTok{ (}\DecValTok{5}\NormalTok{)            }\CommentTok{\# Cria um período de 5 dias}
\end{Highlighting}
\end{Shaded}

\begin{verbatim}
[1] "5d 0H 0M 0S"
\end{verbatim}

\begin{Shaded}
\begin{Highlighting}[]
\FunctionTok{weeks}\NormalTok{ (}\DecValTok{5}\NormalTok{)           }\CommentTok{\# Cria um período de 5 semanas}
\end{Highlighting}
\end{Shaded}

\begin{verbatim}
[1] "35d 0H 0M 0S"
\end{verbatim}

Suponha-se que haja necessidade de saber em qual dia cairá após
acrescentarmos 5 semanas à \texttt{data1} (25/02/2000), criada acima:

\begin{Shaded}
\begin{Highlighting}[]
\NormalTok{data1 }\SpecialCharTok{+} \FunctionTok{weeks}\NormalTok{ (}\DecValTok{5}\NormalTok{)           }
\end{Highlighting}
\end{Shaded}

\begin{verbatim}
[1] "2000-03-31"
\end{verbatim}

Adicionando 1 ano à \texttt{data1} (25/02/2000) com uma função de
duração, tem-se:

\begin{Shaded}
\begin{Highlighting}[]
\NormalTok{data1 }\SpecialCharTok{+} \FunctionTok{dyears}\NormalTok{ (}\DecValTok{1}\NormalTok{)           }
\end{Highlighting}
\end{Shaded}

\begin{verbatim}
[1] "2001-02-24 06:00:00 UTC"
\end{verbatim}

Se for adicionado um ano à mesma data, mas agora com uma função de
período, tem-se:

\begin{Shaded}
\begin{Highlighting}[]
\NormalTok{data1 }\SpecialCharTok{+} \FunctionTok{years}\NormalTok{ (}\DecValTok{1}\NormalTok{)           }
\end{Highlighting}
\end{Shaded}

\begin{verbatim}
[1] "2001-02-25"
\end{verbatim}

Um intervalo de tempo pode ser obtido a partir de uma data inicial e uma
data final. Suponha que uma gestante tenha como data da sua última
menstruação 04/10/2022 e o bebê tenha nascido em 30/06/2023. Qual a
idade gestacional em dias? A sintaxe para calcular um intervalo é dada
pela subtração das duas datas:

\begin{Shaded}
\begin{Highlighting}[]
\NormalTok{data.inicial }\OtherTok{\textless{}{-}} \FunctionTok{dmy}\NormalTok{(}\StringTok{"04/10/2022"}\NormalTok{)}
\NormalTok{data.final }\OtherTok{\textless{}{-}} \FunctionTok{dmy}\NormalTok{(}\StringTok{"30/06/2023"}\NormalTok{)}
\NormalTok{idade\_gesta }\OtherTok{\textless{}{-}}\NormalTok{ data.final }\SpecialCharTok{{-}}\NormalTok{ data.inicial}
\NormalTok{idade\_gesta}
\end{Highlighting}
\end{Shaded}

\begin{verbatim}
Time difference of 269 days
\end{verbatim}

Ou seja a gestação durou 269 dias, constituindo-se em um parto a termo,
entre 37 (259 dias) e 42 semanas (294 dias).

Para mais informações sobre o \texttt{lubridate}, consulte a ajuda do
pacote ou o capítulo 16 do livro \emph{R for Data Science}, Hadley
Wickman e Garrett Grolemund, 2017
{[}https://r4ds.had.co.nz/index.html{]} .

\part{Parte III - Estatística Descritiva}

\chapter{Medidas Resumidoras}\label{sec-resumo}

\section{Dados brutos}\label{sec-dadosbrutos}

Habitualmente, costuma-se armazenar os dados em bancos de dados
(\emph{dataframes} ou \emph{tibbles}). Entretanto, eles estão
registrados de forma aleatória e não classificada. Ao se visualizar um
dataframe, é difícil responder perguntas em relação a qualquer variável,
principalmente, em grandes banco de dados. Eles se constituem uma lista,
um rol de valores colocados na ordem em que foram obtidos. Parecem um
jogo de quebra cabeça antes de ser organizado! A Tabela~\ref{tbl-raw} é
o tipo mais simples de tabela possível. Nela é apresentado um conjunto
de dados, os valores dos pesos de 30 recém-nascidos. Não há nenhum
significado especial para as linhas e colunas, estão dispostos da
maneira como coletados. Eles pouco informam; são apenas os dados na sua
forma inicial sem nenhum tratamento ou ordem. São denominados como
\emph{dados brutos} e se constituem no ponto de partida para uma
análise. Quando são colocados de maneira crescente e permitem uma
compreensão inicial são chamados de dados não agrupados e de dados
agrupados, quando classificados em categorias ou intervalos.

\global\setlength{\Oldarrayrulewidth}{\arrayrulewidth}

\global\setlength{\Oldtabcolsep}{\tabcolsep}

\setlength{\tabcolsep}{2pt}

\renewcommand*{\arraystretch}{1.5}



\providecommand{\ascline}[3]{\noalign{\global\arrayrulewidth #1}\arrayrulecolor[HTML]{#2}\cline{#3}}

\begin{longtable}[c]{|p{0.75in}|p{0.75in}|p{0.75in}|p{0.75in}|p{0.75in}|p{0.75in}}

\caption{\label{tbl-raw}Peso de 30 recém-nascidos de partos
consecutivos}

\tabularnewline

\multicolumn{1}{>{\raggedleft}m{\dimexpr 0.75in+0\tabcolsep}}{\textcolor[HTML]{000000}{\fontsize{11}{11}\selectfont{\global\setmainfont{Arial}{\ }}}} & \multicolumn{1}{>{\raggedleft}m{\dimexpr 0.75in+0\tabcolsep}}{\textcolor[HTML]{000000}{\fontsize{11}{11}\selectfont{\global\setmainfont{Arial}{\ }}}} & \multicolumn{1}{>{\raggedleft}m{\dimexpr 0.75in+0\tabcolsep}}{\textcolor[HTML]{000000}{\fontsize{11}{11}\selectfont{\global\setmainfont{Arial}{\ }}}} & \multicolumn{1}{>{\raggedleft}m{\dimexpr 0.75in+0\tabcolsep}}{\textcolor[HTML]{000000}{\fontsize{11}{11}\selectfont{\global\setmainfont{Arial}{\ }}}} & \multicolumn{1}{>{\raggedleft}m{\dimexpr 0.75in+0\tabcolsep}}{\textcolor[HTML]{000000}{\fontsize{11}{11}\selectfont{\global\setmainfont{Arial}{\ }}}} & \multicolumn{1}{>{\raggedleft}m{\dimexpr 0.75in+0\tabcolsep}}{\textcolor[HTML]{000000}{\fontsize{11}{11}\selectfont{\global\setmainfont{Arial}{\ }}}} \\

\ascline{1.5pt}{000000}{1-6}\endfirsthead 



\multicolumn{1}{>{\raggedleft}m{\dimexpr 0.75in+0\tabcolsep}}{\textcolor[HTML]{000000}{\fontsize{11}{11}\selectfont{\global\setmainfont{Arial}{\ }}}} & \multicolumn{1}{>{\raggedleft}m{\dimexpr 0.75in+0\tabcolsep}}{\textcolor[HTML]{000000}{\fontsize{11}{11}\selectfont{\global\setmainfont{Arial}{\ }}}} & \multicolumn{1}{>{\raggedleft}m{\dimexpr 0.75in+0\tabcolsep}}{\textcolor[HTML]{000000}{\fontsize{11}{11}\selectfont{\global\setmainfont{Arial}{\ }}}} & \multicolumn{1}{>{\raggedleft}m{\dimexpr 0.75in+0\tabcolsep}}{\textcolor[HTML]{000000}{\fontsize{11}{11}\selectfont{\global\setmainfont{Arial}{\ }}}} & \multicolumn{1}{>{\raggedleft}m{\dimexpr 0.75in+0\tabcolsep}}{\textcolor[HTML]{000000}{\fontsize{11}{11}\selectfont{\global\setmainfont{Arial}{\ }}}} & \multicolumn{1}{>{\raggedleft}m{\dimexpr 0.75in+0\tabcolsep}}{\textcolor[HTML]{000000}{\fontsize{11}{11}\selectfont{\global\setmainfont{Arial}{\ }}}} \\

\ascline{1.5pt}{000000}{1-6}\endhead



\multicolumn{1}{>{\raggedleft}m{\dimexpr 0.75in+0\tabcolsep}}{\textcolor[HTML]{000000}{\fontsize{11}{11}\selectfont{\global\setmainfont{Arial}{2,940}}}} & \multicolumn{1}{>{\raggedleft}m{\dimexpr 0.75in+0\tabcolsep}}{\textcolor[HTML]{000000}{\fontsize{11}{11}\selectfont{\global\setmainfont{Arial}{3,575}}}} & \multicolumn{1}{>{\raggedleft}m{\dimexpr 0.75in+0\tabcolsep}}{\textcolor[HTML]{000000}{\fontsize{11}{11}\selectfont{\global\setmainfont{Arial}{3,150}}}} & \multicolumn{1}{>{\raggedleft}m{\dimexpr 0.75in+0\tabcolsep}}{\textcolor[HTML]{000000}{\fontsize{11}{11}\selectfont{\global\setmainfont{Arial}{3,545}}}} & \multicolumn{1}{>{\raggedleft}m{\dimexpr 0.75in+0\tabcolsep}}{\textcolor[HTML]{000000}{\fontsize{11}{11}\selectfont{\global\setmainfont{Arial}{3,365}}}} & \multicolumn{1}{>{\raggedleft}m{\dimexpr 0.75in+0\tabcolsep}}{\textcolor[HTML]{000000}{\fontsize{11}{11}\selectfont{\global\setmainfont{Arial}{2,825}}}} \\





\multicolumn{1}{>{\raggedleft}m{\dimexpr 0.75in+0\tabcolsep}}{\textcolor[HTML]{000000}{\fontsize{11}{11}\selectfont{\global\setmainfont{Arial}{3,060}}}} & \multicolumn{1}{>{\raggedleft}m{\dimexpr 0.75in+0\tabcolsep}}{\textcolor[HTML]{000000}{\fontsize{11}{11}\selectfont{\global\setmainfont{Arial}{2,580}}}} & \multicolumn{1}{>{\raggedleft}m{\dimexpr 0.75in+0\tabcolsep}}{\textcolor[HTML]{000000}{\fontsize{11}{11}\selectfont{\global\setmainfont{Arial}{3,110}}}} & \multicolumn{1}{>{\raggedleft}m{\dimexpr 0.75in+0\tabcolsep}}{\textcolor[HTML]{000000}{\fontsize{11}{11}\selectfont{\global\setmainfont{Arial}{2,415}}}} & \multicolumn{1}{>{\raggedleft}m{\dimexpr 0.75in+0\tabcolsep}}{\textcolor[HTML]{000000}{\fontsize{11}{11}\selectfont{\global\setmainfont{Arial}{4,670}}}} & \multicolumn{1}{>{\raggedleft}m{\dimexpr 0.75in+0\tabcolsep}}{\textcolor[HTML]{000000}{\fontsize{11}{11}\selectfont{\global\setmainfont{Arial}{3,670}}}} \\





\multicolumn{1}{>{\raggedleft}m{\dimexpr 0.75in+0\tabcolsep}}{\textcolor[HTML]{000000}{\fontsize{11}{11}\selectfont{\global\setmainfont{Arial}{1,930}}}} & \multicolumn{1}{>{\raggedleft}m{\dimexpr 0.75in+0\tabcolsep}}{\textcolor[HTML]{000000}{\fontsize{11}{11}\selectfont{\global\setmainfont{Arial}{3,000}}}} & \multicolumn{1}{>{\raggedleft}m{\dimexpr 0.75in+0\tabcolsep}}{\textcolor[HTML]{000000}{\fontsize{11}{11}\selectfont{\global\setmainfont{Arial}{3,115}}}} & \multicolumn{1}{>{\raggedleft}m{\dimexpr 0.75in+0\tabcolsep}}{\textcolor[HTML]{000000}{\fontsize{11}{11}\selectfont{\global\setmainfont{Arial}{2,850}}}} & \multicolumn{1}{>{\raggedleft}m{\dimexpr 0.75in+0\tabcolsep}}{\textcolor[HTML]{000000}{\fontsize{11}{11}\selectfont{\global\setmainfont{Arial}{2,490}}}} & \multicolumn{1}{>{\raggedleft}m{\dimexpr 0.75in+0\tabcolsep}}{\textcolor[HTML]{000000}{\fontsize{11}{11}\selectfont{\global\setmainfont{Arial}{1,465}}}} \\





\multicolumn{1}{>{\raggedleft}m{\dimexpr 0.75in+0\tabcolsep}}{\textcolor[HTML]{000000}{\fontsize{11}{11}\selectfont{\global\setmainfont{Arial}{2,790}}}} & \multicolumn{1}{>{\raggedleft}m{\dimexpr 0.75in+0\tabcolsep}}{\textcolor[HTML]{000000}{\fontsize{11}{11}\selectfont{\global\setmainfont{Arial}{4,445}}}} & \multicolumn{1}{>{\raggedleft}m{\dimexpr 0.75in+0\tabcolsep}}{\textcolor[HTML]{000000}{\fontsize{11}{11}\selectfont{\global\setmainfont{Arial}{3,290}}}} & \multicolumn{1}{>{\raggedleft}m{\dimexpr 0.75in+0\tabcolsep}}{\textcolor[HTML]{000000}{\fontsize{11}{11}\selectfont{\global\setmainfont{Arial}{3,215}}}} & \multicolumn{1}{>{\raggedleft}m{\dimexpr 0.75in+0\tabcolsep}}{\textcolor[HTML]{000000}{\fontsize{11}{11}\selectfont{\global\setmainfont{Arial}{3,245}}}} & \multicolumn{1}{>{\raggedleft}m{\dimexpr 0.75in+0\tabcolsep}}{\textcolor[HTML]{000000}{\fontsize{11}{11}\selectfont{\global\setmainfont{Arial}{3,420}}}} \\





\multicolumn{1}{>{\raggedleft}m{\dimexpr 0.75in+0\tabcolsep}}{\textcolor[HTML]{000000}{\fontsize{11}{11}\selectfont{\global\setmainfont{Arial}{1,750}}}} & \multicolumn{1}{>{\raggedleft}m{\dimexpr 0.75in+0\tabcolsep}}{\textcolor[HTML]{000000}{\fontsize{11}{11}\selectfont{\global\setmainfont{Arial}{2,925}}}} & \multicolumn{1}{>{\raggedleft}m{\dimexpr 0.75in+0\tabcolsep}}{\textcolor[HTML]{000000}{\fontsize{11}{11}\selectfont{\global\setmainfont{Arial}{3,345}}}} & \multicolumn{1}{>{\raggedleft}m{\dimexpr 0.75in+0\tabcolsep}}{\textcolor[HTML]{000000}{\fontsize{11}{11}\selectfont{\global\setmainfont{Arial}{1,105}}}} & \multicolumn{1}{>{\raggedleft}m{\dimexpr 0.75in+0\tabcolsep}}{\textcolor[HTML]{000000}{\fontsize{11}{11}\selectfont{\global\setmainfont{Arial}{3,445}}}} & \multicolumn{1}{>{\raggedleft}m{\dimexpr 0.75in+0\tabcolsep}}{\textcolor[HTML]{000000}{\fontsize{11}{11}\selectfont{\global\setmainfont{Arial}{3,150}}}} \\

\ascline{1.5pt}{000000}{1-6}


\end{longtable}

\arrayrulecolor[HTML]{000000}

\global\setlength{\arrayrulewidth}{\Oldarrayrulewidth}

\global\setlength{\tabcolsep}{\Oldtabcolsep}

\renewcommand*{\arraystretch}{1}

\section{Introdução}\label{introduuxe7uxe3o-1}

Nos relatórios ou artigos científicos, a comunicação dos resultados é
feita através da combinação de medidas resumidoras e visualização dos
dados por meio de tabelas e gráficos, constituindo o que se costuma
chamar de \textbf{Estatística Descritiva}. Neste capítulo, serão
discutidas as principais medidas resumidoras dos dados de duas maneiras:

\begin{itemize}
\tightlist
\item
  Primeiro, um valor em torno do qual os dados têm uma tendência para se
  reunir ou se agrupar, denominado de \textbf{medida sumária de
  localização} ou \textbf{medida de tendência central}.
\item
  Em segundo lugar, um valor que mede o grau em que os dados se
  dispersam, denominado de \textbf{medida de dispersão} ou
  \textbf{variabilidade}
\end{itemize}

Nos próximos capítulos, desta Parte III, estarão em discussão a
construção de Tabelas (Capítulo~\ref{sec-tabelas}) e visualização dos
dados através de gráficos (Capítulo~\ref{sec-graficos}).

\section{Dados usados neste capítulo}\label{sec-dados6}

Para as demonstrações práticas, será usado o banco de dados
\texttt{dadosMater.xlsx} (ver Seção~\ref{sec-dadosMater}). Após
carregá-lo, serão filtrados os partos a termo e selecionada as variáveis
necessárias (idadeMae, anosEst, pesoRN, apgar1). Por última, será
extraída com a função \texttt{slice\_sample()} do pacote \texttt{dplyr}
(Seção~\ref{sec-dplyr}) uma amostra de n = 200. Como cada vez que este
comando for reproduzido, retornará uma nova série de 200 valores
diferentes do anterior. Para tornar o código reproduzível, retornando o
mesmo conjunto de valores, deve-se usar uma ``semente'' (\texttt{seed}),
usando a função \texttt{set.seed()}, cujo argumento é um número que
identificará a série gerada. Após extrair a amostra, esta será atribuída
a um objeto denominado, \texttt{dados}:

\begin{Shaded}
\begin{Highlighting}[]
\FunctionTok{library}\NormalTok{(dplyr)}

\FunctionTok{set.seed}\NormalTok{(}\DecValTok{123}\NormalTok{)}
\NormalTok{dados }\OtherTok{\textless{}{-}}\NormalTok{ readxl}\SpecialCharTok{::}\FunctionTok{read\_excel}\NormalTok{(}\StringTok{"dados/dadosMater.xlsx"}\NormalTok{) }\SpecialCharTok{\%\textgreater{}\%} 
  \FunctionTok{filter}\NormalTok{(ig }\SpecialCharTok{\textgreater{}=} \DecValTok{37} \SpecialCharTok{\&}\NormalTok{ ig }\SpecialCharTok{\textless{}} \DecValTok{42}\NormalTok{) }\SpecialCharTok{\%\textgreater{}\%} 
  \FunctionTok{select}\NormalTok{(idadeMae, anosEst, pesoRN, apgar1) }\SpecialCharTok{\%\textgreater{}\%} 
  \FunctionTok{slice\_sample}\NormalTok{(}\AttributeTok{n=}\DecValTok{200}\NormalTok{)}
\end{Highlighting}
\end{Shaded}

\section{Medidas de tendência
central}\label{medidas-de-tenduxeancia-central}

\subsection{Média}\label{muxe9dia}

A média ( \(\overline{x}\) ) é a mais usada medida de tendência central
para representar um valor típico dentro de um conjunto de números. O
conceito mais comum é a \textbf{média aritmética}, que se calcula
somando todos os valores do conjunto e dividindo pelo número total de
elementos. A média é mais adequada para medidas numéricas simétricas,
pois ela é sensível aos valores extremos (\emph{outliers}).

\[
\overline{x}= \frac{\sum(x_1 + x_2 + x_3 + ... + x_n)}{n}
\]

O R base possui uma função para o cálculo da média, \texttt{mean()},
apresentada na Seção~\ref{sec-funcoes}, onde foi mostrado os seus
argumentos. Se a variável analisada contiver algum valor ausente
(\emph{missing}), deve-se usar o argumento \texttt{na.rm\ =\ TRUE}, para
removê-los, pois, caso contrário, a função retorna um resultado como NA
(\emph{Not Available}). Para evitar transtornos, recomenta-se usar
sempre o argumento.

\begin{Shaded}
\begin{Highlighting}[]
\FunctionTok{mean}\NormalTok{ (dados}\SpecialCharTok{$}\NormalTok{pesoRN, }\AttributeTok{na.rm =} \ConstantTok{TRUE}\NormalTok{)}
\end{Highlighting}
\end{Shaded}

\begin{verbatim}
[1] 3257.95
\end{verbatim}

Para reduzir o número de dígitos decimais, na saída do resultado,
pode-se colocar a função \texttt{mean()}, dentro da função
\texttt{round()}\footnote{Usa-se esta sintaxe:
  \texttt{round(x,\ digits\ =\ 0)}, onde \texttt{x} é o numero que se
  quer arredondar e \texttt{digits} é número de casas decimais. O padrão
  é \texttt{0}, ou seja, arredonda para o inteiro mais próximo.},
atribuindo o resultado da função a um objeto, por exemplo
\texttt{media}.

\begin{Shaded}
\begin{Highlighting}[]
\NormalTok{media }\OtherTok{\textless{}{-}} \FunctionTok{round}\NormalTok{(}\FunctionTok{mean}\NormalTok{ (dados}\SpecialCharTok{$}\NormalTok{pesoRN, }\AttributeTok{na.rm =} \ConstantTok{TRUE}\NormalTok{), }\DecValTok{1}\NormalTok{)}
\FunctionTok{print}\NormalTok{(media)}
\end{Highlighting}
\end{Shaded}

\begin{verbatim}
[1] 3257.9
\end{verbatim}

Ou, usar a função \texttt{round()}, separadamente:

\begin{Shaded}
\begin{Highlighting}[]
\NormalTok{media }\OtherTok{\textless{}{-}} \FunctionTok{mean}\NormalTok{ (dados}\SpecialCharTok{$}\NormalTok{pesoRN, }\AttributeTok{na.rm =} \ConstantTok{TRUE}\NormalTok{)}
\FunctionTok{round}\NormalTok{(media, }\DecValTok{1}\NormalTok{)}
\end{Highlighting}
\end{Shaded}

\begin{verbatim}
[1] 3257.9
\end{verbatim}

\subsection{Mediana}\label{mediana}

A mediana (Md) representa o valor central em uma série ordenada de
valores. Assim, metade dos valores será igual ou menor que o valor
mediano e a outra metade igual ou maior do que ele. Para encontrar a
mediana procede-se da seguinte maneira:

\begin{enumerate}
\def\labelenumi{\arabic{enumi}.}
\tightlist
\item
  Ordenar o conjunto de dados, por exemplo, a variável
  \texttt{dados\$pesoRN}, usando a função \texttt{sort()}:
\end{enumerate}

\begin{Shaded}
\begin{Highlighting}[]
\NormalTok{valores\_ordenados }\OtherTok{\textless{}{-}} \FunctionTok{sort}\NormalTok{(dados}\SpecialCharTok{$}\NormalTok{pesoRN)}
\FunctionTok{print}\NormalTok{(valores\_ordenados)}
\end{Highlighting}
\end{Shaded}

\begin{verbatim}
  [1] 2100 2110 2125 2170 2300 2330 2340 2345 2345 2395 2415 2570 2650 2660 2680
 [16] 2685 2695 2700 2710 2720 2730 2735 2750 2750 2770 2775 2780 2780 2795 2805
 [31] 2830 2845 2845 2850 2860 2870 2875 2880 2900 2900 2910 2920 2930 2935 2940
 [46] 2965 2980 2985 2985 3000 3010 3015 3020 3020 3025 3030 3030 3040 3040 3040
 [61] 3050 3050 3055 3055 3060 3060 3060 3075 3080 3080 3080 3090 3090 3105 3115
 [76] 3120 3120 3135 3145 3150 3150 3150 3150 3160 3160 3165 3165 3170 3180 3180
 [91] 3180 3200 3215 3230 3250 3260 3270 3270 3275 3280 3290 3295 3300 3300 3300
[106] 3300 3300 3305 3305 3315 3320 3320 3320 3325 3330 3330 3335 3340 3340 3345
[121] 3355 3370 3385 3395 3400 3405 3410 3410 3410 3415 3415 3425 3430 3430 3430
[136] 3450 3460 3465 3465 3470 3480 3490 3490 3490 3495 3500 3500 3505 3510 3540
[151] 3545 3545 3550 3550 3560 3570 3580 3585 3585 3600 3610 3610 3625 3625 3635
[166] 3640 3650 3660 3660 3665 3710 3715 3715 3730 3730 3770 3780 3795 3830 3840
[181] 3845 3880 3890 3905 3920 3920 3920 3945 3970 3995 4025 4045 4080 4080 4090
[196] 4160 4315 4350 4370 4485
\end{verbatim}

\begin{enumerate}
\def\labelenumi{\arabic{enumi}.}
\setcounter{enumi}{1}
\tightlist
\item
  Quando o número de valores é par , caso do exemplo, a mediana é a
  média dos dois valores do meio, ou seja, o valor central corresponde a
  média do valor 100 e do valor 101 dos valores ordenados:
\end{enumerate}

\begin{Shaded}
\begin{Highlighting}[]
\NormalTok{mediana }\OtherTok{=}\NormalTok{ (valores\_ordenados[}\DecValTok{100}\NormalTok{] }\SpecialCharTok{+}\NormalTok{ valores\_ordenados[}\DecValTok{101}\NormalTok{])}\SpecialCharTok{/}\DecValTok{2}
\FunctionTok{print}\NormalTok{(mediana)}
\end{Highlighting}
\end{Shaded}

\begin{verbatim}
[1] 3285
\end{verbatim}

\begin{enumerate}
\def\labelenumi{\arabic{enumi}.}
\setcounter{enumi}{2}
\tightlist
\item
  Quando o número de valores no conjunto de dados for ímpar, a mediana é
  o valor do meio.
\end{enumerate}

\begin{itemize}
\tightlist
\item
  No exemplo, tem-se 200 valores. Para transformar a amostra em uma
  amostra com \emph{n} ímpar, remover aleatoriamente uma observação dos
  valores dos \texttt{dados\$pesoRN}:
\end{itemize}

\begin{Shaded}
\begin{Highlighting}[]
\FunctionTok{set.seed}\NormalTok{(}\DecValTok{234}\NormalTok{)}
\NormalTok{index\_aleatorio }\OtherTok{\textless{}{-}} \FunctionTok{sample}\NormalTok{(}\FunctionTok{length}\NormalTok{(dados}\SpecialCharTok{$}\NormalTok{pesoRN), }\DecValTok{1}\NormalTok{)}
\NormalTok{index\_aleatorio}
\end{Highlighting}
\end{Shaded}

\begin{verbatim}
[1] 31
\end{verbatim}

\begin{itemize}
\tightlist
\item
  Ou seja, o valor selecionado é o 31º valor. A seguir, obtem-se o peso
  aleatório que foi selecionado:
\end{itemize}

\begin{Shaded}
\begin{Highlighting}[]
\NormalTok{peso\_aleatorio }\OtherTok{\textless{}{-}}\NormalTok{ dados}\SpecialCharTok{$}\NormalTok{pesoRN[index\_aleatorio]}
\NormalTok{peso\_aleatorio}
\end{Highlighting}
\end{Shaded}

\begin{verbatim}
[1] 3585
\end{verbatim}

\begin{itemize}
\tightlist
\item
  Remover o 31º valor (index\_aleatório) da lista de
  \texttt{dados\$pesoRN} que corresponde ao peso selecionado:
\end{itemize}

\begin{Shaded}
\begin{Highlighting}[]
\NormalTok{pesos\_restantes }\OtherTok{\textless{}{-}}\NormalTok{ dados}\SpecialCharTok{$}\NormalTok{pesoRN[}\SpecialCharTok{{-}}\NormalTok{index\_aleatorio]}
\end{Highlighting}
\end{Shaded}

\begin{itemize}
\tightlist
\item
  Colocar os pesos\_restantes em ordem crescente:
\end{itemize}

\begin{Shaded}
\begin{Highlighting}[]
\NormalTok{restantes\_ordenados }\OtherTok{\textless{}{-}} \FunctionTok{sort}\NormalTok{(pesos\_restantes)}
\FunctionTok{print}\NormalTok{(restantes\_ordenados)}
\end{Highlighting}
\end{Shaded}

\begin{verbatim}
  [1] 2100 2110 2125 2170 2300 2330 2340 2345 2345 2395 2415 2570 2650 2660 2680
 [16] 2685 2695 2700 2710 2720 2730 2735 2750 2750 2770 2775 2780 2780 2795 2805
 [31] 2830 2845 2845 2850 2860 2870 2875 2880 2900 2900 2910 2920 2930 2935 2940
 [46] 2965 2980 2985 2985 3000 3010 3015 3020 3020 3025 3030 3030 3040 3040 3040
 [61] 3050 3050 3055 3055 3060 3060 3060 3075 3080 3080 3080 3090 3090 3105 3115
 [76] 3120 3120 3135 3145 3150 3150 3150 3150 3160 3160 3165 3165 3170 3180 3180
 [91] 3180 3200 3215 3230 3250 3260 3270 3270 3275 3280 3290 3295 3300 3300 3300
[106] 3300 3300 3305 3305 3315 3320 3320 3320 3325 3330 3330 3335 3340 3340 3345
[121] 3355 3370 3385 3395 3400 3405 3410 3410 3410 3415 3415 3425 3430 3430 3430
[136] 3450 3460 3465 3465 3470 3480 3490 3490 3490 3495 3500 3500 3505 3510 3540
[151] 3545 3545 3550 3550 3560 3570 3580 3585 3600 3610 3610 3625 3625 3635 3640
[166] 3650 3660 3660 3665 3710 3715 3715 3730 3730 3770 3780 3795 3830 3840 3845
[181] 3880 3890 3905 3920 3920 3920 3945 3970 3995 4025 4045 4080 4080 4090 4160
[196] 4315 4350 4370 4485
\end{verbatim}

\begin{itemize}
\tightlist
\item
  Cálculo da mediana com os valores restantes ordenados. O valor central
  entre 1 e 199 é o 100º valor:
\end{itemize}

\begin{Shaded}
\begin{Highlighting}[]
\NormalTok{mediana }\OtherTok{\textless{}{-}}\NormalTok{ restantes\_ordenados[}\DecValTok{100}\NormalTok{]}
\NormalTok{mediana}
\end{Highlighting}
\end{Shaded}

\begin{verbatim}
[1] 3280
\end{verbatim}

Imaginem que sempre que se for calcular a mediana houvesse necessidade
de se proceder como realizado acima. Seria tedioso, quase um caos!
Entretanto, usando o R, a situação fica bem mais agradável e intuitiva.
O R facilita esse trabalho, fornecendo a função \texttt{median()}.

Como exemplo, será usada a variável \texttt{apgar1} já incluída no
dataframe \texttt{dados}. Como o Apgar é um escore (21), a medida
resumidora mais adequada, realmente, é a mediana.

\begin{Shaded}
\begin{Highlighting}[]
\FunctionTok{median}\NormalTok{ (dados}\SpecialCharTok{$}\NormalTok{apgar1, }\AttributeTok{na.rm =} \ConstantTok{TRUE}\NormalTok{)}
\end{Highlighting}
\end{Shaded}

\begin{verbatim}
[1] 8
\end{verbatim}

\begin{tcolorbox}[enhanced jigsaw, bottomrule=.15mm, opacitybacktitle=0.6, colframe=quarto-callout-tip-color-frame, arc=.35mm, coltitle=black, toptitle=1mm, colback=white, colbacktitle=quarto-callout-tip-color!10!white, breakable, bottomtitle=1mm, rightrule=.15mm, titlerule=0mm, toprule=.15mm, opacityback=0, leftrule=.75mm, left=2mm, title=\textcolor{quarto-callout-tip-color}{\faLightbulb}\hspace{0.5em}{Exercício}]

Repetrir o cálculo da mediana dos valores
\texttt{dados\textbackslash{}\$peso} e destes dados quando foi extraído
um valor..

\end{tcolorbox}

\subsection{Moda}\label{moda}

Moda (Mo) é o valor que ocorre com maior frequência em um conjunto de
dados. O \emph{R} não possui uma função nativa e direta para calcular a
moda como tem para a média (\texttt{mean()}) e a mediana
(\texttt{median()}). Isso acontece porque a moda pode não ser única em
um conjunto de dados (podem existir múltiplos valores com a mesma
frequência máxima) ou pode nem existir (se todos os valores ocorrerem
apenas uma vez).Tem o menor nível de sofisticação. No entanto, pode-se
facilmente criar uma função própria para calcular a moda ou usar pacotes
que oferecem essa funcionalidade, como o \texttt{DescTools} que oferece
uma função chamada \texttt{Mode()}. Aqui estão algumas maneiras de
calcular a moda em R:

\textbf{Função personalizada}

\begin{Shaded}
\begin{Highlighting}[]
\NormalTok{moda }\OtherTok{\textless{}{-}} \ControlFlowTok{function}\NormalTok{(v) \{}
\NormalTok{  freq\_tab }\OtherTok{\textless{}{-}} \FunctionTok{table}\NormalTok{(v)}
\NormalTok{  max\_freq }\OtherTok{\textless{}{-}} \FunctionTok{max}\NormalTok{(freq\_tab)}
\NormalTok{  moda }\OtherTok{\textless{}{-}} \FunctionTok{names}\NormalTok{(freq\_tab[freq\_tab }\SpecialCharTok{==}\NormalTok{ max\_freq])}
  \FunctionTok{return}\NormalTok{(moda)}
\NormalTok{\}}
\end{Highlighting}
\end{Shaded}

Esta função \texttt{moda()}é constituída por:

\begin{itemize}
\tightlist
\item
  \textbf{table(v)}: Cria uma tabela de frequência dos valores v.
\item
  \textbf{max(freq\_tab)}: Encontra a frequência máxima.
\item
  \textbf{freq\_table{[}freq\_table == max\_freq{]}}: Seleciona as
  entradas da tabela de frequência que são iguais à frequência máxima.
\item
  \textbf{names(\ldots)}: Obtém os nomes (os valores originais) dessas
  entradas, que são as modas.
\end{itemize}

Usando a função criada, a moda da variável \texttt{dados\$apgar1} é
igual a:

\begin{Shaded}
\begin{Highlighting}[]
\FunctionTok{moda}\NormalTok{ (dados}\SpecialCharTok{$}\NormalTok{apgar1) }
\end{Highlighting}
\end{Shaded}

\begin{verbatim}
[1] "8"
\end{verbatim}

A função \texttt{moda()} pode ser salva em seu diretório de trabalho, na
pasta das suas funções próprias. Quando necessário ela pode ser
acessada, como foi visto na Seção~\ref{sec-funcpropria}.

\textbf{Função Mode() do pacote DescTools}

\begin{Shaded}
\begin{Highlighting}[]
\FunctionTok{library}\NormalTok{(DescTools)}
\NormalTok{moda }\OtherTok{\textless{}{-}} \FunctionTok{Mode}\NormalTok{(dados}\SpecialCharTok{$}\NormalTok{apgar1)}
\FunctionTok{print}\NormalTok{(moda)}
\end{Highlighting}
\end{Shaded}

\begin{verbatim}
[1] 8
attr(,"freq")
[1] 85
\end{verbatim}

\subsection{Quantil}\label{sec-quantil}

Uma medida de localização bastante utilizada são os \textbf{quantis} que
são pontos estabelecidos em intervalos regulares que dividem a amostra
em subconjuntos iguais. Se estes subconjuntos são em número de 100, são
denominados de \textbf{percentis}; se são em número de 10, são os
\textbf{decis} e em número de 4, são os \textbf{quartis}. A função
nativa no R para obter o quantil é \texttt{quantile()}.

Para determinar os três quartis do peso dos recém-nascidos
(\texttt{dados\$pesoRN}), usa-se:

\begin{Shaded}
\begin{Highlighting}[]
\FunctionTok{quantile}\NormalTok{ (dados}\SpecialCharTok{$}\NormalTok{pesoRN, }\FunctionTok{c}\NormalTok{ (}\FloatTok{0.25}\NormalTok{, }\FloatTok{0.50}\NormalTok{, }\FloatTok{0.75}\NormalTok{))}
\end{Highlighting}
\end{Shaded}

\begin{verbatim}
    25%     50%     75% 
3007.50 3285.00 3541.25 
\end{verbatim}

Observe que o percentil 50º é igual a mediana. O percentil 75º é o ponto
do conjunto de dados onde 75\% dos recém-nascidos têm um peso inferior a
3541.25g e 25\% está acima deste valor.

\subsection{Média aparada}\label{muxe9dia-aparada}

As médias aparadas são estimadores robustos da tendência central. Para
calcular uma média aparada, é removida uma quantidade predeterminada de
observações em cada lado de uma distribuição e realizada a média das
observações restantes. Um exemplo de média aparada é a própria mediana.

A base R tem como calcular a média aparada acrescentando o argumento
\texttt{trim\ =}, proporção a ser aparada. Se for aparado 20\%, usa-se
\texttt{trim\ =\ 0.2}. isto significa que serão removidos 20\% dos dados
dos dois extremos. No caso da amostra de 15 recém-nascidos, serão
removidos três valores mais baixos e três valores mais altos, passando a
mostra a ter 9 valores, e a média aparada será a média destes 9 valores.

O comando para obter a média aparada é:

\begin{Shaded}
\begin{Highlighting}[]
\FunctionTok{round}\NormalTok{(}\FunctionTok{mean}\NormalTok{ (dados}\SpecialCharTok{$}\NormalTok{pesoRN, }\AttributeTok{na.rm =} \ConstantTok{TRUE}\NormalTok{, }\AttributeTok{trim =} \FloatTok{0.20}\NormalTok{), }\DecValTok{1}\NormalTok{)}
\end{Highlighting}
\end{Shaded}

\begin{verbatim}
[1] 3258.6
\end{verbatim}

\section{Medidas de Dispersão}\label{medidas-de-dispersuxe3o}

\subsection{Amplitude}\label{amplitude}

A amplitude de um grupo de medições é definida como a diferença entre a
maior observação e a menor.\\
No conjunto de dados dos pesos dos recém-nascidos, a amplitude pode ser
obtida, no \emph{R}, com a função \texttt{range()}, que retorna o valor
mínimo e o máximo.

\begin{Shaded}
\begin{Highlighting}[]
\FunctionTok{range}\NormalTok{ (dados}\SpecialCharTok{$}\NormalTok{pesoRN, }\AttributeTok{na.rm =} \ConstantTok{TRUE}\NormalTok{)}
\end{Highlighting}
\end{Shaded}

\begin{verbatim}
[1] 2100 4485
\end{verbatim}

\subsection{Intervalo Interquartil}\label{intervalo-interquartil}

A intervalo interquartil (IIQ), também conhecido como amplitude
interquartil (AIQ) é uma forma de média aparada. É simplesmente a
diferença entre o terceiro e o primeiro quartil, ou seja, a diferença
entre o percentil 75 e o percentil 25. Considere a variável escolaridade
(\texttt{dados\$anosEst}), anos de estudos completos.

Os percentis 25 e 75 são obtidos, usando a função \texttt{quantile()},
vista acima, ou com a função \texttt{summary()} , que retorna os valores
mínimo, primeiro quartil, mediana, média, terceiro quartil e máximo.

\begin{Shaded}
\begin{Highlighting}[]
\FunctionTok{quantile}\NormalTok{ (dados}\SpecialCharTok{$}\NormalTok{anosEst, }\FunctionTok{c}\NormalTok{(}\FloatTok{0.25}\NormalTok{,}\FloatTok{0.75}\NormalTok{))}
\end{Highlighting}
\end{Shaded}

\begin{verbatim}
25% 75% 
  5   9 
\end{verbatim}

\begin{Shaded}
\begin{Highlighting}[]
\FunctionTok{summary}\NormalTok{(dados}\SpecialCharTok{$}\NormalTok{anosEst)}
\end{Highlighting}
\end{Shaded}

\begin{verbatim}
   Min. 1st Qu.  Median    Mean 3rd Qu.    Max. 
   0.00    5.00    7.50    7.29    9.00   16.00 
\end{verbatim}

Portanto, o IIQ está entre 5 a 9 anos de estudo ou, 4 anos de estudos
completos. Em outras palavras, 50\% das mulheres desta amostra têm entre
5 a 9 anos de estudo.

O R possui uma função específica para calcular o intervalo interquartil,
denominada \texttt{IQR()} e incluída no \texttt{R} base. Ela possui os
seguintes argumentos:

\textbf{x} \(\to\) Representa o vetor numérico;\\
\textbf{na.rm} \(\to\) Este assume um valor lógico, TRUE ou FALSE,
indicando se os valores ausentes devem ser removidos ou não;\\
\textbf{type} \(\to\) Representa um número inteiro selecionando um dos
muitos algoritmos de quantil. Este é um parâmetro opcional.

\begin{Shaded}
\begin{Highlighting}[]
\FunctionTok{IQR}\NormalTok{(dados}\SpecialCharTok{$}\NormalTok{anosEst, }\AttributeTok{na.rm =} \ConstantTok{TRUE}\NormalTok{)}
\end{Highlighting}
\end{Shaded}

\begin{verbatim}
[1] 4
\end{verbatim}

\subsection{Variância e Desvio Padrão}\label{sec-variancia}

A variância e o desvio padrão fornecem uma indicação de quão aglomerados
em torno da média os dados de uma amostra estão. Estes tipos de medidas
representam desvios (erros)da média. Quando se verifica o desvio de cada
valor (x) em relação à média \(\overline{x}\), os desvios positivos se
anulam com os negativos, resultando em uma soma igual a zero.

A consequência deste fato é que não é possível resumir os desvios numa
única medida de variabilidade. Para se chegar a uma medida de
variabilidade há necessidade de se eliminar os sinais, antes de somar
todos os desvios em relação à média.

Uma maneira de se fazer isso é elevar todas as diferenças ao quadrado.
Assim, se obtém o desvio em relação à média elevado ao quadrado. A soma
destes valores é denominada de \emph{Soma dos Quadrados (SQ) dos
Desvios} ou \emph{Soma dos Erros ao Quadrado}. Se o interesse é apenas
saber o erro ou desvio médio, divide-se por \emph{n} (tamanho da
amostra). No entanto, em geral o interesse se concentra em usar o desvio
ou erro na amostra para estimar o erro na população. Dessa maneira,
divide-se a Soma dos Quadrados por \(n-1\). Essa medida é conhecida como
\textbf{variância} (\(s^2\)). O divisor, \(n – 1\), é denominado de
\emph{graus de liberdade} (gl) associados à variância.

Os graus de liberdade representam o número de desvios que estão livres
para variar. É um conceito de difícil explicação, mas é possível
compreendê-lo, usando a seguinte explicação:

\begin{tcolorbox}[enhanced jigsaw, bottomrule=.15mm, opacitybacktitle=0.6, colframe=quarto-callout-note-color-frame, arc=.35mm, coltitle=black, toptitle=1mm, colback=white, colbacktitle=quarto-callout-note-color!10!white, breakable, bottomtitle=1mm, rightrule=.15mm, titlerule=0mm, toprule=.15mm, opacityback=0, leftrule=.75mm, left=2mm, title=\textcolor{quarto-callout-note-color}{\faInfo}\hspace{0.5em}{Graus de liberdade}]

Suponha uma maternidade há 50 anos atrás, quando não havia alojamento
conjunto. Nessa época era comum os recém-nascidos normais ficarem em um
berçário. A cada horário de amamentação eles eram levados para os
quartos de suas mães para mamar. Posteriormente, eram trazidos para o
berçário e colocados nos berços até a próxima mamada. Suponha que, em um
determinado momento, havia 15 bebês e que, no berçário, existiam 15
berços (postos) para colocá-los durante o intervalo das mamadas. Quando
o primeiro recém-nascido chega, a enfermeira poderá escolher qualquer um
dos berços para o colocar. Depois, quando o próximo recém-nascido
chegar, ela terá 14 opções de escolha, pois um dos berços está ocupado.
Ainda existe uma boa liberdade de escolha. No entanto, à medida que os
recém-nascidos forem sendo trazidos para o berçário, chegará a um ponto
em que 14 berços estarão ocupados. Agora, a enfermeira não terá
liberdade de escolha, pois só resta um berço. Nesse exemplo existem 14
graus de liberdade (gl)\footnotemark{}. Para o último recém-nascido não
houve liberdade de escolha (71). Portanto, os graus de liberdade são
iguais ao tamanho da amostra menos um (\(n-1\)).

\end{tcolorbox}

\footnotetext{No texto deste livro, \texttt{graus\ de\ liberdade} são
abreviados por \texttt{gl}. Muitas vezes, é usado a abreviação
\texttt{df} de \texttt{degree\ of\ freedom} porque em algumas funções
aparecem como df e optou-se por manter sem a tradução.}

A variância é a razão entre a soma dos quadrados e os graus de liberdade
(observações realizadas menos um).

\[
s^2= \frac{\sum(x_i - \overline{x})^2}{n-1}
\]

No \emph{R} existem embutidas as funções \texttt{sd()} e
\texttt{var()}que facilmente calculam essas medidas de dispersão.

Usando a variável \texttt{dados\$pesoRN}, tem-se:

\begin{Shaded}
\begin{Highlighting}[]
\FunctionTok{var}\NormalTok{(dados}\SpecialCharTok{$}\NormalTok{pesoRN, }\AttributeTok{na.rm =}\ConstantTok{TRUE}\NormalTok{)}
\end{Highlighting}
\end{Shaded}

\begin{verbatim}
[1] 199704.3
\end{verbatim}

O desvio padrão é a raiz quadrada da variância: \(s = \sqrt var\)

\begin{Shaded}
\begin{Highlighting}[]
\FunctionTok{sqrt}\NormalTok{ (}\FunctionTok{var}\NormalTok{(dados}\SpecialCharTok{$}\NormalTok{pesoRN))}
\end{Highlighting}
\end{Shaded}

\begin{verbatim}
[1] 446.8829
\end{verbatim}

Ou, usando a função \texttt{sd()} e arredondando para 1 dígito decimal:

\begin{Shaded}
\begin{Highlighting}[]
\FunctionTok{round}\NormalTok{(}\FunctionTok{sd}\NormalTok{ (dados}\SpecialCharTok{$}\NormalTok{pesoRN, }\AttributeTok{na.rm =} \ConstantTok{TRUE}\NormalTok{), }\DecValTok{1}\NormalTok{)}
\end{Highlighting}
\end{Shaded}

\begin{verbatim}
[1] 446.9
\end{verbatim}

A variância e desvio padrão são medidas de variabilidade e revelam quão
bem a média representa os dados. Informa se ela está \emph{funcionando
bem como modelo}. Pequenos desvios padrão mostram que existe pouca
variabilidade nos dados, que eles se aproximam da média. Quando existe
um grande desvio padrão, a média não é muito precisa para representar os
dados.

O desvio padrão, além de medir a precisão com que a média representa os
dados, também informa sobre o formato dos dados e por isso é uma medida
de dispersão. Em uma amostra onde desvio padrão é pequeno, os dados se
agrupam próximo a média e o formato da distribuição fica mais pontiagudo
(curva em azul, Figura~\ref{fig-dispersao}). Nesse caso a média
representa bem os dados. Em outra amostra, com a mesma média anterior,
mas com os dados mais dispersos entorno da média, o desvio padrão é
maior e o formato da distribuição fica achatado (curva verde, na Figura
Figura~\ref{fig-dispersao}). Nesse caso a média não é uma boa
representação dos dados.

\begin{figure}[H]

\centering{

\pandocbounded{\includegraphics[keepaspectratio]{06-medidasResumidoras_files/figure-pdf/fig-dispersao-1.pdf}}

}

\caption{\label{fig-dispersao}Dispersão dos dados em torno da média.}

\end{figure}%

\subsection{Coeficiente de
Variação}\label{coeficiente-de-variauxe7uxe3o}

O desvio padrão por si só tem limitações. Um desvio padrão igual a 2
pode ser considerado pequeno para um conjunto de valores cuja média é
100. Entretanto, se a média for 5, ele se torna muito grande. Além
disso, o desvio padrão por ser expresso na mesma unidade dos dados, não
permite aplicá-lo na comparação de dois ou mais conjunto de dados que
têm unidades diferentes. Para eliminar essas limitações, é possível
caracterizar a dispersão ou variabilidade dos dados em termos relativos,
usando uma medida denominada Coeficiente de Variação (CV), também
conhecido como como \emph{Desvio Padrão Relativo} ou \emph{Coeficiente
de Variação de Pearson}. É expresso, em geral como uma porcentagem,
sendo definido como a razão do desvio padrão pela média:

\[
CV = \frac{s}{\overline{x}}
\]

Multiplicando o valor da equação por 100 tem-se o CV percentual. O
\emph{R} não possui uma função específica para calcular o CV.

Foi criada uma função específica para isso,já multiplicada por 100.

\begin{Shaded}
\begin{Highlighting}[]
\NormalTok{coef\_var }\OtherTok{\textless{}{-}} \ControlFlowTok{function}\NormalTok{ (valores) \{}
\NormalTok{  (}\FunctionTok{sd}\NormalTok{(valores, }\AttributeTok{na.rm=}\ConstantTok{TRUE}\NormalTok{) }\SpecialCharTok{/} \FunctionTok{mean}\NormalTok{(valores, }\AttributeTok{na.rm=}\ConstantTok{TRUE}\NormalTok{))}\SpecialCharTok{*}\DecValTok{100}\NormalTok{\}}
\end{Highlighting}
\end{Shaded}

Portanto, o CV da variável \texttt{dados\$pesoRN} é igual a:

\begin{Shaded}
\begin{Highlighting}[]
\FunctionTok{round}\NormalTok{ (}\FunctionTok{coef\_var}\NormalTok{ (dados}\SpecialCharTok{$}\NormalTok{pesoRN),}\DecValTok{1}\NormalTok{)}
\end{Highlighting}
\end{Shaded}

\begin{verbatim}
[1] 13.7
\end{verbatim}

Usdando outra variável do banco de dados, por exemplo,
\texttt{dados\$idadeMae}, o CV será igual a:

\begin{Shaded}
\begin{Highlighting}[]
\FunctionTok{round}\NormalTok{(}\FunctionTok{coef\_var}\NormalTok{ (dados}\SpecialCharTok{$}\NormalTok{idadeMae), }\DecValTok{1}\NormalTok{)}
\end{Highlighting}
\end{Shaded}

\begin{verbatim}
[1] 25
\end{verbatim}

O peso do recem-nascido tem um CV = 13.7 \% e a idade materna um CV = 25
\%, mostrando que esta tem uma maior variabilidade. Quanto menor o
desvio padrão, menor o CV e, consequentemente, menor a variabilidade. Um
CV \(\ge\) 50\%, sugere que a variável tem uma distribuição assimétrica.

\subsection{Escolha da medida
resumidora}\label{escolha-da-medida-resumidora}

A seleção da medida de tendência central mais adequada depende de vários
fatores, incluindo a natureza dos dados e do propósito da sumarização.

O tipo da variável tem substancial influência na escolha da medida de
tendência central a ser usada. A moda é mais apropriada para dados
nominais e seu uso com variáveis ordinais resulta em uma perda no poder
em termos de informação que se poderia obter dos dados.

A mediana é mais adequada para variáveis ordinais, embora possa ser
usada para variáveis contínuas, especialmente quando a distribuição dos
dados é assimétrica. A mediana não deveria ser usada com dados nominais
porque os postos assumidos não podem ser obtidos com dados de nível
nominal.

Finalmente, a média somente deve ser usada com dados contínuos
simétricos, se houver assimetria a mediana deve ser preferida.

As medidas de dispersão devem estar associadas a uma medida de tendência
central. Elas caracterizam a variabilidade dos dados na amostra. Com
dados ordinais usar a amplitude ou o intervalo interquartil. O desvio
padrão não é apropriado em dados ordinais devido à natureza não numérica
destes.

Com os dados numéricos deve-se usar o desvio padrão, que utiliza toda a
informação nos dados, ou o intervalo interquartil (IIQ). Quando os dados
forem simétricos, usar a média acompanhada do desvio padrão, caso
contrário, usar a mediana e o IIQ. Não misturar e combinar medidas (22).

\chapter{Tabelas}\label{sec-tabelas}

A apresentação tabular de dados, ou tabulação, é a organização dos dados
por meio de tabelas. Uma tabela é uma disposição sistemática e lógica
dos dados na forma de linhas e colunas com relação às características
dos dados. É uma forma ordenada, compacta, autoexplicativa e eficiente
de mostrar os dados levantados, facilitando a sua compreensão e
interpretação. Permite identificar padrões, tendências e intuições
valiosos.\\
A tabela mais utilizada para descrever os dados é a \textbf{tabela de
frequência}.

\section{Componentes de uma tabela}\label{componentes-de-uma-tabela}

Uma tabela bem estruturada é essencial para comunicar dados de forma
clara e científica, especialmente em publicações acadêmicas. Os
componentes fundamentais que uma tabela (Figura~\ref{fig-tabela1}) deve
conter, para estar pronta para publicação, são os seguintes:

\begin{enumerate}
\def\labelenumi{\arabic{enumi}.}
\item
  \textbf{Título da Tabela} - deve ser claro, conciso e informativo.
  Indica o conteúdo da tabela e o contexto do estudo. Deve estar no topo
  da tabela.
\item
  \textbf{Cabeçalho} - Identifica as variáveis apresentadas. Inclui
  unidade de medidas quando necessário (ex: idade em anos, glicemia em
  md/dL)
\item
  \textbf{Corpo da tabela} - Apresenta os valores estatísticos: média,
  mediana, desvio padrão, intervalo de confiança, etc. Pode incluir
  frequências absolutas e relativas (em proporção ou porcentagem). Os
  dados devem estar alinhados corretamente para facilitar a leitura.
\item
  \textbf{Notas de rodapé} - Explicações adicionais sobre abreviações,
  símbolos ou indicação de testes estatísticos aplicados (ex: teste
  \emph{t} de Student, ANOVA, qui-quadrado). Pode incluir significância
  estatística (\emph{p} \textless{} 0,05).
\item
  \textbf{Fonte dos dados} - Caso os dados sejam secundários ou
  provenientes de outra pesquisa, deve-se citar a fonte.
\end{enumerate}

\begin{figure}

\centering{

\includegraphics[width=0.7\linewidth,height=\textheight,keepaspectratio]{index_files/mediabag/QHPfytN.png}

}

\caption{\label{fig-tabela1}Elementos de uma Tabela}

\end{figure}%

\section{Tabelas de Frequências}\label{tabelas-de-frequuxeancias}

As tabelas de frequências são essenciais em estatística para organizar e
resumir dados. Ajudam a entender como os valores em um conjunto de dados
se distribuem, mostrando a frequência com que cada valor ou intervalo de
valores aparece.\\
Basicamente, uma tabela de frequências responde a uma pergunta simples:
``Quantas vezes cada coisa aconteceu?''.

De um modo geral, uma tabela de frequência deve cumprir algumas
finalidades:

\begin{enumerate}
\def\labelenumi{\arabic{enumi}.}
\tightlist
\item
  \textbf{Clareza e Organização}: Elas transformam dados brutos e
  desorganizados em informações claras e fáceis de interpretar.\\
\item
  \textbf{Identificação de padrões}: Uma tabela deve ajudar a visualizar
  rapidamente os valores mais comuns (a moda) e a distribuição geral dos
  dados.\\
\item
  \textbf{Base para Outras Análises}: São o ponto de partida para criar
  gráficos como histogramas, polígonos de frequência e gráficos de
  setores, que oferecem uma representação visual ainda mais intuitiva.\\
\item
  \textbf{Tomada de decisão}: Permitem que se tire conclusões rápidas
  sobre um conjunto de dados. Por exemplo, na Tabela~\ref{tbl-droga} ,
  se verifica que ``a maioria das gestantes, no Hospital Geral de Caxias
  do Sul, referem não ser drogaditas'' ou ``72\% das gestantes, no
  Hospital Geral de Caxias do Sul, referem não ser drogaditas''.
\end{enumerate}

\subsection{Tipos de Frequências}\label{tipos-de-frequuxeancias}

Geralmente, em uma tabela de frequência aparecem os seguintes tipos de
frequência:

\subsubsection{Frequência Absoluta (f)}\label{frequuxeancia-absoluta-f}

É o número de vezes que um valor específico ocorre no conjunto de dados.
Por exemplo, na Tabela~\ref{tbl-droga}, 913 das parturientes referem não
usar drogas. Podem ser:

\begin{enumerate}
\def\labelenumi{\arabic{enumi}.}
\item
  \ul{\textbf{Distribuição de frequência não agrupada}}\\
  As distribuições de frequência não agrupada listam cada valor
  individual de um conjunto de dados com a sua respetiva contagem
  (frequência), sendo ideais para conjuntos de dados pequenos ou com
  poucas categorias. A Tabela~\ref{tbl-droga} mostra a frequência da
  variável drogadição na gestação, com os diferentes tipos de drogas
  usadas
\item
  \ul{\textbf{Distribuição de frequência agrupada}}

  As distribuições de frequência agrupadas organizam os dados em
  intervalos ou classes, úteis para conjuntos de dados extensos, pois
  condensam a informação em grupos menores., como faixas etárias ,
  estado nutricional da gestante na gravidez, usando o IMC categorizado,
  idade gestacional categorizada em pré-termo, termo e pós-termo.
\end{enumerate}

\subsubsection{Frequência Relativa
(fr)}\label{frequuxeancia-relativa-fr}

É a proporção da frequência absoluta em relação ao total de dados. É
calculada dividindo a frequência absoluta pelo número total de
observações (\(fr=f/n\)).

A frequência relativa pode ser expressa como uma fração, decimal ou,
mais comumente, como uma porcentagem (\emph{frp}). É útil para comparar
distribuições de dados com tamanhos totais diferentes. Por exemplo, na
Tabela~\ref{tbl-droga}, 2,13\% da das parturientes declararam ser
alcoolistas.\\

\subsubsection{Frequência Absoluta Acumulada
(F)}\label{frequuxeancia-absoluta-acumulada-f}

É a soma das frequências absolutas até um determinado ponto na tabela.
Ela mostra quantos dados estão abaixo ou iguais a um certo valor.Na
Tabela~\ref{tbl-droga} não tem importância, são variáveis nominais sem
uma ordem estabelecida. Não faz sentido somar ``pessoas que fumantes''
com ``pessoas não usuárias de drogas''. As frequências cumulativas não
fazem sentido para variáveis nominais porque os valores não têm ordem
--- um valor não é maior ou menor que outro valor.

\subsubsection{Frequência Relativa Acumulada
(Fr)}\label{frequuxeancia-relativa-acumulada-fr}

É a soma das frequências relativas até um determinado ponto. Ela mostra
a proporção de dados que estão abaixo ou iguais a um certo valor.Assim
como an frequência absoluta acumulada, não faz sentido usar a frequência
relativa acumulada com dados nominais, onde a ordem das categorias não
tem valor.

\begin{table}

\caption{\label{tbl-droga}Distribuição de frequência de drogadição em
gestantes}

\centering{

\caption*{
{\large Drogadição em Gestantes\textsuperscript{\textit{1}}}
} 
\fontsize{12.0pt}{14.4pt}\selectfont
\begin{tabular*}{\linewidth}{@{\extracolsep{\fill}}>{\raggedright\arraybackslash}p{\dimexpr 90.00pt -2\tabcolsep-1.5\arrayrulewidth}>{\centering\arraybackslash}p{\dimexpr 112.50pt -2\tabcolsep-1.5\arrayrulewidth}>{\centering\arraybackslash}p{\dimexpr 112.50pt -2\tabcolsep-1.5\arrayrulewidth}}
\toprule
{\bfseries Droga} & Frequência Absoluta\textsuperscript{\textit{2}} & Frequência Relativa (\%) \\ 
\midrule\addlinespace[2.5pt]
Nenhuma & 913 & 72.06 \\ 
Fumo & 301 & 23.76 \\ 
Álcool & 27 & 2.13 \\ 
Medicamentos & 23 & 1.82 \\ 
Crack & 2 & 0.16 \\ 
Cocaína & 1 & 0.08 \\ 
Total & 1267 & 100.00 \\ 
\bottomrule
\end{tabular*}
\begin{minipage}{\linewidth}
\textsuperscript{\textit{1}}Hospital Geral de Caxias do Sul, RS, 2008\\
\textsuperscript{\textit{2}}101 gestantes não informaram a condição de drogadição\\
Fonte: Oliveira Filho, PF (2025)\\
\end{minipage}

}

\end{table}%

\subsection{Regras gerais para construção de tabelas de
frequência}\label{regras-gerais-para-construuxe7uxe3o-de-tabelas-de-frequuxeancia}

As tabelas de frequências seguem algumas regras,

Existem algumas recomendações na construção de uma tabela de frequência
(72):

\begin{itemize}
\item
  Deve ter um título na parte superior que responda as perguntas: ``o
  que? quando? onde?'' relativas ao fato estudado;
\item
  Deve ter um rodapé, na parte inferior da tabela, onde se coloca notas
  necessárias e a fonte dos dados;
\item
  As colunas externas da tabela devem ser abertas, o emprego de linhas
  verticais para a separação das colunas no corpo da tabela é opcional;
\item
  Na parte superior e inferior, as tabelas devem, ser fechadas por
  linhas horizontais;
\item
  Nenhuma casela deve ficar vazia, apresentando um número ou um símbolo.
  Se não se dispuser do dado, colocar reticências \ldots{} e a presença
  de um X representa que o dado foi omitido para evitar a identificação.
\item
  Tabelas devem ter, pelo menos, duas linhas no seu corpo e,
  prioritariamente, pelo menos três colunas. Tabelas com duas colunas
  devem ser evitadas, pois as informações contidas nelas podem, em
  geral, ser apresentadas no corpo do texto da obra.
\end{itemize}

\subsection{Construção de tabelas de frequência no R}\label{sec-tabfreq}

\subsubsection{Dados para os exemplos}\label{sec-dados7}

Para os exemplos, o dataframe \texttt{dadosMater.xlsx} será acionado
para fornecer dados, semelhante ao realizado na Seção~\ref{sec-dados6}
.\\
Após carregar o dataframe, serão selecionadas as variáveis necessárias.
As variáveis \texttt{utiNeo}, \texttt{sexo} serão convertidas a fatores;
a variável \texttt{idadeMae} será categorizada em três níveis
(\textless20 anos, 20-35 anos, \textgreater35 anos), criando uma nova
variável \texttt{categIdade}. Além disso, será criada a variável
\texttt{imc} que também será categorizada em uma nova variável
\texttt{estNutri} em quatro níveis (Baixo Peso, Peso adequado,
Sobrepeso, Obesidade). Por último, será extraída uma amostra de n = 200
e atribuída a um objeto \texttt{dados}:

\begin{Shaded}
\begin{Highlighting}[]
\FunctionTok{library}\NormalTok{(dplyr)}
\FunctionTok{library}\NormalTok{(readxl)}

\FunctionTok{set.seed}\NormalTok{(}\DecValTok{123}\NormalTok{)}
\NormalTok{dados }\OtherTok{\textless{}{-}}\NormalTok{ readxl}\SpecialCharTok{::}\FunctionTok{read\_excel}\NormalTok{(}\StringTok{"dados/dadosMater.xlsx"}\NormalTok{) }\SpecialCharTok{\%\textgreater{}\%} 
  \FunctionTok{select}\NormalTok{(idadeMae, peso, altura, anosEst, fumo, pesoRN, apgar1, utiNeo) }\SpecialCharTok{\%\textgreater{}\%} 
  \FunctionTok{mutate}\NormalTok{(}\AttributeTok{fumo =} \FunctionTok{factor}\NormalTok{(fumo, }
                       \AttributeTok{levels =} \FunctionTok{c}\NormalTok{(}\DecValTok{1}\NormalTok{,}\DecValTok{2}\NormalTok{), }
                       \AttributeTok{labels =} \FunctionTok{c}\NormalTok{(}\StringTok{"fumante"}\NormalTok{, }\StringTok{"não fumante"}\NormalTok{)),}
         \AttributeTok{utiNeo =} \FunctionTok{factor}\NormalTok{(utiNeo, }
                         \AttributeTok{levels =} \FunctionTok{c}\NormalTok{(}\DecValTok{1}\NormalTok{,}\DecValTok{2}\NormalTok{), }
                         \AttributeTok{labels =} \FunctionTok{c}\NormalTok{(}\StringTok{"sim"}\NormalTok{, }\StringTok{"não"}\NormalTok{)),}
         \AttributeTok{categIdade =} \FunctionTok{case\_when}\NormalTok{(}
\NormalTok{           idadeMae }\SpecialCharTok{\textless{}} \DecValTok{20} \SpecialCharTok{\textasciitilde{}} \StringTok{"\textless{} 20 anos"}\NormalTok{,}
\NormalTok{           idadeMae }\SpecialCharTok{\textgreater{}=} \DecValTok{20} \SpecialCharTok{\&}\NormalTok{ idadeMae }\SpecialCharTok{\textless{}=} \DecValTok{35} \SpecialCharTok{\textasciitilde{}} \StringTok{"20 a 35 anos"}\NormalTok{,}
\NormalTok{           idadeMae }\SpecialCharTok{\textgreater{}} \DecValTok{35} \SpecialCharTok{\textasciitilde{}} \StringTok{"\textgreater{} 35 anos"}\NormalTok{),}
         \AttributeTok{categIdade =} \FunctionTok{factor}\NormalTok{(categIdade, }
                             \AttributeTok{levels =} \FunctionTok{c}\NormalTok{(}\StringTok{"\textless{} 20 anos"}\NormalTok{, }\StringTok{"20 a 35 anos"}\NormalTok{, }\StringTok{"\textgreater{} 35 anos"}\NormalTok{)),}
         \AttributeTok{imc =}\NormalTok{ peso}\SpecialCharTok{/}\NormalTok{altura}\SpecialCharTok{\^{}}\DecValTok{2}\NormalTok{,}
         \AttributeTok{estNutri =} \FunctionTok{case\_when}\NormalTok{(}
\NormalTok{           imc }\SpecialCharTok{\textless{}} \FloatTok{18.5} \SpecialCharTok{\textasciitilde{}} \StringTok{"Baixo Peso"}\NormalTok{,}
\NormalTok{           imc }\SpecialCharTok{\textgreater{}=} \FloatTok{18.5} \SpecialCharTok{\&}\NormalTok{ imc }\SpecialCharTok{\textless{}} \DecValTok{25} \SpecialCharTok{\textasciitilde{}} \StringTok{"Peso adequado"}\NormalTok{, }
\NormalTok{           imc }\SpecialCharTok{\textgreater{}=} \DecValTok{25} \SpecialCharTok{\&}\NormalTok{ imc }\SpecialCharTok{\textless{}} \DecValTok{30} \SpecialCharTok{\textasciitilde{}} \StringTok{"Sobrepeso"}\NormalTok{,}
\NormalTok{           imc }\SpecialCharTok{\textgreater{}=} \DecValTok{30} \SpecialCharTok{\textasciitilde{}} \StringTok{"Obesidade"}\NormalTok{),}
         \AttributeTok{estNutri =} \FunctionTok{factor}\NormalTok{(estNutri, }
                           \AttributeTok{levels =} \FunctionTok{c}\NormalTok{(}\StringTok{"Baixo Peso"}\NormalTok{, }\StringTok{"Peso adequado"}\NormalTok{, }
                                      \StringTok{"Sobrepeso"}\NormalTok{, }\StringTok{"Obesidade"}\NormalTok{))) }\SpecialCharTok{\%\textgreater{}\%}
  \FunctionTok{slice\_sample}\NormalTok{(}\AttributeTok{n=}\DecValTok{200}\NormalTok{) }

\FunctionTok{str}\NormalTok{(dados)}
\end{Highlighting}
\end{Shaded}

\begin{verbatim}
tibble [200 x 11] (S3: tbl_df/tbl/data.frame)
 $ idadeMae  : num [1:200] 16 19 27 18 29 38 15 26 27 22 ...
 $ peso      : num [1:200] 68 49 64 53 76 55 64 81 62 59.5 ...
 $ altura    : num [1:200] 1.59 1.62 1.78 1.65 1.64 1.6 1.69 1.65 1.72 1.65 ...
 $ anosEst   : num [1:200] 9 11 13 7 11 4 7 6 8 11 ...
 $ fumo      : Factor w/ 2 levels "fumante","não fumante": 2 2 1 2 2 2 1 2 2 2 ...
 $ pesoRN    : num [1:200] 2940 3060 1930 2790 1750 ...
 $ apgar1    : num [1:200] 8 9 NA 9 NA 4 9 8 8 9 ...
 $ utiNeo    : Factor w/ 2 levels "sim","não": 2 2 2 2 1 2 2 2 2 2 ...
 $ categIdade: Factor w/ 3 levels "< 20 anos","20 a 35 anos",..: 1 1 2 1 2 3 1 2 2 2 ...
 $ imc       : num [1:200] 26.9 18.7 20.2 19.5 28.3 ...
 $ estNutri  : Factor w/ 4 levels "Baixo Peso","Peso adequado",..: 3 2 2 2 3 2 2 3 2 2 ...
\end{verbatim}

\subsubsection{Tabelas de frequência para dados
categóricos}\label{tabelas-de-frequuxeancia-para-dados-categuxf3ricos}

\ul{Função table() e função prop.table}

O R básico possui uma função incorporada, \texttt{table()} que permite a
verificação da frequência absoluta de variáveis categóricas de uma
maneira bem simples. A função \texttt{table()} é a base para criar
tabelas de frequência. Ela conta o número de ocorrências de cada valor
em um vetor (ou coluna de um dataframe). Por exemplo, a frequência
absoluta (f) em cada uma das categorias da variável \texttt{categIdade}:
(idade das parturientes dividida em classes):

\begin{Shaded}
\begin{Highlighting}[]
\NormalTok{f  }\OtherTok{\textless{}{-}} \FunctionTok{table}\NormalTok{ (dados}\SpecialCharTok{$}\NormalTok{categIdade)}
\FunctionTok{print}\NormalTok{ (f)}
\end{Highlighting}
\end{Shaded}

\begin{verbatim}

   < 20 anos 20 a 35 anos    > 35 anos 
          33          144           23 
\end{verbatim}

A função \texttt{prop.table()} é usada para calcular as frequências
relativas a partir de uma tabela de frequência absoluta. Em outras
palavras, ela transforma as contagens em proporções ou porcentagens. A
funçaõ \texttt{round()} recebe a função prop.table para arredondar para
três dígitos a frequência relativa (fr):

\begin{Shaded}
\begin{Highlighting}[]
\NormalTok{fr  }\OtherTok{\textless{}{-}} \FunctionTok{round}\NormalTok{(}\FunctionTok{prop.table}\NormalTok{(f), }\DecValTok{3}\NormalTok{)}
\FunctionTok{print}\NormalTok{(fr)}
\end{Highlighting}
\end{Shaded}

\begin{verbatim}

   < 20 anos 20 a 35 anos    > 35 anos 
       0.165        0.720        0.115 
\end{verbatim}

Multiplicando por 100 a fr, tem-se a frequência relativa em porcentagem
-- frp ou fr(\%). A operação será colocada dentro da função round(),
como feito com a fr, arredondando o resultado para dois dígitos.

\begin{Shaded}
\begin{Highlighting}[]
\NormalTok{frp }\OtherTok{\textless{}{-}} \FunctionTok{round}\NormalTok{(fr}\SpecialCharTok{*}\DecValTok{100}\NormalTok{, }\DecValTok{2}\NormalTok{)}
\FunctionTok{print}\NormalTok{(frp)}
\end{Highlighting}
\end{Shaded}

\begin{verbatim}

   < 20 anos 20 a 35 anos    > 35 anos 
        16.5         72.0         11.5 
\end{verbatim}

Embora as funções \texttt{table()} e \texttt{prop.table()} sejam
separadas, é muito comum usá-las juntas para construir uma
\textbf{tabela de frequência completa}. Portanto, os passos para a
contrução de uma tabela de frequência no R são os segintes

Para atingir este objetivo, a construção de uma tabela de frequência
absoluta e de frequência relativa (\%) deve ser o primeiro passo. Os
passos seguintes são;

\textbf{Passo 1}:Usar as funções \texttt{table()} e
\texttt{prop.table()} juntas como feito acima, para construir as
frequências absoluta e relativa.

\textbf{Passo 2}: Criar um dataframe combinando as funções para uma
melhor visualização:

\begin{Shaded}
\begin{Highlighting}[]
\NormalTok{tab\_completa }\OtherTok{\textless{}{-}} \FunctionTok{data.frame}\NormalTok{(}
     \AttributeTok{f =} \FunctionTok{as.vector}\NormalTok{(f),}
     \AttributeTok{fr =} \FunctionTok{as.factor}\NormalTok{(fr),}
    \AttributeTok{frp =} \FunctionTok{as.vector}\NormalTok{(frp))}

\FunctionTok{print}\NormalTok{(tab\_completa)}
\end{Highlighting}
\end{Shaded}

\begin{verbatim}
               f    fr  frp
< 20 anos     33 0.165 16.5
20 a 35 anos 144  0.72 72.0
> 35 anos     23 0.115 11.5
\end{verbatim}

Este último exemplo mostra a flexibilidade do R para manipular e
apresentar dados de forma clara e organizada, combinando as saídas de
funções básicas em uma única tabela.

Para análises mais avançadas ou para variáveis quantitativas contínuas,
pode-se usar a função \texttt{mutate()} ,junto com a função
\texttt{case\_when()} (73) , usada para criar a variáveis
\texttt{categIdade} e \texttt{estNutri}, na Seção~\ref{sec-dados7}.
Também é possível associar a função \texttt{mutate()} com a função
\texttt{cut()}, como se verá a seguir.

\subsubsection{Tabelas de frequência para dados
numéricos}\label{tabelas-de-frequuxeancia-para-dados-numuxe9ricos}

A construção de tabelas de frequência com dados agrupados em classes é
crucial quando se lida com variáveis quantitativas contínuas, como
idade, altura, peso do recém-nascido, renda, etc. Usar table()
diretamente, nesses casos não faz sentido, pois cada valor pode ser
único e o número de ``categorias'' seria imenso, tornando a tabela quase
inútil. A solução é agrupar os dados em \textbf{intervalos} ou
\textbf{classes}. Para realizar esta operação, pode-se acionar a mesma
função \texttt{mutate()} associada a função \texttt{cut()}, que divide
um vetor numérico em fatores com base em intervalos especificados.

\ul{\textbf{Passo a passo com a função cut()}}

Como exemplo prático, a variável de interesse será representada pelos
pesos dos recém-nascidos (\texttt{pesoRN}) do conjunto de dados,
\texttt{dados}, mencionado na (Seção~\ref{sec-dados7}).

\textbf{Passo 1}: \ul{Definir os intervalos de classe}

Antes, as classes foram estabelecidas de acordo com algum critério
escolhido pelo autor ou por algum critério relacionado à saúde, como por
exemplo a idade materna, onde as gestante com idade abaixo de 20 anos e
acima de 35 anos, gestantes com baixa escolaridade (\textless{} de 5
anos), situação conjugal insegura, etc., constituem fatores de risco na
gravidez (74). Em geral, quando não há um padrão pré-determinado, o
número de classes é estabelecido de acordo com o tamanho da amostra.
Este número pode ser escolhido lembrando-se das oscilações que ocorrem
nos dados e do interesse do pesquisador em mostrar seus dados. Não
existe uma regra totalmente eficiente para determinar o número de
classes. É importante ter bom senso, de maneira que seja possível ver
como os valores se distribuem. Uma regra geral é usar entre 5 e 15
classes (75), mas isso pode variar. Com poucas classes, perde-se
precisão e, com muitas classes, a tabela torna-se muito extensa. Uma
forma comum é usar a \emph{Regra de Sturges} \footnote{Determina que o
  número de classes (\emph{k}) pode ser calculado pela fórmula:
  \(k = 1 + 3.322 * log10(n)\), onde \emph{n} é igual ao número de
  observações.}, para estimar o número de classes. Baseado na regra de
Sturges é sugerido usar a recomendação da Tabela~\ref{tbl-sturges} (76).

\global\setlength{\Oldarrayrulewidth}{\arrayrulewidth}

\global\setlength{\Oldtabcolsep}{\tabcolsep}

\setlength{\tabcolsep}{2pt}

\renewcommand*{\arraystretch}{1.5}



\providecommand{\ascline}[3]{\noalign{\global\arrayrulewidth #1}\arrayrulecolor[HTML]{#2}\cline{#3}}

\begin{longtable}[c]{|p{1.80in}|p{1.22in}}

\caption{\label{tbl-sturges}Numero de classes, baseado em Sturges}

\tabularnewline

\ascline{1.5pt}{666666}{1-2}

\multicolumn{1}{>{\centering}m{\dimexpr 1.8in+0\tabcolsep}}{\textcolor[HTML]{000000}{\fontsize{11}{11}\selectfont{\global\setmainfont{Arial}{\textbf{Nº\ de\ observações\ (n)}}}}} & \multicolumn{1}{>{\centering}m{\dimexpr 1.22in+0\tabcolsep}}{\textcolor[HTML]{000000}{\fontsize{11}{11}\selectfont{\global\setmainfont{Arial}{\textbf{Nº\ de\ classes}}}}} \\

\ascline{1.5pt}{666666}{1-2}\endfirsthead 

\ascline{1.5pt}{666666}{1-2}

\multicolumn{1}{>{\centering}m{\dimexpr 1.8in+0\tabcolsep}}{\textcolor[HTML]{000000}{\fontsize{11}{11}\selectfont{\global\setmainfont{Arial}{\textbf{Nº\ de\ observações\ (n)}}}}} & \multicolumn{1}{>{\centering}m{\dimexpr 1.22in+0\tabcolsep}}{\textcolor[HTML]{000000}{\fontsize{11}{11}\selectfont{\global\setmainfont{Arial}{\textbf{Nº\ de\ classes}}}}} \\

\ascline{1.5pt}{666666}{1-2}\endhead



\multicolumn{1}{>{\centering}m{\dimexpr 1.8in+0\tabcolsep}}{\textcolor[HTML]{000000}{\fontsize{11}{11}\selectfont{\global\setmainfont{Arial}{1}}}} & \multicolumn{1}{>{\centering}m{\dimexpr 1.22in+0\tabcolsep}}{\textcolor[HTML]{000000}{\fontsize{11}{11}\selectfont{\global\setmainfont{Arial}{1}}}} \\





\multicolumn{1}{>{\centering}m{\dimexpr 1.8in+0\tabcolsep}}{\textcolor[HTML]{000000}{\fontsize{11}{11}\selectfont{\global\setmainfont{Arial}{2}}}} & \multicolumn{1}{>{\centering}m{\dimexpr 1.22in+0\tabcolsep}}{\textcolor[HTML]{000000}{\fontsize{11}{11}\selectfont{\global\setmainfont{Arial}{2}}}} \\





\multicolumn{1}{>{\centering}m{\dimexpr 1.8in+0\tabcolsep}}{\textcolor[HTML]{000000}{\fontsize{11}{11}\selectfont{\global\setmainfont{Arial}{3\ a\ 5}}}} & \multicolumn{1}{>{\centering}m{\dimexpr 1.22in+0\tabcolsep}}{\textcolor[HTML]{000000}{\fontsize{11}{11}\selectfont{\global\setmainfont{Arial}{3}}}} \\





\multicolumn{1}{>{\centering}m{\dimexpr 1.8in+0\tabcolsep}}{\textcolor[HTML]{000000}{\fontsize{11}{11}\selectfont{\global\setmainfont{Arial}{6\ a\ 11}}}} & \multicolumn{1}{>{\centering}m{\dimexpr 1.22in+0\tabcolsep}}{\textcolor[HTML]{000000}{\fontsize{11}{11}\selectfont{\global\setmainfont{Arial}{4}}}} \\





\multicolumn{1}{>{\centering}m{\dimexpr 1.8in+0\tabcolsep}}{\textcolor[HTML]{000000}{\fontsize{11}{11}\selectfont{\global\setmainfont{Arial}{12\ a\ 23}}}} & \multicolumn{1}{>{\centering}m{\dimexpr 1.22in+0\tabcolsep}}{\textcolor[HTML]{000000}{\fontsize{11}{11}\selectfont{\global\setmainfont{Arial}{5}}}} \\





\multicolumn{1}{>{\centering}m{\dimexpr 1.8in+0\tabcolsep}}{\textcolor[HTML]{000000}{\fontsize{11}{11}\selectfont{\global\setmainfont{Arial}{24\ a\ 46}}}} & \multicolumn{1}{>{\centering}m{\dimexpr 1.22in+0\tabcolsep}}{\textcolor[HTML]{000000}{\fontsize{11}{11}\selectfont{\global\setmainfont{Arial}{6}}}} \\





\multicolumn{1}{>{\centering}m{\dimexpr 1.8in+0\tabcolsep}}{\textcolor[HTML]{000000}{\fontsize{11}{11}\selectfont{\global\setmainfont{Arial}{47\ a\ 93}}}} & \multicolumn{1}{>{\centering}m{\dimexpr 1.22in+0\tabcolsep}}{\textcolor[HTML]{000000}{\fontsize{11}{11}\selectfont{\global\setmainfont{Arial}{7}}}} \\





\multicolumn{1}{>{\centering}m{\dimexpr 1.8in+0\tabcolsep}}{\textcolor[HTML]{000000}{\fontsize{11}{11}\selectfont{\global\setmainfont{Arial}{94\ a\ 187}}}} & \multicolumn{1}{>{\centering}m{\dimexpr 1.22in+0\tabcolsep}}{\textcolor[HTML]{000000}{\fontsize{11}{11}\selectfont{\global\setmainfont{Arial}{8}}}} \\





\multicolumn{1}{>{\centering}m{\dimexpr 1.8in+0\tabcolsep}}{\textcolor[HTML]{000000}{\fontsize{11}{11}\selectfont{\global\setmainfont{Arial}{188\ a\ 376}}}} & \multicolumn{1}{>{\centering}m{\dimexpr 1.22in+0\tabcolsep}}{\textcolor[HTML]{000000}{\fontsize{11}{11}\selectfont{\global\setmainfont{Arial}{9}}}} \\





\multicolumn{1}{>{\centering}m{\dimexpr 1.8in+0\tabcolsep}}{\textcolor[HTML]{000000}{\fontsize{11}{11}\selectfont{\global\setmainfont{Arial}{377\ a\ 756}}}} & \multicolumn{1}{>{\centering}m{\dimexpr 1.22in+0\tabcolsep}}{\textcolor[HTML]{000000}{\fontsize{11}{11}\selectfont{\global\setmainfont{Arial}{10}}}} \\

\ascline{1.5pt}{666666}{1-2}



\multicolumn{2}{>{\raggedright}m{\dimexpr 3.02in+2\tabcolsep}}{\textcolor[HTML]{000000}{\fontsize{9}{9}\selectfont{\global\setmainfont{Arial}{FONTE:Arango,\ H.\ G.\ (2009)}}}} \\




\end{longtable}

\arrayrulecolor[HTML]{000000}

\global\setlength{\arrayrulewidth}{\Oldarrayrulewidth}

\global\setlength{\tabcolsep}{\Oldtabcolsep}

\renewcommand*{\arraystretch}{1}

Para a variável \texttt{dados\$pesoRN}, como existem 200 observações,
pode-se, de acordo com a Tabela~\ref{tbl-sturges}, usar ao redor de 9
classes. O R tem uma função \texttt{nclass.Sturges\ ()}, que também pode
ser usada:

\begin{Shaded}
\begin{Highlighting}[]
\NormalTok{k }\OtherTok{\textless{}{-}} \FunctionTok{nclass.Sturges}\NormalTok{(dados}\SpecialCharTok{$}\NormalTok{pesoRN)}
\FunctionTok{print}\NormalTok{(k)}
\end{Highlighting}
\end{Shaded}

\begin{verbatim}
[1] 9
\end{verbatim}

O cálculo da função retorna o mesmo resultado.

\textbf{Passo 2}: \ul{Amplitude e limites das classes}

A classe possui um \textbf{limite inferior} e um \textbf{limite
superior}. O importante é que os limites dos intervalos sejam mutuamente
exclusivos, isto é, cada valor deve ser representado em um único
intervalo. Além disso, os intervalos devem ser exaustivos, isto é, devem
conter todos os valores possíveis entre o valor mínimo e o máximo. O
recomendado é que as classes sejam homogêneas, ou seja, tenham a mesma
amplitude\footnote{Quando já existe um critério, como mostrado acima, ou
  o pesquisador tem um determinado interesse, as classes podem ter
  amplitudes diferentes.}. A amplitude dos valores pode ser obtida com a
função \texttt{range()}:

\begin{Shaded}
\begin{Highlighting}[]
\NormalTok{amplitude }\OtherTok{\textless{}{-}} \FunctionTok{range}\NormalTok{(dados}\SpecialCharTok{$}\NormalTok{pesoRN) }
\NormalTok{amplitude}
\end{Highlighting}
\end{Shaded}

\begin{verbatim}
[1]  810 4670
\end{verbatim}

Usando esta amplitude dos dados, é possível ter a largura das classes
(\emph{h}):

\begin{Shaded}
\begin{Highlighting}[]
\NormalTok{h }\OtherTok{\textless{}{-}} \FunctionTok{round}\NormalTok{(}\FunctionTok{diff}\NormalTok{(amplitude)}\SpecialCharTok{/}\NormalTok{k, }\DecValTok{0}\NormalTok{)}
\NormalTok{h}
\end{Highlighting}
\end{Shaded}

\begin{verbatim}
[1] 429
\end{verbatim}

A fórmula é apenas a diferença absoluta dos valores mínimos e máximo ,
calculada pela função \texttt{diff()} \footnote{A função \texttt{diff()}
  no R~calcula as diferenças entre elementos consecutivos de um vetor,
  matriz ou série temporal.~Ela subtrai cada elemento do elemento
  seguinte, retornando um valor absoluto.}, dividida pelo número de
classes (\emph{k}), arredondado com o a função \texttt{round\ ()}.

A partir desses dados, é possível construir as classes: O limite
inferior da primeira classe é 810 e o limite superior é 1239, exclusive;
o limite inferior da segunda classe inclui o valor de 1239 e vai até
1668, excluindo-o, pois ele será limite inferior da terceira classe e
assim por diante

\textbf{Passo 3}: \ul{Agrupar os dados usando \mbox{\texttt{cut()}}}

Seguir esse processo seria longo, tedioso e sujeito a erros. O R permite
agilizar este trabalho com as funções \texttt{mutate()} e função
\texttt{cut()}. Dentro da função \texttt{cut()}, pode ser acionada a
função \texttt{seq()} para estabelecer os pontos de corte.

\begin{enumerate}
\def\labelenumi{\arabic{enumi}.}
\item
  Criar uma sequência dos pontos de corte desejados, usando a função
  \texttt{seq()}:

  \begin{tcolorbox}[enhanced jigsaw, bottomrule=.15mm, opacitybacktitle=0.6, colframe=quarto-callout-tip-color-frame, arc=.35mm, coltitle=black, toptitle=1mm, colback=white, colbacktitle=quarto-callout-tip-color!10!white, breakable, bottomtitle=1mm, rightrule=.15mm, titlerule=0mm, toprule=.15mm, opacityback=0, leftrule=.75mm, left=2mm, title=\textcolor{quarto-callout-tip-color}{\faLightbulb}\hspace{0.5em}{Sintaxe função seq()}]

  Necessita os seguintes argumentos:

  \begin{itemize}
  \item
    \textbf{from} : valor inicial da sequência
  \item
    \textbf{to}: valor final da sequência
  \item
    \textbf{length.out}: número inteiro que especifica o comprimento
    desejado da sequência. Este último argumento é importante, pois se o
    objetivo é construir~ 9 classes, preciso de uma sequência de 10
    valores.
  \end{itemize}

  \end{tcolorbox}
\end{enumerate}

Como os pontos de cortes precisam ser números inteiros, usa-se a função
\texttt{round()} com 0 (zero) dígitos:

\begin{Shaded}
\begin{Highlighting}[]
\NormalTok{valor\_min }\OtherTok{\textless{}{-}} \FunctionTok{min}\NormalTok{(dados}\SpecialCharTok{$}\NormalTok{pesoRN)    }\CommentTok{\# valor inicial}
\NormalTok{valor\_max }\OtherTok{\textless{}{-}} \FunctionTok{max}\NormalTok{(dados}\SpecialCharTok{$}\NormalTok{pesoRN)    }\CommentTok{\# valor final}
\NormalTok{length.out }\OtherTok{=}\NormalTok{ k }\SpecialCharTok{+} \DecValTok{1}                \CommentTok{\# 10 valores para obter 9 classes}

\NormalTok{cortes }\OtherTok{\textless{}{-}} \FunctionTok{round}\NormalTok{(}\FunctionTok{seq}\NormalTok{(valor\_min, valor\_max, }\AttributeTok{length.out =}\NormalTok{ k }\SpecialCharTok{+} \DecValTok{1}\NormalTok{), }\DecValTok{0}\NormalTok{)  }
\end{Highlighting}
\end{Shaded}

\begin{enumerate}
\def\labelenumi{\arabic{enumi}.}
\setcounter{enumi}{1}
\tightlist
\item
  Tendo os pontos de corte estabelecidos, basta agora criar a variável
  \texttt{pesoCateg}, ou seja a variável \texttt{pesoRN} numérica
  classificada em 9 classes:
\end{enumerate}

\begin{Shaded}
\begin{Highlighting}[]
\NormalTok{dados }\OtherTok{\textless{}{-}}\NormalTok{ dados}\SpecialCharTok{\%\textgreater{}\%} 
  \FunctionTok{mutate}\NormalTok{(}\AttributeTok{pesoCateg =} \FunctionTok{cut}\NormalTok{(}
\NormalTok{    pesoRN,}
    \AttributeTok{breaks =}\NormalTok{ cortes,}
    \AttributeTok{include.lowest =} \ConstantTok{TRUE}\NormalTok{,}
    \AttributeTok{right =} \ConstantTok{FALSE}\NormalTok{,}
    \AttributeTok{ordered\_result =} \ConstantTok{TRUE}\NormalTok{))}
\end{Highlighting}
\end{Shaded}

\begin{enumerate}
\def\labelenumi{\arabic{enumi}.}
\setcounter{enumi}{2}
\tightlist
\item
  Criar uma tabela de frequência absoluta com a variável
  \texttt{pesoCateg}:
\end{enumerate}

\begin{Shaded}
\begin{Highlighting}[]
\NormalTok{f }\OtherTok{\textless{}{-}} \FunctionTok{table}\NormalTok{(dados}\SpecialCharTok{$}\NormalTok{pesoCateg)}
\FunctionTok{print}\NormalTok{(f)}
\end{Highlighting}
\end{Shaded}

\begin{verbatim}

     [810,1.24e+03) [1.24e+03,1.67e+03)  [1.67e+03,2.1e+03)  [2.1e+03,2.53e+03) 
                  4                   5                  10                  14 
[2.53e+03,2.95e+03) [2.95e+03,3.38e+03) [3.38e+03,3.81e+03) [3.81e+03,4.24e+03) 
                 47                  62                  42                  12 
[4.24e+03,4.67e+03] 
                  4 
\end{verbatim}

Como diria , o famoso personagem Obelix, amigo inseparável do gaulês
Asterix\footnote{Personagens criados pelos franceses Albert Uderzo e
  Rene Goscinny.} ; ``Os céus caíram sobre nós! O que é isso? Vocês são
loucos romanos!! Por Júpiter!!!''

Nada de especial. simplesmente, o R retornou valores em notação
científica. Não tem nada errado! Apenas, isso torna a legibilidade dos
números mais complicada, menos intuitiva. A primeira classe
{[}810,1.24e+03) ficaria mais fácil de ser lida se estivesse como
{[}810, 1240), ou seja, inclui o valor de 810 e exclui o 1240 e a última
classe {[}4.24e+03,4.67e+03{]} inclui ambos os limites, melhor escritos
como {[}4240, 4670{]}.

\begin{tcolorbox}[enhanced jigsaw, bottomrule=.15mm, opacitybacktitle=0.6, colframe=quarto-callout-note-color-frame, arc=.35mm, coltitle=black, toptitle=1mm, colback=white, colbacktitle=quarto-callout-note-color!10!white, breakable, bottomtitle=1mm, rightrule=.15mm, titlerule=0mm, toprule=.15mm, opacityback=0, leftrule=.75mm, left=2mm, title=\textcolor{quarto-callout-note-color}{\faInfo}\hspace{0.5em}{Nota}]

Observe a lógica para o último intervalo, que é fechado nos dois lados
{[} {]} , incluindo os valores no intervalo de classe. Os outros
obedecem a lógica {[} ), incluindo o valor limite inferior e excluindo o
superior. A forma como isso ocorre é determinado pela função
\texttt{cut()} , que cria os intervalos (variável \texttt{categPeso}),
depende de dois argumentos lógicos:

\begin{itemize}
\item
  \textbf{right = TRUE (padrão):} Os intervalos são do tipo (X, Y{]}.
  Isso significa que o valor X não é incluído, mas o valor Y é. O
  primeiro intervalo é a exceção, sendo {[}X, Y{]}.
\item
  \textbf{right = FALSE:} Os intervalos são do tipo {[}X, Y), como no
  exemplo usado, {[}810,1240). Isso significa que o valor 810 é
  incluído, mas o valor 1240 não é. O último intervalo é a exceção,
  sendo {[}X, Y{]}. No exemplo, {[}4240,4670{]} e significa que os dois
  valores estão incluídos.
\item
  O argumento \texttt{include.lowest} entra em jogo, modificando o
  comportamento padrão:
\end{itemize}

Quando se usa \texttt{right\ =\ FALSE} (como o código usado acima), o R
cria intervalos {[}X, Y), no exemplo, igual a {[}810,1240). O argumento
\texttt{include.lowest\ =\ TRUE} faz com que o primeiro intervalo seja
{[}mínimo, Y), garantindo que o valor mínimo da variável - 810 - seja
incluído no primeiro intervalo.

Resumindo:

1) \textbf{right = FALSE e include.lowest = TRUE:}

\begin{itemize}
\item
  \textbf{Intervalos:} {[}A, B), {[}C, D), {[}E, F), e assim por diante.
\item
  \textbf{O primeiro intervalo} ({[}min(x), \ldots) ) \textbf{inclui o
  valor mínimo}.
\item
  \textbf{O último intervalo} ({[}\ldots, max(x){]}) \textbf{inclui o
  valor máximo}.
\end{itemize}

2) \textbf{right = TRUE} (padrão) \textbf{e include.lowest = FALSE}
(padrão):

\begin{itemize}
\item
  \textbf{Intervalos:} (A, C{]}, (C, D{]}, (E, F{]}.
\item
  \textbf{O primeiro intervalo} {[}min(x), \ldots{]} \textbf{inclui o
  valor mínimo}.
\end{itemize}

\end{tcolorbox}

\begin{enumerate}
\def\labelenumi{\arabic{enumi}.}
\setcounter{enumi}{3}
\tightlist
\item
  Para que os rótulo apareçam sem notação científica\footnote{Se houver
    interesse de remover a notação científica e deixar os gauleses
    felizes.} e no mesmo formato \texttt{{[})} , com exceção da última
  classe que ficará \texttt{{[}{]}}, procede-se do seguinte modo:
\end{enumerate}

\begin{Shaded}
\begin{Highlighting}[]
\NormalTok{rotulos }\OtherTok{\textless{}{-}}  \FunctionTok{paste0}\NormalTok{(}\StringTok{"["}\NormalTok{, }
                   \FunctionTok{format}\NormalTok{(}\FunctionTok{head}\NormalTok{(cortes, }\SpecialCharTok{{-}}\DecValTok{1}\NormalTok{), }
                          \AttributeTok{scientific =} \ConstantTok{FALSE}\NormalTok{),}
                   \StringTok{","}\NormalTok{,}
                   \FunctionTok{format}\NormalTok{(}\FunctionTok{tail}\NormalTok{(cortes, }\SpecialCharTok{{-}}\DecValTok{1}\NormalTok{), }
                          \AttributeTok{scientific =} \ConstantTok{FALSE}\NormalTok{),}
                   \StringTok{")"}\NormalTok{)}
\NormalTok{dados}\SpecialCharTok{$}\NormalTok{pesoCateg }\OtherTok{\textless{}{-}} \FunctionTok{factor}\NormalTok{(rotulos[dados}\SpecialCharTok{$}\NormalTok{pesoCateg], }
                          \AttributeTok{levels =}\NormalTok{ rotulos, }
                          \AttributeTok{ordered =} \ConstantTok{TRUE}\NormalTok{)}
\end{Highlighting}
\end{Shaded}

\textbf{Passo 4}: \ul{Construção da tabela de frequência}

Com os dados agrupados em classes, agora pode-se usar as funções
\texttt{table()} e \texttt{prop.table()} para construir a tabela de
frequência absoluta e relativa, com realizado anteriormente.

\begin{Shaded}
\begin{Highlighting}[]
\CommentTok{\# Frequência Absoluta}
\NormalTok{f }\OtherTok{\textless{}{-}} \FunctionTok{table}\NormalTok{(dados}\SpecialCharTok{$}\NormalTok{pesoCateg)}
\FunctionTok{print}\NormalTok{(f)}
\end{Highlighting}
\end{Shaded}

\begin{verbatim}

[ 810,1239) [1239,1668) [1668,2097) [2097,2526) [2526,2954) [2954,3383) 
          4           5          10          14          47          62 
[3383,3812) [3812,4241) [4241,4670) 
         42          12           4 
\end{verbatim}

\begin{Shaded}
\begin{Highlighting}[]
\CommentTok{\# Frequência Relativa Percentual}
\NormalTok{frp }\OtherTok{\textless{}{-}} \FunctionTok{prop.table}\NormalTok{(f) }\SpecialCharTok{*} \DecValTok{100}
\FunctionTok{print}\NormalTok{(frp)}
\end{Highlighting}
\end{Shaded}

\begin{verbatim}

[ 810,1239) [1239,1668) [1668,2097) [2097,2526) [2526,2954) [2954,3383) 
        2.0         2.5         5.0         7.0        23.5        31.0 
[3383,3812) [3812,4241) [4241,4670) 
       21.0         6.0         2.0 
\end{verbatim}

\ul{Tabela de Frequência Completa}

Para uma apresentação mais clara, combinar tudo em um dataframe,
adicionando as colunas de frequência acumulada.

\begin{Shaded}
\begin{Highlighting}[]
\CommentTok{\# Dataframe com as f e fr como vetores}
\NormalTok{tab\_frequencia }\OtherTok{\textless{}{-}} \FunctionTok{data.frame}\NormalTok{(}\AttributeTok{f =} \FunctionTok{as.vector}\NormalTok{(f),}
                             \AttributeTok{frp =} \FunctionTok{as.vector}\NormalTok{(frp))}
  
\CommentTok{\# Adicionar Frequência Absoluta Acumulada}
\NormalTok{tab\_frequencia}\SpecialCharTok{$}\NormalTok{F }\OtherTok{\textless{}{-}} \FunctionTok{cumsum}\NormalTok{(tab\_frequencia}\SpecialCharTok{$}\NormalTok{f)}

\CommentTok{\# Adicionar Frequência Relativa Acumulada (\%)}
\NormalTok{tab\_frequencia}\SpecialCharTok{$}\NormalTok{Frp }\OtherTok{\textless{}{-}} \FunctionTok{cumsum}\NormalTok{(tab\_frequencia}\SpecialCharTok{$}\NormalTok{frp)}
  
\CommentTok{\# Adicionar os intervalos como uma coluna  }
\NormalTok{tab\_frequencia}\SpecialCharTok{$}\NormalTok{Classes }\OtherTok{\textless{}{-}} \FunctionTok{names}\NormalTok{(}\FunctionTok{table}\NormalTok{(dados}\SpecialCharTok{$}\NormalTok{pesoCateg))}

\CommentTok{\# Reorganizar as colunas para melhor visualização}
\NormalTok{tab\_frequencia }\OtherTok{\textless{}{-}}\NormalTok{ tab\_frequencia[}\FunctionTok{c}\NormalTok{(}\DecValTok{5}\NormalTok{, }\DecValTok{1}\NormalTok{, }\DecValTok{2}\NormalTok{, }\DecValTok{3}\NormalTok{, }\DecValTok{4}\NormalTok{)]}

\FunctionTok{print}\NormalTok{(tab\_frequencia)}
\end{Highlighting}
\end{Shaded}

\begin{verbatim}
      Classes  f  frp   F   Frp
1 [ 810,1239)  4  2.0   4   2.0
2 [1239,1668)  5  2.5   9   4.5
3 [1668,2097) 10  5.0  19   9.5
4 [2097,2526) 14  7.0  33  16.5
5 [2526,2954) 47 23.5  80  40.0
6 [2954,3383) 62 31.0 142  71.0
7 [3383,3812) 42 21.0 184  92.0
8 [3812,4241) 12  6.0 196  98.0
9 [4241,4670)  4  2.0 200 100.0
\end{verbatim}

A saída retorna uma tabela de frequência completa, mostrando a
distribuição dos pesos dos recém-nascidos por classes.

A combinação de mutate(). \texttt{cut()}, \texttt{table()} e
\texttt{prop.table()} é a abordagem padrão e mais robusta para criar
tabelas de frequência com dados agrupados no R.`

Mesmo que seja possível observar com clareza como os pesos dos
recém-nascidos se distribuem, fica pouco intuitivo analisar, levando em
conta as classificações recomendadas pela
\href{https://www.who.int/teams/maternal-newborn-child-adolescent-health-and-ageing/newborn-health/preterm-and-low-birth-weight}{Organização
Mundial da Saúde} (OMS) e Mistério da Saúde do Brasil (MS), através do
\href{https://bvsms.saude.gov.br/bvs/publicacoes/atencao_saude_recem_nascido_profissionais_v1.pdf}{Guia
para os Profissionais de Saúde de Atenção à Saúde do Recém-Nascido}.
Esta tabela de frequência serve mais para se observar o comportamento da
variável \texttt{pesoRN}. Entretanto, essa ação é mais adequadamente
realizada, utilizando-se um gráfico do tipo histograma (veja
Seção~\ref{sec-hist}).\\
A tabela construída será repetida ,usando para classificar, a função
\texttt{case\_when()} e a classificação dos recém-nascidos de acordo com
a OMS.

\ul{\textbf{Passo a passo com a função case\_when()}}

A função \texttt{cut()} foi já usada para classificar a variável
\texttt{pesoRN}, entretanto, no início deste capítulo na
Seção~\ref{sec-dados7} , para realizar esta ação, foi usada a função
\texttt{case\_when()} para agrupar a variável \texttt{idadeMae} e
\texttt{imc} em classes. A principal diferença entre essas funções, é
que cut\texttt{()} é especificamente projetada para agrupar dados em
intervalos numéricos, enquanto \texttt{case\_when()} é uma ferramenta
mais generalista e poderosa do pacote \texttt{dplyr}, que permite criar
classes com base em qualquer tipo de condição lógica. A função
\texttt{case\_when()} pode ser vantajosa por apresentar flexibilidade
total, permitindo definir intervalos com bastante facilidade. Além disso
para pessoas familiarizadas com a sintaxe do \texttt{tidyverse} (veja
Seção~\ref{sec-tidyverse}), a estrutura
\texttt{condição\ \textasciitilde{}\ valor} de \texttt{case\_when()} é
extremamente intuitiva e fácil de entender. Como faz parte do
\texttt{tidyverse}, \texttt{case\_when()} se encaixa perfeitamente em
fluxos de trabalho que usam funções como \texttt{mutate()} e
\texttt{group\_by()}. Os dados dos pesos dos recém-nascidos
(\texttt{pesoRN}) servirão coo exemplo prático para criar uma tabela
mais útil e comparação entre as duas funções.

\textbf{Passo 1}: Criação da variável \texttt{classePeso}, usando o
critério da OMS e das de frequência absoluta e relativa percentual:

\begin{tcolorbox}[enhanced jigsaw, bottomrule=.15mm, opacitybacktitle=0.6, colframe=quarto-callout-note-color-frame, arc=.35mm, coltitle=black, toptitle=1mm, colback=white, colbacktitle=quarto-callout-note-color!10!white, breakable, bottomtitle=1mm, rightrule=.15mm, titlerule=0mm, toprule=.15mm, opacityback=0, leftrule=.75mm, left=2mm, title=\textcolor{quarto-callout-note-color}{\faInfo}\hspace{0.5em}{Classificação dos receém-nascidos de acordo com o peso (OMS)}]

\textbf{Extremo baixo peso ao nascer}: RNs com peso inferior a 1.000
gramas.

\textbf{Muito baixo peso ao nascer}: RNs com peso inferior a 1.500
gramas.

\textbf{Baixo peso ao nascer}: RNs com peso inferior a 2.500 gramas.

\textbf{Adequado peso ao nascer}: RNs com peso inferior a 4000 gramas.

\textbf{Excesso de peso ao nascer}: RNs com peso ≥ 4000 gramas.

\end{tcolorbox}

\begin{Shaded}
\begin{Highlighting}[]
\CommentTok{\# Classificação dos RNs pelo citério da OMS {-}{-} classePeso}
\NormalTok{dados }\OtherTok{\textless{}{-}}\NormalTok{ dados }\SpecialCharTok{\%\textgreater{}\%}  
  \FunctionTok{mutate}\NormalTok{(}\AttributeTok{classePeso =} \FunctionTok{case\_when}\NormalTok{(}
\NormalTok{    pesoRN }\SpecialCharTok{\textless{}} \DecValTok{1000} \SpecialCharTok{\textasciitilde{}} \StringTok{"BP Extremo"}\NormalTok{,}
\NormalTok{    pesoRN }\SpecialCharTok{\textgreater{}=} \DecValTok{1000} \SpecialCharTok{\&}\NormalTok{ pesoRN }\SpecialCharTok{\textless{}} \DecValTok{1500} \SpecialCharTok{\textasciitilde{}} \StringTok{"Muito BP"}\NormalTok{, }
\NormalTok{    pesoRN }\SpecialCharTok{\textgreater{}=} \DecValTok{1500} \SpecialCharTok{\&}\NormalTok{ pesoRN }\SpecialCharTok{\textless{}} \DecValTok{2500} \SpecialCharTok{\textasciitilde{}} \StringTok{"Baixo Peso"}\NormalTok{,}
\NormalTok{    pesoRN }\SpecialCharTok{\textgreater{}=} \DecValTok{2500} \SpecialCharTok{\&}\NormalTok{ pesoRN }\SpecialCharTok{\textless{}} \DecValTok{4000} \SpecialCharTok{\textasciitilde{}} \StringTok{"Peso Normal"}\NormalTok{,}
\NormalTok{    pesoRN }\SpecialCharTok{\textgreater{}=} \DecValTok{4000} \SpecialCharTok{\textasciitilde{}} \StringTok{"Excesso de Peso"}
\NormalTok{  )) }\SpecialCharTok{\%\textgreater{}\%} 
  \FunctionTok{mutate}\NormalTok{(}\AttributeTok{classePeso =} \FunctionTok{factor}\NormalTok{(classePeso, }
                             \AttributeTok{levels =} \FunctionTok{c}\NormalTok{(}\StringTok{"BP Extremo"}\NormalTok{, }\StringTok{"Muito BP"}\NormalTok{, }\StringTok{"Baixo Peso (BP)"}\NormalTok{,}
                                        \StringTok{"Peso Normal"}\NormalTok{, }\StringTok{"Excesso Peso"}\NormalTok{)))  }

\CommentTok{\# Tabela de frequência absoluta}
\NormalTok{f }\OtherTok{\textless{}{-}} \FunctionTok{table}\NormalTok{(dados}\SpecialCharTok{$}\NormalTok{classePeso)}
\FunctionTok{print}\NormalTok{(f)}
\end{Highlighting}
\end{Shaded}

\begin{verbatim}

     BP Extremo        Muito BP Baixo Peso (BP)     Peso Normal    Excesso Peso 
              2               5               0             163               0 
\end{verbatim}

\begin{Shaded}
\begin{Highlighting}[]
\CommentTok{\# Tabela de frequência relativa (em porcentagem)}
\NormalTok{frp }\OtherTok{\textless{}{-}} \FunctionTok{round}\NormalTok{(}\FunctionTok{prop.table}\NormalTok{(f) }\SpecialCharTok{*} \DecValTok{100}\NormalTok{, }\DecValTok{1}\NormalTok{)}
\FunctionTok{print}\NormalTok{(frp)}
\end{Highlighting}
\end{Shaded}

\begin{verbatim}

     BP Extremo        Muito BP Baixo Peso (BP)     Peso Normal    Excesso Peso 
            1.2             2.9             0.0            95.9             0.0 
\end{verbatim}

\textbf{Passo 2}: Combinar as frequências em um dataframe

\begin{Shaded}
\begin{Highlighting}[]
\NormalTok{tab\_completa }\OtherTok{\textless{}{-}} \FunctionTok{data.frame}\NormalTok{(}
  \AttributeTok{f =} \FunctionTok{as.vector}\NormalTok{(f),}
  \AttributeTok{frp =} \FunctionTok{as.vector}\NormalTok{(frp))}
\end{Highlighting}
\end{Shaded}

\textbf{Passo 3}: Adicionar a frequência absoluta e a frequência
relativa (em \%) acumuladas

\begin{Shaded}
\begin{Highlighting}[]
\CommentTok{\# Frequência Absoluta Acumulada}
\NormalTok{tab\_completa}\SpecialCharTok{$}\NormalTok{F }\OtherTok{\textless{}{-}} \FunctionTok{cumsum}\NormalTok{(tab\_completa}\SpecialCharTok{$}\NormalTok{f)}

\CommentTok{\# Frequência Relativa Acumulada (\%)}
\NormalTok{tab\_completa}\SpecialCharTok{$}\NormalTok{Frp }\OtherTok{\textless{}{-}} \FunctionTok{cumsum}\NormalTok{(tab\_completa}\SpecialCharTok{$}\NormalTok{frp)}
\end{Highlighting}
\end{Shaded}

\textbf{Passo 4}: Adicionar os nomes das categorias como uma coluna

\begin{Shaded}
\begin{Highlighting}[]
\NormalTok{tab\_completa}\SpecialCharTok{$}\NormalTok{classePeso }\OtherTok{\textless{}{-}} \FunctionTok{rownames}\NormalTok{ (f)}
\end{Highlighting}
\end{Shaded}

\textbf{Passo 5}: Renomeie a coluna \texttt{classePeso} como
``Classificação''

\begin{Shaded}
\begin{Highlighting}[]
\FunctionTok{library}\NormalTok{(dplyr)}
\NormalTok{tab\_completa }\OtherTok{\textless{}{-}}\NormalTok{ tab\_completa }\SpecialCharTok{\%\textgreater{}\%}
  \FunctionTok{rename}\NormalTok{(}\StringTok{"Classificação"} \OtherTok{=}\NormalTok{ classePeso)}
\end{Highlighting}
\end{Shaded}

\textbf{Passo 6}: Renomeie a coluna \texttt{classePeso} como
``Classificação''

\begin{Shaded}
\begin{Highlighting}[]
\NormalTok{tab\_completa }\OtherTok{\textless{}{-}}\NormalTok{ tab\_completa[}\FunctionTok{c}\NormalTok{(}\StringTok{"Classificação"}\NormalTok{, }\StringTok{"f"}\NormalTok{, }\StringTok{"frp"}\NormalTok{, }\StringTok{"F"}\NormalTok{, }\StringTok{"Frp"}\NormalTok{)]}

\FunctionTok{print}\NormalTok{(tab\_completa)}
\end{Highlighting}
\end{Shaded}

\begin{verbatim}
    Classificação   f  frp   F   Frp
1      BP Extremo   2  1.2   2   1.2
2        Muito BP   5  2.9   7   4.1
3 Baixo Peso (BP)   0  0.0   7   4.1
4     Peso Normal 163 95.9 170 100.0
5    Excesso Peso   0  0.0 170 100.0
\end{verbatim}

Assim, pode-se dizer que a prevalência de baixo peso na maternidade do
Hospital Geral de Caxias do Sul é bem mais alta da prevalência no Brasil
que é de 6,1\% (IC95\%: 4,5-8,3\%) (77).

\section{Tabelas Prontas para
Publicação}\label{tabelas-prontas-para-publicauxe7uxe3o}

O R possui várias alternativas para produzir tabelas científicas,
bonitas e elegantes que possam ser publicadas. Neste livro, serão
mostradas algumas dessas alternativas.

\subsection{Pacote gt}\label{pacote-gt}

O pacote \texttt{gt} é um pacote que permite a apresentação de tabelas
de maneira limpa e organizada (78). funciona através do
\texttt{tidyverse} com os comandos \texttt{pipe\ \%\textgreater{}\%}, o
que facilita sua aplicação.

\subsubsection{Passos para criar uma tabela com o pacote
gt}\label{passos-para-criar-uma-tabela-com-o-pacote-gt}

\textbf{Passo 1}: Criar uma tabela base (dataframe ou tibble), como a
tabela recém criada (\texttt{tab\_completa}) com os dados dos
recém-nascidos. Verificar se os dados são realmente um dataframe ou
tibble:

\begin{Shaded}
\begin{Highlighting}[]
\FunctionTok{class}\NormalTok{(tab\_completa)}
\end{Highlighting}
\end{Shaded}

\begin{verbatim}
[1] "data.frame"
\end{verbatim}

A seguir, transformar o dataframe em uma tabela \texttt{gt} com a função
\texttt{gt()}:

\begin{Shaded}
\begin{Highlighting}[]
\FunctionTok{library}\NormalTok{(gt)}
\NormalTok{tab\_gt }\OtherTok{\textless{}{-}} \FunctionTok{gt}\NormalTok{(}\AttributeTok{data =}\NormalTok{ tab\_completa)}
\NormalTok{tab\_gt}
\end{Highlighting}
\end{Shaded}

\begin{table}

\caption{\label{tbl-tabgt1}Tabela simples com o pacote gt}

\centering{

\fontsize{12.0pt}{14.4pt}\selectfont
\begin{tabular*}{\linewidth}{@{\extracolsep{\fill}}lrrrr}
\toprule
Classificação & f & frp & F & Frp \\ 
\midrule\addlinespace[2.5pt]
BP Extremo & 2 & 1.2 & 2 & 1.2 \\ 
Muito BP & 5 & 2.9 & 7 & 4.1 \\ 
Baixo Peso (BP) & 0 & 0.0 & 7 & 4.1 \\ 
Peso Normal & 163 & 95.9 & 170 & 100.0 \\ 
Excesso Peso & 0 & 0.0 & 170 & 100.0 \\ 
\bottomrule
\end{tabular*}

}

\end{table}%

A Tabela~\ref{tbl-tabgt1} mostra duas partes, os rótulos das colunas e o
corpo da tabela. Entretanto, o pacote \texttt{gt} permite de maneira
fácil modificar ou acrescentar outras partes.\\
As partes que constituem a tabela \texttt{gt} são nomeadas de forma
semelhante ao mostrado na Figura~\ref{fig-tabela1}: cabeçalho
(\emph{table header}), rótulo das linhas (\emph{stub}), corpo
(\emph{body}) e rodapé (\emph{footer}). Reconhecer essas partes é
importante para a adição outras características da tabela.

\textbf{Passo 2}: Personalizar o estilo (Tabela~\ref{tbl-tabgt2})

A família de funções \texttt{tab\_*()} permite adicionar outras partes.
O título e subtítulo podem ser adicionados com a função
\texttt{tab\_header()}. Alterações de estilo são feitas pela função
\texttt{tab\_style()}. Essa função define a formatação a ser aplicada,
que pode ser \texttt{cell\_text()} para formatar texto e
\texttt{locations\ =} que define quais células ou áreas da tabela devem
receber o estilo. As opções incluem:

• \texttt{cells\_column\_labels()}: Para os cabeçalhos das colunas.\\
• \texttt{cells\_title()}: Para o título da tabela.\\
• \texttt{cells\_stub()}: Para o cabeçalho da linha.\\
• \texttt{cells\_body()}: Para o corpo da tabela.\\
• E outras funções específicas, como \texttt{cells\_row\_groups()}.

Para maiores detalhes consulte
\href{Create\%20beautiful\%20tables\%20with\%20gt}{Create beautiful
tables with gt} ou
\href{https://gt.rstudio.com/articles/gt.html}{Introduction to Creating
gt Tables}.

\begin{Shaded}
\begin{Highlighting}[]
\NormalTok{tab\_gt }\OtherTok{\textless{}{-}}\NormalTok{ tab\_gt }\SpecialCharTok{\%\textgreater{}\%}
  \FunctionTok{tab\_header}\NormalTok{(}
    \AttributeTok{title =} \StringTok{"Classificação dos Pesos ao Nascer, de acordo com a OMS"}\NormalTok{,}
    \AttributeTok{subtitle =} \StringTok{"Hospital Geral de Caxias do Sul, 2008"}\NormalTok{) }\SpecialCharTok{\%\textgreater{}\%}
  \FunctionTok{tab\_style}\NormalTok{(}
    \AttributeTok{style =} \FunctionTok{cell\_text}\NormalTok{(}
      \AttributeTok{weight =} \StringTok{"bold"}\NormalTok{,}
      \AttributeTok{size =} \StringTok{"medium"}\NormalTok{,}
      \AttributeTok{font =} \StringTok{"Arial"}\NormalTok{,}
      \AttributeTok{color =} \StringTok{"gray18"}\NormalTok{),}
    \AttributeTok{locations =} \FunctionTok{cells\_title}\NormalTok{(}\AttributeTok{groups =} \StringTok{"title"}\NormalTok{)) }\SpecialCharTok{\%\textgreater{}\%}
  \FunctionTok{tab\_style}\NormalTok{(}
    \AttributeTok{style =} \FunctionTok{cell\_text}\NormalTok{(}
      \AttributeTok{weight =} \StringTok{"bold"}\NormalTok{),}
    \AttributeTok{locations =} \FunctionTok{cells\_column\_labels}\NormalTok{()) }\SpecialCharTok{\%\textgreater{}\%} 
  \FunctionTok{tab\_style}\NormalTok{(}
    \AttributeTok{style =} \FunctionTok{cell\_text}\NormalTok{(}\AttributeTok{color =} \StringTok{"red"}\NormalTok{),}
    \AttributeTok{locations =} \FunctionTok{cells\_body}\NormalTok{(}
      \AttributeTok{columns =}\NormalTok{ Frp,}
      \AttributeTok{rows =} \FunctionTok{near}\NormalTok{(Frp, }\DecValTok{15}\NormalTok{))) }\SpecialCharTok{\%\textgreater{}\%}
    \FunctionTok{cols\_width}\NormalTok{(}
      \DecValTok{1} \SpecialCharTok{\textasciitilde{}} \FunctionTok{px}\NormalTok{(}\DecValTok{150}\NormalTok{),}
      \DecValTok{2}\SpecialCharTok{:}\DecValTok{5} \SpecialCharTok{\textasciitilde{}} \FunctionTok{px}\NormalTok{(}\DecValTok{80}\NormalTok{)}
\NormalTok{    ) }\SpecialCharTok{\%\textgreater{}\%} 
  \FunctionTok{tab\_source\_note}\NormalTok{(}\AttributeTok{source\_note =} \FunctionTok{md}\NormalTok{(}\StringTok{"Fonte: Oliveira Filho, PF (2025)"}\NormalTok{)) }\SpecialCharTok{\%\textgreater{}\%} 
  \FunctionTok{tab\_footnote}\NormalTok{(}
    \AttributeTok{footnote =} \StringTok{"BP = Baixo Peso"}\NormalTok{,}
    \AttributeTok{locations =} \FunctionTok{cells\_body}\NormalTok{(}\AttributeTok{columns =}\NormalTok{ Classificação, }\AttributeTok{rows =} \FunctionTok{c}\NormalTok{(}\DecValTok{1}\NormalTok{,}\DecValTok{2}\NormalTok{))}
\NormalTok{  )}

\NormalTok{tab\_gt}
\end{Highlighting}
\end{Shaded}

\begin{table}

\caption{\label{tbl-tabgt2}Tabela com o pacote gt personalizada}

\centering{

\caption*{
{\large Classificação dos Pesos ao Nascer, de acordo com a OMS} \\ 
{\small Hospital Geral de Caxias do Sul, 2008}
} 
\fontsize{12.0pt}{14.4pt}\selectfont
\begin{tabular*}{\linewidth}{@{\extracolsep{\fill}}>{\raggedright\arraybackslash}p{\dimexpr 112.50pt -2\tabcolsep-1.5\arrayrulewidth}>{\raggedleft\arraybackslash}p{\dimexpr 60.00pt -2\tabcolsep-1.5\arrayrulewidth}>{\raggedleft\arraybackslash}p{\dimexpr 60.00pt -2\tabcolsep-1.5\arrayrulewidth}>{\raggedleft\arraybackslash}p{\dimexpr 60.00pt -2\tabcolsep-1.5\arrayrulewidth}>{\raggedleft\arraybackslash}p{\dimexpr 60.00pt -2\tabcolsep-1.5\arrayrulewidth}}
\toprule
{\bfseries Classificação} & {\bfseries f} & {\bfseries frp} & {\bfseries F} & {\bfseries Frp} \\ 
\midrule\addlinespace[2.5pt]
BP Extremo\textsuperscript{\textit{1}} & 2 & 1.2 & 2 & 1.2 \\ 
Muito BP\textsuperscript{\textit{1}} & 5 & 2.9 & 7 & 4.1 \\ 
Baixo Peso (BP) & 0 & 0.0 & 7 & 4.1 \\ 
Peso Normal & 163 & 95.9 & 170 & 100.0 \\ 
Excesso Peso & 0 & 0.0 & 170 & 100.0 \\ 
\bottomrule
\end{tabular*}
\begin{minipage}{\linewidth}
\textsuperscript{\textit{1}}BP = Baixo Peso\\
Fonte: Oliveira Filho, PF (2025)\\
\end{minipage}

}

\end{table}%

\textbf{Passo 3}: Explicação do código

\begin{enumerate}
\def\labelenumi{\arabic{enumi}.}
\item
  \ul{Estrutura geral}: inicialmente foi construída uma tabela
  \texttt{gt} com os dados da \texttt{tab\_completa}. A partir daí foi
  encadeado, através do operador \texttt{pipe}, várias funções.
\item
  \ul{Cabeçalho da tabela}: a função \texttt{tab\_header()} adiciona um
  título e subtítulo à tabela.

  \begin{itemize}
  \tightlist
  \item
    Aplicado estílo ao título principal com a função
    \texttt{tab\_style()} - negrito (\texttt{bold}), tamanho médio,
    fonte arial e cor cinza escura (\texttt{gray18})
  \end{itemize}
\item
  \ul{Estilo dos rótulos das colunas}: deixa os nomes das colunas em
  negrito, dando maior destaque.
\item
  \ul{Destaque condicional no corpo da tabela}: aplica cor vermelha à
  células da coluna \texttt{Frp} (frequência relativa acumulada em
  porcentagem) somente onde o valor está próximo de 15. Isto indica que
  15\% dos recém-nascidos têm peso ao nascer igual ou abaixo deste
  valor. A função \texttt{near()} é útil para destacar valores com
  tolerância numérica.
\item
  \ul{Largura das colunas}: define a largura das colunas, onde a
  primeira coluna tem 150 px e as demais (2 a 5) 80 px (pixels). Podem
  ser usadas duas unidades de medidas: pixels ou pct (percentual de
  largura da tabela). A conversão de pixels (px) para centímetros (cm)
  depende da resolução da tela medida em pixels por polegada (PPI). O
  padrão mais comum é 96 PPI.

  \[
  cm = (px \ \times 2.54)/PPI
  \]

  Com base em 96 PPI, \(150 px \approx 4\) cm.
\item
  Por úktimo se incluiu a fonte e uma nota explicativa.
\end{enumerate}

\subsection{Pacote flextable}\label{pacote-flextable}

O pacote \texttt{flextable} fornece uma estrutura para criar facilmente
tabelas elegantes e personalizadas para relatórios e publicações. (79).

As finalidades básicas do pacote \texttt{flextable} são:

\begin{itemize}
\item
  Constrói tabelas a partir de \textbf{dataframes} ou \textbf{tibbles}.
\item
  Permite \textbf{formatar títulos, subtítulos, rodapés, cabeçalhos e
  corpo} da tabela.
\item
  Dá suporte a estilos avançados:

  \begin{itemize}
  \item
    cores de células e bordas;
  \item
    alinhamento (horizontal e vertical);
  \item
    negrito/itálico;
  \item
    largura das colunas e altura das linhas;
  \item
    mesclagem de células;
  \item
    numeração e percentuais formatados.
  \end{itemize}
\item
  Gera tabelas que podem ser \textbf{dinâmicas} (em R Markdown, Quarto).
\end{itemize}

\subsubsection{Passos para criar uma tabela com o pacote
flextable}\label{passos-para-criar-uma-tabela-com-o-pacote-flextable}

\textbf{Passo 1}: Da mesma forma como nas tabelas criadas pelo pacote
\texttt{gt}, o \texttt{flextable} também necessita de tabela base
(dataframe ou tibble) (80). Será usada a mesma tabela
(\texttt{tab\_completa}). Uma tabela simples (tbl-ft1) pode ser
facilmente gerada:

\begin{Shaded}
\begin{Highlighting}[]
\FunctionTok{library}\NormalTok{(flextable)}
\FunctionTok{library}\NormalTok{(dplyr)}

\NormalTok{ft }\OtherTok{\textless{}{-}} \FunctionTok{flextable}\NormalTok{(}\AttributeTok{data =}\NormalTok{ tab\_completa)}
\NormalTok{ft}
\end{Highlighting}
\end{Shaded}

\global\setlength{\Oldarrayrulewidth}{\arrayrulewidth}

\global\setlength{\Oldtabcolsep}{\tabcolsep}

\setlength{\tabcolsep}{2pt}

\renewcommand*{\arraystretch}{1.5}



\providecommand{\ascline}[3]{\noalign{\global\arrayrulewidth #1}\arrayrulecolor[HTML]{#2}\cline{#3}}

\begin{longtable}[c]{|p{0.75in}|p{0.75in}|p{0.75in}|p{0.75in}|p{0.75in}}

\caption{\label{tbl-ft1}Tabela simples com o pacote flextable}

\tabularnewline

\ascline{1.5pt}{666666}{1-5}

\multicolumn{1}{>{\raggedright}m{\dimexpr 0.75in+0\tabcolsep}}{\textcolor[HTML]{000000}{\fontsize{11}{11}\selectfont{\global\setmainfont{Arial}{Classificação}}}} & \multicolumn{1}{>{\raggedleft}m{\dimexpr 0.75in+0\tabcolsep}}{\textcolor[HTML]{000000}{\fontsize{11}{11}\selectfont{\global\setmainfont{Arial}{f}}}} & \multicolumn{1}{>{\raggedleft}m{\dimexpr 0.75in+0\tabcolsep}}{\textcolor[HTML]{000000}{\fontsize{11}{11}\selectfont{\global\setmainfont{Arial}{frp}}}} & \multicolumn{1}{>{\raggedleft}m{\dimexpr 0.75in+0\tabcolsep}}{\textcolor[HTML]{000000}{\fontsize{11}{11}\selectfont{\global\setmainfont{Arial}{F}}}} & \multicolumn{1}{>{\raggedleft}m{\dimexpr 0.75in+0\tabcolsep}}{\textcolor[HTML]{000000}{\fontsize{11}{11}\selectfont{\global\setmainfont{Arial}{Frp}}}} \\

\ascline{1.5pt}{666666}{1-5}\endfirsthead 

\ascline{1.5pt}{666666}{1-5}

\multicolumn{1}{>{\raggedright}m{\dimexpr 0.75in+0\tabcolsep}}{\textcolor[HTML]{000000}{\fontsize{11}{11}\selectfont{\global\setmainfont{Arial}{Classificação}}}} & \multicolumn{1}{>{\raggedleft}m{\dimexpr 0.75in+0\tabcolsep}}{\textcolor[HTML]{000000}{\fontsize{11}{11}\selectfont{\global\setmainfont{Arial}{f}}}} & \multicolumn{1}{>{\raggedleft}m{\dimexpr 0.75in+0\tabcolsep}}{\textcolor[HTML]{000000}{\fontsize{11}{11}\selectfont{\global\setmainfont{Arial}{frp}}}} & \multicolumn{1}{>{\raggedleft}m{\dimexpr 0.75in+0\tabcolsep}}{\textcolor[HTML]{000000}{\fontsize{11}{11}\selectfont{\global\setmainfont{Arial}{F}}}} & \multicolumn{1}{>{\raggedleft}m{\dimexpr 0.75in+0\tabcolsep}}{\textcolor[HTML]{000000}{\fontsize{11}{11}\selectfont{\global\setmainfont{Arial}{Frp}}}} \\

\ascline{1.5pt}{666666}{1-5}\endhead



\multicolumn{1}{>{\raggedright}m{\dimexpr 0.75in+0\tabcolsep}}{\textcolor[HTML]{000000}{\fontsize{11}{11}\selectfont{\global\setmainfont{Arial}{BP\ Extremo}}}} & \multicolumn{1}{>{\raggedleft}m{\dimexpr 0.75in+0\tabcolsep}}{\textcolor[HTML]{000000}{\fontsize{11}{11}\selectfont{\global\setmainfont{Arial}{2}}}} & \multicolumn{1}{>{\raggedleft}m{\dimexpr 0.75in+0\tabcolsep}}{\textcolor[HTML]{000000}{\fontsize{11}{11}\selectfont{\global\setmainfont{Arial}{1.2}}}} & \multicolumn{1}{>{\raggedleft}m{\dimexpr 0.75in+0\tabcolsep}}{\textcolor[HTML]{000000}{\fontsize{11}{11}\selectfont{\global\setmainfont{Arial}{2}}}} & \multicolumn{1}{>{\raggedleft}m{\dimexpr 0.75in+0\tabcolsep}}{\textcolor[HTML]{000000}{\fontsize{11}{11}\selectfont{\global\setmainfont{Arial}{1.2}}}} \\





\multicolumn{1}{>{\raggedright}m{\dimexpr 0.75in+0\tabcolsep}}{\textcolor[HTML]{000000}{\fontsize{11}{11}\selectfont{\global\setmainfont{Arial}{Muito\ BP}}}} & \multicolumn{1}{>{\raggedleft}m{\dimexpr 0.75in+0\tabcolsep}}{\textcolor[HTML]{000000}{\fontsize{11}{11}\selectfont{\global\setmainfont{Arial}{5}}}} & \multicolumn{1}{>{\raggedleft}m{\dimexpr 0.75in+0\tabcolsep}}{\textcolor[HTML]{000000}{\fontsize{11}{11}\selectfont{\global\setmainfont{Arial}{2.9}}}} & \multicolumn{1}{>{\raggedleft}m{\dimexpr 0.75in+0\tabcolsep}}{\textcolor[HTML]{000000}{\fontsize{11}{11}\selectfont{\global\setmainfont{Arial}{7}}}} & \multicolumn{1}{>{\raggedleft}m{\dimexpr 0.75in+0\tabcolsep}}{\textcolor[HTML]{000000}{\fontsize{11}{11}\selectfont{\global\setmainfont{Arial}{4.1}}}} \\





\multicolumn{1}{>{\raggedright}m{\dimexpr 0.75in+0\tabcolsep}}{\textcolor[HTML]{000000}{\fontsize{11}{11}\selectfont{\global\setmainfont{Arial}{Baixo\ Peso\ (BP)}}}} & \multicolumn{1}{>{\raggedleft}m{\dimexpr 0.75in+0\tabcolsep}}{\textcolor[HTML]{000000}{\fontsize{11}{11}\selectfont{\global\setmainfont{Arial}{0}}}} & \multicolumn{1}{>{\raggedleft}m{\dimexpr 0.75in+0\tabcolsep}}{\textcolor[HTML]{000000}{\fontsize{11}{11}\selectfont{\global\setmainfont{Arial}{0.0}}}} & \multicolumn{1}{>{\raggedleft}m{\dimexpr 0.75in+0\tabcolsep}}{\textcolor[HTML]{000000}{\fontsize{11}{11}\selectfont{\global\setmainfont{Arial}{7}}}} & \multicolumn{1}{>{\raggedleft}m{\dimexpr 0.75in+0\tabcolsep}}{\textcolor[HTML]{000000}{\fontsize{11}{11}\selectfont{\global\setmainfont{Arial}{4.1}}}} \\





\multicolumn{1}{>{\raggedright}m{\dimexpr 0.75in+0\tabcolsep}}{\textcolor[HTML]{000000}{\fontsize{11}{11}\selectfont{\global\setmainfont{Arial}{Peso\ Normal}}}} & \multicolumn{1}{>{\raggedleft}m{\dimexpr 0.75in+0\tabcolsep}}{\textcolor[HTML]{000000}{\fontsize{11}{11}\selectfont{\global\setmainfont{Arial}{163}}}} & \multicolumn{1}{>{\raggedleft}m{\dimexpr 0.75in+0\tabcolsep}}{\textcolor[HTML]{000000}{\fontsize{11}{11}\selectfont{\global\setmainfont{Arial}{95.9}}}} & \multicolumn{1}{>{\raggedleft}m{\dimexpr 0.75in+0\tabcolsep}}{\textcolor[HTML]{000000}{\fontsize{11}{11}\selectfont{\global\setmainfont{Arial}{170}}}} & \multicolumn{1}{>{\raggedleft}m{\dimexpr 0.75in+0\tabcolsep}}{\textcolor[HTML]{000000}{\fontsize{11}{11}\selectfont{\global\setmainfont{Arial}{100.0}}}} \\





\multicolumn{1}{>{\raggedright}m{\dimexpr 0.75in+0\tabcolsep}}{\textcolor[HTML]{000000}{\fontsize{11}{11}\selectfont{\global\setmainfont{Arial}{Excesso\ Peso}}}} & \multicolumn{1}{>{\raggedleft}m{\dimexpr 0.75in+0\tabcolsep}}{\textcolor[HTML]{000000}{\fontsize{11}{11}\selectfont{\global\setmainfont{Arial}{0}}}} & \multicolumn{1}{>{\raggedleft}m{\dimexpr 0.75in+0\tabcolsep}}{\textcolor[HTML]{000000}{\fontsize{11}{11}\selectfont{\global\setmainfont{Arial}{0.0}}}} & \multicolumn{1}{>{\raggedleft}m{\dimexpr 0.75in+0\tabcolsep}}{\textcolor[HTML]{000000}{\fontsize{11}{11}\selectfont{\global\setmainfont{Arial}{170}}}} & \multicolumn{1}{>{\raggedleft}m{\dimexpr 0.75in+0\tabcolsep}}{\textcolor[HTML]{000000}{\fontsize{11}{11}\selectfont{\global\setmainfont{Arial}{100.0}}}} \\

\ascline{1.5pt}{666666}{1-5}


\end{longtable}

\arrayrulecolor[HTML]{000000}

\global\setlength{\arrayrulewidth}{\Oldarrayrulewidth}

\global\setlength{\tabcolsep}{\Oldtabcolsep}

\renewcommand*{\arraystretch}{1}

\textbf{Passo 2}: As colunas podem ter sua largura ajustadas de forma
manual, passando para o argumento \texttt{width()} um vetor com as
larguras em polegadas.

\begin{Shaded}
\begin{Highlighting}[]
\NormalTok{ft }\OtherTok{\textless{}{-}}\NormalTok{ ft }\SpecialCharTok{\%\textgreater{}\%} \FunctionTok{width}\NormalTok{(}\AttributeTok{width =} \FunctionTok{c}\NormalTok{(}\DecValTok{2}\NormalTok{, }\FloatTok{1.5}\NormalTok{, }\FloatTok{1.5}\NormalTok{, }\FloatTok{1.5}\NormalTok{, }\FloatTok{1.5}\NormalTok{))}
\NormalTok{ft}
\end{Highlighting}
\end{Shaded}

\global\setlength{\Oldarrayrulewidth}{\arrayrulewidth}

\global\setlength{\Oldtabcolsep}{\tabcolsep}

\setlength{\tabcolsep}{2pt}

\renewcommand*{\arraystretch}{1.5}



\providecommand{\ascline}[3]{\noalign{\global\arrayrulewidth #1}\arrayrulecolor[HTML]{#2}\cline{#3}}

\begin{longtable}[c]{|p{2.00in}|p{1.50in}|p{1.50in}|p{1.50in}|p{1.50in}}

\caption{\label{tbl-ft2}Ajuste das colunas}

\tabularnewline

\ascline{1.5pt}{666666}{1-5}

\multicolumn{1}{>{\raggedright}m{\dimexpr 2in+0\tabcolsep}}{\textcolor[HTML]{000000}{\fontsize{11}{11}\selectfont{\global\setmainfont{Arial}{Classificação}}}} & \multicolumn{1}{>{\raggedleft}m{\dimexpr 1.5in+0\tabcolsep}}{\textcolor[HTML]{000000}{\fontsize{11}{11}\selectfont{\global\setmainfont{Arial}{f}}}} & \multicolumn{1}{>{\raggedleft}m{\dimexpr 1.5in+0\tabcolsep}}{\textcolor[HTML]{000000}{\fontsize{11}{11}\selectfont{\global\setmainfont{Arial}{frp}}}} & \multicolumn{1}{>{\raggedleft}m{\dimexpr 1.5in+0\tabcolsep}}{\textcolor[HTML]{000000}{\fontsize{11}{11}\selectfont{\global\setmainfont{Arial}{F}}}} & \multicolumn{1}{>{\raggedleft}m{\dimexpr 1.5in+0\tabcolsep}}{\textcolor[HTML]{000000}{\fontsize{11}{11}\selectfont{\global\setmainfont{Arial}{Frp}}}} \\

\ascline{1.5pt}{666666}{1-5}\endfirsthead 

\ascline{1.5pt}{666666}{1-5}

\multicolumn{1}{>{\raggedright}m{\dimexpr 2in+0\tabcolsep}}{\textcolor[HTML]{000000}{\fontsize{11}{11}\selectfont{\global\setmainfont{Arial}{Classificação}}}} & \multicolumn{1}{>{\raggedleft}m{\dimexpr 1.5in+0\tabcolsep}}{\textcolor[HTML]{000000}{\fontsize{11}{11}\selectfont{\global\setmainfont{Arial}{f}}}} & \multicolumn{1}{>{\raggedleft}m{\dimexpr 1.5in+0\tabcolsep}}{\textcolor[HTML]{000000}{\fontsize{11}{11}\selectfont{\global\setmainfont{Arial}{frp}}}} & \multicolumn{1}{>{\raggedleft}m{\dimexpr 1.5in+0\tabcolsep}}{\textcolor[HTML]{000000}{\fontsize{11}{11}\selectfont{\global\setmainfont{Arial}{F}}}} & \multicolumn{1}{>{\raggedleft}m{\dimexpr 1.5in+0\tabcolsep}}{\textcolor[HTML]{000000}{\fontsize{11}{11}\selectfont{\global\setmainfont{Arial}{Frp}}}} \\

\ascline{1.5pt}{666666}{1-5}\endhead



\multicolumn{1}{>{\raggedright}m{\dimexpr 2in+0\tabcolsep}}{\textcolor[HTML]{000000}{\fontsize{11}{11}\selectfont{\global\setmainfont{Arial}{BP\ Extremo}}}} & \multicolumn{1}{>{\raggedleft}m{\dimexpr 1.5in+0\tabcolsep}}{\textcolor[HTML]{000000}{\fontsize{11}{11}\selectfont{\global\setmainfont{Arial}{2}}}} & \multicolumn{1}{>{\raggedleft}m{\dimexpr 1.5in+0\tabcolsep}}{\textcolor[HTML]{000000}{\fontsize{11}{11}\selectfont{\global\setmainfont{Arial}{1.2}}}} & \multicolumn{1}{>{\raggedleft}m{\dimexpr 1.5in+0\tabcolsep}}{\textcolor[HTML]{000000}{\fontsize{11}{11}\selectfont{\global\setmainfont{Arial}{2}}}} & \multicolumn{1}{>{\raggedleft}m{\dimexpr 1.5in+0\tabcolsep}}{\textcolor[HTML]{000000}{\fontsize{11}{11}\selectfont{\global\setmainfont{Arial}{1.2}}}} \\





\multicolumn{1}{>{\raggedright}m{\dimexpr 2in+0\tabcolsep}}{\textcolor[HTML]{000000}{\fontsize{11}{11}\selectfont{\global\setmainfont{Arial}{Muito\ BP}}}} & \multicolumn{1}{>{\raggedleft}m{\dimexpr 1.5in+0\tabcolsep}}{\textcolor[HTML]{000000}{\fontsize{11}{11}\selectfont{\global\setmainfont{Arial}{5}}}} & \multicolumn{1}{>{\raggedleft}m{\dimexpr 1.5in+0\tabcolsep}}{\textcolor[HTML]{000000}{\fontsize{11}{11}\selectfont{\global\setmainfont{Arial}{2.9}}}} & \multicolumn{1}{>{\raggedleft}m{\dimexpr 1.5in+0\tabcolsep}}{\textcolor[HTML]{000000}{\fontsize{11}{11}\selectfont{\global\setmainfont{Arial}{7}}}} & \multicolumn{1}{>{\raggedleft}m{\dimexpr 1.5in+0\tabcolsep}}{\textcolor[HTML]{000000}{\fontsize{11}{11}\selectfont{\global\setmainfont{Arial}{4.1}}}} \\





\multicolumn{1}{>{\raggedright}m{\dimexpr 2in+0\tabcolsep}}{\textcolor[HTML]{000000}{\fontsize{11}{11}\selectfont{\global\setmainfont{Arial}{Baixo\ Peso\ (BP)}}}} & \multicolumn{1}{>{\raggedleft}m{\dimexpr 1.5in+0\tabcolsep}}{\textcolor[HTML]{000000}{\fontsize{11}{11}\selectfont{\global\setmainfont{Arial}{0}}}} & \multicolumn{1}{>{\raggedleft}m{\dimexpr 1.5in+0\tabcolsep}}{\textcolor[HTML]{000000}{\fontsize{11}{11}\selectfont{\global\setmainfont{Arial}{0.0}}}} & \multicolumn{1}{>{\raggedleft}m{\dimexpr 1.5in+0\tabcolsep}}{\textcolor[HTML]{000000}{\fontsize{11}{11}\selectfont{\global\setmainfont{Arial}{7}}}} & \multicolumn{1}{>{\raggedleft}m{\dimexpr 1.5in+0\tabcolsep}}{\textcolor[HTML]{000000}{\fontsize{11}{11}\selectfont{\global\setmainfont{Arial}{4.1}}}} \\





\multicolumn{1}{>{\raggedright}m{\dimexpr 2in+0\tabcolsep}}{\textcolor[HTML]{000000}{\fontsize{11}{11}\selectfont{\global\setmainfont{Arial}{Peso\ Normal}}}} & \multicolumn{1}{>{\raggedleft}m{\dimexpr 1.5in+0\tabcolsep}}{\textcolor[HTML]{000000}{\fontsize{11}{11}\selectfont{\global\setmainfont{Arial}{163}}}} & \multicolumn{1}{>{\raggedleft}m{\dimexpr 1.5in+0\tabcolsep}}{\textcolor[HTML]{000000}{\fontsize{11}{11}\selectfont{\global\setmainfont{Arial}{95.9}}}} & \multicolumn{1}{>{\raggedleft}m{\dimexpr 1.5in+0\tabcolsep}}{\textcolor[HTML]{000000}{\fontsize{11}{11}\selectfont{\global\setmainfont{Arial}{170}}}} & \multicolumn{1}{>{\raggedleft}m{\dimexpr 1.5in+0\tabcolsep}}{\textcolor[HTML]{000000}{\fontsize{11}{11}\selectfont{\global\setmainfont{Arial}{100.0}}}} \\





\multicolumn{1}{>{\raggedright}m{\dimexpr 2in+0\tabcolsep}}{\textcolor[HTML]{000000}{\fontsize{11}{11}\selectfont{\global\setmainfont{Arial}{Excesso\ Peso}}}} & \multicolumn{1}{>{\raggedleft}m{\dimexpr 1.5in+0\tabcolsep}}{\textcolor[HTML]{000000}{\fontsize{11}{11}\selectfont{\global\setmainfont{Arial}{0}}}} & \multicolumn{1}{>{\raggedleft}m{\dimexpr 1.5in+0\tabcolsep}}{\textcolor[HTML]{000000}{\fontsize{11}{11}\selectfont{\global\setmainfont{Arial}{0.0}}}} & \multicolumn{1}{>{\raggedleft}m{\dimexpr 1.5in+0\tabcolsep}}{\textcolor[HTML]{000000}{\fontsize{11}{11}\selectfont{\global\setmainfont{Arial}{170}}}} & \multicolumn{1}{>{\raggedleft}m{\dimexpr 1.5in+0\tabcolsep}}{\textcolor[HTML]{000000}{\fontsize{11}{11}\selectfont{\global\setmainfont{Arial}{100.0}}}} \\

\ascline{1.5pt}{666666}{1-5}


\end{longtable}

\arrayrulecolor[HTML]{000000}

\global\setlength{\arrayrulewidth}{\Oldarrayrulewidth}

\global\setlength{\tabcolsep}{\Oldtabcolsep}

\renewcommand*{\arraystretch}{1}

O controle manual pode ser útil e fácil. A Tabela~\ref{tbl-ft2} melhorou
o aspecto da Tabela~\ref{tbl-ft1}, mas as colunas ficaram muito largas.
O processo de tentativa e erro para encontrar as larguras ideais
torna-se tedioso, irritante. Felizmente, o \texttt{flextable} permite
que se ignore esse procedimento com a função \texttt{autofit()}, que
tenta selecionar larguras de coluna adequadas automaticamente.

\begin{Shaded}
\begin{Highlighting}[]
\NormalTok{ft }\OtherTok{\textless{}{-}}\NormalTok{ ft }\SpecialCharTok{\%\textgreater{}\%} \FunctionTok{autofit}\NormalTok{()}
\NormalTok{ft}
\end{Highlighting}
\end{Shaded}

\global\setlength{\Oldarrayrulewidth}{\arrayrulewidth}

\global\setlength{\Oldtabcolsep}{\tabcolsep}

\setlength{\tabcolsep}{2pt}

\renewcommand*{\arraystretch}{1.5}



\providecommand{\ascline}[3]{\noalign{\global\arrayrulewidth #1}\arrayrulecolor[HTML]{#2}\cline{#3}}

\begin{longtable}[c]{|p{1.41in}|p{0.54in}|p{0.58in}|p{0.54in}|p{0.67in}}

\caption{\label{tbl-ft3}Ajuste automático das colunas}

\tabularnewline

\ascline{1.5pt}{666666}{1-5}

\multicolumn{1}{>{\raggedright}m{\dimexpr 1.41in+0\tabcolsep}}{\textcolor[HTML]{000000}{\fontsize{11}{11}\selectfont{\global\setmainfont{Arial}{Classificação}}}} & \multicolumn{1}{>{\raggedleft}m{\dimexpr 0.54in+0\tabcolsep}}{\textcolor[HTML]{000000}{\fontsize{11}{11}\selectfont{\global\setmainfont{Arial}{f}}}} & \multicolumn{1}{>{\raggedleft}m{\dimexpr 0.58in+0\tabcolsep}}{\textcolor[HTML]{000000}{\fontsize{11}{11}\selectfont{\global\setmainfont{Arial}{frp}}}} & \multicolumn{1}{>{\raggedleft}m{\dimexpr 0.54in+0\tabcolsep}}{\textcolor[HTML]{000000}{\fontsize{11}{11}\selectfont{\global\setmainfont{Arial}{F}}}} & \multicolumn{1}{>{\raggedleft}m{\dimexpr 0.67in+0\tabcolsep}}{\textcolor[HTML]{000000}{\fontsize{11}{11}\selectfont{\global\setmainfont{Arial}{Frp}}}} \\

\ascline{1.5pt}{666666}{1-5}\endfirsthead 

\ascline{1.5pt}{666666}{1-5}

\multicolumn{1}{>{\raggedright}m{\dimexpr 1.41in+0\tabcolsep}}{\textcolor[HTML]{000000}{\fontsize{11}{11}\selectfont{\global\setmainfont{Arial}{Classificação}}}} & \multicolumn{1}{>{\raggedleft}m{\dimexpr 0.54in+0\tabcolsep}}{\textcolor[HTML]{000000}{\fontsize{11}{11}\selectfont{\global\setmainfont{Arial}{f}}}} & \multicolumn{1}{>{\raggedleft}m{\dimexpr 0.58in+0\tabcolsep}}{\textcolor[HTML]{000000}{\fontsize{11}{11}\selectfont{\global\setmainfont{Arial}{frp}}}} & \multicolumn{1}{>{\raggedleft}m{\dimexpr 0.54in+0\tabcolsep}}{\textcolor[HTML]{000000}{\fontsize{11}{11}\selectfont{\global\setmainfont{Arial}{F}}}} & \multicolumn{1}{>{\raggedleft}m{\dimexpr 0.67in+0\tabcolsep}}{\textcolor[HTML]{000000}{\fontsize{11}{11}\selectfont{\global\setmainfont{Arial}{Frp}}}} \\

\ascline{1.5pt}{666666}{1-5}\endhead



\multicolumn{1}{>{\raggedright}m{\dimexpr 1.41in+0\tabcolsep}}{\textcolor[HTML]{000000}{\fontsize{11}{11}\selectfont{\global\setmainfont{Arial}{BP\ Extremo}}}} & \multicolumn{1}{>{\raggedleft}m{\dimexpr 0.54in+0\tabcolsep}}{\textcolor[HTML]{000000}{\fontsize{11}{11}\selectfont{\global\setmainfont{Arial}{2}}}} & \multicolumn{1}{>{\raggedleft}m{\dimexpr 0.58in+0\tabcolsep}}{\textcolor[HTML]{000000}{\fontsize{11}{11}\selectfont{\global\setmainfont{Arial}{1.2}}}} & \multicolumn{1}{>{\raggedleft}m{\dimexpr 0.54in+0\tabcolsep}}{\textcolor[HTML]{000000}{\fontsize{11}{11}\selectfont{\global\setmainfont{Arial}{2}}}} & \multicolumn{1}{>{\raggedleft}m{\dimexpr 0.67in+0\tabcolsep}}{\textcolor[HTML]{000000}{\fontsize{11}{11}\selectfont{\global\setmainfont{Arial}{1.2}}}} \\





\multicolumn{1}{>{\raggedright}m{\dimexpr 1.41in+0\tabcolsep}}{\textcolor[HTML]{000000}{\fontsize{11}{11}\selectfont{\global\setmainfont{Arial}{Muito\ BP}}}} & \multicolumn{1}{>{\raggedleft}m{\dimexpr 0.54in+0\tabcolsep}}{\textcolor[HTML]{000000}{\fontsize{11}{11}\selectfont{\global\setmainfont{Arial}{5}}}} & \multicolumn{1}{>{\raggedleft}m{\dimexpr 0.58in+0\tabcolsep}}{\textcolor[HTML]{000000}{\fontsize{11}{11}\selectfont{\global\setmainfont{Arial}{2.9}}}} & \multicolumn{1}{>{\raggedleft}m{\dimexpr 0.54in+0\tabcolsep}}{\textcolor[HTML]{000000}{\fontsize{11}{11}\selectfont{\global\setmainfont{Arial}{7}}}} & \multicolumn{1}{>{\raggedleft}m{\dimexpr 0.67in+0\tabcolsep}}{\textcolor[HTML]{000000}{\fontsize{11}{11}\selectfont{\global\setmainfont{Arial}{4.1}}}} \\





\multicolumn{1}{>{\raggedright}m{\dimexpr 1.41in+0\tabcolsep}}{\textcolor[HTML]{000000}{\fontsize{11}{11}\selectfont{\global\setmainfont{Arial}{Baixo\ Peso\ (BP)}}}} & \multicolumn{1}{>{\raggedleft}m{\dimexpr 0.54in+0\tabcolsep}}{\textcolor[HTML]{000000}{\fontsize{11}{11}\selectfont{\global\setmainfont{Arial}{0}}}} & \multicolumn{1}{>{\raggedleft}m{\dimexpr 0.58in+0\tabcolsep}}{\textcolor[HTML]{000000}{\fontsize{11}{11}\selectfont{\global\setmainfont{Arial}{0.0}}}} & \multicolumn{1}{>{\raggedleft}m{\dimexpr 0.54in+0\tabcolsep}}{\textcolor[HTML]{000000}{\fontsize{11}{11}\selectfont{\global\setmainfont{Arial}{7}}}} & \multicolumn{1}{>{\raggedleft}m{\dimexpr 0.67in+0\tabcolsep}}{\textcolor[HTML]{000000}{\fontsize{11}{11}\selectfont{\global\setmainfont{Arial}{4.1}}}} \\





\multicolumn{1}{>{\raggedright}m{\dimexpr 1.41in+0\tabcolsep}}{\textcolor[HTML]{000000}{\fontsize{11}{11}\selectfont{\global\setmainfont{Arial}{Peso\ Normal}}}} & \multicolumn{1}{>{\raggedleft}m{\dimexpr 0.54in+0\tabcolsep}}{\textcolor[HTML]{000000}{\fontsize{11}{11}\selectfont{\global\setmainfont{Arial}{163}}}} & \multicolumn{1}{>{\raggedleft}m{\dimexpr 0.58in+0\tabcolsep}}{\textcolor[HTML]{000000}{\fontsize{11}{11}\selectfont{\global\setmainfont{Arial}{95.9}}}} & \multicolumn{1}{>{\raggedleft}m{\dimexpr 0.54in+0\tabcolsep}}{\textcolor[HTML]{000000}{\fontsize{11}{11}\selectfont{\global\setmainfont{Arial}{170}}}} & \multicolumn{1}{>{\raggedleft}m{\dimexpr 0.67in+0\tabcolsep}}{\textcolor[HTML]{000000}{\fontsize{11}{11}\selectfont{\global\setmainfont{Arial}{100.0}}}} \\





\multicolumn{1}{>{\raggedright}m{\dimexpr 1.41in+0\tabcolsep}}{\textcolor[HTML]{000000}{\fontsize{11}{11}\selectfont{\global\setmainfont{Arial}{Excesso\ Peso}}}} & \multicolumn{1}{>{\raggedleft}m{\dimexpr 0.54in+0\tabcolsep}}{\textcolor[HTML]{000000}{\fontsize{11}{11}\selectfont{\global\setmainfont{Arial}{0}}}} & \multicolumn{1}{>{\raggedleft}m{\dimexpr 0.58in+0\tabcolsep}}{\textcolor[HTML]{000000}{\fontsize{11}{11}\selectfont{\global\setmainfont{Arial}{0.0}}}} & \multicolumn{1}{>{\raggedleft}m{\dimexpr 0.54in+0\tabcolsep}}{\textcolor[HTML]{000000}{\fontsize{11}{11}\selectfont{\global\setmainfont{Arial}{170}}}} & \multicolumn{1}{>{\raggedleft}m{\dimexpr 0.67in+0\tabcolsep}}{\textcolor[HTML]{000000}{\fontsize{11}{11}\selectfont{\global\setmainfont{Arial}{100.0}}}} \\

\ascline{1.5pt}{666666}{1-5}


\end{longtable}

\arrayrulecolor[HTML]{000000}

\global\setlength{\arrayrulewidth}{\Oldarrayrulewidth}

\global\setlength{\tabcolsep}{\Oldtabcolsep}

\renewcommand*{\arraystretch}{1}

Agora, a Tabela~\ref{tbl-ft3} já apresenta um layout bem mais bonito e ,
dependendo, do contexto, quase pronta para publicação.

\textbf{Passo 3}: Ajuste do cabeçalho e rótulos das colunas. Dependendo
da necessidade os rótulos do cabeçalho podem ser facilmente modificados
com a função \texttt{set\_header\_labels\ ()}:

\begin{Shaded}
\begin{Highlighting}[]
\NormalTok{ft }\OtherTok{\textless{}{-}}\NormalTok{ ft }\SpecialCharTok{\%\textgreater{}\%} 
  \FunctionTok{autofit}\NormalTok{() }\SpecialCharTok{\%\textgreater{}\%} 
  \FunctionTok{set\_header\_labels}\NormalTok{(}
    \AttributeTok{frp =} \StringTok{"fr (\%)"}\NormalTok{,}
    \AttributeTok{Frp =} \StringTok{"Fr (\%)"}
\NormalTok{  )}
\NormalTok{ft}
\end{Highlighting}
\end{Shaded}

\global\setlength{\Oldarrayrulewidth}{\arrayrulewidth}

\global\setlength{\Oldtabcolsep}{\tabcolsep}

\setlength{\tabcolsep}{2pt}

\renewcommand*{\arraystretch}{1.5}



\providecommand{\ascline}[3]{\noalign{\global\arrayrulewidth #1}\arrayrulecolor[HTML]{#2}\cline{#3}}

\begin{longtable}[c]{|p{1.41in}|p{0.54in}|p{0.58in}|p{0.54in}|p{0.67in}}

\caption{\label{tbl-ft4}Ajuste do cabeçalho}

\tabularnewline

\ascline{1.5pt}{666666}{1-5}

\multicolumn{1}{>{\raggedright}m{\dimexpr 1.41in+0\tabcolsep}}{\textcolor[HTML]{000000}{\fontsize{11}{11}\selectfont{\global\setmainfont{Arial}{Classificação}}}} & \multicolumn{1}{>{\raggedleft}m{\dimexpr 0.54in+0\tabcolsep}}{\textcolor[HTML]{000000}{\fontsize{11}{11}\selectfont{\global\setmainfont{Arial}{f}}}} & \multicolumn{1}{>{\raggedleft}m{\dimexpr 0.58in+0\tabcolsep}}{\textcolor[HTML]{000000}{\fontsize{11}{11}\selectfont{\global\setmainfont{Arial}{fr\ (\%)}}}} & \multicolumn{1}{>{\raggedleft}m{\dimexpr 0.54in+0\tabcolsep}}{\textcolor[HTML]{000000}{\fontsize{11}{11}\selectfont{\global\setmainfont{Arial}{F}}}} & \multicolumn{1}{>{\raggedleft}m{\dimexpr 0.67in+0\tabcolsep}}{\textcolor[HTML]{000000}{\fontsize{11}{11}\selectfont{\global\setmainfont{Arial}{Fr\ (\%)}}}} \\

\ascline{1.5pt}{666666}{1-5}\endfirsthead 

\ascline{1.5pt}{666666}{1-5}

\multicolumn{1}{>{\raggedright}m{\dimexpr 1.41in+0\tabcolsep}}{\textcolor[HTML]{000000}{\fontsize{11}{11}\selectfont{\global\setmainfont{Arial}{Classificação}}}} & \multicolumn{1}{>{\raggedleft}m{\dimexpr 0.54in+0\tabcolsep}}{\textcolor[HTML]{000000}{\fontsize{11}{11}\selectfont{\global\setmainfont{Arial}{f}}}} & \multicolumn{1}{>{\raggedleft}m{\dimexpr 0.58in+0\tabcolsep}}{\textcolor[HTML]{000000}{\fontsize{11}{11}\selectfont{\global\setmainfont{Arial}{fr\ (\%)}}}} & \multicolumn{1}{>{\raggedleft}m{\dimexpr 0.54in+0\tabcolsep}}{\textcolor[HTML]{000000}{\fontsize{11}{11}\selectfont{\global\setmainfont{Arial}{F}}}} & \multicolumn{1}{>{\raggedleft}m{\dimexpr 0.67in+0\tabcolsep}}{\textcolor[HTML]{000000}{\fontsize{11}{11}\selectfont{\global\setmainfont{Arial}{Fr\ (\%)}}}} \\

\ascline{1.5pt}{666666}{1-5}\endhead



\multicolumn{1}{>{\raggedright}m{\dimexpr 1.41in+0\tabcolsep}}{\textcolor[HTML]{000000}{\fontsize{11}{11}\selectfont{\global\setmainfont{Arial}{BP\ Extremo}}}} & \multicolumn{1}{>{\raggedleft}m{\dimexpr 0.54in+0\tabcolsep}}{\textcolor[HTML]{000000}{\fontsize{11}{11}\selectfont{\global\setmainfont{Arial}{2}}}} & \multicolumn{1}{>{\raggedleft}m{\dimexpr 0.58in+0\tabcolsep}}{\textcolor[HTML]{000000}{\fontsize{11}{11}\selectfont{\global\setmainfont{Arial}{1.2}}}} & \multicolumn{1}{>{\raggedleft}m{\dimexpr 0.54in+0\tabcolsep}}{\textcolor[HTML]{000000}{\fontsize{11}{11}\selectfont{\global\setmainfont{Arial}{2}}}} & \multicolumn{1}{>{\raggedleft}m{\dimexpr 0.67in+0\tabcolsep}}{\textcolor[HTML]{000000}{\fontsize{11}{11}\selectfont{\global\setmainfont{Arial}{1.2}}}} \\





\multicolumn{1}{>{\raggedright}m{\dimexpr 1.41in+0\tabcolsep}}{\textcolor[HTML]{000000}{\fontsize{11}{11}\selectfont{\global\setmainfont{Arial}{Muito\ BP}}}} & \multicolumn{1}{>{\raggedleft}m{\dimexpr 0.54in+0\tabcolsep}}{\textcolor[HTML]{000000}{\fontsize{11}{11}\selectfont{\global\setmainfont{Arial}{5}}}} & \multicolumn{1}{>{\raggedleft}m{\dimexpr 0.58in+0\tabcolsep}}{\textcolor[HTML]{000000}{\fontsize{11}{11}\selectfont{\global\setmainfont{Arial}{2.9}}}} & \multicolumn{1}{>{\raggedleft}m{\dimexpr 0.54in+0\tabcolsep}}{\textcolor[HTML]{000000}{\fontsize{11}{11}\selectfont{\global\setmainfont{Arial}{7}}}} & \multicolumn{1}{>{\raggedleft}m{\dimexpr 0.67in+0\tabcolsep}}{\textcolor[HTML]{000000}{\fontsize{11}{11}\selectfont{\global\setmainfont{Arial}{4.1}}}} \\





\multicolumn{1}{>{\raggedright}m{\dimexpr 1.41in+0\tabcolsep}}{\textcolor[HTML]{000000}{\fontsize{11}{11}\selectfont{\global\setmainfont{Arial}{Baixo\ Peso\ (BP)}}}} & \multicolumn{1}{>{\raggedleft}m{\dimexpr 0.54in+0\tabcolsep}}{\textcolor[HTML]{000000}{\fontsize{11}{11}\selectfont{\global\setmainfont{Arial}{0}}}} & \multicolumn{1}{>{\raggedleft}m{\dimexpr 0.58in+0\tabcolsep}}{\textcolor[HTML]{000000}{\fontsize{11}{11}\selectfont{\global\setmainfont{Arial}{0.0}}}} & \multicolumn{1}{>{\raggedleft}m{\dimexpr 0.54in+0\tabcolsep}}{\textcolor[HTML]{000000}{\fontsize{11}{11}\selectfont{\global\setmainfont{Arial}{7}}}} & \multicolumn{1}{>{\raggedleft}m{\dimexpr 0.67in+0\tabcolsep}}{\textcolor[HTML]{000000}{\fontsize{11}{11}\selectfont{\global\setmainfont{Arial}{4.1}}}} \\





\multicolumn{1}{>{\raggedright}m{\dimexpr 1.41in+0\tabcolsep}}{\textcolor[HTML]{000000}{\fontsize{11}{11}\selectfont{\global\setmainfont{Arial}{Peso\ Normal}}}} & \multicolumn{1}{>{\raggedleft}m{\dimexpr 0.54in+0\tabcolsep}}{\textcolor[HTML]{000000}{\fontsize{11}{11}\selectfont{\global\setmainfont{Arial}{163}}}} & \multicolumn{1}{>{\raggedleft}m{\dimexpr 0.58in+0\tabcolsep}}{\textcolor[HTML]{000000}{\fontsize{11}{11}\selectfont{\global\setmainfont{Arial}{95.9}}}} & \multicolumn{1}{>{\raggedleft}m{\dimexpr 0.54in+0\tabcolsep}}{\textcolor[HTML]{000000}{\fontsize{11}{11}\selectfont{\global\setmainfont{Arial}{170}}}} & \multicolumn{1}{>{\raggedleft}m{\dimexpr 0.67in+0\tabcolsep}}{\textcolor[HTML]{000000}{\fontsize{11}{11}\selectfont{\global\setmainfont{Arial}{100.0}}}} \\





\multicolumn{1}{>{\raggedright}m{\dimexpr 1.41in+0\tabcolsep}}{\textcolor[HTML]{000000}{\fontsize{11}{11}\selectfont{\global\setmainfont{Arial}{Excesso\ Peso}}}} & \multicolumn{1}{>{\raggedleft}m{\dimexpr 0.54in+0\tabcolsep}}{\textcolor[HTML]{000000}{\fontsize{11}{11}\selectfont{\global\setmainfont{Arial}{0}}}} & \multicolumn{1}{>{\raggedleft}m{\dimexpr 0.58in+0\tabcolsep}}{\textcolor[HTML]{000000}{\fontsize{11}{11}\selectfont{\global\setmainfont{Arial}{0.0}}}} & \multicolumn{1}{>{\raggedleft}m{\dimexpr 0.54in+0\tabcolsep}}{\textcolor[HTML]{000000}{\fontsize{11}{11}\selectfont{\global\setmainfont{Arial}{170}}}} & \multicolumn{1}{>{\raggedleft}m{\dimexpr 0.67in+0\tabcolsep}}{\textcolor[HTML]{000000}{\fontsize{11}{11}\selectfont{\global\setmainfont{Arial}{100.0}}}} \\

\ascline{1.5pt}{666666}{1-5}


\end{longtable}

\arrayrulecolor[HTML]{000000}

\global\setlength{\arrayrulewidth}{\Oldarrayrulewidth}

\global\setlength{\tabcolsep}{\Oldtabcolsep}

\renewcommand*{\arraystretch}{1}

A Tabela~\ref{tbl-ft4} mudou pouca coisa, está mais ajustada.

\textbf{Passo 4}: O \texttt{flextable} possui vários temas
(\texttt{theme}) que possibilitam customizar o estilo da tabela. Esses
temas oferecem diferentes aparências para as tabelas. É possível
combinar esses temas com outros comandos de formatação, para criar um
visual personalizado!

\begin{Shaded}
\begin{Highlighting}[]
\NormalTok{ft }\OtherTok{\textless{}{-}}\NormalTok{ ft }\SpecialCharTok{\%\textgreater{}\%} 
  \FunctionTok{autofit}\NormalTok{() }\SpecialCharTok{\%\textgreater{}\%} 
  \FunctionTok{set\_header\_labels}\NormalTok{(}
    \AttributeTok{frp =} \StringTok{"fr (\%)"}\NormalTok{,}
    \AttributeTok{Frp =} \StringTok{"Fr (\%)"}
\NormalTok{  ) }\SpecialCharTok{\%\textgreater{}\%} 
  \FunctionTok{theme\_vanilla}\NormalTok{()}
\NormalTok{ft}
\end{Highlighting}
\end{Shaded}

\global\setlength{\Oldarrayrulewidth}{\arrayrulewidth}

\global\setlength{\Oldtabcolsep}{\tabcolsep}

\setlength{\tabcolsep}{2pt}

\renewcommand*{\arraystretch}{1.5}



\providecommand{\ascline}[3]{\noalign{\global\arrayrulewidth #1}\arrayrulecolor[HTML]{#2}\cline{#3}}

\begin{longtable}[c]{|p{1.41in}|p{0.54in}|p{0.66in}|p{0.54in}|p{0.71in}}

\caption{\label{tbl-ft5}Modificando o tema da tabela flextable}

\tabularnewline

\ascline{1.5pt}{666666}{1-5}

\multicolumn{1}{>{\raggedright}m{\dimexpr 1.41in+0\tabcolsep}}{\textcolor[HTML]{000000}{\fontsize{11}{11}\selectfont{\global\setmainfont{Arial}{\textbf{Classificação}}}}} & \multicolumn{1}{>{\raggedleft}m{\dimexpr 0.54in+0\tabcolsep}}{\textcolor[HTML]{000000}{\fontsize{11}{11}\selectfont{\global\setmainfont{Arial}{\textbf{f}}}}} & \multicolumn{1}{>{\raggedleft}m{\dimexpr 0.66in+0\tabcolsep}}{\textcolor[HTML]{000000}{\fontsize{11}{11}\selectfont{\global\setmainfont{Arial}{\textbf{fr\ (\%)}}}}} & \multicolumn{1}{>{\raggedleft}m{\dimexpr 0.54in+0\tabcolsep}}{\textcolor[HTML]{000000}{\fontsize{11}{11}\selectfont{\global\setmainfont{Arial}{\textbf{F}}}}} & \multicolumn{1}{>{\raggedleft}m{\dimexpr 0.71in+0\tabcolsep}}{\textcolor[HTML]{000000}{\fontsize{11}{11}\selectfont{\global\setmainfont{Arial}{\textbf{Fr\ (\%)}}}}} \\

\ascline{1.5pt}{666666}{1-5}\endfirsthead 

\ascline{1.5pt}{666666}{1-5}

\multicolumn{1}{>{\raggedright}m{\dimexpr 1.41in+0\tabcolsep}}{\textcolor[HTML]{000000}{\fontsize{11}{11}\selectfont{\global\setmainfont{Arial}{\textbf{Classificação}}}}} & \multicolumn{1}{>{\raggedleft}m{\dimexpr 0.54in+0\tabcolsep}}{\textcolor[HTML]{000000}{\fontsize{11}{11}\selectfont{\global\setmainfont{Arial}{\textbf{f}}}}} & \multicolumn{1}{>{\raggedleft}m{\dimexpr 0.66in+0\tabcolsep}}{\textcolor[HTML]{000000}{\fontsize{11}{11}\selectfont{\global\setmainfont{Arial}{\textbf{fr\ (\%)}}}}} & \multicolumn{1}{>{\raggedleft}m{\dimexpr 0.54in+0\tabcolsep}}{\textcolor[HTML]{000000}{\fontsize{11}{11}\selectfont{\global\setmainfont{Arial}{\textbf{F}}}}} & \multicolumn{1}{>{\raggedleft}m{\dimexpr 0.71in+0\tabcolsep}}{\textcolor[HTML]{000000}{\fontsize{11}{11}\selectfont{\global\setmainfont{Arial}{\textbf{Fr\ (\%)}}}}} \\

\ascline{1.5pt}{666666}{1-5}\endhead



\multicolumn{1}{>{\raggedright}m{\dimexpr 1.41in+0\tabcolsep}}{\textcolor[HTML]{000000}{\fontsize{11}{11}\selectfont{\global\setmainfont{Arial}{BP\ Extremo}}}} & \multicolumn{1}{>{\raggedleft}m{\dimexpr 0.54in+0\tabcolsep}}{\textcolor[HTML]{000000}{\fontsize{11}{11}\selectfont{\global\setmainfont{Arial}{2}}}} & \multicolumn{1}{>{\raggedleft}m{\dimexpr 0.66in+0\tabcolsep}}{\textcolor[HTML]{000000}{\fontsize{11}{11}\selectfont{\global\setmainfont{Arial}{1.2}}}} & \multicolumn{1}{>{\raggedleft}m{\dimexpr 0.54in+0\tabcolsep}}{\textcolor[HTML]{000000}{\fontsize{11}{11}\selectfont{\global\setmainfont{Arial}{2}}}} & \multicolumn{1}{>{\raggedleft}m{\dimexpr 0.71in+0\tabcolsep}}{\textcolor[HTML]{000000}{\fontsize{11}{11}\selectfont{\global\setmainfont{Arial}{1.2}}}} \\

\ascline{0.75pt}{666666}{1-5}



\multicolumn{1}{>{\raggedright}m{\dimexpr 1.41in+0\tabcolsep}}{\textcolor[HTML]{000000}{\fontsize{11}{11}\selectfont{\global\setmainfont{Arial}{Muito\ BP}}}} & \multicolumn{1}{>{\raggedleft}m{\dimexpr 0.54in+0\tabcolsep}}{\textcolor[HTML]{000000}{\fontsize{11}{11}\selectfont{\global\setmainfont{Arial}{5}}}} & \multicolumn{1}{>{\raggedleft}m{\dimexpr 0.66in+0\tabcolsep}}{\textcolor[HTML]{000000}{\fontsize{11}{11}\selectfont{\global\setmainfont{Arial}{2.9}}}} & \multicolumn{1}{>{\raggedleft}m{\dimexpr 0.54in+0\tabcolsep}}{\textcolor[HTML]{000000}{\fontsize{11}{11}\selectfont{\global\setmainfont{Arial}{7}}}} & \multicolumn{1}{>{\raggedleft}m{\dimexpr 0.71in+0\tabcolsep}}{\textcolor[HTML]{000000}{\fontsize{11}{11}\selectfont{\global\setmainfont{Arial}{4.1}}}} \\

\ascline{0.75pt}{666666}{1-5}



\multicolumn{1}{>{\raggedright}m{\dimexpr 1.41in+0\tabcolsep}}{\textcolor[HTML]{000000}{\fontsize{11}{11}\selectfont{\global\setmainfont{Arial}{Baixo\ Peso\ (BP)}}}} & \multicolumn{1}{>{\raggedleft}m{\dimexpr 0.54in+0\tabcolsep}}{\textcolor[HTML]{000000}{\fontsize{11}{11}\selectfont{\global\setmainfont{Arial}{0}}}} & \multicolumn{1}{>{\raggedleft}m{\dimexpr 0.66in+0\tabcolsep}}{\textcolor[HTML]{000000}{\fontsize{11}{11}\selectfont{\global\setmainfont{Arial}{0.0}}}} & \multicolumn{1}{>{\raggedleft}m{\dimexpr 0.54in+0\tabcolsep}}{\textcolor[HTML]{000000}{\fontsize{11}{11}\selectfont{\global\setmainfont{Arial}{7}}}} & \multicolumn{1}{>{\raggedleft}m{\dimexpr 0.71in+0\tabcolsep}}{\textcolor[HTML]{000000}{\fontsize{11}{11}\selectfont{\global\setmainfont{Arial}{4.1}}}} \\

\ascline{0.75pt}{666666}{1-5}



\multicolumn{1}{>{\raggedright}m{\dimexpr 1.41in+0\tabcolsep}}{\textcolor[HTML]{000000}{\fontsize{11}{11}\selectfont{\global\setmainfont{Arial}{Peso\ Normal}}}} & \multicolumn{1}{>{\raggedleft}m{\dimexpr 0.54in+0\tabcolsep}}{\textcolor[HTML]{000000}{\fontsize{11}{11}\selectfont{\global\setmainfont{Arial}{163}}}} & \multicolumn{1}{>{\raggedleft}m{\dimexpr 0.66in+0\tabcolsep}}{\textcolor[HTML]{000000}{\fontsize{11}{11}\selectfont{\global\setmainfont{Arial}{95.9}}}} & \multicolumn{1}{>{\raggedleft}m{\dimexpr 0.54in+0\tabcolsep}}{\textcolor[HTML]{000000}{\fontsize{11}{11}\selectfont{\global\setmainfont{Arial}{170}}}} & \multicolumn{1}{>{\raggedleft}m{\dimexpr 0.71in+0\tabcolsep}}{\textcolor[HTML]{000000}{\fontsize{11}{11}\selectfont{\global\setmainfont{Arial}{100.0}}}} \\

\ascline{0.75pt}{666666}{1-5}



\multicolumn{1}{>{\raggedright}m{\dimexpr 1.41in+0\tabcolsep}}{\textcolor[HTML]{000000}{\fontsize{11}{11}\selectfont{\global\setmainfont{Arial}{Excesso\ Peso}}}} & \multicolumn{1}{>{\raggedleft}m{\dimexpr 0.54in+0\tabcolsep}}{\textcolor[HTML]{000000}{\fontsize{11}{11}\selectfont{\global\setmainfont{Arial}{0}}}} & \multicolumn{1}{>{\raggedleft}m{\dimexpr 0.66in+0\tabcolsep}}{\textcolor[HTML]{000000}{\fontsize{11}{11}\selectfont{\global\setmainfont{Arial}{0.0}}}} & \multicolumn{1}{>{\raggedleft}m{\dimexpr 0.54in+0\tabcolsep}}{\textcolor[HTML]{000000}{\fontsize{11}{11}\selectfont{\global\setmainfont{Arial}{170}}}} & \multicolumn{1}{>{\raggedleft}m{\dimexpr 0.71in+0\tabcolsep}}{\textcolor[HTML]{000000}{\fontsize{11}{11}\selectfont{\global\setmainfont{Arial}{100.0}}}} \\

\ascline{1.5pt}{666666}{1-5}


\end{longtable}

\arrayrulecolor[HTML]{000000}

\global\setlength{\arrayrulewidth}{\Oldarrayrulewidth}

\global\setlength{\tabcolsep}{\Oldtabcolsep}

\renewcommand*{\arraystretch}{1}

A Tabela~\ref{tbl-ft5} já está ótima, mas utros temas podem ser usados
como \texttt{theme\_booktabs()} (padrão), \texttt{theme\_vader()},
\texttt{theme\_apa()}, \texttt{theme\_zebra()}, \texttt{theme\_box()},
\texttt{theme\_tron\_legacy\ ()}.

\begin{tcolorbox}[enhanced jigsaw, bottomrule=.15mm, opacitybacktitle=0.6, colframe=quarto-callout-tip-color-frame, arc=.35mm, coltitle=black, toptitle=1mm, colback=white, colbacktitle=quarto-callout-tip-color!10!white, breakable, bottomtitle=1mm, rightrule=.15mm, titlerule=0mm, toprule=.15mm, opacityback=0, leftrule=.75mm, left=2mm, title=\textcolor{quarto-callout-tip-color}{\faLightbulb}\hspace{0.5em}{Exercício}]

Teste a aparência da tabela com diferentes temas.

\end{tcolorbox}

\textbf{Passo 5}: Uma nota de rodapé pode ser adicionada, usando-se uma
função específica, a\texttt{dd\_footer\_lines()} junto com as funções
que determinam o tamanho da fonte:

\begin{Shaded}
\begin{Highlighting}[]
\NormalTok{ft1 }\OtherTok{\textless{}{-}}\NormalTok{ ft }\SpecialCharTok{\%\textgreater{}\%} 
  \FunctionTok{autofit}\NormalTok{() }\SpecialCharTok{\%\textgreater{}\%} 
  \FunctionTok{set\_header\_labels}\NormalTok{(}
    \AttributeTok{frp =} \StringTok{"fr (\%)"}\NormalTok{,}
    \AttributeTok{Frp =} \StringTok{"Fr (\%)"}
\NormalTok{  ) }\SpecialCharTok{\%\textgreater{}\%}  
  \FunctionTok{theme\_vanilla}\NormalTok{() }\SpecialCharTok{\%\textgreater{}\%} 
  \FunctionTok{add\_footer\_lines}\NormalTok{(}\AttributeTok{value =} \StringTok{"FONTE: Hospital Geral, Caxias do Sul, RS, 2008"}\NormalTok{) }\SpecialCharTok{\%\textgreater{}\%} 
  \FunctionTok{fontsize}\NormalTok{(}\AttributeTok{size =} \DecValTok{9}\NormalTok{, }\AttributeTok{part =} \StringTok{"footer"}\NormalTok{) }

\NormalTok{ft1}
\end{Highlighting}
\end{Shaded}

\global\setlength{\Oldarrayrulewidth}{\arrayrulewidth}

\global\setlength{\Oldtabcolsep}{\tabcolsep}

\setlength{\tabcolsep}{2pt}

\renewcommand*{\arraystretch}{1.5}



\providecommand{\ascline}[3]{\noalign{\global\arrayrulewidth #1}\arrayrulecolor[HTML]{#2}\cline{#3}}

\begin{longtable}[c]{|p{1.41in}|p{0.54in}|p{0.68in}|p{0.54in}|p{0.72in}}

\caption{\label{tbl-ft6}Tabela com nota de rodapé}

\tabularnewline

\ascline{1.5pt}{666666}{1-5}

\multicolumn{1}{>{\raggedright}m{\dimexpr 1.41in+0\tabcolsep}}{\textcolor[HTML]{000000}{\fontsize{11}{11}\selectfont{\global\setmainfont{Arial}{\textbf{Classificação}}}}} & \multicolumn{1}{>{\raggedleft}m{\dimexpr 0.54in+0\tabcolsep}}{\textcolor[HTML]{000000}{\fontsize{11}{11}\selectfont{\global\setmainfont{Arial}{\textbf{f}}}}} & \multicolumn{1}{>{\raggedleft}m{\dimexpr 0.68in+0\tabcolsep}}{\textcolor[HTML]{000000}{\fontsize{11}{11}\selectfont{\global\setmainfont{Arial}{\textbf{fr\ (\%)}}}}} & \multicolumn{1}{>{\raggedleft}m{\dimexpr 0.54in+0\tabcolsep}}{\textcolor[HTML]{000000}{\fontsize{11}{11}\selectfont{\global\setmainfont{Arial}{\textbf{F}}}}} & \multicolumn{1}{>{\raggedleft}m{\dimexpr 0.72in+0\tabcolsep}}{\textcolor[HTML]{000000}{\fontsize{11}{11}\selectfont{\global\setmainfont{Arial}{\textbf{Fr\ (\%)}}}}} \\

\ascline{1.5pt}{666666}{1-5}\endfirsthead 

\ascline{1.5pt}{666666}{1-5}

\multicolumn{1}{>{\raggedright}m{\dimexpr 1.41in+0\tabcolsep}}{\textcolor[HTML]{000000}{\fontsize{11}{11}\selectfont{\global\setmainfont{Arial}{\textbf{Classificação}}}}} & \multicolumn{1}{>{\raggedleft}m{\dimexpr 0.54in+0\tabcolsep}}{\textcolor[HTML]{000000}{\fontsize{11}{11}\selectfont{\global\setmainfont{Arial}{\textbf{f}}}}} & \multicolumn{1}{>{\raggedleft}m{\dimexpr 0.68in+0\tabcolsep}}{\textcolor[HTML]{000000}{\fontsize{11}{11}\selectfont{\global\setmainfont{Arial}{\textbf{fr\ (\%)}}}}} & \multicolumn{1}{>{\raggedleft}m{\dimexpr 0.54in+0\tabcolsep}}{\textcolor[HTML]{000000}{\fontsize{11}{11}\selectfont{\global\setmainfont{Arial}{\textbf{F}}}}} & \multicolumn{1}{>{\raggedleft}m{\dimexpr 0.72in+0\tabcolsep}}{\textcolor[HTML]{000000}{\fontsize{11}{11}\selectfont{\global\setmainfont{Arial}{\textbf{Fr\ (\%)}}}}} \\

\ascline{1.5pt}{666666}{1-5}\endhead



\multicolumn{1}{>{\raggedright}m{\dimexpr 1.41in+0\tabcolsep}}{\textcolor[HTML]{000000}{\fontsize{11}{11}\selectfont{\global\setmainfont{Arial}{BP\ Extremo}}}} & \multicolumn{1}{>{\raggedleft}m{\dimexpr 0.54in+0\tabcolsep}}{\textcolor[HTML]{000000}{\fontsize{11}{11}\selectfont{\global\setmainfont{Arial}{2}}}} & \multicolumn{1}{>{\raggedleft}m{\dimexpr 0.68in+0\tabcolsep}}{\textcolor[HTML]{000000}{\fontsize{11}{11}\selectfont{\global\setmainfont{Arial}{1.2}}}} & \multicolumn{1}{>{\raggedleft}m{\dimexpr 0.54in+0\tabcolsep}}{\textcolor[HTML]{000000}{\fontsize{11}{11}\selectfont{\global\setmainfont{Arial}{2}}}} & \multicolumn{1}{>{\raggedleft}m{\dimexpr 0.72in+0\tabcolsep}}{\textcolor[HTML]{000000}{\fontsize{11}{11}\selectfont{\global\setmainfont{Arial}{1.2}}}} \\

\ascline{0.75pt}{666666}{1-5}



\multicolumn{1}{>{\raggedright}m{\dimexpr 1.41in+0\tabcolsep}}{\textcolor[HTML]{000000}{\fontsize{11}{11}\selectfont{\global\setmainfont{Arial}{Muito\ BP}}}} & \multicolumn{1}{>{\raggedleft}m{\dimexpr 0.54in+0\tabcolsep}}{\textcolor[HTML]{000000}{\fontsize{11}{11}\selectfont{\global\setmainfont{Arial}{5}}}} & \multicolumn{1}{>{\raggedleft}m{\dimexpr 0.68in+0\tabcolsep}}{\textcolor[HTML]{000000}{\fontsize{11}{11}\selectfont{\global\setmainfont{Arial}{2.9}}}} & \multicolumn{1}{>{\raggedleft}m{\dimexpr 0.54in+0\tabcolsep}}{\textcolor[HTML]{000000}{\fontsize{11}{11}\selectfont{\global\setmainfont{Arial}{7}}}} & \multicolumn{1}{>{\raggedleft}m{\dimexpr 0.72in+0\tabcolsep}}{\textcolor[HTML]{000000}{\fontsize{11}{11}\selectfont{\global\setmainfont{Arial}{4.1}}}} \\

\ascline{0.75pt}{666666}{1-5}



\multicolumn{1}{>{\raggedright}m{\dimexpr 1.41in+0\tabcolsep}}{\textcolor[HTML]{000000}{\fontsize{11}{11}\selectfont{\global\setmainfont{Arial}{Baixo\ Peso\ (BP)}}}} & \multicolumn{1}{>{\raggedleft}m{\dimexpr 0.54in+0\tabcolsep}}{\textcolor[HTML]{000000}{\fontsize{11}{11}\selectfont{\global\setmainfont{Arial}{0}}}} & \multicolumn{1}{>{\raggedleft}m{\dimexpr 0.68in+0\tabcolsep}}{\textcolor[HTML]{000000}{\fontsize{11}{11}\selectfont{\global\setmainfont{Arial}{0.0}}}} & \multicolumn{1}{>{\raggedleft}m{\dimexpr 0.54in+0\tabcolsep}}{\textcolor[HTML]{000000}{\fontsize{11}{11}\selectfont{\global\setmainfont{Arial}{7}}}} & \multicolumn{1}{>{\raggedleft}m{\dimexpr 0.72in+0\tabcolsep}}{\textcolor[HTML]{000000}{\fontsize{11}{11}\selectfont{\global\setmainfont{Arial}{4.1}}}} \\

\ascline{0.75pt}{666666}{1-5}



\multicolumn{1}{>{\raggedright}m{\dimexpr 1.41in+0\tabcolsep}}{\textcolor[HTML]{000000}{\fontsize{11}{11}\selectfont{\global\setmainfont{Arial}{Peso\ Normal}}}} & \multicolumn{1}{>{\raggedleft}m{\dimexpr 0.54in+0\tabcolsep}}{\textcolor[HTML]{000000}{\fontsize{11}{11}\selectfont{\global\setmainfont{Arial}{163}}}} & \multicolumn{1}{>{\raggedleft}m{\dimexpr 0.68in+0\tabcolsep}}{\textcolor[HTML]{000000}{\fontsize{11}{11}\selectfont{\global\setmainfont{Arial}{95.9}}}} & \multicolumn{1}{>{\raggedleft}m{\dimexpr 0.54in+0\tabcolsep}}{\textcolor[HTML]{000000}{\fontsize{11}{11}\selectfont{\global\setmainfont{Arial}{170}}}} & \multicolumn{1}{>{\raggedleft}m{\dimexpr 0.72in+0\tabcolsep}}{\textcolor[HTML]{000000}{\fontsize{11}{11}\selectfont{\global\setmainfont{Arial}{100.0}}}} \\

\ascline{0.75pt}{666666}{1-5}



\multicolumn{1}{>{\raggedright}m{\dimexpr 1.41in+0\tabcolsep}}{\textcolor[HTML]{000000}{\fontsize{11}{11}\selectfont{\global\setmainfont{Arial}{Excesso\ Peso}}}} & \multicolumn{1}{>{\raggedleft}m{\dimexpr 0.54in+0\tabcolsep}}{\textcolor[HTML]{000000}{\fontsize{11}{11}\selectfont{\global\setmainfont{Arial}{0}}}} & \multicolumn{1}{>{\raggedleft}m{\dimexpr 0.68in+0\tabcolsep}}{\textcolor[HTML]{000000}{\fontsize{11}{11}\selectfont{\global\setmainfont{Arial}{0.0}}}} & \multicolumn{1}{>{\raggedleft}m{\dimexpr 0.54in+0\tabcolsep}}{\textcolor[HTML]{000000}{\fontsize{11}{11}\selectfont{\global\setmainfont{Arial}{170}}}} & \multicolumn{1}{>{\raggedleft}m{\dimexpr 0.72in+0\tabcolsep}}{\textcolor[HTML]{000000}{\fontsize{11}{11}\selectfont{\global\setmainfont{Arial}{100.0}}}} \\

\ascline{1.5pt}{666666}{1-5}



\multicolumn{5}{>{\raggedright}m{\dimexpr 3.88in+8\tabcolsep}}{\textcolor[HTML]{000000}{\fontsize{9}{9}\selectfont{\global\setmainfont{Arial}{FONTE:\ Hospital\ Geral,\ Caxias\ do\ Sul,\ RS,\ 2008}}}} \\




\end{longtable}

\arrayrulecolor[HTML]{000000}

\global\setlength{\arrayrulewidth}{\Oldarrayrulewidth}

\global\setlength{\tabcolsep}{\Oldtabcolsep}

\renewcommand*{\arraystretch}{1}

A Tabela~\ref{tbl-ft6} é igual a Tabela~\ref{tbl-ft5} adicionada de um
rodapé.

\textbf{Passo 6}: Muitas vezes, é útil definir regras de formatação que
se aplicam apenas a um subconjunto da tabela. Por exemplo, algumas
linhas ou colunas devam aparecer em uma cor diferente por um motivo ou
outro. Todas as tabelas flexíveis são compostas por três partes:
cabeçalho (\textbf{header}) na parte superior, uma grade de células no
corpo (\textbf{body}) da tabela e um conjunto de linhas de rodapé
(\textbf{footer}) na parte inferior. Muitas funções na tabela flexível
têm um argumento que se pode usar para selecionar uma (ou todas) essas
três partes. Por exemplo, a função \texttt{bg()} é usada para definir a
cor de fundo. Pode-se adicionar um conteúdo extra ao cabeçalho,
\texttt{add\_header\_lines()} e colorir o fundo:

\begin{Shaded}
\begin{Highlighting}[]
\NormalTok{ft2 }\OtherTok{\textless{}{-}}\NormalTok{ ft }\SpecialCharTok{\%\textgreater{}\%} 
  \FunctionTok{autofit}\NormalTok{() }\SpecialCharTok{\%\textgreater{}\%} 
  \FunctionTok{set\_header\_labels}\NormalTok{(}
    \AttributeTok{frp =} \StringTok{"fr (\%)"}\NormalTok{,}
    \AttributeTok{Frp =} \StringTok{"Fr (\%)"}
\NormalTok{  ) }\SpecialCharTok{\%\textgreater{}\%}  
  \FunctionTok{theme\_booktabs}\NormalTok{() }\SpecialCharTok{\%\textgreater{}\%} 
  \FunctionTok{add\_footer\_lines}\NormalTok{(}\AttributeTok{value =} \StringTok{"FONTE: Hospital Geral, Caxias do Sul, RS, 2008"}\NormalTok{) }\SpecialCharTok{\%\textgreater{}\%} 
  \FunctionTok{fontsize}\NormalTok{(}\AttributeTok{size =} \DecValTok{9}\NormalTok{, }\AttributeTok{part =} \StringTok{"footer"}\NormalTok{) }\SpecialCharTok{\%\textgreater{}\%} 
  \FunctionTok{add\_header\_lines}\NormalTok{(}\StringTok{"TABELA 1: Pesos dos RN de acordo com a OMS"}\NormalTok{) }\SpecialCharTok{\%\textgreater{}\%} 
  \FunctionTok{bg}\NormalTok{(}\AttributeTok{bg =} \StringTok{"gray30"}\NormalTok{, }\AttributeTok{part =}\StringTok{"header"}\NormalTok{) }\SpecialCharTok{\%\textgreater{}\%} 
  \FunctionTok{color}\NormalTok{(}\AttributeTok{part =} \StringTok{"header"}\NormalTok{, }\AttributeTok{color =} \StringTok{"ghostwhite"}\NormalTok{)}

\NormalTok{ft2}
\end{Highlighting}
\end{Shaded}

\global\setlength{\Oldarrayrulewidth}{\arrayrulewidth}

\global\setlength{\Oldtabcolsep}{\tabcolsep}

\setlength{\tabcolsep}{2pt}

\renewcommand*{\arraystretch}{1.5}



\providecommand{\ascline}[3]{\noalign{\global\arrayrulewidth #1}\arrayrulecolor[HTML]{#2}\cline{#3}}

\begin{longtable}[c]{|p{1.41in}|p{0.54in}|p{0.68in}|p{0.54in}|p{0.72in}}

\caption{\label{tbl-ft7}Tabela com nota de rodapé}

\tabularnewline

\hhline{>{\arrayrulecolor[HTML]{666666}\global\arrayrulewidth=1.5pt}->{\arrayrulecolor[HTML]{666666}\global\arrayrulewidth=1.5pt}->{\arrayrulecolor[HTML]{666666}\global\arrayrulewidth=1.5pt}->{\arrayrulecolor[HTML]{666666}\global\arrayrulewidth=1.5pt}->{\arrayrulecolor[HTML]{666666}\global\arrayrulewidth=1.5pt}-}

\multicolumn{5}{>{\cellcolor[HTML]{4D4D4D}\raggedright}m{\dimexpr 3.88in+8\tabcolsep}}{\textcolor[HTML]{F8F8FF}{\fontsize{11}{11}\selectfont{\global\setmainfont{Arial}{TABELA\ 1:\ Pesos\ dos\ RN\ de\ acordo\ com\ a\ OMS}}}} \\

\noalign{\global\arrayrulewidth 0pt}\arrayrulecolor[HTML]{000000}

\hhline{>{\arrayrulecolor[HTML]{666666}\global\arrayrulewidth=1.5pt}->{\arrayrulecolor[HTML]{666666}\global\arrayrulewidth=1.5pt}->{\arrayrulecolor[HTML]{666666}\global\arrayrulewidth=1.5pt}->{\arrayrulecolor[HTML]{666666}\global\arrayrulewidth=1.5pt}->{\arrayrulecolor[HTML]{666666}\global\arrayrulewidth=1.5pt}-}



\multicolumn{1}{>{\cellcolor[HTML]{4D4D4D}\raggedright}m{\dimexpr 1.41in+0\tabcolsep}}{\textcolor[HTML]{F8F8FF}{\fontsize{11}{11}\selectfont{\global\setmainfont{Arial}{Classificação}}}} & \multicolumn{1}{>{\cellcolor[HTML]{4D4D4D}\raggedleft}m{\dimexpr 0.54in+0\tabcolsep}}{\textcolor[HTML]{F8F8FF}{\fontsize{11}{11}\selectfont{\global\setmainfont{Arial}{f}}}} & \multicolumn{1}{>{\cellcolor[HTML]{4D4D4D}\raggedleft}m{\dimexpr 0.68in+0\tabcolsep}}{\textcolor[HTML]{F8F8FF}{\fontsize{11}{11}\selectfont{\global\setmainfont{Arial}{fr\ (\%)}}}} & \multicolumn{1}{>{\cellcolor[HTML]{4D4D4D}\raggedleft}m{\dimexpr 0.54in+0\tabcolsep}}{\textcolor[HTML]{F8F8FF}{\fontsize{11}{11}\selectfont{\global\setmainfont{Arial}{F}}}} & \multicolumn{1}{>{\cellcolor[HTML]{4D4D4D}\raggedleft}m{\dimexpr 0.72in+0\tabcolsep}}{\textcolor[HTML]{F8F8FF}{\fontsize{11}{11}\selectfont{\global\setmainfont{Arial}{Fr\ (\%)}}}} \\

\noalign{\global\arrayrulewidth 0pt}\arrayrulecolor[HTML]{000000}

\hhline{>{\arrayrulecolor[HTML]{666666}\global\arrayrulewidth=1.5pt}->{\arrayrulecolor[HTML]{666666}\global\arrayrulewidth=1.5pt}->{\arrayrulecolor[HTML]{666666}\global\arrayrulewidth=1.5pt}->{\arrayrulecolor[HTML]{666666}\global\arrayrulewidth=1.5pt}->{\arrayrulecolor[HTML]{666666}\global\arrayrulewidth=1.5pt}-}\endfirsthead 

\hhline{>{\arrayrulecolor[HTML]{666666}\global\arrayrulewidth=1.5pt}->{\arrayrulecolor[HTML]{666666}\global\arrayrulewidth=1.5pt}->{\arrayrulecolor[HTML]{666666}\global\arrayrulewidth=1.5pt}->{\arrayrulecolor[HTML]{666666}\global\arrayrulewidth=1.5pt}->{\arrayrulecolor[HTML]{666666}\global\arrayrulewidth=1.5pt}-}

\multicolumn{5}{>{\cellcolor[HTML]{4D4D4D}\raggedright}m{\dimexpr 3.88in+8\tabcolsep}}{\textcolor[HTML]{F8F8FF}{\fontsize{11}{11}\selectfont{\global\setmainfont{Arial}{TABELA\ 1:\ Pesos\ dos\ RN\ de\ acordo\ com\ a\ OMS}}}} \\

\noalign{\global\arrayrulewidth 0pt}\arrayrulecolor[HTML]{000000}

\hhline{>{\arrayrulecolor[HTML]{666666}\global\arrayrulewidth=1.5pt}->{\arrayrulecolor[HTML]{666666}\global\arrayrulewidth=1.5pt}->{\arrayrulecolor[HTML]{666666}\global\arrayrulewidth=1.5pt}->{\arrayrulecolor[HTML]{666666}\global\arrayrulewidth=1.5pt}->{\arrayrulecolor[HTML]{666666}\global\arrayrulewidth=1.5pt}-}



\multicolumn{1}{>{\cellcolor[HTML]{4D4D4D}\raggedright}m{\dimexpr 1.41in+0\tabcolsep}}{\textcolor[HTML]{F8F8FF}{\fontsize{11}{11}\selectfont{\global\setmainfont{Arial}{Classificação}}}} & \multicolumn{1}{>{\cellcolor[HTML]{4D4D4D}\raggedleft}m{\dimexpr 0.54in+0\tabcolsep}}{\textcolor[HTML]{F8F8FF}{\fontsize{11}{11}\selectfont{\global\setmainfont{Arial}{f}}}} & \multicolumn{1}{>{\cellcolor[HTML]{4D4D4D}\raggedleft}m{\dimexpr 0.68in+0\tabcolsep}}{\textcolor[HTML]{F8F8FF}{\fontsize{11}{11}\selectfont{\global\setmainfont{Arial}{fr\ (\%)}}}} & \multicolumn{1}{>{\cellcolor[HTML]{4D4D4D}\raggedleft}m{\dimexpr 0.54in+0\tabcolsep}}{\textcolor[HTML]{F8F8FF}{\fontsize{11}{11}\selectfont{\global\setmainfont{Arial}{F}}}} & \multicolumn{1}{>{\cellcolor[HTML]{4D4D4D}\raggedleft}m{\dimexpr 0.72in+0\tabcolsep}}{\textcolor[HTML]{F8F8FF}{\fontsize{11}{11}\selectfont{\global\setmainfont{Arial}{Fr\ (\%)}}}} \\

\noalign{\global\arrayrulewidth 0pt}\arrayrulecolor[HTML]{000000}

\hhline{>{\arrayrulecolor[HTML]{666666}\global\arrayrulewidth=1.5pt}->{\arrayrulecolor[HTML]{666666}\global\arrayrulewidth=1.5pt}->{\arrayrulecolor[HTML]{666666}\global\arrayrulewidth=1.5pt}->{\arrayrulecolor[HTML]{666666}\global\arrayrulewidth=1.5pt}->{\arrayrulecolor[HTML]{666666}\global\arrayrulewidth=1.5pt}-}\endhead



\multicolumn{1}{>{\raggedright}m{\dimexpr 1.41in+0\tabcolsep}}{\textcolor[HTML]{000000}{\fontsize{11}{11}\selectfont{\global\setmainfont{Arial}{BP\ Extremo}}}} & \multicolumn{1}{>{\raggedleft}m{\dimexpr 0.54in+0\tabcolsep}}{\textcolor[HTML]{000000}{\fontsize{11}{11}\selectfont{\global\setmainfont{Arial}{2}}}} & \multicolumn{1}{>{\raggedleft}m{\dimexpr 0.68in+0\tabcolsep}}{\textcolor[HTML]{000000}{\fontsize{11}{11}\selectfont{\global\setmainfont{Arial}{1.2}}}} & \multicolumn{1}{>{\raggedleft}m{\dimexpr 0.54in+0\tabcolsep}}{\textcolor[HTML]{000000}{\fontsize{11}{11}\selectfont{\global\setmainfont{Arial}{2}}}} & \multicolumn{1}{>{\raggedleft}m{\dimexpr 0.72in+0\tabcolsep}}{\textcolor[HTML]{000000}{\fontsize{11}{11}\selectfont{\global\setmainfont{Arial}{1.2}}}} \\

\noalign{\global\arrayrulewidth 0pt}\arrayrulecolor[HTML]{000000}





\multicolumn{1}{>{\raggedright}m{\dimexpr 1.41in+0\tabcolsep}}{\textcolor[HTML]{000000}{\fontsize{11}{11}\selectfont{\global\setmainfont{Arial}{Muito\ BP}}}} & \multicolumn{1}{>{\raggedleft}m{\dimexpr 0.54in+0\tabcolsep}}{\textcolor[HTML]{000000}{\fontsize{11}{11}\selectfont{\global\setmainfont{Arial}{5}}}} & \multicolumn{1}{>{\raggedleft}m{\dimexpr 0.68in+0\tabcolsep}}{\textcolor[HTML]{000000}{\fontsize{11}{11}\selectfont{\global\setmainfont{Arial}{2.9}}}} & \multicolumn{1}{>{\raggedleft}m{\dimexpr 0.54in+0\tabcolsep}}{\textcolor[HTML]{000000}{\fontsize{11}{11}\selectfont{\global\setmainfont{Arial}{7}}}} & \multicolumn{1}{>{\raggedleft}m{\dimexpr 0.72in+0\tabcolsep}}{\textcolor[HTML]{000000}{\fontsize{11}{11}\selectfont{\global\setmainfont{Arial}{4.1}}}} \\

\noalign{\global\arrayrulewidth 0pt}\arrayrulecolor[HTML]{000000}





\multicolumn{1}{>{\raggedright}m{\dimexpr 1.41in+0\tabcolsep}}{\textcolor[HTML]{000000}{\fontsize{11}{11}\selectfont{\global\setmainfont{Arial}{Baixo\ Peso\ (BP)}}}} & \multicolumn{1}{>{\raggedleft}m{\dimexpr 0.54in+0\tabcolsep}}{\textcolor[HTML]{000000}{\fontsize{11}{11}\selectfont{\global\setmainfont{Arial}{0}}}} & \multicolumn{1}{>{\raggedleft}m{\dimexpr 0.68in+0\tabcolsep}}{\textcolor[HTML]{000000}{\fontsize{11}{11}\selectfont{\global\setmainfont{Arial}{0.0}}}} & \multicolumn{1}{>{\raggedleft}m{\dimexpr 0.54in+0\tabcolsep}}{\textcolor[HTML]{000000}{\fontsize{11}{11}\selectfont{\global\setmainfont{Arial}{7}}}} & \multicolumn{1}{>{\raggedleft}m{\dimexpr 0.72in+0\tabcolsep}}{\textcolor[HTML]{000000}{\fontsize{11}{11}\selectfont{\global\setmainfont{Arial}{4.1}}}} \\

\noalign{\global\arrayrulewidth 0pt}\arrayrulecolor[HTML]{000000}





\multicolumn{1}{>{\raggedright}m{\dimexpr 1.41in+0\tabcolsep}}{\textcolor[HTML]{000000}{\fontsize{11}{11}\selectfont{\global\setmainfont{Arial}{Peso\ Normal}}}} & \multicolumn{1}{>{\raggedleft}m{\dimexpr 0.54in+0\tabcolsep}}{\textcolor[HTML]{000000}{\fontsize{11}{11}\selectfont{\global\setmainfont{Arial}{163}}}} & \multicolumn{1}{>{\raggedleft}m{\dimexpr 0.68in+0\tabcolsep}}{\textcolor[HTML]{000000}{\fontsize{11}{11}\selectfont{\global\setmainfont{Arial}{95.9}}}} & \multicolumn{1}{>{\raggedleft}m{\dimexpr 0.54in+0\tabcolsep}}{\textcolor[HTML]{000000}{\fontsize{11}{11}\selectfont{\global\setmainfont{Arial}{170}}}} & \multicolumn{1}{>{\raggedleft}m{\dimexpr 0.72in+0\tabcolsep}}{\textcolor[HTML]{000000}{\fontsize{11}{11}\selectfont{\global\setmainfont{Arial}{100.0}}}} \\

\noalign{\global\arrayrulewidth 0pt}\arrayrulecolor[HTML]{000000}





\multicolumn{1}{>{\raggedright}m{\dimexpr 1.41in+0\tabcolsep}}{\textcolor[HTML]{000000}{\fontsize{11}{11}\selectfont{\global\setmainfont{Arial}{Excesso\ Peso}}}} & \multicolumn{1}{>{\raggedleft}m{\dimexpr 0.54in+0\tabcolsep}}{\textcolor[HTML]{000000}{\fontsize{11}{11}\selectfont{\global\setmainfont{Arial}{0}}}} & \multicolumn{1}{>{\raggedleft}m{\dimexpr 0.68in+0\tabcolsep}}{\textcolor[HTML]{000000}{\fontsize{11}{11}\selectfont{\global\setmainfont{Arial}{0.0}}}} & \multicolumn{1}{>{\raggedleft}m{\dimexpr 0.54in+0\tabcolsep}}{\textcolor[HTML]{000000}{\fontsize{11}{11}\selectfont{\global\setmainfont{Arial}{170}}}} & \multicolumn{1}{>{\raggedleft}m{\dimexpr 0.72in+0\tabcolsep}}{\textcolor[HTML]{000000}{\fontsize{11}{11}\selectfont{\global\setmainfont{Arial}{100.0}}}} \\

\noalign{\global\arrayrulewidth 0pt}\arrayrulecolor[HTML]{000000}

\hhline{>{\arrayrulecolor[HTML]{666666}\global\arrayrulewidth=1.5pt}->{\arrayrulecolor[HTML]{666666}\global\arrayrulewidth=1.5pt}->{\arrayrulecolor[HTML]{666666}\global\arrayrulewidth=1.5pt}->{\arrayrulecolor[HTML]{666666}\global\arrayrulewidth=1.5pt}->{\arrayrulecolor[HTML]{666666}\global\arrayrulewidth=1.5pt}-}



\multicolumn{5}{>{\raggedright}m{\dimexpr 3.88in+8\tabcolsep}}{\textcolor[HTML]{000000}{\fontsize{9}{9}\selectfont{\global\setmainfont{Arial}{FONTE:\ Hospital\ Geral,\ Caxias\ do\ Sul,\ RS,\ 2008}}}} \\

\noalign{\global\arrayrulewidth 0pt}\arrayrulecolor[HTML]{FFFFFF}




\end{longtable}

\arrayrulecolor[HTML]{000000}

\global\setlength{\arrayrulewidth}{\Oldarrayrulewidth}

\global\setlength{\tabcolsep}{\Oldtabcolsep}

\renewcommand*{\arraystretch}{1}

A Tabela~\ref{tbl-ft7} mostra um título com fundo escuro e letras
brancas.

\textbf{Passo 7}: Se for especificado uma seleção de linha e uma seleção
de coluna, a regra de formatação será aplicada às células que
satisfizerem ambos os critérios. Por exemplo, será selecionada a coluna
5 (frequência relativa acumulada ) e as linhas 1 a 3 coloridas em um
gradiente de azul para representar todos os recém-nascidos abaixo de
2500 g (baixo peso).

Antes cria-se uma função geradora de cores numéricas, criada com a
função \texttt{col\_numeric()} do pacote \texttt{scales} que cria um
mapeamenteo de cores para valores numéricos. A paleta define um
gradiente de cores que vai do transparente para o azul. O intervalo de
valores é definido pelo argumento \texttt{domain\ =\ c(0,50)}. Isto
significa que qualquer número é convertido em uma cor proporcional
dentro do gradiente. Os valores mais baixos serão transparentes (próximo
a 0) e os valores mais altos (próximos a 50) serão azuis
(Tabela~\ref{tbl-ft8}).

\begin{Shaded}
\begin{Highlighting}[]
\FunctionTok{library}\NormalTok{(scales)}
\NormalTok{colourer }\OtherTok{\textless{}{-}}\NormalTok{ scales}\SpecialCharTok{::}\FunctionTok{col\_numeric}\NormalTok{(}
  \AttributeTok{palette =} \FunctionTok{c}\NormalTok{(}\StringTok{"transparent"}\NormalTok{, }\StringTok{"deepskyblue4"}\NormalTok{),}
  \AttributeTok{domain =} \FunctionTok{c}\NormalTok{(}\DecValTok{0}\NormalTok{, }\DecValTok{50}\NormalTok{))}

\NormalTok{ft3}\OtherTok{\textless{}{-}}\NormalTok{ ft }\SpecialCharTok{\%\textgreater{}\%} 
  \FunctionTok{autofit}\NormalTok{() }\SpecialCharTok{\%\textgreater{}\%} 
  \FunctionTok{set\_header\_labels}\NormalTok{(}
    \AttributeTok{frp =} \StringTok{"fr (\%)"}\NormalTok{,}
    \AttributeTok{Frp =} \StringTok{"Fr (\%)"}
\NormalTok{  ) }\SpecialCharTok{\%\textgreater{}\%}  
  \FunctionTok{theme\_booktabs}\NormalTok{() }\SpecialCharTok{\%\textgreater{}\%} 
  \FunctionTok{add\_footer\_lines}\NormalTok{(}\AttributeTok{value =} \StringTok{"FONTE: Hospital Geral, Caxias do Sul, RS, 2008"}\NormalTok{) }\SpecialCharTok{\%\textgreater{}\%} 
  \FunctionTok{fontsize}\NormalTok{(}\AttributeTok{size =} \DecValTok{9}\NormalTok{, }\AttributeTok{part =} \StringTok{"footer"}\NormalTok{) }\SpecialCharTok{\%\textgreater{}\%} 
  \FunctionTok{add\_header\_lines}\NormalTok{(}\StringTok{"TABELA 1: Pesos dos RN de acordo com a OMS"}\NormalTok{) }\SpecialCharTok{\%\textgreater{}\%} 
  \FunctionTok{bg}\NormalTok{(}\AttributeTok{i =} \DecValTok{1}\SpecialCharTok{:}\DecValTok{3}\NormalTok{, }\AttributeTok{j =} \DecValTok{5}\NormalTok{, }\AttributeTok{bg =}\NormalTok{colourer)}

\NormalTok{ft3}
\end{Highlighting}
\end{Shaded}

\global\setlength{\Oldarrayrulewidth}{\arrayrulewidth}

\global\setlength{\Oldtabcolsep}{\tabcolsep}

\setlength{\tabcolsep}{2pt}

\renewcommand*{\arraystretch}{1.5}



\providecommand{\ascline}[3]{\noalign{\global\arrayrulewidth #1}\arrayrulecolor[HTML]{#2}\cline{#3}}

\begin{longtable}[c]{|p{1.41in}|p{0.54in}|p{0.68in}|p{0.54in}|p{0.72in}}

\caption{\label{tbl-ft8}Tabela com seleção esécial de células}

\tabularnewline

\hhline{>{\arrayrulecolor[HTML]{666666}\global\arrayrulewidth=1.5pt}->{\arrayrulecolor[HTML]{666666}\global\arrayrulewidth=1.5pt}->{\arrayrulecolor[HTML]{666666}\global\arrayrulewidth=1.5pt}->{\arrayrulecolor[HTML]{666666}\global\arrayrulewidth=1.5pt}->{\arrayrulecolor[HTML]{666666}\global\arrayrulewidth=1.5pt}-}

\multicolumn{5}{>{\raggedright}m{\dimexpr 3.88in+8\tabcolsep}}{\textcolor[HTML]{000000}{\fontsize{11}{11}\selectfont{\global\setmainfont{Arial}{TABELA\ 1:\ Pesos\ dos\ RN\ de\ acordo\ com\ a\ OMS}}}} \\

\noalign{\global\arrayrulewidth 0pt}\arrayrulecolor[HTML]{000000}

\hhline{>{\arrayrulecolor[HTML]{666666}\global\arrayrulewidth=1.5pt}->{\arrayrulecolor[HTML]{666666}\global\arrayrulewidth=1.5pt}->{\arrayrulecolor[HTML]{666666}\global\arrayrulewidth=1.5pt}->{\arrayrulecolor[HTML]{666666}\global\arrayrulewidth=1.5pt}->{\arrayrulecolor[HTML]{666666}\global\arrayrulewidth=1.5pt}-}



\multicolumn{1}{>{\raggedright}m{\dimexpr 1.41in+0\tabcolsep}}{\textcolor[HTML]{000000}{\fontsize{11}{11}\selectfont{\global\setmainfont{Arial}{Classificação}}}} & \multicolumn{1}{>{\raggedleft}m{\dimexpr 0.54in+0\tabcolsep}}{\textcolor[HTML]{000000}{\fontsize{11}{11}\selectfont{\global\setmainfont{Arial}{f}}}} & \multicolumn{1}{>{\raggedleft}m{\dimexpr 0.68in+0\tabcolsep}}{\textcolor[HTML]{000000}{\fontsize{11}{11}\selectfont{\global\setmainfont{Arial}{fr\ (\%)}}}} & \multicolumn{1}{>{\raggedleft}m{\dimexpr 0.54in+0\tabcolsep}}{\textcolor[HTML]{000000}{\fontsize{11}{11}\selectfont{\global\setmainfont{Arial}{F}}}} & \multicolumn{1}{>{\raggedleft}m{\dimexpr 0.72in+0\tabcolsep}}{\textcolor[HTML]{000000}{\fontsize{11}{11}\selectfont{\global\setmainfont{Arial}{Fr\ (\%)}}}} \\

\noalign{\global\arrayrulewidth 0pt}\arrayrulecolor[HTML]{000000}

\hhline{>{\arrayrulecolor[HTML]{666666}\global\arrayrulewidth=1.5pt}->{\arrayrulecolor[HTML]{666666}\global\arrayrulewidth=1.5pt}->{\arrayrulecolor[HTML]{666666}\global\arrayrulewidth=1.5pt}->{\arrayrulecolor[HTML]{666666}\global\arrayrulewidth=1.5pt}->{\arrayrulecolor[HTML]{666666}\global\arrayrulewidth=1.5pt}-}\endfirsthead 

\hhline{>{\arrayrulecolor[HTML]{666666}\global\arrayrulewidth=1.5pt}->{\arrayrulecolor[HTML]{666666}\global\arrayrulewidth=1.5pt}->{\arrayrulecolor[HTML]{666666}\global\arrayrulewidth=1.5pt}->{\arrayrulecolor[HTML]{666666}\global\arrayrulewidth=1.5pt}->{\arrayrulecolor[HTML]{666666}\global\arrayrulewidth=1.5pt}-}

\multicolumn{5}{>{\raggedright}m{\dimexpr 3.88in+8\tabcolsep}}{\textcolor[HTML]{000000}{\fontsize{11}{11}\selectfont{\global\setmainfont{Arial}{TABELA\ 1:\ Pesos\ dos\ RN\ de\ acordo\ com\ a\ OMS}}}} \\

\noalign{\global\arrayrulewidth 0pt}\arrayrulecolor[HTML]{000000}

\hhline{>{\arrayrulecolor[HTML]{666666}\global\arrayrulewidth=1.5pt}->{\arrayrulecolor[HTML]{666666}\global\arrayrulewidth=1.5pt}->{\arrayrulecolor[HTML]{666666}\global\arrayrulewidth=1.5pt}->{\arrayrulecolor[HTML]{666666}\global\arrayrulewidth=1.5pt}->{\arrayrulecolor[HTML]{666666}\global\arrayrulewidth=1.5pt}-}



\multicolumn{1}{>{\raggedright}m{\dimexpr 1.41in+0\tabcolsep}}{\textcolor[HTML]{000000}{\fontsize{11}{11}\selectfont{\global\setmainfont{Arial}{Classificação}}}} & \multicolumn{1}{>{\raggedleft}m{\dimexpr 0.54in+0\tabcolsep}}{\textcolor[HTML]{000000}{\fontsize{11}{11}\selectfont{\global\setmainfont{Arial}{f}}}} & \multicolumn{1}{>{\raggedleft}m{\dimexpr 0.68in+0\tabcolsep}}{\textcolor[HTML]{000000}{\fontsize{11}{11}\selectfont{\global\setmainfont{Arial}{fr\ (\%)}}}} & \multicolumn{1}{>{\raggedleft}m{\dimexpr 0.54in+0\tabcolsep}}{\textcolor[HTML]{000000}{\fontsize{11}{11}\selectfont{\global\setmainfont{Arial}{F}}}} & \multicolumn{1}{>{\raggedleft}m{\dimexpr 0.72in+0\tabcolsep}}{\textcolor[HTML]{000000}{\fontsize{11}{11}\selectfont{\global\setmainfont{Arial}{Fr\ (\%)}}}} \\

\noalign{\global\arrayrulewidth 0pt}\arrayrulecolor[HTML]{000000}

\hhline{>{\arrayrulecolor[HTML]{666666}\global\arrayrulewidth=1.5pt}->{\arrayrulecolor[HTML]{666666}\global\arrayrulewidth=1.5pt}->{\arrayrulecolor[HTML]{666666}\global\arrayrulewidth=1.5pt}->{\arrayrulecolor[HTML]{666666}\global\arrayrulewidth=1.5pt}->{\arrayrulecolor[HTML]{666666}\global\arrayrulewidth=1.5pt}-}\endhead



\multicolumn{1}{>{\raggedright}m{\dimexpr 1.41in+0\tabcolsep}}{\textcolor[HTML]{000000}{\fontsize{11}{11}\selectfont{\global\setmainfont{Arial}{BP\ Extremo}}}} & \multicolumn{1}{>{\raggedleft}m{\dimexpr 0.54in+0\tabcolsep}}{\textcolor[HTML]{000000}{\fontsize{11}{11}\selectfont{\global\setmainfont{Arial}{2}}}} & \multicolumn{1}{>{\raggedleft}m{\dimexpr 0.68in+0\tabcolsep}}{\textcolor[HTML]{000000}{\fontsize{11}{11}\selectfont{\global\setmainfont{Arial}{1.2}}}} & \multicolumn{1}{>{\raggedleft}m{\dimexpr 0.54in+0\tabcolsep}}{\textcolor[HTML]{000000}{\fontsize{11}{11}\selectfont{\global\setmainfont{Arial}{2}}}} & \multicolumn{1}{>{\cellcolor[HTML]{FAFBFC}\raggedleft}m{\dimexpr 0.72in+0\tabcolsep}}{\textcolor[HTML]{000000}{\fontsize{11}{11}\selectfont{\global\setmainfont{Arial}{1.2}}}} \\

\noalign{\global\arrayrulewidth 0pt}\arrayrulecolor[HTML]{000000}





\multicolumn{1}{>{\raggedright}m{\dimexpr 1.41in+0\tabcolsep}}{\textcolor[HTML]{000000}{\fontsize{11}{11}\selectfont{\global\setmainfont{Arial}{Muito\ BP}}}} & \multicolumn{1}{>{\raggedleft}m{\dimexpr 0.54in+0\tabcolsep}}{\textcolor[HTML]{000000}{\fontsize{11}{11}\selectfont{\global\setmainfont{Arial}{5}}}} & \multicolumn{1}{>{\raggedleft}m{\dimexpr 0.68in+0\tabcolsep}}{\textcolor[HTML]{000000}{\fontsize{11}{11}\selectfont{\global\setmainfont{Arial}{2.9}}}} & \multicolumn{1}{>{\raggedleft}m{\dimexpr 0.54in+0\tabcolsep}}{\textcolor[HTML]{000000}{\fontsize{11}{11}\selectfont{\global\setmainfont{Arial}{7}}}} & \multicolumn{1}{>{\cellcolor[HTML]{EDF2F5}\raggedleft}m{\dimexpr 0.72in+0\tabcolsep}}{\textcolor[HTML]{000000}{\fontsize{11}{11}\selectfont{\global\setmainfont{Arial}{4.1}}}} \\

\noalign{\global\arrayrulewidth 0pt}\arrayrulecolor[HTML]{000000}





\multicolumn{1}{>{\raggedright}m{\dimexpr 1.41in+0\tabcolsep}}{\textcolor[HTML]{000000}{\fontsize{11}{11}\selectfont{\global\setmainfont{Arial}{Baixo\ Peso\ (BP)}}}} & \multicolumn{1}{>{\raggedleft}m{\dimexpr 0.54in+0\tabcolsep}}{\textcolor[HTML]{000000}{\fontsize{11}{11}\selectfont{\global\setmainfont{Arial}{0}}}} & \multicolumn{1}{>{\raggedleft}m{\dimexpr 0.68in+0\tabcolsep}}{\textcolor[HTML]{000000}{\fontsize{11}{11}\selectfont{\global\setmainfont{Arial}{0.0}}}} & \multicolumn{1}{>{\raggedleft}m{\dimexpr 0.54in+0\tabcolsep}}{\textcolor[HTML]{000000}{\fontsize{11}{11}\selectfont{\global\setmainfont{Arial}{7}}}} & \multicolumn{1}{>{\cellcolor[HTML]{EDF2F5}\raggedleft}m{\dimexpr 0.72in+0\tabcolsep}}{\textcolor[HTML]{000000}{\fontsize{11}{11}\selectfont{\global\setmainfont{Arial}{4.1}}}} \\

\noalign{\global\arrayrulewidth 0pt}\arrayrulecolor[HTML]{000000}





\multicolumn{1}{>{\raggedright}m{\dimexpr 1.41in+0\tabcolsep}}{\textcolor[HTML]{000000}{\fontsize{11}{11}\selectfont{\global\setmainfont{Arial}{Peso\ Normal}}}} & \multicolumn{1}{>{\raggedleft}m{\dimexpr 0.54in+0\tabcolsep}}{\textcolor[HTML]{000000}{\fontsize{11}{11}\selectfont{\global\setmainfont{Arial}{163}}}} & \multicolumn{1}{>{\raggedleft}m{\dimexpr 0.68in+0\tabcolsep}}{\textcolor[HTML]{000000}{\fontsize{11}{11}\selectfont{\global\setmainfont{Arial}{95.9}}}} & \multicolumn{1}{>{\raggedleft}m{\dimexpr 0.54in+0\tabcolsep}}{\textcolor[HTML]{000000}{\fontsize{11}{11}\selectfont{\global\setmainfont{Arial}{170}}}} & \multicolumn{1}{>{\raggedleft}m{\dimexpr 0.72in+0\tabcolsep}}{\textcolor[HTML]{000000}{\fontsize{11}{11}\selectfont{\global\setmainfont{Arial}{100.0}}}} \\

\noalign{\global\arrayrulewidth 0pt}\arrayrulecolor[HTML]{000000}





\multicolumn{1}{>{\raggedright}m{\dimexpr 1.41in+0\tabcolsep}}{\textcolor[HTML]{000000}{\fontsize{11}{11}\selectfont{\global\setmainfont{Arial}{Excesso\ Peso}}}} & \multicolumn{1}{>{\raggedleft}m{\dimexpr 0.54in+0\tabcolsep}}{\textcolor[HTML]{000000}{\fontsize{11}{11}\selectfont{\global\setmainfont{Arial}{0}}}} & \multicolumn{1}{>{\raggedleft}m{\dimexpr 0.68in+0\tabcolsep}}{\textcolor[HTML]{000000}{\fontsize{11}{11}\selectfont{\global\setmainfont{Arial}{0.0}}}} & \multicolumn{1}{>{\raggedleft}m{\dimexpr 0.54in+0\tabcolsep}}{\textcolor[HTML]{000000}{\fontsize{11}{11}\selectfont{\global\setmainfont{Arial}{170}}}} & \multicolumn{1}{>{\raggedleft}m{\dimexpr 0.72in+0\tabcolsep}}{\textcolor[HTML]{000000}{\fontsize{11}{11}\selectfont{\global\setmainfont{Arial}{100.0}}}} \\

\noalign{\global\arrayrulewidth 0pt}\arrayrulecolor[HTML]{000000}

\hhline{>{\arrayrulecolor[HTML]{666666}\global\arrayrulewidth=1.5pt}->{\arrayrulecolor[HTML]{666666}\global\arrayrulewidth=1.5pt}->{\arrayrulecolor[HTML]{666666}\global\arrayrulewidth=1.5pt}->{\arrayrulecolor[HTML]{666666}\global\arrayrulewidth=1.5pt}->{\arrayrulecolor[HTML]{666666}\global\arrayrulewidth=1.5pt}-}



\multicolumn{5}{>{\raggedright}m{\dimexpr 3.88in+8\tabcolsep}}{\textcolor[HTML]{000000}{\fontsize{9}{9}\selectfont{\global\setmainfont{Arial}{FONTE:\ Hospital\ Geral,\ Caxias\ do\ Sul,\ RS,\ 2008}}}} \\

\noalign{\global\arrayrulewidth 0pt}\arrayrulecolor[HTML]{FFFFFF}




\end{longtable}

\arrayrulecolor[HTML]{000000}

\global\setlength{\arrayrulewidth}{\Oldarrayrulewidth}

\global\setlength{\tabcolsep}{\Oldtabcolsep}

\renewcommand*{\arraystretch}{1}

Muitas outras funções existem para alterar o layout, dependendo do que o
pesquisador deseja mostrar. Para isso, o \texttt{flextable} tem funções
para altera as bordas, as fontes, formato, cores, alinhamento, etc. que
podem ser pesquisadas na
\href{https://cran.r-project.org/web/packages/flextable/flextable.pdf}{documentação}
do pacote.

\subsection{Pacote gtsummary}\label{pacote-gtsummary}

O pacote \texttt{gtsummary} foi criado @daniel2021gtsummary como
complemento do pacote \texttt{gt}. Para usar o pacote , ele deve estar
instalado e carregado:

\begin{Shaded}
\begin{Highlighting}[]
\FunctionTok{library}\NormalTok{(gtsummary)}
\end{Highlighting}
\end{Shaded}

O pacote \texttt{gtsummary} possui uma função \texttt{tbl\_summary()}
que permite sumarizar um dataframe.\\
Ela calcula e apresenta automaticamente estatísticas resumidas para
variáveis contínuas, categóricas e dicotômicas dentro de uma estrutura
de dados, formatadas para tabelas prontas para publicação. Detecta
automaticamente os tipos de variáveis e aplica estatísticas resumidas
padrão apropriadas (por exemplo, mediana e intervalo interquarti, média
e desvio padrão para variáveis contínuas, contagens e porcentagens para
variáveis categóricas).\\
Para maiores detalhes consulte a
\href{\%5Bdocumentação\%5D(https://cran.r-project.org/web/packages/flextable/flextable.pdf)}{vinheta}
da função.\\
\#\#\#\# Dados para o exemplo

Como exemplo, será usado um dataframe originário do conjunto de dados
\texttt{dadosMater.xlsx} (veja Seção~\ref{sec-dados7}), contento uma
amostra de 200 observações com as seguintes variáveis relacionadas aos
recém-nascidos:

\begin{Shaded}
\begin{Highlighting}[]
\FunctionTok{library}\NormalTok{(dplyr)}
\FunctionTok{library}\NormalTok{(readxl)}

\FunctionTok{set.seed}\NormalTok{(}\DecValTok{1234}\NormalTok{)}
\NormalTok{dados }\OtherTok{\textless{}{-}}\NormalTok{ readxl}\SpecialCharTok{::}\FunctionTok{read\_excel}\NormalTok{(}\StringTok{"dados/dadosMater.xlsx"}\NormalTok{) }\SpecialCharTok{\%\textgreater{}\%} 
  \FunctionTok{mutate}\NormalTok{(}\AttributeTok{eCivil =} \FunctionTok{factor}\NormalTok{(eCivil, }
                         \AttributeTok{levels =} \FunctionTok{c}\NormalTok{(}\DecValTok{1}\NormalTok{,}\DecValTok{2}\NormalTok{), }
                         \AttributeTok{labels =} \FunctionTok{c}\NormalTok{(}\StringTok{"Sem companheiro"}\NormalTok{, }\StringTok{"Com companheiro"}\NormalTok{)),}
         \AttributeTok{tipoParto =} \FunctionTok{factor}\NormalTok{(tipoParto, }
                            \AttributeTok{levels =} \FunctionTok{c}\NormalTok{(}\DecValTok{1}\NormalTok{,}\DecValTok{2}\NormalTok{), }
                            \AttributeTok{labels =} \FunctionTok{c}\NormalTok{(}\StringTok{"Normal"}\NormalTok{, }\StringTok{"Cesareo"}\NormalTok{)),}
         \AttributeTok{fumo =} \FunctionTok{factor}\NormalTok{(fumo, }
                       \AttributeTok{levels =} \FunctionTok{c}\NormalTok{(}\DecValTok{1}\NormalTok{,}\DecValTok{2}\NormalTok{), }
                       \AttributeTok{labels =} \FunctionTok{c}\NormalTok{(}\StringTok{"Fumante"}\NormalTok{, }\StringTok{"Não fumante"}\NormalTok{)),}
         \AttributeTok{obito =} \FunctionTok{factor}\NormalTok{(obito, }
                        \AttributeTok{levels =} \FunctionTok{c}\NormalTok{(}\DecValTok{1}\NormalTok{,}\DecValTok{2}\NormalTok{), }
                        \AttributeTok{labels =} \FunctionTok{c}\NormalTok{(}\StringTok{"Sim"}\NormalTok{, }\StringTok{"Não"}\NormalTok{)),}
         \AttributeTok{categIdade =} \FunctionTok{case\_when}\NormalTok{(}
\NormalTok{           idadeMae }\SpecialCharTok{\textless{}} \DecValTok{20} \SpecialCharTok{\textasciitilde{}} \StringTok{"\textless{} 20 anos"}\NormalTok{,}
\NormalTok{           idadeMae }\SpecialCharTok{\textgreater{}=} \DecValTok{20} \SpecialCharTok{\&}\NormalTok{ idadeMae }\SpecialCharTok{\textless{}=} \DecValTok{35} \SpecialCharTok{\textasciitilde{}} \StringTok{"20 a 35 anos"}\NormalTok{,}
\NormalTok{           idadeMae }\SpecialCharTok{\textgreater{}} \DecValTok{35} \SpecialCharTok{\textasciitilde{}} \StringTok{"\textgreater{} 35 anos"}\NormalTok{),}
         \AttributeTok{categIdade =} \FunctionTok{factor}\NormalTok{(categIdade, }
                             \AttributeTok{levels =} \FunctionTok{c}\NormalTok{(}\StringTok{"\textless{} 20 anos"}\NormalTok{, }\StringTok{"20 a 35 anos"}\NormalTok{, }\StringTok{"\textgreater{} 35 anos"}\NormalTok{)),}
         \AttributeTok{sexo =} \FunctionTok{factor}\NormalTok{(sexo, }
                       \AttributeTok{levels =} \FunctionTok{c}\NormalTok{(}\DecValTok{1}\NormalTok{,}\DecValTok{2}\NormalTok{), }
                       \AttributeTok{labels =} \FunctionTok{c}\NormalTok{(}\StringTok{"Masculino"}\NormalTok{, }\StringTok{"Feminino"}\NormalTok{)),}
         \AttributeTok{categIg =} \FunctionTok{case\_when}\NormalTok{(}
\NormalTok{           ig }\SpecialCharTok{\textless{}} \DecValTok{37} \SpecialCharTok{\textasciitilde{}} \StringTok{"RN Pré{-}termo"}\NormalTok{,}
\NormalTok{           ig }\SpecialCharTok{\textgreater{}=} \DecValTok{37} \SpecialCharTok{\&}\NormalTok{ ig }\SpecialCharTok{\textless{}} \DecValTok{42} \SpecialCharTok{\textasciitilde{}} \StringTok{"RN a Termo"}\NormalTok{,}
\NormalTok{           ig }\SpecialCharTok{\textgreater{}=} \DecValTok{42} \SpecialCharTok{\textasciitilde{}} \StringTok{"RN Pós{-}termo"}\NormalTok{),}
         \AttributeTok{categIg =} \FunctionTok{factor}\NormalTok{(categIg, }
                          \AttributeTok{levels =} \FunctionTok{c}\NormalTok{(}\StringTok{"RN Pré{-}termo"}\NormalTok{, }\StringTok{"RN a Termo"}\NormalTok{, }\StringTok{"RN Pós{-}termo"}\NormalTok{))) }\SpecialCharTok{\%\textgreater{}\%}  
\NormalTok{  dplyr}\SpecialCharTok{::}\FunctionTok{select}\NormalTok{(categIdade, eCivil, anosEst, renda, fumo, tipoParto, sexo,}
\NormalTok{                pesoRN, compRN, apgar1, obito) }\SpecialCharTok{\%\textgreater{}\%} 
  \FunctionTok{slice\_sample}\NormalTok{(}\AttributeTok{n=}\DecValTok{200}\NormalTok{)}
\end{Highlighting}
\end{Shaded}

\subsubsection{Construção da tabela}\label{construuxe7uxe3o-da-tabela}

\begin{Shaded}
\begin{Highlighting}[]
\FunctionTok{tbl\_summary}\NormalTok{(dados, }
            \AttributeTok{by =}\NormalTok{ sexo,}
            \AttributeTok{missing =} \StringTok{"ifany"}\NormalTok{,}
            \AttributeTok{type =} \FunctionTok{list}\NormalTok{(pesoRN }\SpecialCharTok{\textasciitilde{}} \StringTok{"continuous2"}\NormalTok{,}
\NormalTok{                        compRN }\SpecialCharTok{\textasciitilde{}} \StringTok{"continuous2"}\NormalTok{),}
            \FunctionTok{list}\NormalTok{(categIdade }\SpecialCharTok{\textasciitilde{}} \StringTok{"Faixa Etária"}\NormalTok{,}
\NormalTok{                 eCivil }\SpecialCharTok{\textasciitilde{}} \StringTok{"Estado Civil"}\NormalTok{,}
\NormalTok{                 anosEst }\SpecialCharTok{\textasciitilde{}}\StringTok{"Anos de estudo completos"}\NormalTok{,}
\NormalTok{                 renda }\SpecialCharTok{\textasciitilde{}} \StringTok{"Renda Familiar (SM)"}\NormalTok{,}
\NormalTok{                 fumo }\SpecialCharTok{\textasciitilde{}} \StringTok{"Tabagismo"}\NormalTok{,}
\NormalTok{                 tipoParto }\SpecialCharTok{\textasciitilde{}}\StringTok{"Tipo de Parto"}\NormalTok{,}
\NormalTok{                 pesoRN }\SpecialCharTok{\textasciitilde{}} \StringTok{"Peso RN (g)"}\NormalTok{,}
\NormalTok{                 compRN }\SpecialCharTok{\textasciitilde{}} \StringTok{"Comp RN (cm)"}\NormalTok{,}
\NormalTok{                 apgar1 }\SpecialCharTok{\textasciitilde{}} \StringTok{"Apgar primeiro min"}\NormalTok{,}
\NormalTok{                 obito }\SpecialCharTok{\textasciitilde{}}\StringTok{"Óbito"}\NormalTok{)) }\SpecialCharTok{\%\textgreater{}\%} 
  \FunctionTok{add\_n}\NormalTok{() }\SpecialCharTok{\%\textgreater{}\%} 
  \FunctionTok{add\_p}\NormalTok{(}\AttributeTok{test =} \FunctionTok{all\_continuous}\NormalTok{() }\SpecialCharTok{\textasciitilde{}} \StringTok{"t.test"}\NormalTok{,}
        \AttributeTok{pvalue\_fun =} \SpecialCharTok{\textasciitilde{}}\FunctionTok{style\_pvalue}\NormalTok{(., }\AttributeTok{digits =} \DecValTok{2}\NormalTok{)) }\SpecialCharTok{\%\textgreater{}\%} 
  \FunctionTok{modify\_header}\NormalTok{(}\AttributeTok{label =} \StringTok{"**Variáveis**"}\NormalTok{) }\SpecialCharTok{\%\textgreater{}\%} 
  \FunctionTok{bold\_labels}\NormalTok{()}
\end{Highlighting}
\end{Shaded}

\begin{table}
\fontsize{12.0pt}{14.4pt}\selectfont
\begin{tabular*}{\linewidth}{@{\extracolsep{\fill}}lcccc}
\toprule
\textbf{Variáveis} & \textbf{N} & \textbf{Masculino}  N = 108\textsuperscript{\textit{1}} & \textbf{Feminino}  N = 92\textsuperscript{\textit{1}} & \textbf{p-value}\textsuperscript{\textit{2}} \\ 
\midrule\addlinespace[2.5pt]
{\bfseries Faixa Etária} & 200 &  &  & 0.39 \\ 
    < 20 anos &  & 16 (15\%) & 10 (11\%) &  \\ 
    20 a 35 anos &  & 78 (72\%) & 74 (80\%) &  \\ 
    > 35 anos &  & 14 (13\%) & 8 (8.7\%) &  \\ 
{\bfseries Estado Civil} & 200 &  &  & 0.82 \\ 
    Sem companheiro &  & 29 (27\%) & 26 (28\%) &  \\ 
    Com companheiro &  & 79 (73\%) & 66 (72\%) &  \\ 
{\bfseries Anos de estudo completos} & 200 & 8.00 (5.00, 10.00) & 7.50 (5.00, 11.00) & 0.55 \\ 
{\bfseries Renda Familiar (SM)} & 200 & 1.93 (1.45, 2.89) & 1.92 (1.45, 2.53) & 0.86 \\ 
{\bfseries Tabagismo} & 200 &  &  & 0.046 \\ 
    Fumante &  & 15 (14\%) & 23 (25\%) &  \\ 
    Não fumante &  & 93 (86\%) & 69 (75\%) &  \\ 
{\bfseries Tipo de Parto} & 200 &  &  & 0.17 \\ 
    Normal &  & 54 (50\%) & 55 (60\%) &  \\ 
    Cesareo &  & 54 (50\%) & 37 (40\%) &  \\ 
{\bfseries Peso RN (g)} & 200 &  &  & 0.21 \\ 
    Median (Q1, Q3) &  & 3,080 (2,703, 3,428) & 2,873 (2,525, 3,325) &  \\ 
{\bfseries Comp RN (cm)} & 200 &  &  & 0.34 \\ 
    Median (Q1, Q3) &  & 48.0 (45.3, 49.0) & 47.0 (44.8, 48.5) &  \\ 
{\bfseries Apgar primeiro min} & 173 & 8.00 (8.00, 9.00) & 9.00 (8.00, 9.00) & 0.077 \\ 
    Unknown &  & 15 & 12 &  \\ 
{\bfseries Óbito} & 200 &  &  & >0.99 \\ 
    Sim &  & 1 (0.9\%) & 0 (0\%) &  \\ 
    Não &  & 107 (99\%) & 92 (100\%) &  \\ 
\bottomrule
\end{tabular*}
\begin{minipage}{\linewidth}
\textsuperscript{\textit{1}}n (\%); Median (Q1, Q3)\\
\textsuperscript{\textit{2}}Pearson's Chi-squared test; Welch Two Sample t-test; Fisher's exact test\\
\end{minipage}
\end{table}

\chapter{Gráficos}\label{sec-graficos}

Para descrever os dados e visualizar o que está acontecendo,
recomenda-se utilizar um gráfico adequado. O que é adequado depende
principalmente do tipo de dados, bem como das características
particulares do que se quer explorar. Além disso, um gráfico em um
relatório sempre é um fator de ``impacto''. Ou seja, pode ter um efeito
positivo no leitor ou fazê-lo abandonar a leitura. Finalmente, um
gráfico de frequência pode ser utilizado para ilustrar, explicar uma
situação complexa onde palavras ou uma tabela podem ser confusos,
extensos ou de outro modo insuficiente. Por outro lado, deve-se evitar
usar gráficos onde poucas palavras expressam claramente o que se quer
mostrar. Aconselha-se que, ao analisar os dados, é importante
inspecioná-los como se fossem uma imagem, uma fotografia, ver como eles
se parecem, qual o seu aspecto, e só então pensar em interpretar os
aspectos vitais da estatística (81).\\
O R básico fornece uma grande variedade de funções para visualizar
dados, elas de uma maneira relativamente simples permitem a construção
de gráficos que facilitam a interpretação tanto de variáveis categórica
como numéricas. Existe uma farta bibliografia para a construção de
gráficos, utilizando o R básico, mas a
\href{https://r-graph-gallery.com/}{The R Graph Gallery} responde a
maioria das dúvidas, apesar de ter seu foco em \texttt{tidyverse} (veja
Seção~\ref{sec-tidyverse}) e \texttt{ggplot2.}\\
Neste livro, a ênfase será no pacote \texttt{ggplot2} (82). Este pacote
é uma ferramenta extremamente versátil que oferece uma estrutura com
grande flexibilidade para exibir os dados através de gráficos. O
\texttt{gpglot2} não é apenas uma instrumento para criar gráficos, mas
uma maneira de pensar sobre a visualização de dados de uma forma mais
estruturada e poderosa.

\section{Pacotes necessários neste
capítulo}\label{pacotes-necessuxe1rios-neste-capuxedtulo}

Certifique-se que estes pacotes estejam instalados e carregados,
utilizando o pacote \texttt{pacman} (consulte a Seção~\ref{sec-pacman}):

\begin{Shaded}
\begin{Highlighting}[]
\NormalTok{pacman}\SpecialCharTok{::}\FunctionTok{p\_load}\NormalTok{(tidyverse, readxl, scales, ggsci,  paletteer, knitr, RColorBrewer, scico)}
\end{Highlighting}
\end{Shaded}

\section{Fonte de dados para este capítulo}\label{sec-dados8}

Os dados serão provenientes do conjunto de dados apresentado na
Seção~\ref{sec-dadosMater}, denominado de \texttt{dadosMater.xlsx} e que
pode ser encontrado para baixar
\href{https://github.com/petronioliveira/Arquivos/blob/main/dadosMater.xlsx}{aqui}.Serão
selecionadas as variáveis de interesse e adicionadas variáveis que
categorizam a idade da mãe e a intensidade do tabagismo durante a
gestação\footnote{\textbf{categIdade} \(\to\) \textless{} 20 anos, 20 a
  35 anos e \textgreater{} 35 anos; \textbf{categFumo} \(\to\) Não
  fumante, Fumante leve: \textless= 10 cigarros/dia, Fumante moderada:
  \textgreater{} 10 a \textless{} 20 cigarros/dia, Fumante pesada:
  \textgreater= 20 cigarros/ dia}. Além disso, serão feitas
transformações para fatores das variáveis numéricas que na realidade são
fatores. O código inicia com a semente \texttt{set.seed()} para garantir
a repetibilidade.

\begin{Shaded}
\begin{Highlighting}[]
\FunctionTok{set.seed}\NormalTok{(}\DecValTok{123}\NormalTok{)}
\NormalTok{dados }\OtherTok{\textless{}{-}}\NormalTok{ readxl}\SpecialCharTok{::}\FunctionTok{read\_excel}\NormalTok{(}\StringTok{"dados/dadosMater.xlsx"}\NormalTok{) }\SpecialCharTok{\%\textgreater{}\%}
  \FunctionTok{select}\NormalTok{(idadeMae,fumo, quantFumo, ig, pesoRN, compRN, para, sexo) }\SpecialCharTok{\%\textgreater{}\%} 
  \FunctionTok{mutate}\NormalTok{(}\AttributeTok{fumo =} \FunctionTok{factor}\NormalTok{(fumo, }
                       \AttributeTok{levels =} \FunctionTok{c}\NormalTok{(}\DecValTok{1}\NormalTok{,}\DecValTok{2}\NormalTok{), }
                       \AttributeTok{labels =} \FunctionTok{c}\NormalTok{(}\StringTok{"Fumante"}\NormalTok{, }\StringTok{"Não fumante"}\NormalTok{)),}
         \AttributeTok{sexo =} \FunctionTok{factor}\NormalTok{(sexo, }
                       \AttributeTok{levels =} \FunctionTok{c}\NormalTok{(}\DecValTok{1}\NormalTok{,}\DecValTok{2}\NormalTok{), }
                       \AttributeTok{labels =} \FunctionTok{c}\NormalTok{(}\StringTok{"Masculino"}\NormalTok{, }\StringTok{"Feminino"}\NormalTok{)),}
         \AttributeTok{categIdade =} \FunctionTok{case\_when}\NormalTok{(}
\NormalTok{           idadeMae }\SpecialCharTok{\textless{}} \DecValTok{20} \SpecialCharTok{\textasciitilde{}} \StringTok{"\textless{} 20 anos"}\NormalTok{,}
\NormalTok{           idadeMae }\SpecialCharTok{\textgreater{}=} \DecValTok{20} \SpecialCharTok{\&}\NormalTok{ idadeMae }\SpecialCharTok{\textless{}=} \DecValTok{35} \SpecialCharTok{\textasciitilde{}} \StringTok{"20 a 35 anos"}\NormalTok{,}
\NormalTok{           idadeMae }\SpecialCharTok{\textgreater{}} \DecValTok{35} \SpecialCharTok{\textasciitilde{}} \StringTok{"\textgreater{} 35 anos"}\NormalTok{),}
         \AttributeTok{categIdade =} \FunctionTok{factor}\NormalTok{(categIdade, }
                             \AttributeTok{levels =} \FunctionTok{c}\NormalTok{(}\StringTok{"\textless{} 20 anos"}\NormalTok{, }
                                        \StringTok{"20 a 35 anos"}\NormalTok{,}
                                        \StringTok{"\textgreater{} 35 anos"}\NormalTok{)),}
         \AttributeTok{categFumo =} \FunctionTok{case\_when}\NormalTok{(}
\NormalTok{           quantFumo }\SpecialCharTok{==} \DecValTok{0} \SpecialCharTok{\textasciitilde{}} \StringTok{"nao\_fumante"}\NormalTok{,}
\NormalTok{           quantFumo }\SpecialCharTok{\textless{}=} \DecValTok{10} \SpecialCharTok{\textasciitilde{}}\StringTok{"fumante\_leve"}\NormalTok{,}
\NormalTok{           quantFumo }\SpecialCharTok{\textgreater{}} \DecValTok{10}  \SpecialCharTok{\&}\NormalTok{ quantFumo }\SpecialCharTok{\textless{}} \DecValTok{20} \SpecialCharTok{\textasciitilde{}} \StringTok{"fumante\_moderada"}\NormalTok{,}
\NormalTok{           quantFumo }\SpecialCharTok{\textgreater{}=} \DecValTok{20} \SpecialCharTok{\textasciitilde{}} \StringTok{"fumante\_pesada"}\NormalTok{),}
         \AttributeTok{categFumo =} \FunctionTok{factor}\NormalTok{(categFumo, }
                            \AttributeTok{levels =} \FunctionTok{c}\NormalTok{(}\StringTok{"nao\_fumante"}\NormalTok{, }
                                       \StringTok{"fumante\_leve"}\NormalTok{,}
                                       \StringTok{"fumante\_moderada"}\NormalTok{,}
                                       \StringTok{"fumante\_pesada"}\NormalTok{))) }

\FunctionTok{str}\NormalTok{(dados)}
\end{Highlighting}
\end{Shaded}

\begin{verbatim}
tibble [1,368 x 10] (S3: tbl_df/tbl/data.frame)
 $ idadeMae  : num [1:1368] 42 29 19 31 34 29 30 34 17 32 ...
 $ fumo      : Factor w/ 2 levels "Fumante","Não fumante": 2 2 2 2 2 1 1 2 2 2 ...
 $ quantFumo : num [1:1368] 0 0 0 0 0 10 20 0 0 0 ...
 $ ig        : num [1:1368] 29 33 33 33 33 33 33 33 34 34 ...
 $ pesoRN    : num [1:1368] 1035 2300 1580 1840 2475 ...
 $ compRN    : num [1:1368] 35.5 45 39 41 47 41 44 44 47 48 ...
 $ para      : num [1:1368] 5 0 0 1 2 1 2 1 0 4 ...
 $ sexo      : Factor w/ 2 levels "Masculino","Feminino": 2 2 2 2 2 2 2 2 2 2 ...
 $ categIdade: Factor w/ 3 levels "< 20 anos","20 a 35 anos",..: 3 2 1 2 2 2 2 2 1 2 ...
 $ categFumo : Factor w/ 4 levels "nao_fumante",..: 1 1 1 1 1 2 4 1 1 1 ...
\end{verbatim}

\section{Pacote ggplot2}\label{sec-ggplot2}

O \texttt{ggplot2} é um pacote da linguagem R voltado para a
visualização de dados, oferecendo uma abordagem poderosa e elegante
baseada na \emph{Gramática dos Gráficos} (83).

\subsection{Gramática dos Gráficos}\label{sec-grammar}

O R base usa funções específicas para a construção de um gráfico, por
exemplo, \texttt{hist()} para criar um histograma ou a função plot() que
é uma função mais genérica que produz um gráfico de dispersão, no R,
quando são passados a ela dois vetores numéricos ou boxplots quando são
fornecidos dados categóricos.

Tomando as variáveis \texttt{compRN} e \texttt{pesoRN} de uma amostra de
de 100 observações do conjunto \texttt{dados}, com filtro para as
gestações a termo \footnote{Gestações com idade gestacional igual ou
  acima de 37 semanas e abaixo de 42 semanas.}, será construído um
gráfico de dispersão (Figura~\ref{fig-scatter1}) com a
função\texttt{plot()} do R base:

\begin{Shaded}
\begin{Highlighting}[]
\FunctionTok{plot}\NormalTok{ (}\AttributeTok{x =}\FunctionTok{jitter}\NormalTok{(dadosRNT100}\SpecialCharTok{$}\NormalTok{compRN,}\DecValTok{10}\NormalTok{),}
      \AttributeTok{y =}\NormalTok{ dadosRNT100}\SpecialCharTok{$}\NormalTok{pesoRN,}
      \AttributeTok{ylab =} \StringTok{"Peso de Recém{-}nascido (g)"}\NormalTok{,}
      \AttributeTok{xlab =} \StringTok{"Comprimento do Recém{-}nascido (cm)"}\NormalTok{,}
      \AttributeTok{cex.axis =} \FloatTok{0.8}\NormalTok{,}
      \AttributeTok{las =} \DecValTok{1}\NormalTok{,}
      \AttributeTok{bty =} \StringTok{"L"}\NormalTok{,}
      \AttributeTok{pch =} \DecValTok{19}\NormalTok{,}
      \AttributeTok{cex =} \FloatTok{1.5}\NormalTok{,}
      \AttributeTok{col =} \StringTok{"cyan4"}\NormalTok{)}
\end{Highlighting}
\end{Shaded}

\begin{figure}[H]

\centering{

\includegraphics[width=0.8\linewidth,height=\textheight,keepaspectratio]{08-graficos_files/figure-pdf/fig-scatter1-1.pdf}

}

\caption{\label{fig-scatter1}Gráfico de dispersão produzido com uma
função específica plot(), do R base}

\end{figure}%

Este gráfico não difere muito do gráfico da Figura~\ref{fig-scatter5} no
seu aspecto final, produzido no \texttt{ggplot2}. Entretanto, a
filosofia de construção dos gráficos é muito diferente.

Em vez de se pensar em funções específicas para cada tipo de gráfico, o
\texttt{ggplot2} permite que se construa gráficos combinando diferentes
componentes. Pode-se pensar nisso como se fosse a montagem de um
quebra-cabeça, onde cada peça representa uma parte do gráfico. Essa
abordagem modular e intuitiva é o que torna o \texttt{ggplot2} tão
flexível e poderoso.

Os principais componentes combinados para criar um gráfico no
\texttt{ggplot2} são:

\begin{itemize}
\item
  \textbf{Dados (Data)}: cada camada deve ter dados associados. É o
  conjunto de dados que se quer visualizar. Geralmente, ele deve estar
  em um formato \textbf{tidy} (arrumado), onde cada coluna é uma
  variável e cada linha é uma observação (Seção~\ref{sec-tibble}).
\item
  \textbf{Mapeamentos Estéticos}: Os mapeamentos estéticos são definidos
  com a função \texttt{aes()}. A parte onde se associa as variáveis do
  conjunto de dados a propriedades visuais do gráfico, como os eixos
  \emph{x} e \emph{y}, a cor, o tamanho, a forma e a transparência. Por
  exemplo, é possível mapear a variável peso do recém-nascido para o
  eixo \emph{y} e a variável comprimento do recém-nascido para o eixo
  \emph{x}. Os mapeamentos estéticos podem ser fornecidos na
  \texttt{ggplot()} , chamada inicial, em camadas individuais ou em uma
  combinação de ambos. Todas essas chamadas criam a mesma especificação
  de plotagem.

  Escalas (\emph{Scales}) podem ser usadas para controlar a forma dos
  mapeamentos estéticos. Pode-se usar escalas para ajustar cores,
  tamanhos e a aparência dos eixos.
\item
  \textbf{Geometrias (Geoms)}: Formas geométricas que representam os
  dados. É aqui que se define o tipo de gráfico a ser criado. Alguns
  exemplos são:

  \begin{itemize}
  \item
    \texttt{geom\_point()} para um gráfico de dispersão (pontos).
  \item
    \texttt{geom\_line()} para um gráfico de linha.
  \item
    \texttt{geom\_bar()} para um gráfico de barras.
  \item
    \texttt{geom\_errorbar()}: barras de erro.
  \item
    \texttt{geom\_bar(stat\ =\ "identity")}: um gráfico de barras de
    resumos pré-calculados
  \item
    \texttt{geom\_histogram()} para um histograma.
  \item
    \texttt{geom\_boxplot()} para um boxplot.
  \end{itemize}

  Pode-se adicionar múltiplas camadas que permitem combinar diferentes
  tipos de geoms em uma única visualização. Por exemplo, é possível
  colocar uma linha de tendência (\texttt{geom\_smooth()}) em cima de um
  gráfico de dispersão (\texttt{geom\_point()}), veja a
  Figura~\ref{fig-scatter12}.
\item
  \textbf{Estatísticas}: O \texttt{ggplot2} não permite a colocação
  direta de estatísticas dentro dos geoms. A forma mais comum de
  adicionar estatísticas em um gráfico é usando a função
  \texttt{stat\_}. As funções \texttt{stat\_} calculam as estatísticas
  (como média, mediana, contagem) e, em seguida, as representam no
  gráfico, criando uma camada de dados calculados.

  \begin{itemize}
  \item
    \texttt{stat\_summary}: adiciona um resumo estatístico a cada grupo.
    Usada para adicionar a média, mediana ou desvio padrão em um geom já
    existente, como um gráfico de dispersão ou de barras.
  \item
    \texttt{stat\_smooth}: Adiciona uma linha de tendência (como
    regressão linear) com um intervalo de confiança.
  \item
    \texttt{stat\_bin}: Calcula a contagem de cada \emph{bin} e os plota
    como um histograma.
  \end{itemize}

  A partir das últimas versões do \texttt{ggplot2}, é possível mapear
  variáveis calculadas pelo \texttt{stat\_} diretamente em uma estética,
  como \emph{y}, \emph{size} ou \emph{label}. Isso dá uma grande
  flexibilidade.

  Em vez de usar as funções \texttt{stat\_}, é possível calcular as
  estatísticas em um passo separado usando o \texttt{dplyr} e, em
  seguida, plotar esses dados pré-calculados usando \texttt{ggplot2}.

  \begin{itemize}
  \tightlist
  \item
    Calcule as estatísticas usando \texttt{group\_by()} e
    \texttt{summarize()} do \texttt{dplyr}.
  \item
    Crie o gráfico usando os novos dados calculados.
  \end{itemize}
\item
  \textbf{Ajustes de posição}: aplicam pequenos ajustes na posição dos
  elementos dentro de uma camada. Por exemplos, há três ajustes que são
  úteis em gráficos de pontos

  \begin{itemize}
  \item
    \texttt{position\_nudge()}: mover pontos por um deslocamento fixo.
  \item
    \texttt{position\_jitter()}: adicione um pouco de ruído aleatório a
    cada posição (Figura~\ref{fig-scatter4}).
  \item
    \texttt{position\_jitterdodge()}: desviar de pontos dentro de grupos
    e depois adicionar um pouco de ruído aleatório. Na construção do
    gráfico de dispersão será mostrado exemplo desses ajustes
    (\textbf{?@sec-atributos}).
  \end{itemize}

  Para o gráfico de barras, pode-se aplicar alguns ajustes:\\

  \begin{itemize}
  \tightlist
  \item
    \texttt{position\_stack()}: empilhar barras (ou áreas) sobrepostas
    umas sobre as outras.
  \item
    \texttt{position\_fill()}: empilhe barras sobrepostas, dimensionando
    para que o topo esteja sempre em 1.
  \item
    \texttt{position\_dodge()}: coloque barras sobrepostas (ou boxplots)
    lado a lado.
  \end{itemize}
\item
  \textbf{Facetas (Facets)}: Permitem criar múltiplos subgráficos
  baseados em uma ou mais variáveis categóricas. É uma ótima maneira de
  explorar as relações entre diferentes grupos de dados de forma visual
  (Figura~\ref{fig-scatter11}).
\item
  \textbf{Temas (Themes)}: Controlam a aparência geral do gráfico, como
  a cor de fundo, a fonte, as linhas da grade e a aparência dos títulos.
  O \texttt{ggplot2} permite construir gráficos complexos camada por
  camada, possibilitando a criação de gráficos sofisticados (84). Em
  cada uma das camadas deve-se ter preocupação em controlar os
  componentes do gráfico.
\end{itemize}

\subsection{Vantagens do ggplot2}\label{vantagens-do-ggplot2}

\begin{itemize}
\item
  \textbf{Consistência e Flexibilidade}: A abordagem de camadas e a
  gramática de gráficos permitem criar uma variedade enorme de
  visualizações de forma consistente.
\item
  \textbf{Qualidade Visual}: Os gráficos produzidos pelo ggplot2 são
  esteticamente agradáveis e prontos para publicações.
\item
  \textbf{Intuitividade}: Uma vez que se entenda o conceito, é fácil
  construir gráficos complexos de forma gradual.
\item
  \textbf{Extensibilidade}: O pacote pode ser estendido com outros
  pacotes que adicionam novas geoms, temas ou funcionalidades, como o
  gganimate para animações ou o ggrepel para evitar sobreposição de
  rótulos.
\end{itemize}

\section{Gráfico de dispersão}\label{sec-scatter}

Um gráfico de dispersão (\emph{Scatterplot}) exibe a relação entre duas
variáveis numéricas. Cada ponto representa uma observação. Suas posições
nos eixos \emph{x} (horizontal) e \emph{y} (vertical) representam os
valores das duas variáveis.\\
O gráfico de dispersão permite identificar padrões, tendências e a força
de uma possível correlação entre essas variáveis. Frequentemente, vem
acompanhado por um cálculo do coeficiente de correlação
(\textbf{?@sec-cor}), que , em geral, mede a relação linear.

Pretende-se, na construção de um gráfico de dispersão introduzir a
lógica do \texttt{ggplot2}. Os dados usados para o exemplo, serão os
mesmos usados na construção do gráfico de dispersão com a função nativa
\texttt{plot()}(Seção~\ref{sec-grammar}).

\subsection{Gráfico de dispersão
básico}\label{gruxe1fico-de-dispersuxe3o-buxe1sico}

Este é o exemplo mais simples que começa com o mapeamento de duas
variáveis para os eixos \emph{x} e \emph{y.} A função central do pacote
\texttt{ggplot2} é \texttt{ggplot()}, que recebe os dados por meio do
argumento \texttt{data}. Em seguida, a função estética \texttt{aes()}
define os mapeamentos dos eixos \emph{x} e \emph{y}, iniciando o gráfico
com uma \textbf{camada base} --- ainda vazia, mesmo que os dados já
tenham sido fornecidos. Essa camada base corresponde a um painel cinza
(Figura~\ref{fig-baselayer}), sobre o qual outras camadas serão
adicionadas. Funciona como um terreno pronto para receber uma
construção, que será erguida com o uso de uma função geométrica.

\begin{Shaded}
\begin{Highlighting}[]
\FunctionTok{ggplot}\NormalTok{(}\AttributeTok{data =}\NormalTok{ dadosRNT100, }\FunctionTok{aes}\NormalTok{(}\AttributeTok{x =}\NormalTok{ compRN, }\AttributeTok{y =}\NormalTok{ pesoRN))}
\end{Highlighting}
\end{Shaded}

\begin{figure}[H]

\centering{

\includegraphics[width=0.8\linewidth,height=\textheight,keepaspectratio]{08-graficos_files/figure-pdf/fig-baselayer-1.pdf}

}

\caption{\label{fig-baselayer}Camada base do ggplot}

\end{figure}%

\subsection{Geometria}\label{geometria}

A seguir, adiciona-se \footnote{O adicionar aqui é literal, pois isto é
  feito com o sinal (+).} a camada dos pontos que usa a geometria
\texttt{geom\_point()} para criar um gráfico de dispersão.

\begin{Shaded}
\begin{Highlighting}[]
\FunctionTok{ggplot}\NormalTok{(}\AttributeTok{data =}\NormalTok{ dadosRNT100,}
       \AttributeTok{mapping =} \FunctionTok{aes}\NormalTok{(}\AttributeTok{x =}\NormalTok{ compRN, }\AttributeTok{y =}\NormalTok{ pesoRN)) }\SpecialCharTok{+}
  \FunctionTok{geom\_point}\NormalTok{()}
\end{Highlighting}
\end{Shaded}

\begin{figure}[H]

\centering{

\includegraphics[width=0.8\linewidth,height=\textheight,keepaspectratio]{08-graficos_files/figure-pdf/fig-scatter2-1.pdf}

}

\caption{\label{fig-scatter2}Gráfico de dispersão simples}

\end{figure}%

A Figura~\ref{fig-scatter2} mostra um gráfico de dispersão ainda sem um
aspecto elegante, mas com as informações necessárias. Tem este fundo
escuro que não é do agrado da maioria, além de não apresentar o rótulos
das variáveis de forma mais clara, mais adequada.

O mesmo resultado da Figura~\ref{fig-scatter2} pode ser obtido,
colocando o mapeamento com a estética \texttt{aes()} dentro do
\texttt{geom\_point()}:

\begin{Shaded}
\begin{Highlighting}[]
\FunctionTok{ggplot}\NormalTok{(}\AttributeTok{data =}\NormalTok{ dadosRNT100) }\SpecialCharTok{+} 
  \FunctionTok{geom\_point}\NormalTok{(}\FunctionTok{aes}\NormalTok{(}\AttributeTok{x =}\NormalTok{ compRN, }\AttributeTok{y =}\NormalTok{ pesoRN))}
\end{Highlighting}
\end{Shaded}

\subsection{Customização do gráfico de dispersão}\label{sec-custom}

A geometria \texttt{geom\_point()} múltiplas opções de customização,
através de seus argumentos:

\begin{itemize}
\tightlist
\item
  \textbf{color}: a cor do traço, o contorno do círculo
\item
  \textbf{stroke}: a largura do traço no ponto
\item
  \textbf{fill}: cor da parte interna do ponto
\item
  \textbf{shape}: forma do marcador (Figura~\ref{fig-shape})
\item
  \textbf{alpha}: transparência do ponto, varia de 0 a 1, 0 é totalmente
  transparente; 1 = opaco.
\item
  \textbf{size}: tamanho do ponto
\end{itemize}

\begin{figure}

\centering{

\includegraphics[width=0.5\linewidth,height=\textheight,keepaspectratio]{index_files/mediabag/2irU1vJ.png}

}

\caption{\label{fig-shape}Formato dos pontos disponíveis no R}

\end{figure}%

Para definir um tamanho uniforme para todos os pontos do gráfico, basta
especificar um valor numérico no argumento \texttt{size} da função
\texttt{geom\_point()}, como por exemplo \texttt{size\ =\ 1.5} (padrão).
Para melhor visualização, será escolhido \texttt{size\ =\ 3} ou
\texttt{4}. O mesmo princípio se aplica à cor. No argumento
\texttt{color} (vai colorir o contorno dos pontos), colocar, por
exemplo, \texttt{color\ =\ “gray20”} ou, se o formato (\texttt{shape}),
no argumento \texttt{fill} para a cor de preenchimento do ponto. A
escolha das cores depende do gosto pessoal, na Seção~\ref{sec-coresr},
serão mostrados alguns princípios que auxiliam esse processo. Para
alterar o formato dos pontos, usar o argumento \texttt{shape}, conforme
as opções na figura Figura~\ref{fig-shape}. Somente os formatos 21 a 25
permitem preenchimento. No exemplo, será usado \texttt{shape\ =\ 21.}
Nesse caso, é possível adicionar o argumento para definir a cor interna
do ponto, ou seja, uma cor fixa (fill = ``tomato'') ou uma variável
categórica, como \texttt{sexo}. Ao utilizar uma variável como
\texttt{fill\ =\ sexo}, o \texttt{ggplot2} preencherá os pontos com
cores diferentes automaticamente, de acordo com os níveis dessa
variável.

\begin{Shaded}
\begin{Highlighting}[]
\FunctionTok{ggplot}\NormalTok{(}\AttributeTok{data =}\NormalTok{ dadosRNT100,}
       \AttributeTok{mapping =} \FunctionTok{aes}\NormalTok{(}\AttributeTok{x =}\NormalTok{ compRN, }\AttributeTok{y =}\NormalTok{ pesoRN)) }\SpecialCharTok{+}
  \FunctionTok{geom\_point}\NormalTok{(}\AttributeTok{color =} \StringTok{"gray20"}\NormalTok{,}
             \AttributeTok{fill =}\StringTok{"tomato"}\NormalTok{,}
             \AttributeTok{alpha =} \DecValTok{1}\NormalTok{,}
             \AttributeTok{shape =} \DecValTok{21}\NormalTok{, }
             \AttributeTok{size =} \DecValTok{3}\NormalTok{,}
             \AttributeTok{stroke =}\DecValTok{1}\NormalTok{)}
\end{Highlighting}
\end{Shaded}

\begin{figure}[H]

\centering{

\includegraphics[width=0.8\linewidth,height=\textheight,keepaspectratio]{08-graficos_files/figure-pdf/fig-scatter3-1.pdf}

}

\caption{\label{fig-scatter3}Gráfico de dispersão colorido}

\end{figure}%

\subsection{Lidando com a sobreposição dos pontos e os
rótulos}\label{lidando-com-a-sobreposiuxe7uxe3o-dos-pontos-e-os-ruxf3tulos}

Na Figura~\ref{fig-scatter3} a modificação realizada melhorou o aspecto
do gráfico. Entretanto, os pontos estão se sobrepondo, porque o
comprimento dos recém-nascidos está registrado como uma variável
numérica discreta e existem vários com o mesmo comprimento. Nesse caso,
a solução para evitar a sobreposição, é provocar um pequeno deslocamento
aleatório dos pontos, tornando o gráfico mais legível. Isto é feito,
embutindo o \emph{jitter} (espalhamento) com um argumento dentro do
\texttt{geom\_point()}, o
\texttt{position\ =\ position\_jitter\ (width\ =\ 0.2,\ height\ =\ 0)}.
O argumento \texttt{width} controla o deslocamento horizontal (eixo
\emph{x}); \texttt{height} controla o deslocamento vertical (eixo
\emph{y}). No exemplo (Figura~\ref{fig-scatter4}), os pontos serão
espalhados horizontalmente, mantendo a posição vertical. Será usado o
argumento \texttt{width\ =\ 0.2} que espalha os pontos de forma leve,
sem alterar o eixo \emph{y}.

Nesta modificação, será colocada mais uma camada para trabalhar com os
rótulos dos eixos x e y., usando as funções \texttt{ylab()} e
\texttt{xlab()}.

\begin{Shaded}
\begin{Highlighting}[]
\FunctionTok{ggplot}\NormalTok{(}\AttributeTok{data =}\NormalTok{ dadosRNT100, }
       \AttributeTok{mapping =} \FunctionTok{aes}\NormalTok{(}\AttributeTok{x =}\NormalTok{ compRN, }\AttributeTok{y =}\NormalTok{ pesoRN, }\AttributeTok{fill =}\NormalTok{ sexo)) }\SpecialCharTok{+}
  \FunctionTok{geom\_point}\NormalTok{(}\AttributeTok{position =} \FunctionTok{position\_jitter}\NormalTok{(}\AttributeTok{width =} \FloatTok{0.2}\NormalTok{, }\AttributeTok{height =} \DecValTok{0}\NormalTok{),}
             \AttributeTok{color =} \StringTok{"gray20"}\NormalTok{,}
             \AttributeTok{fill =}\StringTok{"tomato2"}\NormalTok{,}
             \AttributeTok{shape =} \DecValTok{21}\NormalTok{,}
             \AttributeTok{alpha =} \DecValTok{1}\NormalTok{,}
             \AttributeTok{size =} \DecValTok{3}\NormalTok{,}
             \AttributeTok{stroke =} \DecValTok{1}\NormalTok{) }\SpecialCharTok{+}
  \FunctionTok{ylab}\NormalTok{(}\StringTok{"Peso do Recém{-}nascido (g)"}\NormalTok{) }\SpecialCharTok{+}
  \FunctionTok{xlab}\NormalTok{(}\StringTok{"Comprimento do Recém{-}nascido (cm)"}\NormalTok{) }
\end{Highlighting}
\end{Shaded}

\begin{figure}[H]

\centering{

\includegraphics[width=0.8\linewidth,height=\textheight,keepaspectratio]{08-graficos_files/figure-pdf/fig-scatter4-1.pdf}

}

\caption{\label{fig-scatter4}Gráfico de dispersão com jitter}

\end{figure}%

\subsection{Mudando o tema}\label{mudando-o-tema}

A Figura~\ref{fig-scatter4} já é um gráfico bem aceitável, praticamente
sem defeitos, apesar de o autor implicar muito com o fundo cinza --
\texttt{theme\_gray()}. Essa cor acinzentada padrão do \texttt{ggplot2}
pode ser alterada pela definição de outro tema integrado, entre muitos,
como \texttt{theme\_classic()} que é um tema de aparência clássica, com
linhas dos eixos \emph{x} e \emph{y} e sem linhas de grade, semelhante
ao da Figura~\ref{fig-scatter1}, criado com a função nativa
\texttt{plot()}. Outro tema interessante é o \texttt{theme\_bw()} que
usa um fundo branco e linhas finas de grade cinza.

Para ver outras possibilidades acesse
\href{https://ggplot2.tidyverse.org/reference/ggtheme.html}{Completes
themes - ggplt2}. Foi adicionado o argumento \texttt{base\_size\ =\ 13},
para modificar o tamaho das letras.

O gráfico da Figura~\ref{fig-scatter4} com a adição do
\texttt{theme\_classic()} e aumento dp tamanho das letras, pode ser
observado na Figura~\ref{fig-scatter5}.

\begin{Shaded}
\begin{Highlighting}[]
\FunctionTok{ggplot}\NormalTok{(}\AttributeTok{data =}\NormalTok{ dadosRNT100,}
       \AttributeTok{mapping =} \FunctionTok{aes}\NormalTok{(}\AttributeTok{x =}\NormalTok{ compRN, }\AttributeTok{y =}\NormalTok{ pesoRN)) }\SpecialCharTok{+}
  \FunctionTok{geom\_point}\NormalTok{(}\AttributeTok{position =} \FunctionTok{position\_jitter}\NormalTok{(}\AttributeTok{width =} \FloatTok{0.2}\NormalTok{, }\AttributeTok{height =} \DecValTok{0}\NormalTok{),}
             \AttributeTok{color =} \StringTok{"gray20"}\NormalTok{,}
             \AttributeTok{fill =}\StringTok{"steelblue"}\NormalTok{,}
             \AttributeTok{shape =} \DecValTok{21}\NormalTok{, }
             \AttributeTok{alpha =} \DecValTok{1}\NormalTok{,}
             \AttributeTok{size =} \DecValTok{3}\NormalTok{,}
             \AttributeTok{stroke =}\DecValTok{1}\NormalTok{) }\SpecialCharTok{+}
  \FunctionTok{ylab}\NormalTok{(}\StringTok{"Peso do Recém{-}nascido (g)"}\NormalTok{) }\SpecialCharTok{+}
  \FunctionTok{xlab}\NormalTok{(}\StringTok{"Comprimento do Recém{-}nascido (cm)"}\NormalTok{)}\SpecialCharTok{+}
  \FunctionTok{theme\_classic}\NormalTok{(}\AttributeTok{base\_size =} \DecValTok{13}\NormalTok{)}
\end{Highlighting}
\end{Shaded}

\begin{figure}[H]

\centering{

\includegraphics[width=0.8\linewidth,height=\textheight,keepaspectratio]{08-graficos_files/figure-pdf/fig-scatter5-1.pdf}

}

\caption{\label{fig-scatter5}Gráfico de dispersão com jitter e
theme\_bw()}

\end{figure}%

\subsection{Mudando as cores}\label{sec-coresr}

Na Seção~\ref{sec-custom} foi introduzido o uso de cores no R. Agora,
apesar deste tema praticamente não ter limites, serão mostrados alguns
princípios do manuseio das cores no \texttt{ggplot2.} É possível
visualizar as cores para usar no \texttt{ggplot2} de uma maneira fácil,
acessando, por exemplo,
\href{https://r-graph-gallery.com/42-colors-names.html}{An overview of
color names in R} ou um guia completo
\href{http://www.cookbook-r.com/Graphs/Colors_(ggplot2)/}{Colors
(ggplot2)}.\\
A escolha das cores pode ser feita especificando o seu nome em inglês.
Essa escolha é pessoal. O R possui 657 cores integradas que permitem uma
gama ampla de opções. Pode-se chamar uma cor pelo nome e a função
\texttt{colors()} exibe todos elas. Uma outra maneira de especificar as
cores, é usar o sistema RGB ou hexadecimal. O código hexadecimal da cor
branca é \texttt{\#FFFFFF}F, da ``gray58'' é \texttt{\#949494}, da
``yellow4'' é \texttt{\#999900}, etc. Opcionalmente, a cor pode ser
transparente, usando o formato ``\#RRGGBBAA''. \texttt{Alpha} refere-se
à transparência de um \texttt{geom\_}. Os valores de \texttt{alpha}
variam de 0 a 1, com valores mais baixos correspondendo a cores mais
transparentes. O argumento \texttt{alpha} também pode ser modificado por
meio da estética de \texttt{color} ou \texttt{fill} se qualquer uma das
estéticas fornecer valores de cor usando uma especificação RGB.

\subsubsection{Mapeando Cores com aes()}\label{mapeando-cores-com-aes}

A forma mais comum de usar cores é mapeando uma variável para a estética
\texttt{color} ou \texttt{fill} dentro da função \texttt{aes()}. Quando
se faz isso, o \texttt{ggplot2} atribui cores automaticamente, criando
uma legenda. Esta legenda pode ser modificada de posição com a função
theme(legend.position = ``bottom''), que colocará a legenda na parte
inferior do gráfico\footnote{A legenda também pode ser colocada em
  outras partes do gráfico: ``top'' (superior), ``left'' (esquerda),
  ``rigth'' (direita) e ser removida (``none'').}. No caso deste
gráfico, o título da legenda pode ser removido, porque ele é óbvio. Para
isso, basta usar a função labs() que manauseia os títulos, usando
\texttt{color} ou \texttt{fill}, dependendo se for a cor do contorno ou
do preenchimento do ponto.

\begin{itemize}
\item
  \textbf{color}: Usado para o contorno de formas, como a borda dos
  pontos ( ou como será visto adiante, as linhas de um gráfico de linha,
  ou a borda de um boxplot, de um gráfico de barra).
\item
  \textbf{fill}: Usado para preencher formas, como pontos
  (\texttt{shape} 21 a 25) ou como as barras de um histograma ou as
  caixas de um boxplot (ver adiante).
\end{itemize}

Inicialmente, serão manipuladas as cores do gráfico da
Figura~\ref{fig-scatter5}, utilizando uma variável categórica,
\texttt{fill\ =\ sexo}, dentro da função \texttt{aes()}.

\begin{Shaded}
\begin{Highlighting}[]
\FunctionTok{ggplot}\NormalTok{(}\AttributeTok{data =}\NormalTok{ dadosRNT100, }
       \AttributeTok{mapping =} \FunctionTok{aes}\NormalTok{(}\AttributeTok{x =}\NormalTok{ compRN, }\AttributeTok{y =}\NormalTok{ pesoRN, }\AttributeTok{fill =}\NormalTok{ sexo)) }\SpecialCharTok{+}
  \FunctionTok{geom\_point}\NormalTok{(}\AttributeTok{position =} \FunctionTok{position\_jitter}\NormalTok{(}\AttributeTok{width =} \FloatTok{0.2}\NormalTok{, }\AttributeTok{height =} \DecValTok{0}\NormalTok{),}
             \AttributeTok{shape =} \DecValTok{21}\NormalTok{,}
             \AttributeTok{alpha =} \DecValTok{1}\NormalTok{,}
             \AttributeTok{size =} \DecValTok{4}\NormalTok{,}
             \AttributeTok{stroke =} \DecValTok{1}\NormalTok{) }\SpecialCharTok{+}
  \FunctionTok{labs}\NormalTok{(}\AttributeTok{title=}\StringTok{""}\NormalTok{,}
       \AttributeTok{x =} \StringTok{"Comprimento do Recém{-}nascido (cm)"}\NormalTok{,}
       \AttributeTok{y =} \StringTok{"Peso do Recém{-}nascido (g)"}\NormalTok{,}
       \AttributeTok{fill =} \StringTok{""}\NormalTok{) }\SpecialCharTok{+}
  \FunctionTok{theme\_classic}\NormalTok{(}\AttributeTok{base\_size =} \DecValTok{13}\NormalTok{) }\SpecialCharTok{+}
  \FunctionTok{theme}\NormalTok{(}\AttributeTok{legend.position =} \StringTok{"bottom"}\NormalTok{)}
\end{Highlighting}
\end{Shaded}

\begin{figure}[H]

\centering{

\includegraphics[width=0.8\linewidth,height=\textheight,keepaspectratio]{08-graficos_files/figure-pdf/fig-scatter6-1.pdf}

}

\caption{\label{fig-scatter6}Gráfico de dispersão com cores de acordo
com o sexo}

\end{figure}%

Neste caso, a variável categórica \texttt{sexo} foi mapeada para a
estética fill. O \texttt{ggplot2} atribuiu uma cor diferente para cada
\texttt{sexo} e criou uma legenda automaticamente
(Figura~\ref{fig-scatter6}). No exemplo da Figura~\ref{fig-scatter5},
foi mostrado que quando a cor for única, o mapeamento da mesma se dá
dentro do \texttt{geom\_point()}.

\begin{tcolorbox}[enhanced jigsaw, bottomrule=.15mm, opacitybacktitle=0.6, colframe=quarto-callout-caution-color-frame, arc=.35mm, coltitle=black, toptitle=1mm, colback=white, colbacktitle=quarto-callout-caution-color!10!white, breakable, bottomtitle=1mm, rightrule=.15mm, titlerule=0mm, toprule=.15mm, opacityback=0, leftrule=.75mm, left=2mm, title=\textcolor{quarto-callout-caution-color}{\faFire}\hspace{0.5em}{Atenção!}]

Se estiver usando \texttt{shape} que \textbf{não aceita preenchimento}
(como 16 ou 19), então deve-se usar \texttt{color\ =\ sexo} e
\texttt{scale\_color\_manual().}

\end{tcolorbox}

\begin{tcolorbox}[enhanced jigsaw, bottomrule=.15mm, opacitybacktitle=0.6, colframe=quarto-callout-tip-color-frame, arc=.35mm, coltitle=black, toptitle=1mm, colback=white, colbacktitle=quarto-callout-tip-color!10!white, breakable, bottomtitle=1mm, rightrule=.15mm, titlerule=0mm, toprule=.15mm, opacityback=0, leftrule=.75mm, left=2mm, title=\textcolor{quarto-callout-tip-color}{\faLightbulb}\hspace{0.5em}{Exercício}]

O que acontece se a cor determinada por uma variável categórica
(\texttt{fill\ =\ sexo} ou \texttt{color=\ sexo}) for colocada fora do
\texttt{aes()}?

\end{tcolorbox}

\ul{Resposta}\footnote{Não acontece nada! O~\emph{R}~ignora o código e
  retorna um gráfico com as sua cor padrão, a preta.}

\subsubsection{Paletas de Cores}\label{paletas-de-cores}

Para ter controle total sobre as cores, deve-se usar funções de escala
(\texttt{scale}). Existem diferentes funções de escala para variáveis
categóricas e numéricas. Na Figura~\ref{fig-scatter6}, o
\texttt{ggplot2} definiu as cores automaticamente. Agora, as cores serão
personalizadas, usando a função \texttt{scale\_color\_manual()}
(Figura~\ref{fig-scatter7}).

\begin{Shaded}
\begin{Highlighting}[]
\FunctionTok{ggplot}\NormalTok{(dadosRNT100, }
       \FunctionTok{aes}\NormalTok{(}\AttributeTok{x =}\NormalTok{ compRN, }\AttributeTok{y =}\NormalTok{ pesoRN, }\AttributeTok{fill =}\NormalTok{ sexo)) }\SpecialCharTok{+}
  \FunctionTok{geom\_point}\NormalTok{(}\AttributeTok{position =} \FunctionTok{position\_jitter}\NormalTok{(}\AttributeTok{width =} \FloatTok{0.2}\NormalTok{, }\AttributeTok{height =} \DecValTok{0}\NormalTok{),}
             \AttributeTok{shape =} \DecValTok{21}\NormalTok{, }
             \AttributeTok{color =} \StringTok{"gray20"}\NormalTok{, }
             \AttributeTok{size =} \DecValTok{4}\NormalTok{, }
             \AttributeTok{stroke =} \DecValTok{1}\NormalTok{) }\SpecialCharTok{+}
  \FunctionTok{scale\_fill\_manual}\NormalTok{(}\AttributeTok{values =} \FunctionTok{c}\NormalTok{(}\AttributeTok{Masculino =} \StringTok{"cyan"}\NormalTok{,}
                               \AttributeTok{Feminino =} \StringTok{"pink3"}\NormalTok{)) }\SpecialCharTok{+}
  \FunctionTok{labs}\NormalTok{(}\AttributeTok{title=}\StringTok{""}\NormalTok{,}
       \AttributeTok{x =} \StringTok{"Comprimento do Recém{-}nascido (cm)"}\NormalTok{,}
       \AttributeTok{y =} \StringTok{"Peso do Recém{-}nascido (g)"}\NormalTok{,}
       \AttributeTok{fill =} \StringTok{""}\NormalTok{) }\SpecialCharTok{+}
  \FunctionTok{theme\_classic}\NormalTok{(}\AttributeTok{base\_size =} \DecValTok{13}\NormalTok{) }\SpecialCharTok{+}
  \FunctionTok{theme}\NormalTok{(}\AttributeTok{legend.position =} \StringTok{"bottom"}\NormalTok{)}
\end{Highlighting}
\end{Shaded}

\begin{figure}[H]

\centering{

\includegraphics[width=0.8\linewidth,height=\textheight,keepaspectratio]{08-graficos_files/figure-pdf/fig-scatter7-1.pdf}

}

\caption{\label{fig-scatter7}Gráfico de dispersão com cores
personalizadas de acordo com o sexo}

\end{figure}%

Observar que, na Figura~\ref{fig-scatter6}, os pontos que representam os
meninos estavam cor rosa e as meninas com cor azul. Na figura
Figura~\ref{fig-scatter7}, isto foi modificado \textbf{manualmente}.
Para realizar o mesmo processo, é muito comum usar pacotes de
\textbf{paletas de cores}.

\paragraph{Pacote ggsci}\label{pacote-ggsci}

O \texttt{ggsci} é um pacote que oferece uma coleção de paletas de alta
qualidade inspiradas em cores usadas em revistas científicas,
bibliotecas de visualização de dados, filmes de ficção científica e
programas de TV. As paletas de cores no \texttt{ggsci} estão disponíveis
como escalas \texttt{ggplot2}, Para todas usa-se as seguintes funções:

\begin{itemize}
\tightlist
\item
  \texttt{scale\_color\_nomedapaleta()} e
\item
  \texttt{scale\_fill\_nomedapaleta\ ()}.
\end{itemize}

Por exemplo, para a paleta do \emph{Lancet}, usa-se para o
preenchimento: s\texttt{cale\_fill\_lancet()} . O pacote \texttt{ggsci}
deve ser instalado e carregado para usar estas paletas.\\
Para visualizar as opções do pacote \texttt{ggsci} acessar
\href{https://cran.r-project.org/web/packages/ggsci/vignettes/ggsci.html}{Scientific
Journal and Sci-Fi Themed Color Palettes for ggplot2}.

Como exemplo, será usado o código que gerou o gráfico da
Figura~\ref{fig-scatter7} com alterações, usando a paleta do periódico
\emph{Lancet}.

\begin{Shaded}
\begin{Highlighting}[]
\FunctionTok{library}\NormalTok{(ggsci)}
\FunctionTok{ggplot}\NormalTok{(dadosRNT100, }
       \FunctionTok{aes}\NormalTok{(}\AttributeTok{x =}\NormalTok{ compRN, }\AttributeTok{y =}\NormalTok{ pesoRN, }\AttributeTok{fill =}\NormalTok{ sexo)) }\SpecialCharTok{+}
  \FunctionTok{geom\_point}\NormalTok{(}\AttributeTok{position =} \FunctionTok{position\_jitter}\NormalTok{(}\AttributeTok{width =} \FloatTok{0.2}\NormalTok{, }\AttributeTok{height =} \DecValTok{0}\NormalTok{),}
             \AttributeTok{shape =} \DecValTok{21}\NormalTok{, }
             \AttributeTok{color =} \StringTok{"gray20"}\NormalTok{, }
             \AttributeTok{size =} \DecValTok{4}\NormalTok{, }
             \AttributeTok{stroke =} \DecValTok{1}\NormalTok{) }\SpecialCharTok{+}
  \FunctionTok{scale\_fill\_lancet}\NormalTok{(}\AttributeTok{alpha =} \FloatTok{0.6}\NormalTok{) }\SpecialCharTok{+}
  \FunctionTok{labs}\NormalTok{(}\AttributeTok{title=}\StringTok{""}\NormalTok{,}
       \AttributeTok{x =} \StringTok{"Comprimento do Recém{-}nascido (cm)"}\NormalTok{,}
       \AttributeTok{y =} \StringTok{"Peso do Recém{-}nascido (g)"}\NormalTok{,}
       \AttributeTok{fill =} \StringTok{""}\NormalTok{) }\SpecialCharTok{+}
  \FunctionTok{theme\_classic}\NormalTok{(}\AttributeTok{base\_size =} \DecValTok{13}\NormalTok{) }\SpecialCharTok{+}
  \FunctionTok{theme}\NormalTok{(}\AttributeTok{legend.position =} \StringTok{"bottom"}\NormalTok{)}
\end{Highlighting}
\end{Shaded}

\begin{figure}[H]

\centering{

\includegraphics[width=0.8\linewidth,height=\textheight,keepaspectratio]{08-graficos_files/figure-pdf/fig-scatter8-1.pdf}

}

\caption{\label{fig-scatter8}Gráfico de dispersão com cores
personalizadas usando o ggsci}

\end{figure}%

As cores usadas são agora as da paleta do \emph{Lancet}
(Figura~\ref{fig-scatter8}). Como eles são muito vivas e para que
aparecesse que o ponto foi preenchido, foi utilizada uma transparência
de alpha = 0.6. A paleta do \emph{Lancet} pode ser visualizada com a
função \texttt{show\_col(pal\_lancet())}(Figura~\ref{fig-lancet})

\begin{Shaded}
\begin{Highlighting}[]
\FunctionTok{library}\NormalTok{(ggsci)}
\FunctionTok{show\_col}\NormalTok{(}\FunctionTok{pal\_lancet}\NormalTok{())}
\end{Highlighting}
\end{Shaded}

\begin{figure}[H]

\centering{

\includegraphics[width=0.6\linewidth,height=\textheight,keepaspectratio]{08-graficos_files/figure-pdf/fig-lancet-1.pdf}

}

\caption{\label{fig-lancet}Paleta do Lancet do pacote ggsci}

\end{figure}%

\begin{tcolorbox}[enhanced jigsaw, bottomrule=.15mm, opacitybacktitle=0.6, colframe=quarto-callout-tip-color-frame, arc=.35mm, coltitle=black, toptitle=1mm, colback=white, colbacktitle=quarto-callout-tip-color!10!white, breakable, bottomtitle=1mm, rightrule=.15mm, titlerule=0mm, toprule=.15mm, opacityback=0, leftrule=.75mm, left=2mm, title=\textcolor{quarto-callout-tip-color}{\faLightbulb}\hspace{0.5em}{Exercício}]

Visualizar outras paletas do pacote \texttt{ggsci}, usando a função
\texttt{show\_col()} do pacote scales. Por exemplo:

show\_col(pal\_jama())

show\_col(pal\_bmj())

show\_col(pal\_aaas())

show\_col(pal\_simpsons())

\end{tcolorbox}

\paragraph{Pacote RColorBrewer}\label{pacote-rcolorbrewer}

Existem outros pacotes, como o \texttt{RColorBrewer}, que oferecem
paletas visualmente agradáveis e podem facilmente ser exploradas. Por
alguns, é considerado uma ferramenta indispensável para gerenciar cores
com R (85). Para visualizar (Figura~\ref{fig-colorbrewer}) as paletas do
pacote \texttt{RColorBrewer}.

\begin{Shaded}
\begin{Highlighting}[]
\FunctionTok{library}\NormalTok{(RColorBrewer)}

\FunctionTok{par}\NormalTok{(}\AttributeTok{mar=}\FunctionTok{c}\NormalTok{(}\DecValTok{2}\NormalTok{, }\DecValTok{4}\NormalTok{, }\DecValTok{2}\NormalTok{, }\DecValTok{3}\NormalTok{))            }\CommentTok{\# modifica o tamanho das margens}
\FunctionTok{display.brewer.all}\NormalTok{()}
\FunctionTok{par}\NormalTok{(}\AttributeTok{mar=}\FunctionTok{c}\NormalTok{(}\FloatTok{5.1}\NormalTok{, }\FloatTok{4.1}\NormalTok{, }\FloatTok{4.1}\NormalTok{, }\FloatTok{2.1}\NormalTok{))    }\CommentTok{\# retorna ao tamanho original das margens}
\end{Highlighting}
\end{Shaded}

\begin{figure}[H]

\centering{

\includegraphics[width=0.9\linewidth,height=\textheight,keepaspectratio]{08-graficos_files/figure-pdf/fig-colorbrewer-1.pdf}

}

\caption{\label{fig-colorbrewer}Paleta do RColorBrewer}

\end{figure}%

Como exemplo, será repetido o gráfico da Figura~\ref{fig-scatter8} com
uma paleta de cores do RColorBrewer, \texttt{Pastel2}.

\begin{Shaded}
\begin{Highlighting}[]
\FunctionTok{ggplot}\NormalTok{(dadosRNT100, }
       \FunctionTok{aes}\NormalTok{(}\AttributeTok{x =}\NormalTok{ compRN, }\AttributeTok{y =}\NormalTok{ pesoRN, }\AttributeTok{fill =}\NormalTok{ sexo)) }\SpecialCharTok{+}
  \FunctionTok{geom\_point}\NormalTok{(}\AttributeTok{position =} \FunctionTok{position\_jitter}\NormalTok{(}\AttributeTok{width =} \FloatTok{0.2}\NormalTok{, }\AttributeTok{height =} \DecValTok{0}\NormalTok{),}
             \AttributeTok{shape =} \DecValTok{21}\NormalTok{, }
             \AttributeTok{color =} \StringTok{"gray20"}\NormalTok{, }
             \AttributeTok{size =} \DecValTok{4}\NormalTok{, }
             \AttributeTok{stroke =} \DecValTok{1}\NormalTok{) }\SpecialCharTok{+}
  \FunctionTok{scale\_fill\_brewer}\NormalTok{(}\AttributeTok{palette =} \StringTok{"Pastel2"}\NormalTok{) }\SpecialCharTok{+}
  \FunctionTok{labs}\NormalTok{(}\AttributeTok{title=}\StringTok{""}\NormalTok{,}
       \AttributeTok{x =} \StringTok{"Comprimento do Recém{-}nascido (cm)"}\NormalTok{,}
       \AttributeTok{y =} \StringTok{"Peso do Recém{-}nascido (g)"}\NormalTok{,}
       \AttributeTok{fill =} \StringTok{""}\NormalTok{) }\SpecialCharTok{+}
  \FunctionTok{theme\_classic}\NormalTok{(}\AttributeTok{base\_size =} \DecValTok{13}\NormalTok{) }\SpecialCharTok{+}
  \FunctionTok{theme}\NormalTok{(}\AttributeTok{legend.position =} \StringTok{"bottom"}\NormalTok{)}
\end{Highlighting}
\end{Shaded}

\begin{figure}[H]

\centering{

\includegraphics[width=0.8\linewidth,height=\textheight,keepaspectratio]{08-graficos_files/figure-pdf/fig-scatter9-1.pdf}

}

\caption{\label{fig-scatter9}Gráfico de dispersão com cores
personalizadas usando o RColorBrewer}

\end{figure}%

\paragraph{Paleta paletteer}\label{paleta-paletteer}

O pacote \texttt{paletteer} no R reúne um grande número de paletas de
cores de diversos pacotes do R dedicados a cores. Fornece uma interface
simples e consistente para acessar essas paletas, facilitando o
trabalho. Oferece mais de 2000 paletas de cores de vários pacotes do R,
como \texttt{ggthemes}, \texttt{wesanderson}, \texttt{lisa},
\texttt{scico}, entre outros. Tudo acessível por uma interface simples e
poderosa, facilitando a criação de visualizações bonitas e informativas
(86).\\
Todas as paletas podem ser acessadas a partir das três funções
\texttt{paletteer\_c()}, \texttt{paletteer\_d()} e
\texttt{paletteer\_dynamic()} usando a sintaxe:
\texttt{nome\_do\_pacote::nome\_da\_paleta} \footnote{A função
  \texttt{paletteer\_c()} é usada para variáveis contínuas; a
  \texttt{paletteer\_d()} para variáveis categóricas e a
  \texttt{paletteer\_dinamic()} é pouco usada, mas serve para paletas
  que mudam conforme o número de categorias.}.

Paletas discretas são paletas com um número fixo de cores. Elas são
úteis para visualizar dados categóricos. Por exemplo, uma paleta que vai
do vermelho ao laranja, do verde ao preto é uma paleta discreta.

Exemplo com a paleta \texttt{nbapalettes::supersonics\_holiday},
mapeando os pontos em um tamanho maior (\texttt{size\ =\ 7}) para chamar
atenção das cores.

\begin{Shaded}
\begin{Highlighting}[]
\FunctionTok{library}\NormalTok{(paletteer)}

\FunctionTok{ggplot}\NormalTok{(dadosRNT100, }
       \FunctionTok{aes}\NormalTok{(}\AttributeTok{x =}\NormalTok{ compRN, }\AttributeTok{y =}\NormalTok{ pesoRN, }\AttributeTok{fill =}\NormalTok{ sexo)) }\SpecialCharTok{+}
  \FunctionTok{geom\_point}\NormalTok{(}\AttributeTok{position =} \FunctionTok{position\_jitter}\NormalTok{(}\AttributeTok{width =} \FloatTok{0.2}\NormalTok{, }\AttributeTok{height =} \DecValTok{0}\NormalTok{),}
             \AttributeTok{shape =} \DecValTok{21}\NormalTok{, }
             \AttributeTok{color =} \StringTok{"gray20"}\NormalTok{, }
             \AttributeTok{size =} \DecValTok{6}\NormalTok{, }
             \AttributeTok{stroke =} \FloatTok{1.5}\NormalTok{) }\SpecialCharTok{+}
  \FunctionTok{scale\_fill\_paletteer\_d}\NormalTok{(}\StringTok{"nbapalettes::supersonics\_holiday"}\NormalTok{) }\SpecialCharTok{+}
  \FunctionTok{labs}\NormalTok{(}\AttributeTok{title=}\StringTok{""}\NormalTok{,}
       \AttributeTok{x =} \StringTok{"Comprimento do Recém{-}nascido (cm)"}\NormalTok{,}
       \AttributeTok{y =} \StringTok{"Peso do Recém{-}nascido (g)"}\NormalTok{,}
       \AttributeTok{fill =} \StringTok{""}\NormalTok{) }\SpecialCharTok{+}
  \FunctionTok{theme\_classic}\NormalTok{(}\AttributeTok{base\_size =} \DecValTok{13}\NormalTok{) }\SpecialCharTok{+}
  \FunctionTok{theme}\NormalTok{(}\AttributeTok{legend.position =} \StringTok{"bottom"}\NormalTok{) }
\end{Highlighting}
\end{Shaded}

\begin{figure}[H]

\centering{

\includegraphics[width=0.8\linewidth,height=\textheight,keepaspectratio]{08-graficos_files/figure-pdf/fig-scatter10-1.pdf}

}

\caption{\label{fig-scatter10}Gráfico de dispersão com cores
personalizadas usando o paletteer}

\end{figure}%

Para uma visualização rápida de algumas paletas com o
\texttt{paleteteer}, pode-se digitar em um script do RStudio ou Positron
o seguinte comando:

\begin{Shaded}
\begin{Highlighting}[]
\NormalTok{paletteer}\SpecialCharTok{::}\FunctionTok{paletteer\_d}\NormalTok{(}\StringTok{"lisa::FridaKahlo"}\NormalTok{)}
\end{Highlighting}
\end{Shaded}

\begin{verbatim}
<colors>
#121510FF #6D8325FF #D6CFB7FF #E5AD4FFF #BD5630FF 
\end{verbatim}

\begin{Shaded}
\begin{Highlighting}[]
\NormalTok{paletteer}\SpecialCharTok{::}\FunctionTok{paletteer\_d}\NormalTok{(}\StringTok{"nbapalettes::supersonics\_holiday"}\NormalTok{)}
\end{Highlighting}
\end{Shaded}

\begin{verbatim}
<colors>
#D50032FF #F6BE00FF #00573FFF #010101FF 
\end{verbatim}

No console, aparecerão as cores com os códigos hexadecimais. Se as cores
não estiverem visíveis como aqui, então acesse o website
\href{https://html-color-codes.info/}{HTML Coloe Codes}, onde facilente
é feita essa conversão.

Isto é apenas o caminho, existem uma enorme quantidade de paletas (mais
de 2500 paletas!) e o \texttt{paletteer} é uma espécie de facilitador
para se ter acesso a elas. Escolher as cores para um gráfico é uma
tarefa desafiadora e demorada, o que geralmente leva à insatisfação.Em
um dos seus sites educacionais \href{https://www.yan-holtz.com/}{Yan
Holtz} disponibiliza um
\href{https://r-graph-gallery.com/color-palette-finder}{localizador de
paletas de cores} que torna este trabalho mais palatável.

\subsection{Facetamento}\label{sec-facet}

Na Seção~\ref{sec-coresr}, foi mostrado como comparar grupos, através da
cor, usando as estéticas \texttt{fill} ou \texttt{color} \footnote{Além
  da cor, os grupos em um gráfico também podem ser diferenciados por
  meio das estéticas \textbf{shape} e \textbf{size}. Para isso,
  substituir o mapeamento \texttt{fill\ =\ sexo} por
  \texttt{shape\ =\ sexo} ou \texttt{size\ =\ sexo}.}. Outra técnica
para diferenciar grupos em um gráfico é o \textbf{facetamento}. O
facetamento cria gráficos dividindo os dados em subconjuntos e exibindo
o mesmo gráfico para cada subconjunto (Figura~\ref{fig-scatter11}) .
Para facetar um gráfico, basta adicionar uma especificação de
facetamento com a função \texttt{facet\_wrap()}, que recebe o nome de
uma variável categórica precedido pelo sinal gráfico
\texttt{til\ (\textasciitilde{})}.

Como exemplo prático, será aproveitado o código que gerou a
Figura~\ref{fig-scatter10} com pequenas alterações \footnote{Além da
  função específica para o facetamento, a cor agora é determinada com
  preenchimento dos pontos ( \texttt{fill=”tomato”}) colocado dentro do
  \texttt{geom\_point()} e houve uma diminuição do tamanho dos pontos
  (\texttt{size\ =\ 4}) .} e aplicação do \texttt{facet\_wrap()}.

\begin{Shaded}
\begin{Highlighting}[]
\FunctionTok{ggplot}\NormalTok{(dadosRNT100, }
       \FunctionTok{aes}\NormalTok{(}\AttributeTok{x =}\NormalTok{ compRN, }\AttributeTok{y =}\NormalTok{ pesoRN)) }\SpecialCharTok{+}
  \FunctionTok{geom\_point}\NormalTok{(}\AttributeTok{position =} \FunctionTok{position\_jitter}\NormalTok{(}\AttributeTok{width =} \FloatTok{0.2}\NormalTok{, }\AttributeTok{height =} \DecValTok{0}\NormalTok{),}
             \AttributeTok{shape =} \DecValTok{21}\NormalTok{,}
             \AttributeTok{fill =} \StringTok{"tomato"}\NormalTok{,}
             \AttributeTok{color =} \StringTok{"gray20"}\NormalTok{, }
             \AttributeTok{size =} \DecValTok{4}\NormalTok{, }
             \AttributeTok{stroke =} \DecValTok{1}\NormalTok{) }\SpecialCharTok{+}
  \FunctionTok{labs}\NormalTok{(}\AttributeTok{x =} \StringTok{"Comprimento do Recém{-}nascido (cm)"}\NormalTok{, }
       \AttributeTok{y =} \StringTok{"Peso do Recém{-}nascido (g)"}\NormalTok{) }\SpecialCharTok{+}
  \FunctionTok{theme\_bw}\NormalTok{(}\AttributeTok{base\_size =} \DecValTok{13}\NormalTok{) }\SpecialCharTok{+}
  \FunctionTok{facet\_wrap}\NormalTok{(}\SpecialCharTok{\textasciitilde{}}\NormalTok{sexo)}
\end{Highlighting}
\end{Shaded}

\begin{figure}[H]

\centering{

\includegraphics[width=0.8\linewidth,height=\textheight,keepaspectratio]{08-graficos_files/figure-pdf/fig-scatter11-1.pdf}

}

\caption{\label{fig-scatter11}Gráfico de dispersão com facetamento por
categoria}

\end{figure}%

O facetamento (Figura~\ref{fig-scatter11}) permite verificar que a
relação entre o comprimento e o peso dos recém-nascidos é nitidamente
linear e semelhante entre os sexos.

\subsection{Reta de Regressão}\label{reta-de-regressuxe3o}

A função \texttt{geom\_smooth()} é uma forma geométrica do
\texttt{ggplot2} usada para visualizar tendências ou padrões entre duas
variáveis numéricas. Ele adiciona uma linha suavizada ao gráfico, que
ajuda a entender a relação entre os dados , especialmente quando há
muitos pontos ou quando a relação não é linear. O
\texttt{geom\_smooth()} ajusta uma curva aos dados, usando métodos
estatísticos. A regressão linear usa \texttt{method\ =\ “lm”}
\footnote{Outros métodos: \texttt{“loess”}, \texttt{“glm”},
  \texttt{“gam”}. Para mais informações, consulte a
  ajuda~\texttt{?loess}, \texttt{?gam} ou \texttt{?glm} ou
  \texttt{NULL}, onde a escolha é automática: usa \texttt{“loess”} para
  \textless{} 1000 pontos e \texttt{“gam”} para \textgreater{} 1000.}.
Por padrão, exibe o intervalo de confiança (veja
Capítulo~\ref{sec-estimacao}), que mostra a incerteza da estimativa da
reta. Esta técnica ajuda a identificar padrões que não podem ser
visíveis apenas com os pontos brutos.

O código da figura Figura~\ref{fig-scatter8} será tomado como base sem a
divisão por sexo \footnote{A Figura~\ref{fig-scatter11} mostrou que a
  correlação não parece diferir entre os sexos.}, com modificações, para
gerar a reta de regressão.

\begin{Shaded}
\begin{Highlighting}[]
\FunctionTok{library}\NormalTok{(ggsci)}
\FunctionTok{ggplot}\NormalTok{(dadosRNT100, }
       \FunctionTok{aes}\NormalTok{(}\AttributeTok{x =}\NormalTok{ compRN, }\AttributeTok{y =}\NormalTok{ pesoRN)) }\SpecialCharTok{+}
  \FunctionTok{geom\_point}\NormalTok{(}\AttributeTok{position =} \FunctionTok{position\_jitter}\NormalTok{(}\AttributeTok{width =} \FloatTok{0.2}\NormalTok{, }\AttributeTok{height =} \DecValTok{0}\NormalTok{),}
             \AttributeTok{shape =} \DecValTok{21}\NormalTok{,}
             \AttributeTok{fill =} \StringTok{"tomato"}\NormalTok{,}
             \AttributeTok{color =} \StringTok{"gray20"}\NormalTok{, }
             \AttributeTok{size =} \DecValTok{4}\NormalTok{, }
             \AttributeTok{stroke =} \DecValTok{1}\NormalTok{) }\SpecialCharTok{+}
  \FunctionTok{geom\_smooth}\NormalTok{(}\AttributeTok{method =} \StringTok{"lm"}\NormalTok{, }
              \AttributeTok{se =}\ConstantTok{TRUE}\NormalTok{,}
              \AttributeTok{color=} \StringTok{"darkred"}\NormalTok{) }\SpecialCharTok{+}
  \FunctionTok{scale\_fill\_lancet}\NormalTok{(}\AttributeTok{alpha =} \FloatTok{0.6}\NormalTok{) }\SpecialCharTok{+}
  \FunctionTok{xlab}\NormalTok{(}\StringTok{"Comprimento do Recém{-}nascido (cm)"}\NormalTok{) }\SpecialCharTok{+}
  \FunctionTok{ylab}\NormalTok{(}\StringTok{"Peso do Recém{-}nascido (g)"}\NormalTok{) }\SpecialCharTok{+}
  \FunctionTok{theme\_bw}\NormalTok{(}\AttributeTok{base\_size =} \DecValTok{13}\NormalTok{)}
\end{Highlighting}
\end{Shaded}

\begin{verbatim}
`geom_smooth()` using formula = 'y ~ x'
\end{verbatim}

\begin{figure}[H]

\centering{

\includegraphics[width=0.8\linewidth,height=\textheight,keepaspectratio]{08-graficos_files/figure-pdf/fig-scatter12-1.pdf}

}

\caption{\label{fig-scatter12}Gráfico de dispersão com reta de
regressão}

\end{figure}%

O gráfico da Figura~\ref{fig-scatter12}, mostra um ajuste dos pontos a
uma reta, com inclinação ascendente, ou seja uma correlação positiva, à
medida que o comprimento do recém-nascido aumenta, aumenta o seu peso ao
nascer. Pela forte inclinação da reta. pressupoe-se que esta correlação
é alta.\\
A distância dos pontos à reta é o erro ou resíduo. A melhor reta
ajustada é aquela em que a soma dos quadrados da distância de cada ponto
(soma dos quadrados residual) em relação à reta é minimizada (veja
também \textbf{?@sec-rls}).

\section{Histograma}\label{sec-hist}

O \textbf{histograma} é uma ferramenta gráfica que fornece informações
sobre o formato da distribuição e dispersão dos dados, permitindo
verificar se existe ou não simetria. É usado para dados contínuos.

No histograma, as frequências observadas são representadas por
intervalos de classes de ocorrência que estão no eixo \emph{x} e a
altura das barras, representando a frequência de cada intervalo, no eixo
\emph{y}. A área de cada barra é proporcional à porcentagem de
observações de cada intervalo. O \texttt{geom\_histogram()} é a
geometria para a construção de um histograma. Aqui, há necessidade
apenas do eixo \emph{x}, pois existe uma única variável. A execução do
comando retorna a distribuição dessa variável.

Os dados para plotar um histograma, serão provenientes do dataframe
\texttt{dados} (Seção~\ref{sec-dados8}), utilizando um filtro para as
gestações a termo, designado como \texttt{dadosRNT}.

\begin{Shaded}
\begin{Highlighting}[]
\FunctionTok{set.seed}\NormalTok{(}\DecValTok{123}\NormalTok{)}
\NormalTok{dadosRNT }\OtherTok{\textless{}{-}}\NormalTok{ dados }\SpecialCharTok{\%\textgreater{}\%} 
  \FunctionTok{filter}\NormalTok{(ig }\SpecialCharTok{\textgreater{}=} \DecValTok{37} \SpecialCharTok{\&}\NormalTok{ ig }\SpecialCharTok{\textless{}} \DecValTok{42}\NormalTok{) }
\end{Highlighting}
\end{Shaded}

Será construído um histograma simples da variável \texttt{pesoRN} (peso
dos recém-nascidos a termo), usando os eguinte código:

\begin{Shaded}
\begin{Highlighting}[]
\FunctionTok{ggplot}\NormalTok{(dadosRNT, }\FunctionTok{aes}\NormalTok{(}\AttributeTok{x=}\NormalTok{pesoRN)) }\SpecialCharTok{+} 
  \FunctionTok{geom\_histogram}\NormalTok{()}\SpecialCharTok{+}
  \FunctionTok{labs}\NormalTok{(}\AttributeTok{x =} \StringTok{" Peso dos Recém{-}Nascidos (g)"}\NormalTok{, }
       \AttributeTok{y =} \StringTok{"Frequência"}\NormalTok{)  }\SpecialCharTok{+}
  \FunctionTok{theme\_bw}\NormalTok{(}\AttributeTok{base\_size =} \DecValTok{13}\NormalTok{)}
\end{Highlighting}
\end{Shaded}

\begin{figure}[H]

\centering{

\includegraphics[width=0.8\linewidth,height=\textheight,keepaspectratio]{08-graficos_files/figure-pdf/fig-hist1-1.pdf}

}

\caption{\label{fig-hist1}Histograma simples}

\end{figure}%

A aparência do histograma da Figura~\ref{fig-hist1} permite ter uma
idéia da distribuição e simetria dos dados, mas não está com um aspecto
agradável, amigável. Mesmo que ele expresse corretamente a sua mensagem,
essa mensagem pode ser prejudicada por uma má aparência.\\
O histograma recebe uma variável numérica (no caso, o peso dos
recém-nascidos) e a divide em vários ``compartimentos'', os intervalos,
representados pelas barras. A escolha do tamanho (amplitude) do
intervalo é de extrema importância para a aparência do histograma.\\
O \texttt{geom\_histogram()} tem um argumento, denominado
\texttt{binwidth} que permite alterar a amplitude do intervalo. O
\texttt{binwidth} é um intervalo e sua unidade é igual a da variável que
se está ``histogramando''. No exemplo, foi usado o peso do recém-nascido
(g). Se o objetivo são intervalos de 200 em 200 gramas, o
\texttt{binwidth\ =\ 200}. Uma outra maneira, é usar bins que agrupa em
intervalos de mesmo tamanho. Por exemplo \texttt{bins\ =\ 15}, o
\texttt{geom\_histogram()} dividirá em 15 intervalos iguais, gerando um
histograma semelhante ao da Figura~\ref{fig-hist2}. Junto com a
alteração dos intervalos, vamos modificar a cor de preenchimento
(\texttt{fill}) e bordas (\texttt{color}) das barras
(Figura~\ref{fig-hist2} ).

\begin{Shaded}
\begin{Highlighting}[]
\FunctionTok{ggplot}\NormalTok{(dadosRNT, }\FunctionTok{aes}\NormalTok{(}\AttributeTok{x=}\NormalTok{pesoRN)) }\SpecialCharTok{+} 
  \FunctionTok{geom\_histogram}\NormalTok{(}\AttributeTok{binwidth =} \DecValTok{200}\NormalTok{,}
                 \AttributeTok{fill =} \StringTok{"chartreuse"}\NormalTok{,}
                 \AttributeTok{color =} \StringTok{"darkgreen"}\NormalTok{)}\SpecialCharTok{+}
  \FunctionTok{labs}\NormalTok{(}\AttributeTok{x =} \StringTok{"Peso dos Recém{-}Nascidos (g)"}\NormalTok{, }
       \AttributeTok{y =} \StringTok{"Frequência"}\NormalTok{)  }\SpecialCharTok{+}
  \FunctionTok{theme\_bw}\NormalTok{(}\AttributeTok{base\_size =} \DecValTok{13}\NormalTok{)}
\end{Highlighting}
\end{Shaded}

\begin{figure}[H]

\centering{

\includegraphics[width=0.8\linewidth,height=\textheight,keepaspectratio]{08-graficos_files/figure-pdf/fig-hist2-1.pdf}

}

\caption{\label{fig-hist2}Histograma simples modificado}

\end{figure}%

Com frequência se observa um histograma com curva normal sobreposta
(Figura~\ref{fig-hist3}) para facilitar a comparação dos dados com a
distribuição normal. Isso pode ser conseguido com um código que usa
função \texttt{stat\_function()} para a construção da curva normal,
baseada nos dados (média e desvio padrão da variável \texttt{pesoRN}) e
a função \texttt{after\_stat(density)}, colocada na estética do
histograma, no eixo \emph{y}, para substituir a frequência pela
densidade de probabilidade (veja Seção~\ref{sec-distprob}). O restante
do código somente estabelece que a linha da curva será tracejada
(\texttt{linetype\ =\ “dashed”}), de cor vermelha
(\texttt{color\ =\ “red”}) e com tamanho 1 (\texttt{linewidth\ =\ 1}).

\begin{Shaded}
\begin{Highlighting}[]
\FunctionTok{library}\NormalTok{(ggplot2)}
\FunctionTok{library}\NormalTok{(dplyr)}

\FunctionTok{ggplot}\NormalTok{(dadosRNT) }\SpecialCharTok{+} 
  \FunctionTok{geom\_histogram}\NormalTok{(}\FunctionTok{aes}\NormalTok{(}\AttributeTok{x =}\NormalTok{ pesoRN, }
                     \AttributeTok{y =} \FunctionTok{after\_stat}\NormalTok{(density)),}
                 \AttributeTok{binwidth =} \DecValTok{200}\NormalTok{,}
                 \AttributeTok{fill =} \StringTok{"chartreuse"}\NormalTok{,}
                 \AttributeTok{colour =} \StringTok{"darkgreen"}\NormalTok{) }\SpecialCharTok{+}
  \FunctionTok{stat\_function}\NormalTok{(}\AttributeTok{fun =}\NormalTok{ dnorm, }
                \AttributeTok{args =} \FunctionTok{list}\NormalTok{(}\AttributeTok{mean =} \FunctionTok{mean}\NormalTok{(dadosRNT}\SpecialCharTok{$}\NormalTok{pesoRN),}
                            \AttributeTok{sd =} \FunctionTok{sd}\NormalTok{(dadosRNT}\SpecialCharTok{$}\NormalTok{pesoRN)),}
                \AttributeTok{linetype =} \StringTok{"dashed"}\NormalTok{,}
                \AttributeTok{linewidth =} \DecValTok{1}\NormalTok{,}
                \AttributeTok{color =} \StringTok{"red"}\NormalTok{) }\SpecialCharTok{+}
  \FunctionTok{labs}\NormalTok{(}\AttributeTok{x =} \StringTok{"Peso do Recém{-}Nascido (g)"}\NormalTok{, }
       \AttributeTok{y =} \StringTok{"Densidade de Probabilidade"}\NormalTok{)  }\SpecialCharTok{+}
  \FunctionTok{theme\_bw}\NormalTok{(}\AttributeTok{base\_size =} \DecValTok{13}\NormalTok{)}
\end{Highlighting}
\end{Shaded}

\begin{figure}[H]

\centering{

\includegraphics[width=0.8\linewidth,height=\textheight,keepaspectratio]{08-graficos_files/figure-pdf/fig-hist3-1.pdf}

}

\caption{\label{fig-hist3}Histograma com curva normal sobreposta}

\end{figure}%

Na Figura~\ref{fig-hist3}, se observa que os pesos dos recém-nascidos se
ajustam razoavelmente à curva normal.

\section{Boxplot}\label{boxplot}

O boxplot é uma representação gráfica de um resumo eficaz, de fácil
compreensão, de uma ou mais varáveis numéricas. Fornece uma análise
visual da posição, dispersão, simetria, caudas e valores discrepantes
(outliers) do conjunto de dados (Figura~\ref{fig-boxplot}).

\begin{itemize}
\item
  \textbf{Posição} -- Em relação à posição dos dados, observa-se a linha
  central do retângulo (a mediana ou segundo quartil).
\item
  \textbf{Dispersão} -- A dispersão dos dados pode ser representada pelo
  intervalo interquartil (IIQ), tamanho da caixa, que é a diferença
  entre o terceiro quartil (3ºQ) e o primeiro quartil (1ºQ), ou ainda
  pela amplitude que é calculada da seguinte maneira: valor máximo --
  valor mínimo. Embora a amplitude seja de fácil entendimento, o
  intervalo interquartil é uma estatística mais robusta para medir
  variabilidade uma vez que não sofre influência de outliers.
\item
  \textbf{Simetria} -- Um conjunto de dados que tem uma distribuição
  simétrica, terá a linha da mediana no centro do retângulo. Quando a
  linha da mediana está próxima ao primeiro quartil, os dados são
  assimétricos positivos e quando a posição da linha da mediana é
  próxima ao terceiro quartil, os dados são assimétricos negativos. Vale
  lembrar que a mediana é a medida de tendência central mais indicada
  quando os dados possuem distribuição assimétrica, uma vez que a média
  aritmética é influenciada pelos valores extremos.\\
\item
  \textbf{Caudas} -- As linhas que vão do retângulo até aos outliers
  podem fornecer o comprimento das caudas da distribuição.\\
\item
  \textbf{Valores atípicos} (\emph{Outliers}) -- Os valores atípicos
  indicam possíveis valores discrepantes. No boxplot, as observações são
  consideradas atípicas quando estão abaixo ou acima dos limites
  superior e inferior. O limite de detecção de valores atípicos
  (outliers) é construído utilizando o intervalo interquartil, dado pela
  distância entre o primeiro e o terceiro quartil. Sendo assim, os
  limites inferior e superior de detecção de outlier são dados por:

  \begin{itemize}
  \item
    o Limite Inferior: 1ºQ -- (1,5 * IIQ);
  \item
    o Limite Superior: 3ºQ + (1,5 * IIQ). Tanto o limite superior como o
    inferior são representados por (º).
  \item
    os Valores extremos: são valores que estão acima ou abaixo de 3
    vezes o IIQ e são representados por (*).
  \end{itemize}
\end{itemize}

\begin{figure}

\centering{

\includegraphics[width=0.7\linewidth,height=\textheight,keepaspectratio]{index_files/mediabag/Dh2UszX.png}

}

\caption{\label{fig-boxplot}Anatomia de um Boxplot}

\end{figure}%

Os boxplots são construídos com o \texttt{geom\_boxplot()}. Deve-se
especificar uma variável quantitativa para o eixo \emph{y} e uma
variável qualitativa para o eixo x (grupo). Se não houver, variável
\emph{x} e tem-se apenas um vetor de valores numéricos, então, ignora-se
a variável \emph{x}.\\
Para o exemplo de construção de um boxplot, será usada a variável
\texttt{pesoRN} do conjunto de dados \texttt{dadosRNT00} carregados na
Seção~\ref{sec-grammar}.

\begin{Shaded}
\begin{Highlighting}[]
\FunctionTok{ggplot}\NormalTok{(dadosRNT100, }
       \FunctionTok{aes}\NormalTok{(}\AttributeTok{x =} \StringTok{""}\NormalTok{, }\AttributeTok{y =}\NormalTok{ pesoRN)) }\SpecialCharTok{+}
  \FunctionTok{geom\_boxplot}\NormalTok{(}\AttributeTok{fill =} \StringTok{"skyblue"}\NormalTok{, }\AttributeTok{alpha =} \FloatTok{0.6}\NormalTok{) }\SpecialCharTok{+}
  \FunctionTok{labs}\NormalTok{ (}\AttributeTok{x =} \ConstantTok{NULL}\NormalTok{, }\AttributeTok{y =} \StringTok{"Peso do Recém{-}nascido (g)"}\NormalTok{) }\SpecialCharTok{+}
  \FunctionTok{theme\_bw}\NormalTok{(}\AttributeTok{base\_size =} \DecValTok{13}\NormalTok{)}
\end{Highlighting}
\end{Shaded}

\begin{figure}[H]

\centering{

\includegraphics[width=0.8\linewidth,height=\textheight,keepaspectratio]{08-graficos_files/figure-pdf/fig-bxp1-1.pdf}

}

\caption{\label{fig-bxp1}Boxplot simples}

\end{figure}%

O \texttt{geom\_boxplot()} vazio gera um gráfico sem cores. Colocando o
preenchimento \texttt{fill\ =\ “skyblue”},tem-se um boxplot de cor azul
céu (Figura~\ref{fig-bxp1}).

Na aparência do boxplot , é clássico o seu formato com os ``bigodes''
terminando em ``T'', como mostra a Figura~\ref{fig-boxplot}, e não um
traço simples. Para modificar isso, pode-se criar uma nova camada de
barra de erro, usando a função \texttt{geom\_errorbar()}, antes de
\texttt{geom\_boxblot()}. Assim, como o boxplot passa ser a camada mais
superficial, ele impede que se visualize a barra de erro na caixa
(Figura~\ref{fig-bxp2}), desde que ele seja opaco (remover ou zerar o
argumento \texttt{alpha}). A função \texttt{geom\_errorbar()}
normalmente é usada para barras de erro, no entanto, aqui ela está sendo
utilizada com \texttt{stat\ =\ "boxplot"} \footnote{O padrão é
  \texttt{stat\ =\ "identity"}, o que significa que os valores das
  barras de erro devem ser fornecidos diretamente no conjunto de dados,
  sem cálculos adicionais.}, o que significa que os cálculos de
estatística do boxplot serão aplicados à barra de erro. O argumento
\texttt{width\ =\ 0.1} ajusta a largura das barras de erro, tornando-as
mais estreitas.

\begin{Shaded}
\begin{Highlighting}[]
\FunctionTok{ggplot}\NormalTok{(dadosRNT100, }
       \FunctionTok{aes}\NormalTok{(}\AttributeTok{x =} \StringTok{""}\NormalTok{, }\AttributeTok{y =}\NormalTok{ pesoRN)) }\SpecialCharTok{+}
  \FunctionTok{geom\_errorbar}\NormalTok{(}\AttributeTok{stat =} \StringTok{"boxplot"}\NormalTok{, }\AttributeTok{width =} \FloatTok{0.1}\NormalTok{) }\SpecialCharTok{+}
  \FunctionTok{geom\_boxplot}\NormalTok{(}\AttributeTok{fill =} \StringTok{"skyblue"}\NormalTok{) }\SpecialCharTok{+}
  \FunctionTok{labs}\NormalTok{ (}\AttributeTok{x =} \ConstantTok{NULL}\NormalTok{, }\AttributeTok{y =} \StringTok{"Peso do Recém{-}nascido (g)"}\NormalTok{) }\SpecialCharTok{+}
  \FunctionTok{theme\_bw}\NormalTok{(}\AttributeTok{base\_size =} \DecValTok{13}\NormalTok{)}
\end{Highlighting}
\end{Shaded}

\begin{figure}[H]

\centering{

\includegraphics[width=0.8\linewidth,height=\textheight,keepaspectratio]{08-graficos_files/figure-pdf/fig-bxp2-1.pdf}

}

\caption{\label{fig-bxp2}Boxplot simples com bigodes em T}

\end{figure}%

\subsection{Múltiplos boxplots}\label{muxfaltiplos-boxplots}

Os boxplots são bastante úteis quando se compara dois grupos,
tornando-se uma ferramenta conveniente para compreender rapidamente as
diferenças entre esses grupos. Ao usar os boxplots para comparar grupos,
deve-se ter cuidado, pois os resumos podem levar à perda de informação
que pode induzir erros de interpretação. Considere os boxplots da
Figura~\ref{fig-bxp3}, comparando o comprimentos de recém-nascidos a
termo masculinos e femininos, a mediana dos meninos é mais alta do a das
meninas. As meninas apresentam a mediana fora do centro das caixas,
indicando um certo grau de assimetria. Mesmo sendo possível obter
informações importantes sobre os dados, usando um boxplot, não se pode
discernir a distribuição subjacente dos pontos de dados individuais
dentro de cada grupo ou o número total de observações.\\
As cores dos boxplots serão definidas manualmente dentro do
\texttt{geom\_boxplot()}. Além disso, será adicionada a função
\texttt{theme(legend.postion=\ "none)} para remover a legenda, pois ela
é redundante neste gráfico, uma vez que os sexos já estão mencionados no
eixo \emph{x}.

\begin{Shaded}
\begin{Highlighting}[]
\FunctionTok{ggplot}\NormalTok{(dadosRNT100, }\FunctionTok{aes}\NormalTok{(}\AttributeTok{x =}\NormalTok{ sexo, }\AttributeTok{y =}\NormalTok{ compRN)) }\SpecialCharTok{+}
  \FunctionTok{geom\_errorbar}\NormalTok{(}\AttributeTok{stat =} \StringTok{"boxplot"}\NormalTok{, }\AttributeTok{width =} \FloatTok{0.1}\NormalTok{) }\SpecialCharTok{+}
  \FunctionTok{geom\_boxplot}\NormalTok{(}\AttributeTok{fill=} \FunctionTok{c}\NormalTok{(}\StringTok{"goldenrod2"}\NormalTok{, }\StringTok{"orangered3"}\NormalTok{)) }\SpecialCharTok{+}
  \FunctionTok{labs}\NormalTok{(}\AttributeTok{x =} \ConstantTok{NULL}\NormalTok{, }\AttributeTok{y =} \StringTok{"Comprimento dos Recém{-}Nascidos (cm)"}\NormalTok{) }\SpecialCharTok{+}
  \FunctionTok{theme}\NormalTok{(}\AttributeTok{legend.postion=} \StringTok{"none"}\NormalTok{) }\SpecialCharTok{+}  
  \FunctionTok{theme\_bw}\NormalTok{(}\AttributeTok{base\_size =} \DecValTok{13}\NormalTok{)}
\end{Highlighting}
\end{Shaded}

\begin{figure}[H]

\centering{

\includegraphics[width=0.8\linewidth,height=\textheight,keepaspectratio]{08-graficos_files/figure-pdf/fig-bxp3-1.pdf}

}

\caption{\label{fig-bxp3}Comparação de dois grupos com boxplots}

\end{figure}%

Se necessários mais informações, é possível adicionar
\emph{jitter}\footnote{Prestar atenção para o fato de que o
  \emph{jitter} é aleatório, por isso em cada execução do código os
  pontos se distribuem em posições diferentes.} no boxplot da
(Figura~\ref{fig-bxp4}) para torná-lo mais esclarecedor e visualizar
melhor a distribuição dos dados.

\begin{Shaded}
\begin{Highlighting}[]
\FunctionTok{ggplot}\NormalTok{(dadosRNT100, }\FunctionTok{aes}\NormalTok{(}\AttributeTok{x =}\NormalTok{ sexo, }\AttributeTok{y =}\NormalTok{ compRN)) }\SpecialCharTok{+}
  \FunctionTok{geom\_errorbar}\NormalTok{(}\AttributeTok{stat =} \StringTok{"boxplot"}\NormalTok{, }\AttributeTok{width =} \FloatTok{0.1}\NormalTok{) }\SpecialCharTok{+}
  \FunctionTok{geom\_boxplot}\NormalTok{(}\AttributeTok{fill=} \FunctionTok{c}\NormalTok{(}\StringTok{"goldenrod2"}\NormalTok{, }\StringTok{"orangered3"}\NormalTok{)) }\SpecialCharTok{+}
  \FunctionTok{geom\_jitter}\NormalTok{(}\AttributeTok{color=}\StringTok{"grey30"}\NormalTok{, }\AttributeTok{size=}\FloatTok{1.5}\NormalTok{) }\SpecialCharTok{+}
  \FunctionTok{labs}\NormalTok{(}\AttributeTok{x =} \ConstantTok{NULL}\NormalTok{, }\AttributeTok{y =} \StringTok{"Comprimento dos Recém{-}Nascidos (cm)"}\NormalTok{) }\SpecialCharTok{+}
  \FunctionTok{theme}\NormalTok{(}\AttributeTok{legend.postion=} \StringTok{"none"}\NormalTok{) }\SpecialCharTok{+}  
  \FunctionTok{theme\_bw}\NormalTok{(}\AttributeTok{base\_size =} \DecValTok{13}\NormalTok{)}
\end{Highlighting}
\end{Shaded}

\begin{figure}[H]

\centering{

\includegraphics[width=0.8\linewidth,height=\textheight,keepaspectratio]{08-graficos_files/figure-pdf/fig-bxp4-1.pdf}

}

\caption{\label{fig-bxp4}Boxplots com jitter}

\end{figure}%

\subsection{Boxplots horizontais}\label{sec-bxphorizontal}

Para criar boxplots horizontais, adiciona-se a função
\texttt{coord\_flip()} à função \texttt{geom\_boxplot()} para inverter
os eixos. Em um boxplot padrão, a variável categórica está no eixo
\emph{x} e a variável numérica no eixo \emph{y}. Com
\texttt{coord\_flip()}, as variáveis são invertidas, colocando a
variável categórica no eixo \emph{y} e a numérica no eixo \emph{x},
resultando no boxplot horizontal da Figura~\ref{fig-bxp5}.

\begin{Shaded}
\begin{Highlighting}[]
\FunctionTok{library}\NormalTok{(ggplot2)}
\FunctionTok{library}\NormalTok{(dplyr)}

\FunctionTok{ggplot}\NormalTok{(dadosRNT100, }\FunctionTok{aes}\NormalTok{(}\AttributeTok{x =}\NormalTok{ sexo, }\AttributeTok{y =}\NormalTok{ compRN)) }\SpecialCharTok{+}
  \FunctionTok{geom\_errorbar}\NormalTok{(}\AttributeTok{stat =} \StringTok{"boxplot"}\NormalTok{, }\AttributeTok{width =} \FloatTok{0.1}\NormalTok{) }\SpecialCharTok{+}
  \FunctionTok{geom\_boxplot}\NormalTok{(}\AttributeTok{fill=} \FunctionTok{c}\NormalTok{(}\StringTok{"goldenrod2"}\NormalTok{, }\StringTok{"orangered3"}\NormalTok{)) }\SpecialCharTok{+}
  \FunctionTok{coord\_flip}\NormalTok{() }\SpecialCharTok{+}
  \FunctionTok{labs}\NormalTok{(}\AttributeTok{x =} \ConstantTok{NULL}\NormalTok{, }\AttributeTok{y =} \StringTok{"Comprimento dos Recém{-}Nascidos (cm)"}\NormalTok{) }\SpecialCharTok{+}
  \FunctionTok{theme}\NormalTok{(}\AttributeTok{legend.postion=} \StringTok{"none"}\NormalTok{) }\SpecialCharTok{+}  
  \FunctionTok{theme\_bw}\NormalTok{(}\AttributeTok{base\_size =} \DecValTok{13}\NormalTok{)}
\end{Highlighting}
\end{Shaded}

\begin{figure}[H]

\centering{

\includegraphics[width=0.8\linewidth,height=\textheight,keepaspectratio]{08-graficos_files/figure-pdf/fig-bxp5-1.pdf}

}

\caption{\label{fig-bxp5}Boxplots horizontais}

\end{figure}%

\section{Gráfico de violino}\label{sec-violin}

Os gráficos de violino permitem visualizar a distribuição de uma
variável numérica para um ou vários grupos. No \texttt{ggplot2}, são
construídos com o \texttt{geom\_violin()} e, com frequência, substituem
os boxplots, principalmente, quando se tem uma amostra muito grande e
usar o \emph{jitter} no boxplot pode não ser eficaz, pois os pontos
podem se sobrepor e tornar a figura inelegível.

Cada ``violino'' representa uma variável de agrupamento. A forma
representa a estimativa de densidade de probabilidade da variável:
quanto mais pontos de dados em um intervalo específico, mais largo será
o violino para esse intervalo. É muito parecido com um boxplot, mas
permite uma compreensão mais profunda da distribuição.

O gráfico de violino é uma técnica poderosa de visualização de dados,
pois permite comparar a classificação de vários grupos e sua
distribuição. São particularmente adequados quando a quantidade de dados
é grande e é impossível mostrar observações individuais. Para conjuntos
de dados pequenos, um boxplot com jitter é provavelmente uma opção
melhor, pois realmente mostra todas as informações.

Para o exemplo prático, será usada uma amostra proveniente do conjunto
de dados \texttt{dados} (veja Seção~\ref{sec-dados8}), com filtrado para
as gestações a termo, \texttt{dadosRNT}. Serão utilizadas as variáveis
\texttt{pesoRN} e \texttt{categFumo}, tabagismo entre as gestantes, de
acordo com a intensidade (não fumante, fumante leve. fumante moderada,
fumante pesada) . O objetivo é observar visualmente o impacto do
tabagismo sobre os pesos dos recém-nascidos .

Para construir o gráfico de violino, serão usados os argumentos
\texttt{trim\ =\ FALSE}, para não aparar as caudas, e
\texttt{draw\_quantiles\ =\ c(0.25,\ 0.5,\ 0.75)}, para traçar os
quartis (Figura~\ref{fig-violino1}). As cores das categoria foram
definidas pelo \texttt{ggplot2}.\\
Reiterando, função \texttt{theme(legend.position\ =\ "none")} será
colocada para evitar que a legenda das categorias apareça, uma vez que
ela é explicita no gráfico.

\begin{Shaded}
\begin{Highlighting}[]
\FunctionTok{ggplot}\NormalTok{(dadosRNT, }\FunctionTok{aes}\NormalTok{(}\AttributeTok{x=}\NormalTok{categFumo, }\AttributeTok{y=}\NormalTok{pesoRN,         }
                       \AttributeTok{fill=}\NormalTok{categFumo)) }\SpecialCharTok{+} 
  \FunctionTok{geom\_violin}\NormalTok{(}\AttributeTok{trim =} \ConstantTok{FALSE}\NormalTok{,}
              \AttributeTok{draw\_quantiles =} \FunctionTok{c}\NormalTok{(}\FloatTok{0.25}\NormalTok{, }\FloatTok{0.5}\NormalTok{, }\FloatTok{0.75}\NormalTok{)) }\SpecialCharTok{+}
  \FunctionTok{labs}\NormalTok{(}\AttributeTok{x =} \StringTok{"Tabagismo Materno"}\NormalTok{, }
       \AttributeTok{y =} \StringTok{"Peso do Recém{-}Nascido (g)"}\NormalTok{)  }\SpecialCharTok{+}
  \FunctionTok{theme\_bw}\NormalTok{(}\AttributeTok{base\_size =} \DecValTok{13}\NormalTok{) }\SpecialCharTok{+}
  \FunctionTok{theme}\NormalTok{(}\AttributeTok{legend.position =} \StringTok{"none"}\NormalTok{)}
\end{Highlighting}
\end{Shaded}

\begin{figure}[H]

\centering{

\includegraphics[width=0.8\linewidth,height=\textheight,keepaspectratio]{08-graficos_files/figure-pdf/fig-violino1-1.pdf}

}

\caption{\label{fig-violino1}Gráfico de violino com os quartis}

\end{figure}%

Uma alteração interessante que pode ser feita no gráfico de violino, é
colocar um boxplot, dentro do mesmo (Figura~\ref{fig-violino2}), faz o
efeito do argumento \texttt{draw\_quantiles()}, usado na
Figura~\ref{fig-violino1}. Facilita a interpretação e, na opinião do
autor, é mais bonito e elegante. O argumento \texttt{width\ =\ 0.5}, na
função \texttt{geom\_boxplot()}, estabelece a largura do boxplot,
evitando que o boxplot se estenda para fora do ``violino''.

\begin{Shaded}
\begin{Highlighting}[]
\FunctionTok{ggplot}\NormalTok{(dadosRNT, }\FunctionTok{aes}\NormalTok{(}\AttributeTok{x=}\NormalTok{categFumo, }\AttributeTok{y=}\NormalTok{pesoRN, }\AttributeTok{fill=}\NormalTok{categFumo)) }\SpecialCharTok{+} 
  \FunctionTok{geom\_violin}\NormalTok{(}\AttributeTok{trim =} \ConstantTok{FALSE}\NormalTok{) }\SpecialCharTok{+}
  \FunctionTok{geom\_boxplot}\NormalTok{(}\AttributeTok{width =} \FloatTok{0.5}\NormalTok{) }\SpecialCharTok{+}
  \FunctionTok{labs}\NormalTok{(}\AttributeTok{x =} \StringTok{"Tabagismo Materno"}\NormalTok{, }
       \AttributeTok{y =} \StringTok{"Peso do Recém{-}Nascido (g)"}\NormalTok{)  }\SpecialCharTok{+}
  \FunctionTok{theme\_bw}\NormalTok{(}\AttributeTok{base\_size =} \DecValTok{13}\NormalTok{) }\SpecialCharTok{+}
  \FunctionTok{theme}\NormalTok{(}\AttributeTok{legend.position =} \StringTok{"none"}\NormalTok{)}
\end{Highlighting}
\end{Shaded}

\begin{figure}[H]

\centering{

\includegraphics[width=0.8\linewidth,height=\textheight,keepaspectratio]{08-graficos_files/figure-pdf/fig-violino2-1.pdf}

}

\caption{\label{fig-violino2}Gráfico de violino com boxplots}

\end{figure}%

A observação da Figura~\ref{fig-violino2} mostra uma tendência do peso
do recém-nascido diminuir à medida que intensidade do fumo aumenta. O
gráfico sugere que esta tendência não é sugnificativa, pois as caixas se
sobrepõem.

Para obter uma versão horizontal da Figura~\ref{fig-violino1}, chama-se
a função \texttt{coord\_flip()} \footnote{Consulte a construção da
  Figura~\ref{fig-bxp5}}que permite inverter os eixos \emph{x} e
\emph{y} e, assim, tornar a interpretação mais intuitiva, mais amigável
(?).

Para \ul{interpretar um gráfico de violino}, observar o seguinte:

\begin{itemize}
\item
  Forma do violino, observando a largura em diferentes pontos para
  entender onde os dados se concentram.
\item
  A linha mediana e a caixa do boxplot associado indicam a mediana e o
  intervalo interquartil, respectivamente.
\item
  Se o violino é simétrico em torno da mediana, a distribuição dos dados
  é aproximadamente simétrica.
\item
  Se a parte superior do violino é mais larga, os dados podem ser
  assimétricos, inclinados para valores maiores.
\item
  Em múltiplas categorias, pode-se comparar rapidamente as
  distribuições. Diferentes formas e larguras entre as categorias
  fornecem uma visão clara das variações entre elas.
\end{itemize}

\section{Gráfico de barras}\label{gruxe1fico-de-barras}

O gráfico de barras é uma análogo do histograma, onde as barras, ao
contrário deste, são separadas. Os gráficos de barra exibem a
distribuição (frequências) de uma variável categórica através de barras
verticais ou horizontais, ou sobrepostas. A função \texttt{geom\_bar()}
permite delinear o gráfico de barras da Figura~\ref{fig-bar1}.

Para os exemplos práticos, será usada uma amostra proveniente do
conjunto de dados dados (Seção~\ref{sec-dados8}), manipulando as mesmas
variáveis: \texttt{categFumo} , tabagismo materno, e \texttt{categIdade}
, idade materna categorizada. Em outros exemplos de grágicos barras,
serão usadas as variáveis \texttt{fumo} (fumante e não fumante),
\texttt{para} (número de filhos anteriores) e \texttt{sexo} do
recém-nascido.

O gráfico de barras inicial servirá para para visualizar a prevalência
(\textbf{?@sec-prevalencia}) de fumo na gestação categorizada pela
intensidade do fumo.

\begin{Shaded}
\begin{Highlighting}[]
\FunctionTok{ggplot}\NormalTok{(}\AttributeTok{data =}\NormalTok{ dados) }\SpecialCharTok{+} 
  \FunctionTok{geom\_bar}\NormalTok{(}\FunctionTok{aes}\NormalTok{(}\AttributeTok{x =}\NormalTok{ categFumo, }
               \AttributeTok{y =} \FunctionTok{after\_stat}\NormalTok{(count}\SpecialCharTok{/}\FunctionTok{sum}\NormalTok{(count))))}\SpecialCharTok{+} 
  \FunctionTok{labs}\NormalTok{(}\AttributeTok{x =} \StringTok{"Tabagismo Materno"}\NormalTok{, }
       \AttributeTok{y =} \StringTok{"Proporção por categoria"}\NormalTok{)  }\SpecialCharTok{+}
  \FunctionTok{theme\_bw}\NormalTok{(}\AttributeTok{base\_size =} \DecValTok{13}\NormalTok{)}
\end{Highlighting}
\end{Shaded}

\begin{figure}[H]

\centering{

\includegraphics[width=0.8\linewidth,height=\textheight,keepaspectratio]{08-graficos_files/figure-pdf/fig-bar1-1.pdf}

}

\caption{\label{fig-bar1}Gráfico de barras padrão do ggplot2}

\end{figure}%

As cores de preenchimento das barras podem ser alteradas, de acordo com
a variável categórica. As cores serão estabelecidas de acordo com o
padrão do ggplot2 (Figura~\ref{fig-bar2}). O gráfico retornará uma
legenda, mostrando o que representa cada cor. Ela é desnecessária porque
fica explicito, no eixo \emph{x}, o que cada barra representa. Portanto,
é uma boa conduta remover a legenda com a função
\texttt{theme\ (legend.position\ =\ "none")}, como já visto em outras
ocasiões (boxplots e gráfico de violino):

\begin{Shaded}
\begin{Highlighting}[]
\FunctionTok{ggplot}\NormalTok{(}\AttributeTok{data =}\NormalTok{ dados) }\SpecialCharTok{+} 
  \FunctionTok{geom\_bar}\NormalTok{(}\FunctionTok{aes}\NormalTok{(}\AttributeTok{x =}\NormalTok{ categFumo, }
               \AttributeTok{y =} \FunctionTok{after\_stat}\NormalTok{(count}\SpecialCharTok{/}\FunctionTok{sum}\NormalTok{(count)),}
               \AttributeTok{fill =}\NormalTok{ categFumo))}\SpecialCharTok{+} 
                 \FunctionTok{labs}\NormalTok{(}\AttributeTok{x =} \StringTok{"Tabagismo Materno"}\NormalTok{, }
                      \AttributeTok{y =} \StringTok{"Proporção por categoria"}\NormalTok{)  }\SpecialCharTok{+}
                 \FunctionTok{theme\_bw}\NormalTok{(}\AttributeTok{base\_size =} \DecValTok{13}\NormalTok{) }\SpecialCharTok{+}
  \FunctionTok{theme}\NormalTok{(}\AttributeTok{legend.position =} \StringTok{"none"}\NormalTok{)}
\end{Highlighting}
\end{Shaded}

\begin{figure}[H]

\centering{

\includegraphics[width=0.8\linewidth,height=\textheight,keepaspectratio]{08-graficos_files/figure-pdf/fig-bar2-1.pdf}

}

\caption{\label{fig-bar2}Gráfico de barras com as cores das barras
estabelecidas pelo ggplot2}

\end{figure}%

As cores padrão do ggplot2 podem ser alteradas, como foi visto na
Seção~\ref{sec-coresr}, escolhendo manualmente, ou usando uma paleta,
como as apresentadas pelo pacote \texttt{ggsci}, \texttt{RColorBrewer}
ou \texttt{paletteer}. No exemplo, será usada a paleta do periódico
\emph{New England Journal of Medicine} (NEJM), Figura~\ref{fig-bar3}.

\begin{Shaded}
\begin{Highlighting}[]
\FunctionTok{library}\NormalTok{(ggsci)}

\FunctionTok{ggplot}\NormalTok{(}\AttributeTok{data =}\NormalTok{ dados) }\SpecialCharTok{+} 
  \FunctionTok{geom\_bar}\NormalTok{(}\FunctionTok{aes}\NormalTok{(}\AttributeTok{x =}\NormalTok{ categFumo, }
               \AttributeTok{y =} \FunctionTok{after\_stat}\NormalTok{(count}\SpecialCharTok{/}\FunctionTok{sum}\NormalTok{(count)),}
               \AttributeTok{fill =}\NormalTok{ categFumo))}\SpecialCharTok{+} 
  \FunctionTok{scale\_fill\_nejm}\NormalTok{() }\SpecialCharTok{+}
  \FunctionTok{labs}\NormalTok{(}\AttributeTok{x =} \StringTok{"Tabagismo Materno"}\NormalTok{, }
       \AttributeTok{y =} \StringTok{"Proporção por categoria"}\NormalTok{)  }\SpecialCharTok{+}
  \FunctionTok{theme\_bw}\NormalTok{(}\AttributeTok{base\_size =} \DecValTok{13}\NormalTok{) }\SpecialCharTok{+}
  \FunctionTok{theme}\NormalTok{(}\AttributeTok{legend.position =} \StringTok{"none"}\NormalTok{)}
\end{Highlighting}
\end{Shaded}

\begin{figure}[H]

\centering{

\includegraphics[width=0.8\linewidth,height=\textheight,keepaspectratio]{08-graficos_files/figure-pdf/fig-bar3-1.pdf}

}

\caption{\label{fig-bar3}Gráfico de barras com cores da paleta NEJM}

\end{figure}%

\subsection{Proporção ou
porcentagem}\label{proporuxe7uxe3o-ou-porcentagem}

Na Figura~\ref{fig-bar3}, a unidade do eixo \emph{y} encontra-se como
uma proporção \texttt{y\ =\ after\_stat(count/sum(count)}, ou seja, y =
\emph{frequência por categoria}/\emph{total de observações}.\\
É possível modificar para porcentagem (Figura~\ref{fig-bar3a}),
empregando a função \texttt{percent\_format()} do
\texttt{pacote\ scales}. O código é praticamente igual, apenas
acrescentar o argumento
\texttt{labels\ =\ percent\_format(accuracy\ =\ 0.1,\ decimal.mark\ =\ “,”)}
dentro da função \texttt{scale\_y\_continuous()}.

\begin{Shaded}
\begin{Highlighting}[]
\FunctionTok{library}\NormalTok{(ggsci)}
\FunctionTok{library}\NormalTok{(scales)}

\FunctionTok{ggplot}\NormalTok{(}\AttributeTok{data =}\NormalTok{ dados) }\SpecialCharTok{+} 
  \FunctionTok{geom\_bar}\NormalTok{(}\FunctionTok{aes}\NormalTok{(}\AttributeTok{x =}\NormalTok{ categFumo, }
               \AttributeTok{y =} \FunctionTok{after\_stat}\NormalTok{(count}\SpecialCharTok{/}\FunctionTok{sum}\NormalTok{(count)),}
               \AttributeTok{fill =}\NormalTok{ categFumo))}\SpecialCharTok{+} 
  \FunctionTok{scale\_fill\_nejm}\NormalTok{() }\SpecialCharTok{+}
  \FunctionTok{scale\_y\_continuous}\NormalTok{ (}\AttributeTok{labels =} \FunctionTok{percent\_format}\NormalTok{ (}\AttributeTok{accuracy =} \FloatTok{0.1}\NormalTok{,}
                                               \AttributeTok{decimal.mark =} \StringTok{","}\NormalTok{)) }\SpecialCharTok{+}
  \FunctionTok{labs}\NormalTok{(}\AttributeTok{x =} \StringTok{"Tabagismo Materno"}\NormalTok{, }
       \AttributeTok{y =} \StringTok{"Porcentagem por categoria"}\NormalTok{)  }\SpecialCharTok{+}
  \FunctionTok{theme\_bw}\NormalTok{(}\AttributeTok{base\_size =} \DecValTok{13}\NormalTok{) }\SpecialCharTok{+}
  \FunctionTok{theme}\NormalTok{(}\AttributeTok{legend.position =} \StringTok{"none"}\NormalTok{)}
\end{Highlighting}
\end{Shaded}

\begin{figure}[H]

\centering{

\includegraphics[width=0.8\linewidth,height=\textheight,keepaspectratio]{08-graficos_files/figure-pdf/fig-bar3a-1.pdf}

}

\caption{\label{fig-bar3a}Gráfico de barras com porcentagem no eixo y}

\end{figure}%

\subsection{Controle da largura da barra com
width}\label{controle-da-largura-da-barra-com-width}

Para controlar a largura e o espaço entre as barras, num gráfico de
barras no \texttt{ggplot2}, usar o argumento \texttt{width} dentro da
função \texttt{geom\_bar()}, definindo um valor entre 0 e 1 (ou um valor
fixo). Um valor de 1 representa a largura total, ou seja, não haverá
espaço entre as barras, como no histograma.

Como exemplo, será alterado a largura das barras do gráfico da figura
Figura~\ref{fig-bar3} . Se o objetivo é que as barras sejam mais
estreitas e com mais espaço entre elas, deve-se definir um valor para
\texttt{width} inferior a 0.9 (padrão). Na Figura~\ref{fig-bar3b}, será
usado \texttt{width\ =\ 0.5}.

\begin{Shaded}
\begin{Highlighting}[]
\FunctionTok{library}\NormalTok{(ggplot2)}
\FunctionTok{library}\NormalTok{(dplyr)}
\FunctionTok{library}\NormalTok{(ggsci)}

\FunctionTok{ggplot}\NormalTok{(}\AttributeTok{data =}\NormalTok{ dados) }\SpecialCharTok{+} 
  \FunctionTok{geom\_bar}\NormalTok{(}\FunctionTok{aes}\NormalTok{(}\AttributeTok{x =}\NormalTok{ categFumo, }
               \AttributeTok{y =} \FunctionTok{after\_stat}\NormalTok{(count}\SpecialCharTok{/}\FunctionTok{sum}\NormalTok{(count)),}
               \AttributeTok{fill =}\NormalTok{ categFumo),}
           \AttributeTok{width =} \FloatTok{0.5}\NormalTok{)}\SpecialCharTok{+} 
  \FunctionTok{scale\_fill\_nejm}\NormalTok{() }\SpecialCharTok{+}
  \FunctionTok{labs}\NormalTok{(}\AttributeTok{x =} \StringTok{"Tabagismo Materno"}\NormalTok{, }
       \AttributeTok{y =} \StringTok{"Proporção por categoria"}\NormalTok{)  }\SpecialCharTok{+}
  \FunctionTok{theme\_bw}\NormalTok{(}\AttributeTok{base\_size =} \DecValTok{13}\NormalTok{) }\SpecialCharTok{+}
  \FunctionTok{theme}\NormalTok{(}\AttributeTok{legend.position =} \StringTok{"none"}\NormalTok{)}
\end{Highlighting}
\end{Shaded}

\begin{figure}[H]

\centering{

\includegraphics[width=0.8\linewidth,height=\textheight,keepaspectratio]{08-graficos_files/figure-pdf/fig-bar3b-1.pdf}

}

\caption{\label{fig-bar3b}Gráfico de barras com cores da paleta NEJM}

\end{figure}%

\begin{tcolorbox}[enhanced jigsaw, bottomrule=.15mm, opacitybacktitle=0.6, colframe=quarto-callout-tip-color-frame, arc=.35mm, coltitle=black, toptitle=1mm, colback=white, colbacktitle=quarto-callout-tip-color!10!white, breakable, bottomtitle=1mm, rightrule=.15mm, titlerule=0mm, toprule=.15mm, opacityback=0, leftrule=.75mm, left=2mm, title=\textcolor{quarto-callout-tip-color}{\faLightbulb}\hspace{0.5em}{Exercício}]

Qual a diferença entre \texttt{geom\_bar()} e \texttt{geom\_col()}?
Construir o mesmo gráfico de barras da Figura~\ref{fig-bar3b}, usando a
\texttt{geom\_col()}.

\end{tcolorbox}

\ul{Resposta} \footnote{A \texttt{geom\_bar()} conta as ocorrências e
  usa a altura para representar essa contagem, enquanto
  \texttt{geom\_col()} usa a altura para representar o valor
  especificado na estética \emph{y} . Ambas as funções aceitam o
  argumento~\texttt{width}. Enquanto o \texttt{geom\_bar()} com
  \texttt{after\_start()} calcula os valores automaticamente, o
  \texttt{geom\_col()} exige que se forneça os valores de \emph{y} (as
  proporções) diretamente.}

\subsection{Gráfico de barras
empilhadas}\label{gruxe1fico-de-barras-empilhadas}

O gráfico de barras empilhadas é ideal para visualizar a proporção de
cada grupo dentro de uma categoria. A altura total da barra representa a
contagem total para a variável no eixo \emph{x}, e as cores dentro da
barra mostram a distribuição da segunda variável.

Para criá-lo, mapear a primeira variável para o \textbf{eixo x} e a
segunda variável para a \textbf{estética fill}. A função
\texttt{geom\_bar()} faz o empilhamento por padrão.

Com os mesmos dados, usados até nos gráficos de barras
(Seção~\ref{sec-dados8}), agora serão trabalhadas as variáveis
\texttt{categIdade} e \texttt{fumo} com o objetivo de ver a proporção de
tabagismo por faixa etária. Como aprimoramentos, se pretende colocar as
porcentagens de fumantes em cada uma das faixas etária no topo das
barras.

Em primeiro lugar, calcular as \emph{proporções} de fumante em cada uma
das faixas etárias e a \emph{posição vertical dos rótulos no eixo y}.

\begin{Shaded}
\begin{Highlighting}[]
\NormalTok{proporcoes\_fumo }\OtherTok{\textless{}{-}}\NormalTok{ dados }\SpecialCharTok{\%\textgreater{}\%}
  \FunctionTok{group\_by}\NormalTok{(categIdade, fumo) }\SpecialCharTok{\%\textgreater{}\%}
  \FunctionTok{summarise}\NormalTok{(}\AttributeTok{n =} \FunctionTok{n}\NormalTok{(), }\AttributeTok{.groups =} \StringTok{"drop"}\NormalTok{) }\SpecialCharTok{\%\textgreater{}\%}
  \FunctionTok{group\_by}\NormalTok{(categIdade) }\SpecialCharTok{\%\textgreater{}\%}
  \FunctionTok{mutate}\NormalTok{(}\AttributeTok{posicao\_y =} \FunctionTok{sum}\NormalTok{(n) }\SpecialCharTok{{-}}\NormalTok{ (}\FloatTok{0.5} \SpecialCharTok{*}\NormalTok{ n),}
         \AttributeTok{total\_faixa =} \FunctionTok{sum}\NormalTok{(n),}
         \AttributeTok{proporcao\_fumo =}\NormalTok{ n }\SpecialCharTok{/}\NormalTok{ total\_faixa) }\SpecialCharTok{\%\textgreater{}\%} 
  \FunctionTok{filter}\NormalTok{(fumo }\SpecialCharTok{==} \StringTok{"Fumante"}\NormalTok{)}
\NormalTok{proporcoes\_fumo}
\end{Highlighting}
\end{Shaded}

\begin{verbatim}
# A tibble: 3 x 6
# Groups:   categIdade [3]
  categIdade   fumo        n posicao_y total_faixa proporcao_fumo
  <fct>        <fct>   <int>     <dbl>       <int>          <dbl>
1 < 20 anos    Fumante    35      202.         219          0.160
2 20 a 35 anos Fumante   235      874.         992          0.237
3 > 35 anos    Fumante    31      142.         157          0.197
\end{verbatim}

Após realizado o cálculo das proporções de fumantes em cada faixa
etária, constrói-se o gráfico. Para colocar as porcentagens no
gráfico\footnote{Este acréscimo das porcentagens dentro do gráfico é
  opcional. Para fazer o mesmo gráfico sem esta informação, não há
  necessidade dos cálculos das proporções e nem do \texttt{geom\_text()}
  que deve ser removido.}, será usada a
\texttt{geometria\ geom\_text()}, informando essas porcentagens e a
localização no eixo \emph{x} e \emph{y}, resultando na
Figura~\ref{fig-bar4}.

\begin{Shaded}
\begin{Highlighting}[]
\FunctionTok{ggplot}\NormalTok{(dados, }\FunctionTok{aes}\NormalTok{(}\AttributeTok{x =}\NormalTok{ categIdade, }\AttributeTok{fill =}\NormalTok{ fumo)) }\SpecialCharTok{+}
  \FunctionTok{geom\_bar}\NormalTok{(}\AttributeTok{color =}\StringTok{"black"}\NormalTok{) }\SpecialCharTok{+}
  \FunctionTok{scale\_fill\_manual}\NormalTok{(}\AttributeTok{values =} \FunctionTok{c}\NormalTok{(}\StringTok{"gray90"}\NormalTok{,}\StringTok{"skyblue"}\NormalTok{)) }\SpecialCharTok{+}
  \FunctionTok{labs}\NormalTok{(}\AttributeTok{x =} \StringTok{"Faixa Etária"}\NormalTok{, }
       \AttributeTok{y =} \StringTok{"Frequência"}\NormalTok{,}
       \AttributeTok{fill =} \ConstantTok{NULL}\NormalTok{)  }\SpecialCharTok{+}
  \FunctionTok{geom\_text}\NormalTok{(}\AttributeTok{data =}\NormalTok{ proporcoes\_fumo,}
            \FunctionTok{aes}\NormalTok{(}\AttributeTok{x =}\NormalTok{ categIdade,}
                \AttributeTok{y =}\NormalTok{ posicao\_y,}
                \AttributeTok{label =}\NormalTok{ scales}\SpecialCharTok{::}\FunctionTok{percent}\NormalTok{(proporcao\_fumo, }\AttributeTok{accuracy =} \FloatTok{0.1}\NormalTok{)),}
            \AttributeTok{size =} \DecValTok{4}\NormalTok{,}
            \AttributeTok{color =} \StringTok{"black"}\NormalTok{ ) }\SpecialCharTok{+} 
  \FunctionTok{scale\_y\_continuous}\NormalTok{ (}\AttributeTok{expand =} \FunctionTok{expansion}\NormalTok{(}\AttributeTok{add =} \FunctionTok{c}\NormalTok{(}\DecValTok{0}\NormalTok{,}\FloatTok{0.05}\NormalTok{))) }\SpecialCharTok{+}
  \FunctionTok{labs}\NormalTok{(}\AttributeTok{x =} \StringTok{"Faixa Etária"}\NormalTok{,}
       \AttributeTok{y =} \StringTok{"Frequência"}\NormalTok{,}
       \AttributeTok{fill =} \ConstantTok{NULL}\NormalTok{) }\SpecialCharTok{+}
  \FunctionTok{theme\_classic}\NormalTok{(}\AttributeTok{base\_size =} \DecValTok{13}\NormalTok{) }\SpecialCharTok{+}
  \FunctionTok{theme}\NormalTok{(}\AttributeTok{legend.position =} \StringTok{"top"}\NormalTok{)}
\end{Highlighting}
\end{Shaded}

\begin{figure}[H]

\centering{

\includegraphics[width=0.8\linewidth,height=\textheight,keepaspectratio]{08-graficos_files/figure-pdf/fig-bar4-1.pdf}

}

\caption{\label{fig-bar4}Gráfico de barras empilhadas}

\end{figure}%

\subsection{Gráfico de barras lado a
lado}\label{gruxe1fico-de-barras-lado-a-lado}

O gráfico de barras lado a lado (ou agrupado) é útil para comparar
diretamente a contagem de cada grupo entre as categorias. As barras de
uma mesma categoria são dispostas lado a lado para facilitar a
comparação visual. Para criá-lo, se faz de maneira semelhante das barras
empilhadas. Mapear a primeira variável para o eixo \emph{x} e a segunda
para a estética \texttt{fill} e adicionar o argumento
\texttt{position\ =\ "dodge"} dentro do \texttt{geom\_bar()}. Este
argumento diz ao \texttt{ggplot2} para não empilhar as barras, mas sim
colocá-las lado a lado.\\
O \texttt{geom\_text\ ()} usa \texttt{position\_dodge()} para replicar o
comportamento das barras. O argumento \texttt{width\ =\ 0.9} é o valor
padrão para a largura das barras no \texttt{ggplot2} e garante um
alinhamento perfeito. O gráfico de barras lado a lado é, portanto
construído assim:

\begin{enumerate}
\def\labelenumi{\arabic{enumi})}
\tightlist
\item
  Incialmente, calcula-se as proporções das categorias e a posição
  \emph{y} dos rótulos:
\end{enumerate}

\begin{Shaded}
\begin{Highlighting}[]
\NormalTok{prop\_fumo }\OtherTok{\textless{}{-}}\NormalTok{ dados }\SpecialCharTok{\%\textgreater{}\%}
  \FunctionTok{group\_by}\NormalTok{(categIdade, fumo) }\SpecialCharTok{\%\textgreater{}\%}
  \FunctionTok{summarise}\NormalTok{(}\AttributeTok{n =} \FunctionTok{n}\NormalTok{(), }\AttributeTok{.groups =} \StringTok{"drop"}\NormalTok{) }\SpecialCharTok{\%\textgreater{}\%}
  \FunctionTok{group\_by}\NormalTok{(categIdade) }\SpecialCharTok{\%\textgreater{}\%}
  \FunctionTok{mutate}\NormalTok{(}
    \CommentTok{\# Calcula a posição vertical do ponto médio de cada segmento}
    \AttributeTok{posicao\_y =}\NormalTok{ (n) }\SpecialCharTok{{-}} \DecValTok{18}\NormalTok{,}
    \AttributeTok{total\_faixa =} \FunctionTok{sum}\NormalTok{(n),}
    \AttributeTok{proporcao\_fumo =}\NormalTok{ n }\SpecialCharTok{/}\NormalTok{ total\_faixa)}
\NormalTok{prop\_fumo}
\end{Highlighting}
\end{Shaded}

\begin{verbatim}
# A tibble: 6 x 6
# Groups:   categIdade [3]
  categIdade   fumo            n posicao_y total_faixa proporcao_fumo
  <fct>        <fct>       <int>     <dbl>       <int>          <dbl>
1 < 20 anos    Fumante        35        17         219          0.160
2 < 20 anos    Não fumante   184       166         219          0.840
3 20 a 35 anos Fumante       235       217         992          0.237
4 20 a 35 anos Não fumante   757       739         992          0.763
5 > 35 anos    Fumante        31        13         157          0.197
6 > 35 anos    Não fumante   126       108         157          0.803
\end{verbatim}

\begin{enumerate}
\def\labelenumi{\arabic{enumi})}
\setcounter{enumi}{1}
\tightlist
\item
  Com esses dados, constrói-se o gráfico:
\end{enumerate}

\begin{Shaded}
\begin{Highlighting}[]
\FunctionTok{ggplot}\NormalTok{(dados, }\FunctionTok{aes}\NormalTok{(}\AttributeTok{x =}\NormalTok{ categIdade, }\AttributeTok{fill =}\NormalTok{ fumo)) }\SpecialCharTok{+}
  \FunctionTok{geom\_bar}\NormalTok{(}\AttributeTok{position =} \StringTok{"dodge"}\NormalTok{,}
           \AttributeTok{color =}\StringTok{"black"}\NormalTok{) }\SpecialCharTok{+}
  \FunctionTok{scale\_fill\_manual}\NormalTok{(}\AttributeTok{values =} \FunctionTok{c}\NormalTok{(}\StringTok{"gray90"}\NormalTok{,}\StringTok{"skyblue"}\NormalTok{)) }\SpecialCharTok{+}
  \FunctionTok{labs}\NormalTok{(}\AttributeTok{x =} \StringTok{"Faixa Etária"}\NormalTok{, }
       \AttributeTok{y =} \StringTok{"Frequência"}\NormalTok{) }\SpecialCharTok{+}
  \FunctionTok{geom\_text}\NormalTok{(}\AttributeTok{data =}\NormalTok{ prop\_fumo,}
            \FunctionTok{aes}\NormalTok{(}\AttributeTok{x =}\NormalTok{ categIdade,}
                \AttributeTok{y =}\NormalTok{ posicao\_y,}
                \AttributeTok{label =}\NormalTok{ scales}\SpecialCharTok{::}\FunctionTok{percent}\NormalTok{(proporcao\_fumo, }\AttributeTok{accuracy =} \FloatTok{0.1}\NormalTok{)),}
            \AttributeTok{size =} \FloatTok{3.5}\NormalTok{,}
            \AttributeTok{color =} \StringTok{"black"}\NormalTok{ ,}
            \AttributeTok{position =} \FunctionTok{position\_dodge}\NormalTok{(}\AttributeTok{width =} \FloatTok{0.9}\NormalTok{)) }\SpecialCharTok{+} 
  \FunctionTok{scale\_y\_continuous}\NormalTok{ (}\AttributeTok{expand =} \FunctionTok{expansion}\NormalTok{(}\AttributeTok{add =} \FunctionTok{c}\NormalTok{(}\DecValTok{0}\NormalTok{,}\FloatTok{0.10}\NormalTok{))) }\SpecialCharTok{+}
  \FunctionTok{labs}\NormalTok{(}\AttributeTok{x =} \StringTok{"Faixa Etária"}\NormalTok{,}
       \AttributeTok{y =} \StringTok{"Frequência"}\NormalTok{,}
       \AttributeTok{fill =} \StringTok{""}\NormalTok{) }\SpecialCharTok{+}
  \FunctionTok{theme\_classic}\NormalTok{(}\AttributeTok{base\_size =} \DecValTok{12}\NormalTok{) }\SpecialCharTok{+}
  \FunctionTok{theme}\NormalTok{(}\AttributeTok{legend.position =} \StringTok{"top"}\NormalTok{)}
\end{Highlighting}
\end{Shaded}

\begin{figure}[H]

\centering{

\includegraphics[width=0.8\linewidth,height=\textheight,keepaspectratio]{08-graficos_files/figure-pdf/fig-bar5-1.pdf}

}

\caption{\label{fig-bar5}Gráfico de barras lado a lado}

\end{figure}%

A Figura~\ref{fig-bar5} mostra as porcentagens de fumantes em cada uma
das categorias de uma forma bem clara.

\subsection{Gráfico de barra para uma variável numérica
discreta}\label{gruxe1fico-de-barra-para-uma-variuxe1vel-numuxe9rica-discreta}

Uma variável numérica discreta é um tipo de variável que assume valores
inteiros, resultantes de contagens, e que não podem assumir valores
fracionários entre eles. A medição ocorre através da contagem das
observações (Seção~\ref{sec-tipovariavel}). Para a representação
gráfica, utiliza-se um gráfico de barras . O resultado é semelhante a um
histograma com as barras separadas. Para o exemplo, serão usados os
mesmos dados empregados na construção dos gráficos de barras empilhadas
e lado a lado, provenientes do conjunto de dados
\texttt{dadosMater.xlsx}. Nesses dados, existe a variável para que
representa o número de filhos que a gestante teve antes do atual. É uma
variável numérica discreta que será mostrada visualmente por gráfico de
barras simples (Figura~\ref{fig-bar6}).

\begin{tcolorbox}[enhanced jigsaw, bottomrule=.15mm, opacitybacktitle=0.6, colframe=quarto-callout-important-color-frame, arc=.35mm, coltitle=black, toptitle=1mm, colback=white, colbacktitle=quarto-callout-important-color!10!white, breakable, bottomtitle=1mm, rightrule=.15mm, titlerule=0mm, toprule=.15mm, opacityback=0, leftrule=.75mm, left=2mm, title=\textcolor{quarto-callout-important-color}{\faExclamation}\hspace{0.5em}{Importante}]

Apesar da variável para ser uma variável numérica discreta, para a
construção do gráfico ela foi \textbf{transformada em um fator} para
informar ao \texttt{ggplot2} que todos os rótulos do eixo \emph{x} (nº
de filhos anteriores) devem aparecer, inclusive o referente a 7 filhos
anteriores, apesar de não existir na amostra.

\end{tcolorbox}

\begin{Shaded}
\begin{Highlighting}[]
\CommentTok{\# Criar uma tabela com contagem completa, incluindo zeros}
\NormalTok{contagem }\OtherTok{\textless{}{-}} \FunctionTok{as.data.frame}\NormalTok{(}\FunctionTok{table}\NormalTok{(}\FunctionTok{factor}\NormalTok{(dados}\SpecialCharTok{$}\NormalTok{para, }\AttributeTok{levels =} \DecValTok{0}\SpecialCharTok{:}\DecValTok{11}\NormalTok{)))}

\CommentTok{\# Calcular proporção de cada barra}
\NormalTok{contagem}\SpecialCharTok{$}\NormalTok{prop }\OtherTok{\textless{}{-}}\NormalTok{ contagem}\SpecialCharTok{$}\NormalTok{Freq }\SpecialCharTok{/} \FunctionTok{sum}\NormalTok{(contagem}\SpecialCharTok{$}\NormalTok{Freq)}

\CommentTok{\# Plotar}
\FunctionTok{ggplot}\NormalTok{(contagem, }\FunctionTok{aes}\NormalTok{(}\AttributeTok{x =}\NormalTok{ Var1, }\AttributeTok{y =}\NormalTok{ prop)) }\SpecialCharTok{+}
  \FunctionTok{geom\_bar}\NormalTok{(}\AttributeTok{stat =} \StringTok{"identity"}\NormalTok{, }
           \AttributeTok{fill =} \StringTok{"tomato"}\NormalTok{, }\AttributeTok{color =} \StringTok{"gray30"}\NormalTok{) }\SpecialCharTok{+}
  \FunctionTok{geom\_text}\NormalTok{(}\FunctionTok{aes}\NormalTok{(}\AttributeTok{label =}\NormalTok{ scales}\SpecialCharTok{::}\FunctionTok{percent}\NormalTok{(prop, }\AttributeTok{accuracy =} \FloatTok{0.1}\NormalTok{),}
                \AttributeTok{y =}\NormalTok{ prop }\SpecialCharTok{+} \FloatTok{0.01}\NormalTok{), }\AttributeTok{size =} \FloatTok{3.5}\NormalTok{, }\AttributeTok{color =} \StringTok{"black"}\NormalTok{) }\SpecialCharTok{+}
  \FunctionTok{scale\_y\_continuous}\NormalTok{ (}\AttributeTok{expand =} \FunctionTok{expansion}\NormalTok{(}\AttributeTok{add =} \FunctionTok{c}\NormalTok{(}\DecValTok{0}\NormalTok{,}\FloatTok{0.05}\NormalTok{))) }\SpecialCharTok{+}
  \FunctionTok{labs}\NormalTok{(}\AttributeTok{x =} \StringTok{"Número de filhos anteriores ao atual"}\NormalTok{, }
       \AttributeTok{y =} \StringTok{"Proporção"}\NormalTok{) }\SpecialCharTok{+}
  \FunctionTok{theme\_classic}\NormalTok{(}\AttributeTok{base\_size =} \DecValTok{13}\NormalTok{)}
\end{Highlighting}
\end{Shaded}

\begin{figure}[H]

\centering{

\includegraphics[width=0.8\linewidth,height=\textheight,keepaspectratio]{08-graficos_files/figure-pdf/fig-bar6-1.pdf}

}

\caption{\label{fig-bar6}Gráfico de barras de uma variável discreta}

\end{figure}%

\subsection{Gráfico de barra de erro}\label{gruxe1fico-de-barra-de-erro}

Um gráfico de barra de erro é uma ferramenta visual que mostra a
variabilidade de dados em um ponto específico. Ele consiste em pontos ou
barras que representam as médias (ou outras estatísticas) de um conjunto
de dados, com linhas verticais (ou horizontais) que indicam o intervalo
de confiança, o desvio padrão ou o erro padrão da média. Essas linhas
verticais são conhecidas como ``barras de erro''. Usado para comparar as
médias de diferentes grupos, mostrando a variabilidade dentro de cada
grupo. É visto com frequência em pesquisas científicas e publicações
para apresentar os resultados experimentais com suas respectivas
variabilidades. As barras de erro dão uma ideia geral de quão precisa é
uma medição. O cenário para a construção de um gráfico de barra de erro
é o tabagismo na gestação e o peso dos recém-nascidos, onde as barras
representarão a média do peso ao nascer (g) e as barras de erro com
intervalo de confiança de 95\% (veja Capítulo~\ref{sec-estimacao}),
calculado usando média \(\pm\) margem de erro, onde a margem de
\(erro = 1.96 × erro \ padrão\).

\begin{enumerate}
\def\labelenumi{\arabic{enumi})}
\tightlist
\item
  Resumo dos dados: Após carregar os dados necessários
  (\texttt{dadosRNT}), se fará um resumo dos mesmos que informará as
  respectivas médias, desvios padrão e margens de erro por grupo.
\end{enumerate}

\begin{Shaded}
\begin{Highlighting}[]
\NormalTok{  resumo }\OtherTok{\textless{}{-}}\NormalTok{ dadosRNT }\SpecialCharTok{\%\textgreater{}\%} 
    \FunctionTok{group\_by}\NormalTok{(sexo, fumo) }\SpecialCharTok{\%\textgreater{}\%} 
    \FunctionTok{summarise}\NormalTok{(}\AttributeTok{n =} \FunctionTok{n}\NormalTok{(),}
              \AttributeTok{media =} \FunctionTok{mean}\NormalTok{(pesoRN, }\AttributeTok{na.rm =} \ConstantTok{TRUE}\NormalTok{),}
              \AttributeTok{dp =} \FunctionTok{sd}\NormalTok{(pesoRN, }\AttributeTok{na.rm =} \ConstantTok{TRUE}\NormalTok{),}
              \AttributeTok{me =} \FloatTok{1.96} \SpecialCharTok{*}\NormalTok{ dp}\SpecialCharTok{/}\FunctionTok{sqrt}\NormalTok{(n),}
              \AttributeTok{min =}\FunctionTok{min}\NormalTok{(pesoRN, }\AttributeTok{na.rm =} \ConstantTok{TRUE}\NormalTok{),}
              \AttributeTok{max =}\FunctionTok{max}\NormalTok{(pesoRN, }\AttributeTok{na.rm =} \ConstantTok{TRUE}\NormalTok{),}
              \AttributeTok{.groups =} \StringTok{\textquotesingle{}drop\textquotesingle{}}\NormalTok{)}
  \FunctionTok{print}\NormalTok{(resumo)}
\end{Highlighting}
\end{Shaded}

\begin{verbatim}
# A tibble: 4 x 8
  sexo      fumo            n media    dp    me   min   max
  <fct>     <fct>       <int> <dbl> <dbl> <dbl> <dbl> <dbl>
1 Masculino Fumante       122 3162.  464.  82.4  1440  4410
2 Masculino Não fumante   470 3303.  453.  40.9  1425  4950
3 Feminino  Fumante       110 2998.  503.  94.1  1715  4620
4 Feminino  Não fumante   383 3190.  435.  43.6  2090  4485
\end{verbatim}

\begin{enumerate}
\def\labelenumi{\arabic{enumi})}
\setcounter{enumi}{1}
\item
  Construção do gráfico (Figura~\ref{fig-bar7}) , conforme explicado
  abaixo:

  a) O gráfico inicia com a colocação a variável \texttt{sexo} no eixo
  \emph{x} e média da variável \texttt{pesoRN} no \emph{y} e
  estabelecendo cores diferentes para o fator \texttt{sexo};

  b) \texttt{geom\_bar(stat\ =\ “identity”} usa os valores reais da
  média para a altura das barras, \texttt{color\ =\ “black”} estabelece
  a cor preta para o contorno das barras e \texttt{position\_dodge(0.9)}
  separa as barras lado a lado para cada grupo de \texttt{fumo};

  c) \texttt{geom\_point\ (position\ =\ position\_dodge(0.9)} adiciona
  um ponto sobre cada barra (pode ser útil para destacar a média). É
  opcional;

  d) \texttt{geom\_errorbar()} adiciona a barra de erro acima da média,
  com base na barra de erro \texttt{me}. Poderia ter sido usado o desvio
  padrão. O erro está opcionalmente colocado acima da barra, mas poderia
  ser acima e abaixo ;

  e) \texttt{labs()} define os rótulos dos eixos e da legenda;

  f) \texttt{coord\_cartesian(ylim\ =\ c(0,\ 3500))} limita o eixo
  \emph{y} de 0 a 3500, sem cortar dados fora desse intervalo.
  Entretanto, para reduzir a altura das barras, pode-se cortar dados,
  por exemplo começar em1000, 1500 ou, mesmo, 2000, uma vez que não
  existem recém-nascidos, nesta amostra, com menos de 2000 g e o foco é
  a média e o IC95\%;

  g) \texttt{scale\_fill\_manual(values\ =\ c("gray80"},
  \texttt{"darkslategray1"))} estabelece as cores para os níveis de
  \texttt{fumo};

  h)
  \texttt{scale\_y\_continuous(\ breaks\ =\ seq(0,\ 3500,\ 500)\ ,\ expand\ =\ expansion(add\ =\ c(0,\ 0.05)))}
  - a primeira parte define os rótulos do eixo \emph{y} de 500 em 500,
  começando em 0 \footnote{Modificar se no \texttt{coord\_cartesian()} o
    valor inicial for alterado, por exemplo, 1000, 1500 ou 2000.}, a
  segunda adiciona um pequeno espaço acima das barras para não cortar os
  rótulos (adiante será discutido sobre isso);

  i) Por último colocou o tema clássico\footnote{Pode ser qualquer tema,
    este fica bem por ser bem limpo, sem grades, apenas eixo \emph{x} e
    \emph{y}.} do \texttt{ggplot2} com fonte maior para melhorar a
  leitura.
\end{enumerate}

\begin{Shaded}
\begin{Highlighting}[]
\FunctionTok{ggplot}\NormalTok{(resumo, }\FunctionTok{aes}\NormalTok{(}\AttributeTok{x=}\NormalTok{sexo, }\AttributeTok{y=}\NormalTok{media, }\AttributeTok{fill=}\NormalTok{fumo)) }\SpecialCharTok{+}      
  \FunctionTok{geom\_bar}\NormalTok{(}\AttributeTok{stat=}\StringTok{"identity"}\NormalTok{, }\AttributeTok{color=}\StringTok{"black"}\NormalTok{, }
           \AttributeTok{position=}\FunctionTok{position\_dodge}\NormalTok{(}\FloatTok{0.9}\NormalTok{)) }\SpecialCharTok{+}
  \FunctionTok{geom\_point}\NormalTok{(}\AttributeTok{position=}\FunctionTok{position\_dodge}\NormalTok{(}\FloatTok{0.9}\NormalTok{)) }\SpecialCharTok{+}
  \FunctionTok{geom\_errorbar}\NormalTok{(}\FunctionTok{aes}\NormalTok{(}\AttributeTok{ymin =}\NormalTok{ media, }\AttributeTok{ymax =}\NormalTok{ media}\SpecialCharTok{+}\NormalTok{me), }\AttributeTok{width=}\FloatTok{0.2}\NormalTok{,}
                \AttributeTok{position=}\FunctionTok{position\_dodge}\NormalTok{(.}\DecValTok{9}\NormalTok{)) }\SpecialCharTok{+}
  \FunctionTok{labs}\NormalTok{(}\AttributeTok{x=}\StringTok{""}\NormalTok{, }
       \AttributeTok{y =} \StringTok{"Peso do Recém{-}Nascido(g)"}\NormalTok{,}
       \AttributeTok{fill =} \StringTok{""}\NormalTok{) }\SpecialCharTok{+}
  \FunctionTok{coord\_cartesian}\NormalTok{(}\AttributeTok{ylim =} \FunctionTok{c}\NormalTok{(}\DecValTok{1500}\NormalTok{, }\DecValTok{3500}\NormalTok{)) }\SpecialCharTok{+}
  \FunctionTok{scale\_fill\_manual}\NormalTok{(}\AttributeTok{values =} \FunctionTok{c}\NormalTok{(}\StringTok{"gray80"}\NormalTok{, }
                               \StringTok{"darkslategray1"}\NormalTok{)) }\SpecialCharTok{+}
  \FunctionTok{scale\_y\_continuous}\NormalTok{ (}\AttributeTok{breaks =} \FunctionTok{seq}\NormalTok{(}\DecValTok{1500}\NormalTok{, }\DecValTok{3500}\NormalTok{, }\DecValTok{500}\NormalTok{),}
                      \AttributeTok{expand =} \FunctionTok{expansion}\NormalTok{(}\AttributeTok{add =} \FunctionTok{c}\NormalTok{(}\DecValTok{0}\NormalTok{,}\FloatTok{0.05}\NormalTok{))) }\SpecialCharTok{+}
  \FunctionTok{theme\_classic}\NormalTok{(}\AttributeTok{base\_size =} \DecValTok{13}\NormalTok{) }\SpecialCharTok{+}
  \FunctionTok{theme}\NormalTok{(}\AttributeTok{legend.position =} \StringTok{"top"}\NormalTok{)}
\end{Highlighting}
\end{Shaded}

\begin{figure}[H]

\centering{

\includegraphics[width=0.8\linewidth,height=\textheight,keepaspectratio]{08-graficos_files/figure-pdf/fig-bar7-1.pdf}

}

\caption{\label{fig-bar7}Gráfico de barras de erro}

\end{figure}%

\section{Manipulando outras partes dos
gráficos}\label{manipulando-outras-partes-dos-gruxe1ficos}

\subsection{Mudando o nome dos eixos, o nome e a ordem dos
rótulos}\label{mudando-o-nome-dos-eixos-o-nome-e-a-ordem-dos-ruxf3tulos}

Para modificar o nome dos eixos, utiliza-se, com frequência, as funções
\texttt{xlab()} e \texttt{ylab()} como no gráfico da
Figura~\ref{fig-scatter4} . O mesmo trabalho de alteração dos rótulos
pode ser feito com a função labs() como no gráfico da figura
Figura~\ref{fig-scatter10} .

O nome e ordem dos rótulos podem ser modificados, usando a função
\texttt{scale\_x\_discrete()} com os argumentos \texttt{limits\ =} que
coloca os níveis na ordem desejada e \texttt{labels\ =} que coloca os
novos nomes\footnote{Pode-se aproveitar aqui para trocar os nomes ou ,
  simplesmente, corrigir acentuação que, às vezes, não foi colocada no
  dataframe.} na ordem estabelecida pelo argumento \texttt{limits\ =}.

Observando, por exemplo, o gráfico da Figura~\ref{fig-bar3}, onde se
usou a paleta do NEJM, verifica-se que os rótulos do eixo \emph{x} estão
como: \texttt{fumante\_leve}, \texttt{fumante\_moderada},
\texttt{fumante\_pesada} e \texttt{nao\_fumante}. Esses nomes não estão
prontos para publicação e o ideal é que sejam modificados para
\texttt{Leve}, \texttt{Moderado}, \texttt{Pesado} e \texttt{Não}, uma
vez que o título do eixo x será modificado para
\texttt{Tabagismo\ Materno}. Aproveitando, pode-se modificar a ordem das
categorias, colocando, por exemplo, as fumantes pesadas como primeira
categoria na Figura~\ref{fig-bar8}, a seguir as fumantes moderadas,
leves e não fumantes para ter uma lógico crescente da intensidade de
tabagismo materno.

\begin{enumerate}
\def\labelenumi{\arabic{enumi})}
\tightlist
\item
  Cálculo das proporções de tabagismo em cada uma das categorias para
  adicionar ao gráfico, melhorando as informações
\end{enumerate}

\begin{Shaded}
\begin{Highlighting}[]
\NormalTok{prop\_fumo }\OtherTok{\textless{}{-}}\NormalTok{ dados }\SpecialCharTok{\%\textgreater{}\%}
  \FunctionTok{group\_by}\NormalTok{(categFumo) }\SpecialCharTok{\%\textgreater{}\%}
  \FunctionTok{summarise}\NormalTok{(}\AttributeTok{n =} \FunctionTok{n}\NormalTok{(), }\AttributeTok{.groups =} \StringTok{"drop"}\NormalTok{) }\SpecialCharTok{\%\textgreater{}\%}
  \FunctionTok{mutate}\NormalTok{(}
    \CommentTok{\# Calcula a posição vertical do ponto médio de cada segmento}
    \AttributeTok{posicao\_y =}\NormalTok{ (n) }\SpecialCharTok{+} \DecValTok{30}\NormalTok{,}
    \AttributeTok{total\_faixa =} \FunctionTok{sum}\NormalTok{(n),}
    \AttributeTok{proporcao\_fumo =}\NormalTok{ n }\SpecialCharTok{/}\NormalTok{ total\_faixa)}
\NormalTok{prop\_fumo}
\end{Highlighting}
\end{Shaded}

\begin{verbatim}
# A tibble: 4 x 5
  categFumo            n posicao_y total_faixa proporcao_fumo
  <fct>            <int>     <dbl>       <int>          <dbl>
1 nao_fumante       1067      1097        1368         0.780 
2 fumante_leve       157       187        1368         0.115 
3 fumante_moderada    37        67        1368         0.0270
4 fumante_pesada     107       137        1368         0.0782
\end{verbatim}

\begin{enumerate}
\def\labelenumi{\arabic{enumi})}
\setcounter{enumi}{1}
\tightlist
\item
  Construção do gráfico que resultará na Figura~\ref{fig-bar8}.
\end{enumerate}

\begin{Shaded}
\begin{Highlighting}[]
\FunctionTok{ggplot}\NormalTok{(}\AttributeTok{data =}\NormalTok{ dados, }\FunctionTok{aes}\NormalTok{(}\AttributeTok{x =}\NormalTok{ categFumo, }\AttributeTok{fill =}\NormalTok{ categFumo)) }\SpecialCharTok{+} 
  \FunctionTok{geom\_bar}\NormalTok{(}\AttributeTok{position =}\StringTok{"dodge"}\NormalTok{, }\AttributeTok{color =}\StringTok{"black"}\NormalTok{)}\SpecialCharTok{+} 
  \FunctionTok{scale\_fill\_nejm}\NormalTok{() }\SpecialCharTok{+}
  \FunctionTok{labs}\NormalTok{(}\AttributeTok{x =} \StringTok{"Tabagismo Materno"}\NormalTok{, }
       \AttributeTok{y =} \StringTok{"Frequência"}\NormalTok{)  }\SpecialCharTok{+}
  \FunctionTok{scale\_x\_discrete}\NormalTok{(}\AttributeTok{limits =} \FunctionTok{c}\NormalTok{(}\StringTok{"fumante\_pesada"}\NormalTok{, }
                              \StringTok{"fumante\_moderada"}\NormalTok{, }
                              \StringTok{"fumante\_leve"}\NormalTok{,}
                              \StringTok{"nao\_fumante"}\NormalTok{),}
                   \AttributeTok{labels =} \FunctionTok{c}\NormalTok{(}\StringTok{"Pesado"}\NormalTok{, }
                              \StringTok{"Moderado"}\NormalTok{, }
                              \StringTok{"Leve"}\NormalTok{,}
                              \StringTok{"Não"}\NormalTok{)) }\SpecialCharTok{+}
  \FunctionTok{geom\_text}\NormalTok{(}\AttributeTok{data =}\NormalTok{ prop\_fumo,}
            \FunctionTok{aes}\NormalTok{(}\AttributeTok{x =}\NormalTok{ categFumo,}
                \AttributeTok{y =}\NormalTok{ posicao\_y,}
                \AttributeTok{label =}\NormalTok{ scales}\SpecialCharTok{::}\FunctionTok{percent}\NormalTok{(proporcao\_fumo, }\AttributeTok{accuracy =} \FloatTok{0.1}\NormalTok{)),}
            \AttributeTok{size =} \DecValTok{4}\NormalTok{,}
            \AttributeTok{color =} \StringTok{"black"}\NormalTok{ ) }\SpecialCharTok{+}
  \FunctionTok{theme\_bw}\NormalTok{(}\AttributeTok{base\_size=}\DecValTok{13}\NormalTok{) }\SpecialCharTok{+}
  \FunctionTok{theme}\NormalTok{(}\AttributeTok{legend.position =} \StringTok{"none"}\NormalTok{)}
\end{Highlighting}
\end{Shaded}

\begin{figure}[H]

\centering{

\includegraphics[width=0.8\linewidth,height=\textheight,keepaspectratio]{08-graficos_files/figure-pdf/fig-bar8-1.pdf}

}

\caption{\label{fig-bar8}Gráfico de barras modificado}

\end{figure}%

\subsection{Título e subtítulo do
gráfico}\label{tuxedtulo-e-subtuxedtulo-do-gruxe1fico}

Nem sempre necessários, o título, o subtítulo ou uma nota de rodapé
podem ser adicionados ao gráfico através da função \texttt{labs()},
usada anteriormente (por ex. na Figura~\ref{fig-scatter10}) para colocar
rótulos nos eixos \emph{x} e \emph{y}. Além de argumentos para colocar
título e subtítulo, a função \texttt{labs()} tem argumento para nota de
rodapé, \texttt{caption}.

Como exemplo, será plotado um gráfico com boxplots que ilustrem o
impacto do tabagismo materno sobre o peso do recém-nascido. Os dados
serão provenientes da amostra dadosRNT (Seção~\ref{sec-violin}).\\
Serão usados todos os argumento da função \texttt{labs()}, e se repetirá
o que foi feito na construção do gráfico da Figura~\ref{fig-bar8},
alterando os nomes dos rótulos do eixo \emph{x}. O código do gráfico da
Figura~\ref{fig-bxp6} vai ser atribuído a um objeto denominado
\texttt{bxp}:

\begin{Shaded}
\begin{Highlighting}[]
\NormalTok{bxp }\OtherTok{\textless{}{-}} \FunctionTok{ggplot}\NormalTok{(dadosRNT, }\FunctionTok{aes}\NormalTok{(}\AttributeTok{x =}\NormalTok{ categFumo, }
                            \AttributeTok{y =}\NormalTok{ pesoRN,}
                            \AttributeTok{fill =}\NormalTok{ categFumo)) }\SpecialCharTok{+}
  \FunctionTok{stat\_boxplot}\NormalTok{(}\AttributeTok{geom =} \StringTok{"errorbar"}\NormalTok{, }\AttributeTok{width =} \FloatTok{0.1}\NormalTok{) }\SpecialCharTok{+}
  \FunctionTok{geom\_boxplot}\NormalTok{() }\SpecialCharTok{+}
  \FunctionTok{scale\_fill\_brewer}\NormalTok{(}\AttributeTok{palette =} \StringTok{"Pastel2"}\NormalTok{)  }\SpecialCharTok{+}
  \FunctionTok{stat\_summary}\NormalTok{(}\AttributeTok{fun =} \StringTok{"mean"}\NormalTok{, }
                \AttributeTok{colour =} \StringTok{"red"}\NormalTok{, }
                \AttributeTok{size =} \DecValTok{3}\NormalTok{, }
                \AttributeTok{geom =} \StringTok{"point"}\NormalTok{) }\SpecialCharTok{+}
  \FunctionTok{labs}\NormalTok{(}\AttributeTok{title =} \StringTok{"Tabagismo Materno e Peso do Recém Nascido"}\NormalTok{, }
       \AttributeTok{subtitle =} \StringTok{"Maternidade do Hospital Geral de Caxias do Sul, 2008"}\NormalTok{,}
       \AttributeTok{x =} \StringTok{"Tabagismo Materno"}\NormalTok{,}
       \AttributeTok{y =} \StringTok{"Peso dos Recém{-}Nascidos (g)"}\NormalTok{,}
       \AttributeTok{caption =} \StringTok{"O ponto vermelho é a média de cada grupo"}\NormalTok{) }\SpecialCharTok{+}
  \FunctionTok{scale\_x\_discrete}\NormalTok{(}\AttributeTok{limits =} \FunctionTok{c}\NormalTok{(}\StringTok{"nao\_fumante"}\NormalTok{, }
                              \StringTok{"fumante\_leve"}\NormalTok{, }
                              \StringTok{"fumante\_moderada"}\NormalTok{, }
                              \StringTok{"fumante\_pesada"}\NormalTok{),}
                     \AttributeTok{labels =} \FunctionTok{c}\NormalTok{(}\StringTok{"Não"}\NormalTok{, }\StringTok{"Leve"}\NormalTok{, }
                                \StringTok{"Moderado"}\NormalTok{, }\StringTok{"Pesado"}\NormalTok{)) }\SpecialCharTok{+}
  \FunctionTok{theme\_bw}\NormalTok{(}\AttributeTok{base\_size =} \DecValTok{13}\NormalTok{) }\SpecialCharTok{+}
  \FunctionTok{theme}\NormalTok{(}\AttributeTok{legend.position =} \StringTok{"none"}\NormalTok{)}
  
\FunctionTok{print}\NormalTok{(bxp)}
\end{Highlighting}
\end{Shaded}

\begin{figure}[H]

\centering{

\includegraphics[width=0.8\linewidth,height=\textheight,keepaspectratio]{08-graficos_files/figure-pdf/fig-bxp6-1.pdf}

}

\caption{\label{fig-bxp6}Boxplots do impacto do tabagismo materno no
peso do recém-nascido}

\end{figure}%

\subsection{Modificação dos limites dos
eixos}\label{modificauxe7uxe3o-dos-limites-dos-eixos}

O pacote \texttt{ggplot2} possui uma família de funções \texttt{scale\_}
para modificar as propriedades referentes às escalas do gráfico. Como é
possível ter escalas de números, categorias, cores, datas, entre outras,
é disponibilizada uma função específica para cada tipo de escala.

Cada tipo fundamental é manipulado por uma das três funções construtoras
de escala: \texttt{continuous\_scale()}, \texttt{discrete\_scale()} e
\texttt{binned\_scale()}.

No gráfico da Figura~\ref{fig-bxp6}, os pesos dos recém-nascidos estão
dispostos em uma escala que varia a cada 1000 g. Para modificar esses
limites, pode-se usar a função \texttt{scale\_y\_continuous()} para ter
intervalos de 500 g.

O gráfico da Figura~\ref{fig-bxp6} foi designado para um objeto
denominado \texttt{bxp}. Isto facilita o trabalho, pois não há
necessidade de repetir todo o código que gerou o gráfico, apenas as
modificações:

\begin{Shaded}
\begin{Highlighting}[]
\NormalTok{bxp }\SpecialCharTok{+}
  \FunctionTok{theme}\NormalTok{(}\AttributeTok{plot.title =} \FunctionTok{element\_text}\NormalTok{(}\AttributeTok{size =} \DecValTok{14}\NormalTok{,}
                                  \AttributeTok{face =} \StringTok{"bold"}\NormalTok{),}
        \AttributeTok{plot.subtitle =} \FunctionTok{element\_text}\NormalTok{(}\AttributeTok{size =} \DecValTok{12}\NormalTok{,}
                                     \AttributeTok{face =} \StringTok{"bold"}\NormalTok{,}
                                     \AttributeTok{color =} \StringTok{"darkgreen"}\NormalTok{)) }\SpecialCharTok{+}
  \FunctionTok{scale\_y\_continuous}\NormalTok{(}\AttributeTok{breaks =} \FunctionTok{seq}\NormalTok{(}\DecValTok{1000}\NormalTok{, }\DecValTok{5000}\NormalTok{, }\DecValTok{500}\NormalTok{))}
\end{Highlighting}
\end{Shaded}

\begin{figure}[H]

\centering{

\includegraphics[width=0.8\linewidth,height=\textheight,keepaspectratio]{08-graficos_files/figure-pdf/fig-bxp7-1.pdf}

}

\caption{\label{fig-bxp7}Modificação dos limites do eixo y: 500 em 500}

\end{figure}%

Junto com a modificação dos limites dos eixo da Figura~\ref{fig-bxp7},
manipulou-se o título e subtítulo, usando a função \texttt{theme()} e
foi aumentado o tamanho da fonte, usou-se negrito e a cor do subtítulo
passou a ser ``darkgreen''.

\subsection{Modificação da
expansão}\label{modificauxe7uxe3o-da-expansuxe3o}

Voltando aos gráficos de barra, todos, com exceção do gráfico da
Figura~\ref{fig-bar7}, tem algo que incomoda ao autor: abaixo do valor 0
(zero) existe uma expansão, ou seja um espaço abaixo do 0. Isto,
visualmente, é desagradável.

Para que as barras tenham início exatamente no 0 (zero), pode-se
empregar a função \texttt{scale\_y\_continuous()} com o argumento
\texttt{expand\ =\ expansion\ (add\ =\ c(0,0.05))}, significando que não
se expande nada abaixo do 0 e se adiciona 5 unidades para cima, criando
uma margem superior. Comparar a Figura~\ref{fig-bar9} com a
Figura~\ref{fig-bar3}. O rótulos do eixo \emph{y} também foram
corrigidos.

\begin{Shaded}
\begin{Highlighting}[]
\FunctionTok{library}\NormalTok{(ggsci)}
\FunctionTok{library}\NormalTok{(scales)}

\FunctionTok{ggplot}\NormalTok{(}\AttributeTok{data =}\NormalTok{ dados) }\SpecialCharTok{+} 
  \FunctionTok{geom\_bar}\NormalTok{(}\FunctionTok{aes}\NormalTok{(}\AttributeTok{x =}\NormalTok{ categFumo, }
               \AttributeTok{y =} \FunctionTok{after\_stat}\NormalTok{(count}\SpecialCharTok{/}\FunctionTok{sum}\NormalTok{(count)),}
               \AttributeTok{fill =}\NormalTok{ categFumo))}\SpecialCharTok{+} 
  \FunctionTok{scale\_fill\_nejm}\NormalTok{() }\SpecialCharTok{+}
  \FunctionTok{scale\_y\_continuous}\NormalTok{ (}\AttributeTok{labels =} \FunctionTok{percent\_format}\NormalTok{ (}\AttributeTok{accuracy =} \FloatTok{0.1}\NormalTok{,}
                                               \AttributeTok{decimal.mark =} \StringTok{","}\NormalTok{)) }\SpecialCharTok{+}
  \FunctionTok{scale\_y\_continuous}\NormalTok{ (}\AttributeTok{expand =} \FunctionTok{expansion}\NormalTok{(}\AttributeTok{add =} \FunctionTok{c}\NormalTok{(}\DecValTok{0}\NormalTok{,}\FloatTok{0.05}\NormalTok{))) }\SpecialCharTok{+}
  \FunctionTok{scale\_x\_discrete}\NormalTok{(}\AttributeTok{limits =} \FunctionTok{c}\NormalTok{(}\StringTok{"nao\_fumante"}\NormalTok{, }
                              \StringTok{"fumante\_leve"}\NormalTok{, }
                              \StringTok{"fumante\_moderada"}\NormalTok{, }
                              \StringTok{"fumante\_pesada"}\NormalTok{),}
                   \AttributeTok{labels =} \FunctionTok{c}\NormalTok{(}\StringTok{"Não"}\NormalTok{, }\StringTok{"Leve"}\NormalTok{, }
                              \StringTok{"Moderado"}\NormalTok{, }\StringTok{"Pesado"}\NormalTok{)) }\SpecialCharTok{+}
  \FunctionTok{labs}\NormalTok{(}\AttributeTok{x =} \StringTok{"Tabagismo Materno"}\NormalTok{, }
       \AttributeTok{y =} \StringTok{"Proporção por categoria"}\NormalTok{)  }\SpecialCharTok{+}
  \FunctionTok{theme\_bw}\NormalTok{(}\AttributeTok{base\_size =} \DecValTok{13}\NormalTok{) }\SpecialCharTok{+}
  \FunctionTok{theme}\NormalTok{(}\AttributeTok{legend.position =} \StringTok{"none"}\NormalTok{)}
\end{Highlighting}
\end{Shaded}

\begin{figure}[H]

\centering{

\includegraphics[width=0.8\linewidth,height=\textheight,keepaspectratio]{08-graficos_files/figure-pdf/fig-bar9-1.pdf}

}

\caption{\label{fig-bar9}Gráfico de barras com remoção do espaço abaixo
de 0 (zero)}

\end{figure}%

\section{Ajustando o layout e as margens no
ggplot2}\label{ajustando-o-layout-e-as-margens-no-ggplot2}

\subsection{Modificação das margens com a função
theme()}\label{modificauxe7uxe3o-das-margens-com-a-funuxe7uxe3o-theme}

Usando o gráfico da Figura~\ref{fig-scatter5}, repetido aqui e
designando-o a um objeto com nome de gdsip:

\begin{Shaded}
\begin{Highlighting}[]
\NormalTok{gdisp }\OtherTok{\textless{}{-}} \FunctionTok{ggplot}\NormalTok{(}\AttributeTok{data =}\NormalTok{ dadosRNT100,}
                \FunctionTok{aes}\NormalTok{(}\AttributeTok{x =}\NormalTok{ compRN, }\AttributeTok{y =}\NormalTok{ pesoRN)) }\SpecialCharTok{+}
  \FunctionTok{geom\_point}\NormalTok{(}\AttributeTok{position =} \FunctionTok{position\_jitter}\NormalTok{(}\AttributeTok{width =} \FloatTok{0.2}\NormalTok{, }\AttributeTok{height =} \DecValTok{0}\NormalTok{),}
             \AttributeTok{color =} \StringTok{"gray20"}\NormalTok{,}
             \AttributeTok{fill =}\StringTok{"steelblue"}\NormalTok{,}
             \AttributeTok{shape =} \DecValTok{21}\NormalTok{, }
             \AttributeTok{alpha =} \DecValTok{1}\NormalTok{,}
             \AttributeTok{size =} \DecValTok{3}\NormalTok{,}
             \AttributeTok{stroke =}\DecValTok{1}\NormalTok{) }\SpecialCharTok{+}
  \FunctionTok{ylab}\NormalTok{(}\StringTok{"Peso do Recém{-}nascido (g)"}\NormalTok{) }\SpecialCharTok{+}
  \FunctionTok{xlab}\NormalTok{(}\StringTok{"Comprimento do Recém{-}nascido (cm)"}\NormalTok{)}\SpecialCharTok{+}
  \FunctionTok{theme\_classic}\NormalTok{(}\AttributeTok{base\_size =} \DecValTok{13}\NormalTok{)}
\FunctionTok{print}\NormalTok{(gdisp)}
\end{Highlighting}
\end{Shaded}

\begin{figure}[H]

{\centering \includegraphics[width=0.8\linewidth,height=\textheight,keepaspectratio]{08-graficos_files/figure-pdf/unnamed-chunk-12-1.pdf}

}

\caption{Gráfico de dispersão}

\end{figure}%

O objeto \texttt{gdisp} contém o gráfico de dispersão da
Figura~\ref{fig-scatter5} e pode ser modificado sem ter que digitar
todos os comandos novamente. Será usado para exemplicar como promover um
aumento das margens, partindo do padrão do tema usado:

\begin{Shaded}
\begin{Highlighting}[]
\FunctionTok{theme\_classic}\NormalTok{()}\SpecialCharTok{$}\NormalTok{plot.margin}
\end{Highlighting}
\end{Shaded}

\begin{verbatim}
[1] 5.5points 5.5points 5.5points 5.5points
\end{verbatim}

Para aumentar as margens usa-se:

\begin{Shaded}
\begin{Highlighting}[]
\NormalTok{gdisp }\SpecialCharTok{+}
  \FunctionTok{theme}\NormalTok{(}\AttributeTok{plot.margin =} \FunctionTok{margin}\NormalTok{(}\AttributeTok{t =} \DecValTok{80}\NormalTok{, }\AttributeTok{r =} \DecValTok{80}\NormalTok{, }\AttributeTok{b =} \DecValTok{80}\NormalTok{, }\AttributeTok{l =} \DecValTok{80}\NormalTok{))}
\end{Highlighting}
\end{Shaded}

\begin{figure}[H]

\centering{

\includegraphics[width=0.8\linewidth,height=\textheight,keepaspectratio]{08-graficos_files/figure-pdf/fig-scatter13-1.pdf}

}

\caption{\label{fig-scatter13}Gráfico de dispersão com margem reduzida}

\end{figure}%

A função \texttt{margin()} define as margens em pontos \footnote{A
  margem pode ser definida também em centímetros (cm), usando o mesmo
  comando, mas especificando que é em ``cm''. Por exemplo, para aumentar
  2 cm em todos os lados:
  \texttt{theme(plot.margin\ =\ unit(c(2,\ 2,\ 2,\ 20,\ "cm")}}. Um
ponto (pt) equivale a 1/72 de polegada, ou aproximadamente 0,35
milímetros. É a mesma unidade usada para definir tamanho de fonte.
portanto, quando se observa um valor de 80, significa que o gráfico terá
80 pontos de margem no topo, à direita, à esquerda e na base. Isto dá
aproximadamente 28 mm de espaço em cada lado. As letras \texttt{t},
\texttt{r}, \texttt{b} e \texttt{l} equivalem, respectivamente a
\texttt{top}, \texttt{right}, \texttt{bottom} e \texttt{left}.

\begin{tcolorbox}[enhanced jigsaw, bottomrule=.15mm, opacitybacktitle=0.6, colframe=quarto-callout-note-color-frame, arc=.35mm, coltitle=black, toptitle=1mm, colback=white, colbacktitle=quarto-callout-note-color!10!white, breakable, bottomtitle=1mm, rightrule=.15mm, titlerule=0mm, toprule=.15mm, opacityback=0, leftrule=.75mm, left=2mm, title=\textcolor{quarto-callout-note-color}{\faInfo}\hspace{0.5em}{Dica prática}]

Ao exportar gráficos para PDF ou PNG e quiser controlar o layout com
precisão (por exemplo, para publicação), ajustar as margens com
\texttt{margin()} é essencial para evitar que elementos fiquem cortados
ou apertados demais.

\end{tcolorbox}

\part{Parte IV - Probabilidades e Distribuições}

\chapter{Introdução à Teoria das
Probabilidades}\label{introduuxe7uxe3o-uxe0-teoria-das-probabilidades}

\section{Pacotes necessários neste
capítulo}\label{pacotes-necessuxe1rios-neste-capuxedtulo-1}

\begin{Shaded}
\begin{Highlighting}[]
\NormalTok{pacman}\SpecialCharTok{::}\FunctionTok{p\_load}\NormalTok{(dplyr, readxl, scales, ggplot2)}
\end{Highlighting}
\end{Shaded}

\section{Introdução}\label{introduuxe7uxe3o-2}

A teoria das probabilidades é a base sobre a qual a estatística é
desenvolvida. Os jogos de azar deram um grande impulso ao conhecimento
da moderna teoria das probabilidades, principalmente, pelo trabalho de
Blaise Pascal (1623-1662), em parceria com Pierre de Fermat (1601-1665).
Eles foram estimulados por um escritor francês e matemático amador,
Antoine Gombaud (1607-1684), conhecido como Chevalier de Méré, que era
muito interessado em jogos de azar (87).

A Teoria das probabilidades permite que seja possível modelar
populações, experimentos ou qualquer situação que possa ser considerada
aleatória. Estes modelos possibilitam fazer inferência sobre populações
a partir da observação de uma amostra dessa população. Ao usar apenas
uma parte da população, inevitavelmente, é cometido um erro, o
\textbf{erro amostral}. Este erro amostral pode ser dimensionado pela
teoria das probabilidades.

Existem duas interpretações alternativas de probabilidades: a
\textbf{frequentista} e a \textbf{bayesiana} (88). Neste livro, será
discutida, basicamente, a definição de probabilidade frequentista. O
processo bayesiano de formulação de um modelo probabilístico faz uso do
conhecimento subjetivo, estabelecendo uma especificação \textbf{a
priori}, combinado com a informação objetiva ou empírica. A teoria
bayesiana é a estrutura integradora dessas duas fontes de informação,
derivando como resultado a distribuição \textbf{a posteriori} dos
parâmetros de interesse. Na \textbf{?@sec-diagbayes}, sobre análise de
testes diagnósticos, serão abordados alguns aspectos relacionados à
teoria bayesiana em medicina.

\section{Processo aleatório}\label{processo-aleatuxf3rio}

Um processo ou experimento é dito \textbf{aleatório} quando em uma
situação se sabe quais os resultados que podem acontecer, mas não se
sabe qual resultado particular irá acontecer. Por exemplo, quando uma
moeda é lançada, se conhece que a probabilidade de o desfecho
\texttt{cara} ocorrer é de 50\%, mas se desconhece o que irá ocorrer até
que a moeda esteja no chão.

O número de caras que podem surgir em vários lançamentos da moeda é
chamado de \textbf{variável aleatória}, ou seja, uma variável que pode
assumir mais de um valor com determinadas probabilidades (89). Da mesma
forma, um dado lançado pode mostrar seis faces, numeradas de um a seis,
com igual probabilidade de 16,7\%. Portanto, quando a probabilidade é
associada a todos os conjuntos de valores possíveis de uma variável,
diz-se que ela é aleatória. O conjunto de todos os possíveis resultados
de um experimento aleatório é denominado \textbf{espaço amostral}.

Na área da saúde, trabalha-se com uma infinidade de variáveis
aleatórias, por exemplo, o número de filhos de uma mulher, o número de
mortos diários em uma epidemia, o número de vacinados em uma campanha,
etc. Essas variáveis são variáveis aleatórias discretas, pois apenas
permitem ser quantificadas por processo de contagem. Por outro lado, o
peso ou a altura de uma mulher são ditos variáveis aleatórias contínuas,
pois podem assumir qualquer valor real entre uma medida e outra,
dependendo da precisão do aparelho usado.

Em geral, variáveis aleatórias são representadas por letras maiúsculas,
como X, Y e Z e sua a probabilidade, por exemplo, pode ser denotada por:
\(P(X)\).

\section{Definição frequentista}\label{definiuxe7uxe3o-frequentista}

A probabilidade se relaciona a eventos futuros ou que ainda não
ocorreram, desta forma a probabilidade pode ser entendida como uma
medida de incerteza em relação ao evento. A probabilidade de um evento
ocorrer, em determinadas circunstâncias, pode ser definida como a
proporção de vezes que o evento é observado quando o experimento é
repetido um número infinitamente grande de vezes (88). Pode-se dizer que
a visão frequentista define a probabilidade como uma \textbf{frequência
de longo prazo}.

A chamada \textbf{Lei dos Grandes Números} diz que à medida que
múltiplas observações são coletadas, a proporção observada de
ocorrências de um determinado desfecho, após \emph{n} ensaios, converge
para a probabilidade real P desse desfecho. Ou seja, quanto mais vezes
for repetido uma experiência, a melhor estimativa de probabilidade tende
a ocorrer. Suponha que seja lançada uma moeda honesta repetidas vezes.
Por definição, essa é uma moeda que tem \(P(cara)=0,5\). O que se
observaria? O autor fez 20 lançamentos seguidos com uma mesma moeda e
obteve o seguinte resultado, onde 1 = cara (Figura~\ref{fig-moeda20}):

\begin{figure}[H]

\centering{

\includegraphics[width=0.8\linewidth,height=0.8\textheight]{index_files/mediabag/57v7i9n.png}

}

\caption{\label{fig-moeda20}Vinte lançamentos seguidos de uma moeda}

\end{figure}%

Neste caso, 10 (50\%) desses lançamentos deram cara. Agora, suponha que
foram feitos registros do número de caras (\(n_1\)) dos primeiros
lançamentos (\emph{N}) e calculadas as proporções de caras (\(n_1⁄N\))
todas as vezes. O resultado está na Figura~\ref{fig-propmoeda}.

\begin{figure}[H]

\centering{

\includegraphics[width=0.8\linewidth,height=0.8\textheight]{index_files/mediabag/joXyqmz.png}

}

\caption{\label{fig-propmoeda}Proporção em 20 lançamentos de moeda}

\end{figure}%

Observa-se, nessa sequência, que a proporção de caras flutua muito,
variando de 0,17 a 0,75. Se o número de lançamentos for aumentando
tem-se a sensação de que a proporção se aproxima da ``correta''. Por
exemplo, com 100 jogadas, obteve-se 53 caras (0,53); com 150 jogadas, 79
(0,53) e com 200 jogadas, 111 (0,56). Quando N se aproximar do infinito
(\(N \to\infty\)) a proporção de caras convergirá para 0,50. A definição
frequentista de probabilidade segue essa definição. Ninguém consegue um
número infinito de lançamentos de moedas, mas um computador pode simular
milhares de lançamentos. A Figura~\ref{fig-nmoeda} mostra o que acontece
com a proporção \(n_1⁄N\) à medida que \emph{N} aumenta em lançamentos
de moedas. As simulações foram repetidas 4 vezes somente para ter
certeza de que o que aconteceu não foi obra do acaso.

\begin{figure}[H]

\centering{

\includegraphics[width=0.8\linewidth,height=0.8\textheight]{index_files/mediabag/H1eeJNj.png}

}

\caption{\label{fig-nmoeda}Proporção à medida que N aumenta em
lançamentos de moedas}

\end{figure}%

Embora nenhuma das simulações tenha realmente terminado com um valor
exato de 0,5, elas se aproximaram, oscilando muito pouco em torno desse
valor.

\subsection{Aplicando a visão frequentista no dia a
dia}\label{sec-aplicafreq}

A definição frequentista também pode ser aplicada no cotidiano.
Utilizando a altura de 1368 mulheres, uma medida numérica contínua,
incluída no conjunto de dados \texttt{dadosMater.xlsx} (veja
Seção~\ref{sec-dadosMater}). Essas alturas serão selecionadas e
colocadas em um objeto, denominado \texttt{dados}.

\begin{Shaded}
\begin{Highlighting}[]
\NormalTok{ dados }\OtherTok{\textless{}{-}} \FunctionTok{read\_excel}\NormalTok{(}\StringTok{"dados/dadosMater.xlsx"}\NormalTok{) }\SpecialCharTok{\%\textgreater{}\%} 
  \FunctionTok{select}\NormalTok{(altura)}
\end{Highlighting}
\end{Shaded}

Usando a função \texttt{summary()}, será feito um resumo da variável
\texttt{altura}:

\begin{Shaded}
\begin{Highlighting}[]
\FunctionTok{summary}\NormalTok{(dados}\SpecialCharTok{$}\NormalTok{altura)}
\end{Highlighting}
\end{Shaded}

\begin{verbatim}
   Min. 1st Qu.  Median    Mean 3rd Qu.    Max. 
  1.400   1.550   1.600   1.598   1.650   1.850 
\end{verbatim}

A mediana da altura das gestantes é 1.6 m. Em um longo conjunto de
sorteios, a probabilidade de uma mulher ter altura acima devalor é 50\%.
O percentil 75 (3º quartil) é igual a 1.65 m, a probabilidade de estar
acima deste valor, portanto, é 25\%. É possível encontrar a
probabilidade de a altura estar acima, abaixo ou entre quaisquer
valores. Quando se faz a mensuração de uma variável contínua, fica-se
limitado ao método usado. Portanto, quando se diz que uma mulher tem 160
cm, significa dizer que está entre 159,5 e 160,5 cm, dependendo da
precisão do instrumento de medição. Dessa maneira, o interesse está na
probabilidade de a variável aleatória assumir valores entre certos
limites.

A probabilidade de encontrar um valor exatamente igual à média (159.8)
cm é quase igual a zero. Como se verá a seguir, isto pode ser
verificado, no \emph{R}, com bastante facilidade,calculando a distância
que esta medida está da média em número de desvios padrão (escore
\emph{Z}):

\begin{Shaded}
\begin{Highlighting}[]
\NormalTok{Z }\OtherTok{\textless{}{-}}\NormalTok{ (}\FloatTok{1.60} \SpecialCharTok{{-}} \FunctionTok{mean}\NormalTok{(dados}\SpecialCharTok{$}\NormalTok{altura))}\SpecialCharTok{/}\FunctionTok{sd}\NormalTok{(dados}\SpecialCharTok{$}\NormalTok{altura)}
\NormalTok{Z}
\end{Highlighting}
\end{Shaded}

\begin{verbatim}
[1] 0.03103551
\end{verbatim}

Observe que o valor de 1,60 m está muito próximo da média e isto é um
indicativo de que essa variável tem uma distribuição praticamente
simétrica. Sabendo a distância, em números de desvios padrão, que 1,60 m
está da média, qual a probabilidade de encontrar, na maternidade do HGCS
\footnote{Hospital Geral de Caxias do Sul, Hospital de Ensino da
  Universidade de Caxias do Sul, RS}, uma parturiente que tenha
exatamente esta altura?\\
Para responder a essa pergunta, será usada a função \texttt{pnorm()}
(veja Seção~\ref{sec-dnp}) que utiliza o escore \emph{Z}, a média e o
desvio padrão para encontrar essa proporção que, multiplicada por 100,
fornece a percentagem.

\begin{Shaded}
\begin{Highlighting}[]
\NormalTok{ p }\OtherTok{\textless{}{-}} \FunctionTok{pnorm}\NormalTok{ (Z, }\FunctionTok{mean}\NormalTok{(dados}\SpecialCharTok{$}\NormalTok{altura),}\FunctionTok{sd}\NormalTok{(dados }\SpecialCharTok{$}\NormalTok{altura))}
\NormalTok{ p}
\end{Highlighting}
\end{Shaded}

\begin{verbatim}
[1] 7.387473e-127
\end{verbatim}

O \emph{R} por padrão retorna números grandes como notação científica. O
resultado dessa operação é um número tão grande que para escrevê-lo sem
este tipo de notação, seriam necessários 127 dígitos decimais. O
resultado não caberia em apenas uma linha. Ficaria assim, suprimindo a
notação científica \footnote{Para remover a notação científica, usar a
  função options (scipen = 999) e, para desfazer essa ação, trocar o 999
  por 0.}:

\begin{Shaded}
\begin{Highlighting}[]
 \FunctionTok{options}\NormalTok{(}\AttributeTok{scipen =}\DecValTok{999}\NormalTok{)}
\NormalTok{ p}
\end{Highlighting}
\end{Shaded}

\begin{verbatim}
[1] 0.0000000000000000000000000000000000000000000000000000000000000000000000000000000000000000000000000000000000000000000000000000007387473
\end{verbatim}

\begin{Shaded}
\begin{Highlighting}[]
 \FunctionTok{options}\NormalTok{(}\AttributeTok{scipen =} \DecValTok{0}\NormalTok{)}
\end{Highlighting}
\end{Shaded}

Ou seja, um número tão próximo de zero que poderia muito bem ser zero!

\section{Propriedades das
probabilidades}\label{propriedades-das-probabilidades}

As seguintes propriedades simples decorrem da definição de
probabilidade.

Sendo \texttt{E} um evento aleatório, a \(P[E]\) está entre 0 e 1, ou
seja \(0\le P[E]\le 1\). Quando o evento certamente não ocorre, a
probabilidade é 0, quando sempre ocorre a probabilidade é 1. Quando a
probabilidade for igual a 0,50 tem-se máxima incerteza.

\begin{enumerate}
\def\labelenumi{\arabic{enumi}.}
\tightlist
\item
  \emph{Regra de adição (regra do ``ou'')}
\end{enumerate}

Dois eventos A e B são mutuamente exclusivos, ou seja, quando A
acontece, B não pode acontecer. Então, a probabilidade de que um ou
outro aconteça é a soma de suas probabilidades. Por exemplo, um dado
lançado pode mostrar um ou dois, mas não ambos. A probabilidade de
mostrar um ou dois é igual a \(1/6 + 1/6 = 1/3\).

\[
P[A ou B]=P[A]+P[B]
\]

Se A e B não são mutuamente exclusivos, ou seja, quando A acontece pode
também ocorrer B. Por exemplo, o nascimento de uma menina pode ser
concomitante com o fato de ser branca.

\[
P[A ou B]=P[A]+P[B]-P[A \space e \space B]
\]

\begin{enumerate}
\def\labelenumi{\arabic{enumi}.}
\setcounter{enumi}{1}
\tightlist
\item
  \emph{Regra de multiplicação (regra do ``e'')}
\end{enumerate}

Suponha que dois eventos (A e B) sejam independentes, ou seja, saber que
um aconteceu não nos diz nada sobre se o outro aconteceu. Então, a
probabilidade de que ambos aconteçam é o produto de suas probabilidades.
Por exemplo, suponha que jogamos duas moedas. Uma moeda não influencia a
outra, portanto os resultados dos dois lançamentos são independentes e a
probabilidade de ocorrerem duas caras é 050 × 0,50 = 0,25.

\[
P[A \quad e\quad B]=P[A]×P[B]
\]

Se os eventos são dependentes, a probabilidade que ambos aconteçam é
igual a:

\[
P[A \quad e \quad B]=P[A]×P[B \rvert A]
\] Com essas propriedades simples e outras mais complexas, é possível
construir algumas ferramentas matemáticas extremamente poderosas, mas
isso não faz parte do objetivo deste livro e não se entrará em detalhes.

\section{Distribuição de Probabilidades}\label{sec-distprob}

Um conjunto de eventos que são mutuamente excludentes e que inclui todos
os eventos que podem acontecer, é chamado de exaustivo. A soma de suas
probabilidades é 1. O conjunto dessas probabilidades constitui uma
\emph{distribuição de probabilidade}.

Existem diversos modelos probabilísticos que procuram descrever vários
tipos de variáveis aleatórias discretas ou contínuas. Estas
distribuições também são chamadas de \emph{modelos probabilísticos
estocásticos} que são definidas por duas funções matemáticas: a
\emph{função de probabilidade} (fp) para variáveis discretas, que
atribui a cada valor a sua probabilidade de ocorrência (\texttt{P(X=x}))
e \emph{função densidade de probabilidade} (\texttt{fdp}) para variáveis
contínuas.

A função de probabilidade é a função que atribui probabilidades a cada
um dos possíveis valores da variável aleatória discreta, usando, em
geral, as frequências relativas, apresentadas em uma tabela de
frequência. O \emph{modelo de Bernoulli} ou \emph{Binomial} e o
\emph{modelo de Poisson} são exemplos de modelo probabilístico de
variáveis discretas.

A função densidade de probabilidade é a função que atribui probabilidade
a qualquer intervalo de número reais, ou seja, um conjunto de valores
não enumerável (infinito). Não é possível atribuir probabilidades para
um determinado valor, é possível apenas para um intervalo. Por exemplo,
o peso dos recém-nascidos. Para atribuir probabilidade a intervalos de
valores é utilizada uma função e as probabilidades são representadas por
áreas. Existem diversos modelos contínuos de probabilidade, mas o mais
importante deles, é o \emph{modelo normal}, também conhecido como
\emph{modelo gaussiano}.

\section{Distribuição Normal}\label{sec-normal}

O \emph{modelo probabilístico normal} ou \emph{gaussiano} é extremamente
importante em estatística, pois serve como um fundamento para técnicas
de inferência. Variáveis como os pesos dos recém-nascidos a termo, as
alturas das mulheres adultas, a renda familiar em reais e muitas outras
variáveis, na natureza, se ajustam ao modelo da distribuição normal.

O modelo de distribuição normal sempre descreve uma curva simétrica,
unimodal e em forma de sino (\textbf{?@fig-curvanormal}).

\begin{figure}[H]

{\centering \includegraphics[width=0.7\linewidth,height=0.7\textheight]{09-probabilidades_files/figure-pdf/curvanormal-1.pdf}

}

\caption{Curva Normal}

\end{figure}%

Uma distribuição normal é descrita por meio de dois parâmetros: a média
da distribuição \(\mu\) e o desvio padrão da distribuição \(\sigma\). Em
função dessa informação, observe como a distribuição normal funciona se
esses parâmetros forem alterados.

Como é fácil prever, alterar a média desloca a curva de sino para a
esquerda ou para a direita, enquanto a alteração do desvio padrão
estende ou achata a curva, ou seja, muda a dispersão da distribuição.

A \textbf{?@fig-threecurves}, mostra a distribuição normal com média 0 e
desvio padrão 1, na curva à direita, a distribuição normal com média 1.5
e desvio padrão 1. Sobrepondo-se à curva da esquerda observa-se uma
curva mais achatada (verde) que tem média 0 e desvio padrão 1.5.
Observa-se, como mencionado, que modificando os parâmetros da curva,
altera-se a posição ou o formato da mesma.

\begin{figure}[H]

{\centering \includegraphics[width=0.7\linewidth,height=0.7\textheight]{09-probabilidades_files/figure-pdf/threecurves-1.pdf}

}

\caption{Curvas normais com modificação dos parâmetros}

\end{figure}%

\subsection{Características da distribuição
normal}\label{caracteruxedsticas-da-distribuiuxe7uxe3o-normal}

A curva normal apresenta as seguintes características:

\begin{itemize}
\item
  A média e o desvio padrão descrevem exatamente uma distribuição
  normal, eles são chamados de parâmetros da distribuição. Se uma
  distribuição normal tem média \(\mu\) e desvio padrão \(\sigma\),
  pode-se escrever a distribuição como \(N (\mu,\sigma)\). As três
  distribuições dos gráficos da \textbf{?@fig-threecurves} podem ser
  escritas como:

  \begin{itemize}
  \tightlist
  \item
    Curva azul \(\to\) \(N(\mu = 0,\sigma = 1)\)
  \item
    Curva verde \(\to\) \(N(\mu = 0,\sigma = 1.5)\)
  \item
    Curva vermelha \(\to\) \(N(\mu = 1.5,\sigma = 1)\)
  \end{itemize}
\item
  Na distribuição normal, a média, a mediana e a moda coincidem.
\item
  A curva normal é simétrica em torno da média (\(\mu\)).
\item
  As extremidades da curva, em ambos os lados da média, se estendem cada
  vez mais próximas do eixo \emph{x} (abscissa) sem jamais tocá-lo. É
  assintótica.
\item
  Os pontos de inflexão da curva são \(\mu - \sigma\) e
  \(\mu + \sigma\).
\item
  A área total sob a curva é 1 ou 100\%.
\end{itemize}

\subsection{Distribuição normal padronizada}\label{sec-dnp}

Cada variável aleatória contínua tem a sua média e seu desvio padrão e,
portanto, a sua curva normal correspondente.

Para facilitar a comparação entre variáveis, foi criado o conceito de
\textbf{curva normal padronizada}, que é uma curva normal com média 0 e
desvio padrão 1. A distribuição normal padrão também pode ser chamada de
\emph{distribuição normal centrada} ou \emph{reduzida}.

Para calcular probabilidades associadas a distribuição normal,
costuma-se converter a variável aleatória original \emph{X}, em unidades
reduzidas ou padronizadas, denominadas de \textbf{escore Z} ou
\textbf{escore padrão}. Essa transformação é realizada pela equação que
indica o número de desvios padrão envolvidos no afastamento do valor
\emph{x} em relação à média da população:

\[
Z =\frac{x-\mu}{\sigma}
\] onde:

\begin{itemize}
\tightlist
\item
  \emph{Z} \(\to\) escore Z\\
\item
  \emph{x} \(\to\) valor qualquer da variável aleatória \emph{X}\\
\item
  \(\mu\) \(\to\) média da variável \emph{X}\\
\item
  \(\sigma\) \(\to\) desvio padrão da variável \emph{X}
\end{itemize}

Qualquer distribuição de uma variável aleatória normal pode ser
padronizada, usando o escore \emph{Z}. Isto permite que se calcule a
probabilidade de se encontrar determinados intervalos de valores (90).

Como exemplo, se retornará à \texttt{altura} das mulheres. É,
praticamente, impossível saber o valor da média populacional, por isso.
costuma-se usar a média aritmética como um estimador da média
populacional. Dessa forma, a variável \texttt{dados\$altura} poderá
utilizada com estimativa da média populacional. Em primeiro lugar, se
construirá um \texttt{tibble} de nome \texttt{resumo}:

\begin{Shaded}
\begin{Highlighting}[]
\NormalTok{ resumo }\OtherTok{\textless{}{-}}\NormalTok{ dados }\SpecialCharTok{\%\textgreater{}\%} 
\NormalTok{   dplyr}\SpecialCharTok{::}\FunctionTok{summarise}\NormalTok{(}\AttributeTok{n =} \FunctionTok{n}\NormalTok{(),}
                    \AttributeTok{media =} \FunctionTok{mean}\NormalTok{(altura, }\AttributeTok{na.rm =} \ConstantTok{TRUE}\NormalTok{),}
                    \AttributeTok{dp =} \FunctionTok{sd}\NormalTok{(altura, }\AttributeTok{na.rm =} \ConstantTok{TRUE}\NormalTok{),}
                    \AttributeTok{min =} \FunctionTok{min}\NormalTok{(altura, }\AttributeTok{na.rm =} \ConstantTok{TRUE}\NormalTok{),}
                    \AttributeTok{max =} \FunctionTok{max}\NormalTok{(altura, }\AttributeTok{na.rm =} \ConstantTok{TRUE}\NormalTok{))}
\NormalTok{ resumo}
\end{Highlighting}
\end{Shaded}

\begin{verbatim}
# A tibble: 1 x 5
      n media     dp   min   max
  <int> <dbl>  <dbl> <dbl> <dbl>
1  1368  1.60 0.0655   1.4  1.85
\end{verbatim}

Assim, pode-se verificar quantos desvios padrão uma mulher, pertencente
a essa amostra, com 1,725m está afastada da média:

\begin{Shaded}
\begin{Highlighting}[]
\NormalTok{Z }\OtherTok{\textless{}{-}}\NormalTok{ (}\FloatTok{1.725} \SpecialCharTok{{-}}\NormalTok{ resumo}\SpecialCharTok{$}\NormalTok{media)}\SpecialCharTok{/}\NormalTok{resumo}\SpecialCharTok{$}\NormalTok{dp}
\FunctionTok{round}\NormalTok{(Z, }\DecValTok{2}\NormalTok{)}
\end{Highlighting}
\end{Shaded}

\begin{verbatim}
[1] 1.94
\end{verbatim}

Esta mulher está distante praticamente 2 desvios padrão acima da média
da sua população. Portanto, ela é considerada alta. Por que?

Para responder a essa pergunta, há necessidade de calcular a
probabilidade de encontrar uma mulher com esta altura, nesta população.
Primeiro, calcula-se a probabilidade de encontrar uma mulher com esta
altura, nesta amostra. No \emph{R}, existem as funções \texttt{dnorm()},
\texttt{pnorm()} e \texttt{qnorm()}, que permitem calcular a densidade
de probabilidade, a distribuição cumulativa e a função quantílica da
distribuição normal para um conjunto de valores. Além dessas, há a
função \texttt{rnorm()} que permite obter observações aleatórias que
seguem uma distribuição normal (91).

\subsubsection{Função pnorm()}\label{funuxe7uxe3o-pnorm}

A função \texttt{pnorm()} fornece a \emph{Função de Distribuição
Cumulativa} (CDF) da distribuição Normal, que é a probabilidade de que a
variável \emph{X} contenha um valor menor ou igual a \emph{x}.

\ul{Argumentos}:

\begin{itemize}
\tightlist
\item
  \textbf{q} \(\rightarrow\) vetor de quantis\\
\item
  \textbf{mean} \(\rightarrow\) média\\
\item
  \textbf{sd} \(\rightarrow\) desvio padrão\\
\item
  \textbf{lower.tail} \(\rightarrow\) Se \texttt{TRUE}, as
  probabilidades são \(P(X\le x)\), caso contrário \(P(X > x)\)
\end{itemize}

Se for usado \(mean = 0\) e \(sd = 1\), o valor de q = Z, caso
contrário, toma-se os valores da média, o desvio padrão da população e o
valor de \emph{x}. Com esta função, é possível responder a pergunta
feita anteriormente em relação a probabilidade de encontrar uma mulher
com mais de 1,725m, equivalente a 1.94 desvios padrão acima da média, em
uma população com média = 1.5979678 e desvio padrão = 0.0654787.

\begin{Shaded}
\begin{Highlighting}[]
\NormalTok{p }\OtherTok{\textless{}{-}} \FunctionTok{pnorm}\NormalTok{(Z, }\AttributeTok{mean =} \DecValTok{0}\NormalTok{, }\AttributeTok{sd =} \DecValTok{1}\NormalTok{, }\AttributeTok{lower.tail =} \ConstantTok{FALSE}\NormalTok{)}
\NormalTok{p}
\end{Highlighting}
\end{Shaded}

\begin{verbatim}
[1] 0.02618654
\end{verbatim}

Ou, usando os valores:

\begin{Shaded}
\begin{Highlighting}[]
\FunctionTok{pnorm}\NormalTok{(}\FloatTok{1.725}\NormalTok{, }\AttributeTok{mean =}\NormalTok{ resumo}\SpecialCharTok{$}\NormalTok{media, }\AttributeTok{sd =}\NormalTok{ resumo}\SpecialCharTok{$}\NormalTok{dp, }\AttributeTok{lower.tail =} \ConstantTok{FALSE}\NormalTok{)}
\end{Highlighting}
\end{Shaded}

\begin{verbatim}
[1] 0.02618654
\end{verbatim}

Observa-se que, nesta amostra, apenas 2.6\% das mulheres têm acima de
1,725m, razão de ser considerada uma mulher alta. Ou seja, é pouco
provável encontrar mulheres acima dessa altura, nesta amostra.

Para representar graficamente essa pequena probabilidade, será
construída uma curva com essa pequena área sombreada, colorida em azul
claro. Para isso, uma função própria, denominada
\texttt{normal\_intervalo\_sombreado()} plotará a figura. Os argumentos
dessa função são: a = limite inferior do intervalo; b = limite superior
do intervalo; mean = média (padrão = 0); sd = desvio padrão (padrão =
1). Ela pode ser obtida
\href{https://github.com/petronioliveira/Arquivos/blob/main/normal_intervalo_sombreado.R}{aqui}
para ser baixada em seu diretório de trabalho para uso posterior
\footnote{Veja Seção~\ref{sec-funcpropria} como se constrói uma função.
  A função apresentada aqui é mais complexa..}. Para carregar a função
para uso usa-se a função nativa source() que serve para executar o
código contido em um arquivo com a extensão .R.

A Figura~\ref{fig-prob1725} representa com clareza esta pequena
probabilidade.

\begin{figure}

\centering{

\includegraphics[width=0.7\linewidth,height=0.7\textheight]{09-probabilidades_files/figure-pdf/fig-prob1725-1.pdf}

}

\caption{\label{fig-prob1725}Probabilidade de encontrar mulheres com
mais de 1,725m}

\end{figure}%

\subsubsection{Função qnorm()}\label{funuxe7uxe3o-qnorm}

A função \texttt{qnorm()} permite encontrar o quantil \emph{q} para
qualquer probabilidade \emph{p}. Portanto, a função \texttt{qnorm} é o
inverso da função \texttt{pnorm()}.

\ul{Argumentos}:

\begin{itemize}
\tightlist
\item
  \emph{p} \(\to\) vetor de probabilidades\\
\item
  \emph{mean} \(\to\) média\\
\item
  \emph{sd} \(\to\) desvio padrão\\
\item
  \emph{lower.tail} \(\to\) Se \texttt{TRUE}, as probabilidades são
  (\(P \le x\)), caso contrário \(P(X > x)\)
\end{itemize}

No exemplo anterior, a probabilidade de se encontrar mulheres, na
maternidade, com mais de 1,725m foi de 2.6\%. Poderia ser calculado com
a função \texttt{qnorm()} qual o escore \emph{Z} correspondente:

\begin{Shaded}
\begin{Highlighting}[]
\FunctionTok{qnorm}\NormalTok{(p, }\AttributeTok{mean =} \DecValTok{0}\NormalTok{, }\AttributeTok{sd =} \DecValTok{1}\NormalTok{, }\AttributeTok{lower.tail =} \ConstantTok{FALSE}\NormalTok{)}
\end{Highlighting}
\end{Shaded}

\begin{verbatim}
[1] 1.940054
\end{verbatim}

\subsubsection{Função dnorm()}\label{funuxe7uxe3o-dnorm}

Essa função retorna o valor da \emph{função de densidade de
probabilidade} (pdf) da distribuição normal dada uma certa variável
aleatória X, uma média populacional \(\mu\) e o desvio padrão
populacional \(\sigma\).

\ul{Argumentos}:

\begin{itemize}
\tightlist
\item
  \textbf{x} \(\to\) vetor de quantis\\
\item
  \textbf{mean} \(\to\) média\\
\item
  \textbf{sd} \(\to\) desvio padrão
\end{itemize}

Embora \emph{x} represente a variável independente da \emph{pdf} para a
distribuição normal, também é útil pensar em \emph{x} como um escore
\emph{Z}. Por exemplo, a densidade de probabilidade quando \emph{x} = 0
é igual:

\begin{Shaded}
\begin{Highlighting}[]
\FunctionTok{dnorm}\NormalTok{(}\AttributeTok{x =} \DecValTok{0}\NormalTok{, }\AttributeTok{mean =} \DecValTok{0}\NormalTok{, }\AttributeTok{sd =} \DecValTok{1}\NormalTok{)}
\end{Highlighting}
\end{Shaded}

\begin{verbatim}
[1] 0.3989423
\end{verbatim}

Para se construir uma curva de densidade de probabilidades normal (
\(X \sim N(μ=0,σ=1)\)), basta aplicar a função \texttt{dnorm()} a uma
sequência contínua de escores \emph{Z}. O vetor de escores Z é obtido
com a função \texttt{seq()}, como mostrado a seguir:

\begin{Shaded}
\begin{Highlighting}[]
\NormalTok{escores\_z }\OtherTok{\textless{}{-}} \FunctionTok{seq}\NormalTok{(}\SpecialCharTok{{-}}\DecValTok{3}\NormalTok{,}\DecValTok{3}\NormalTok{, }\AttributeTok{by =} \FloatTok{0.05}\NormalTok{)}
\NormalTok{escores\_z}
\end{Highlighting}
\end{Shaded}

\begin{verbatim}
  [1] -3.00 -2.95 -2.90 -2.85 -2.80 -2.75 -2.70 -2.65 -2.60 -2.55 -2.50 -2.45
 [13] -2.40 -2.35 -2.30 -2.25 -2.20 -2.15 -2.10 -2.05 -2.00 -1.95 -1.90 -1.85
 [25] -1.80 -1.75 -1.70 -1.65 -1.60 -1.55 -1.50 -1.45 -1.40 -1.35 -1.30 -1.25
 [37] -1.20 -1.15 -1.10 -1.05 -1.00 -0.95 -0.90 -0.85 -0.80 -0.75 -0.70 -0.65
 [49] -0.60 -0.55 -0.50 -0.45 -0.40 -0.35 -0.30 -0.25 -0.20 -0.15 -0.10 -0.05
 [61]  0.00  0.05  0.10  0.15  0.20  0.25  0.30  0.35  0.40  0.45  0.50  0.55
 [73]  0.60  0.65  0.70  0.75  0.80  0.85  0.90  0.95  1.00  1.05  1.10  1.15
 [85]  1.20  1.25  1.30  1.35  1.40  1.45  1.50  1.55  1.60  1.65  1.70  1.75
 [97]  1.80  1.85  1.90  1.95  2.00  2.05  2.10  2.15  2.20  2.25  2.30  2.35
[109]  2.40  2.45  2.50  2.55  2.60  2.65  2.70  2.75  2.80  2.85  2.90  2.95
[121]  3.00
\end{verbatim}

Agora, usando a função \texttt{dnorm()}, será construído conjunto de
valores de densidade de probabilidade correspondentes aos escores
\emph{Z} obtidos anteriormente:

\begin{Shaded}
\begin{Highlighting}[]
\NormalTok{valores\_d }\OtherTok{\textless{}{-}} \FunctionTok{dnorm}\NormalTok{(escores\_z, }\AttributeTok{mean =} \DecValTok{0}\NormalTok{, }\AttributeTok{sd =} \DecValTok{1}\NormalTok{)}
\end{Highlighting}
\end{Shaded}

Estes valores serão plotados para construir a curva normal
(Figura~\ref{fig-pdf}):

\begin{figure}

\centering{

\includegraphics[width=0.7\linewidth,height=0.7\textheight]{09-probabilidades_files/figure-pdf/fig-pdf-1.pdf}

}

\caption{\label{fig-pdf}Função densidade de probabilidade}

\end{figure}%

Os argumentos básicos a serem informados da função \texttt{axis()} são:
\texttt{side=}, \texttt{at=} e \texttt{labels=}. Esses argumentos
determinam qual eixo será preenchido, qual a posição dos valores no eixo
e a sequência de valores a ser preenchida, respectivamente. O argumento
\texttt{side=} recebe valores que vão de 1 a 4: 1 = eixo inferior, 2 =
eixo lateral esquerdo, 3 = eixo superior, 4= eixo lateral direito. Ou
seja, partindo do eixo inferior (eixo \emph{x}), os valores aumentam até
4 seguindo o sentindo horário para os quatros lados do gráfico. No
exemplo, foi modificado o eixo \emph{x}, logo \texttt{side\ =\ 1}. O
argumento \texttt{at\ =} estabelece os pontos (densidades de
probabilidade) do eixo \emph{x} que receberão os rótulos, especificados
no argumento \texttt{label\ =}.

Como se pode ver, \texttt{dnorm()} fornece a ``altura'' do \emph{pdf} da
distribuição normal em qualquer escore \emph{Z} que se forneça como
argumento.

\subsubsection{Função rnorm()}\label{sec-rnorm}

A função \texttt{rnorm()} gera \emph{n} números aleatórios com
distribuição normal com média \(\mu\) e desvio padrão \(\sigma\).

\ul{Argumentos}:

\begin{itemize}
\tightlist
\item
  \textbf{n} \(\to\) número de observações a serem geradas\\
\item
  \textbf{mean} \(\to\) média\\
\item
  \textbf{sd} \(\to\) desvio padrão
\end{itemize}

Com esta função é possível, por exemplo, gerar 10 observações de uma
distribuição normal:

\begin{Shaded}
\begin{Highlighting}[]
\FunctionTok{rnorm}\NormalTok{(}\DecValTok{10}\NormalTok{)}
\end{Highlighting}
\end{Shaded}

\begin{verbatim}
 [1] -0.24190304  1.42886055  0.18943442  1.06444281  0.05194556  1.79574794
 [7]  0.07196058 -1.13470089 -1.36724289 -0.90399659
\end{verbatim}

No entanto, deve-se notar que, se uma ``semente'' (\texttt{seed}) não
for especificada, a saída não será reproduzível, ou seja, cada vez que o
comando for executado, retornará um novo conjunto de observações:

\begin{Shaded}
\begin{Highlighting}[]
\FunctionTok{rnorm}\NormalTok{(}\DecValTok{10}\NormalTok{)}
\end{Highlighting}
\end{Shaded}

\begin{verbatim}
 [1] -1.7690462 -1.0313077 -0.1759992 -0.4401495 -0.5794208  0.0524025
 [7]  1.4753766 -0.4119497  2.9809746 -1.5410742
\end{verbatim}

Cada vez que este comando for reproduzido, retornará uma nova série de
10 números diferentes do anterior. Para tornar o código reproduzível,
retornando o mesmo conjunto de valores, deve-se usar uma ``semente''
(\emph{seed}), usando a função \emph{set.seed()}, cujo argumento é um
número que identificará a série gerada, no exemplo, pela função
\texttt{rnorm()}. O valor do número (``semente'') não é importante, é
apenas um identificador. Para ilustrar, será construído dois conjuntos
de 10 números que serão recebidos pelos objetos \emph{x} e \emph{y}.
Para gerar o conjunto de números \emph{x}, será usado o número
\texttt{123} como ``semente''. A ``semente'' funciona como uma espécie
de marca. Para o \emph{y} não será usado a função \texttt{set.seed()}:

\begin{Shaded}
\begin{Highlighting}[]
\NormalTok{n }\OtherTok{\textless{}{-}} \DecValTok{10}
\FunctionTok{set.seed}\NormalTok{ (}\DecValTok{123}\NormalTok{)}
\NormalTok{x }\OtherTok{\textless{}{-}} \FunctionTok{rnorm}\NormalTok{ (n)}
\NormalTok{x}
\end{Highlighting}
\end{Shaded}

\begin{verbatim}
 [1] -0.56047565 -0.23017749  1.55870831  0.07050839  0.12928774  1.71506499
 [7]  0.46091621 -1.26506123 -0.68685285 -0.44566197
\end{verbatim}

\begin{Shaded}
\begin{Highlighting}[]
\NormalTok{y }\OtherTok{\textless{}{-}} \FunctionTok{rnorm}\NormalTok{(n)}
\NormalTok{y}
\end{Highlighting}
\end{Shaded}

\begin{verbatim}
 [1]  1.2240818  0.3598138  0.4007715  0.1106827 -0.5558411  1.7869131
 [7]  0.4978505 -1.9666172  0.7013559 -0.4727914
\end{verbatim}

Comparando os conjuntos com a função \texttt{identical()} do R base,
observa-se que os conjuntos são diferentes:

\begin{Shaded}
\begin{Highlighting}[]
\FunctionTok{identical}\NormalTok{(x, y)}
\end{Highlighting}
\end{Shaded}

\begin{verbatim}
[1] FALSE
\end{verbatim}

Agora, repetindo os mesmos comandos, mas usando antes a mesma
``semente'', observa-se que os conjuntos são idênticos.

\begin{Shaded}
\begin{Highlighting}[]
\FunctionTok{set.seed}\NormalTok{ (}\DecValTok{123}\NormalTok{)}
\NormalTok{x }\OtherTok{\textless{}{-}} \FunctionTok{rnorm}\NormalTok{ (n)}
\NormalTok{x}
\end{Highlighting}
\end{Shaded}

\begin{verbatim}
 [1] -0.56047565 -0.23017749  1.55870831  0.07050839  0.12928774  1.71506499
 [7]  0.46091621 -1.26506123 -0.68685285 -0.44566197
\end{verbatim}

\begin{Shaded}
\begin{Highlighting}[]
\FunctionTok{set.seed}\NormalTok{ (}\DecValTok{123}\NormalTok{)}
\NormalTok{y }\OtherTok{\textless{}{-}} \FunctionTok{rnorm}\NormalTok{(n)}
\NormalTok{y}
\end{Highlighting}
\end{Shaded}

\begin{verbatim}
 [1] -0.56047565 -0.23017749  1.55870831  0.07050839  0.12928774  1.71506499
 [7]  0.46091621 -1.26506123 -0.68685285 -0.44566197
\end{verbatim}

\begin{Shaded}
\begin{Highlighting}[]
\FunctionTok{identical}\NormalTok{(x, y)}
\end{Highlighting}
\end{Shaded}

\begin{verbatim}
[1] TRUE
\end{verbatim}

\ul{Outros usos da função rnorm()}:

A função \texttt{rnorm()}será usada para gerar três vetores diferentes
de números aleatórios de uma distribuição normal.

\begin{Shaded}
\begin{Highlighting}[]
\FunctionTok{set.seed}\NormalTok{(}\DecValTok{1234}\NormalTok{)}
\NormalTok{n10 }\OtherTok{\textless{}{-}} \FunctionTok{rnorm}\NormalTok{(}\DecValTok{10}\NormalTok{, }\AttributeTok{mean =} \DecValTok{0}\NormalTok{, }\AttributeTok{sd =} \DecValTok{1}\NormalTok{)}
\NormalTok{n100 }\OtherTok{\textless{}{-}} \FunctionTok{rnorm}\NormalTok{(}\DecValTok{100}\NormalTok{, }\AttributeTok{mean =} \DecValTok{0}\NormalTok{, }\AttributeTok{sd =} \DecValTok{1}\NormalTok{)}
\NormalTok{n10000 }\OtherTok{\textless{}{-}}  \FunctionTok{rnorm}\NormalTok{(}\DecValTok{10000}\NormalTok{, }\AttributeTok{mean =} \DecValTok{0}\NormalTok{, }\AttributeTok{sd =} \DecValTok{1}\NormalTok{)}
\end{Highlighting}
\end{Shaded}

Em sequência, serão construídos histogramas (Figura~\ref{fig-pdf1}),
onde se pode observar que, aumentando o número de observações, tem-se
gráficos que irão progressivamente se aproximando da verdadeira função
de densidade normal.

\begin{figure}

\centering{

\includegraphics[width=0.8\linewidth,height=0.8\textheight]{09-probabilidades_files/figure-pdf/fig-pdf1-1.pdf}

}

\caption{\label{fig-pdf1}Histogramas construídos com amostras geradas
pela função rnorm}

\end{figure}%

Observe também que à medida que \emph{n} aumenta, a distribuição dos
dados caracteriza-se como uma distribuição normal. Na
Seção~\ref{sec-dam}, este assunto voltará à cena.

\subsection{Regra Empírica
68-95-99.7}\label{regra-empuxedrica-68-95-99.7}

A regra empírica diz que, se uma população de um conjunto de dados tem
uma distribuição normal com média 0 e desvio padrão 1
(\(X\sim N(\mu=0,\sigma=1)\)) pode-se afirmar que aproximadamente, 68\%,
95\% e 99,7\% dos valores encontram-se, respectivamente, dentro de
\(\pm\) 1, 2 e 3 desvio padrão acima e abaixo média.

Essa regra pode ser usada para descrever uma população e ajudar a
decidir se uma amostra de dados veio de uma distribuição normal. Se uma
amostra é grande o suficiente e a observação do histograma tem um
formato parecido com um sino, é possível verificar se os dados seguem as
especificações 68-95-99,7\%. Se sim, é razoável concluir que os dados
vieram de uma distribuição normal.

\ul{Exemplo}:

Com a função \texttt{rnorm()}, será gerado um vetor de 1000 números e um
histograma com curva normal sobreposta:

\begin{figure}

\centering{

\includegraphics[width=0.7\linewidth,height=0.7\textheight]{09-probabilidades_files/figure-pdf/fig-histdp-1.pdf}

}

\caption{\label{fig-histdp}Histograma com curva normal sobreposta}

\end{figure}%

No histograma (Figura~\ref{fig-histdp}), a probabilidade entre os
escores Z - 1 e + 1 (entre as duas linhas tracejadas A e B) é igual a
aproximadamente 68\%. Pode-se calcular isto facilmente, usando a função
\texttt{pnorm()}:

\ul{Probabilidade abaixo de z = 1, abaixo do ponto B}:

\begin{Shaded}
\begin{Highlighting}[]
\NormalTok{B }\OtherTok{\textless{}{-}} \FunctionTok{pnorm}\NormalTok{ (}\DecValTok{1}\NormalTok{, }\DecValTok{0}\NormalTok{, }\DecValTok{1}\NormalTok{)}
\NormalTok{B }\OtherTok{\textless{}{-}} \FunctionTok{round}\NormalTok{(B, }\DecValTok{3}\NormalTok{)}\SpecialCharTok{*}\DecValTok{100}
\NormalTok{B}
\end{Highlighting}
\end{Shaded}

\begin{verbatim}
[1] 84.1
\end{verbatim}

\ul{Probabilidade abaixo de z = -1, abaixo do ponto A}:

\begin{Shaded}
\begin{Highlighting}[]
\NormalTok{ A }\OtherTok{\textless{}{-}} \FunctionTok{pnorm}\NormalTok{ (}\SpecialCharTok{{-}}\DecValTok{1}\NormalTok{, }\DecValTok{0}\NormalTok{, }\DecValTok{1}\NormalTok{)}
\NormalTok{ A }\OtherTok{\textless{}{-}} \FunctionTok{round}\NormalTok{(A, }\DecValTok{3}\NormalTok{)}\SpecialCharTok{*}\DecValTok{100}
\NormalTok{ A}
\end{Highlighting}
\end{Shaded}

\begin{verbatim}
[1] 15.9
\end{verbatim}

Logo , a área abaixo da curva entre A e B é igual a:

\begin{Shaded}
\begin{Highlighting}[]
\NormalTok{ prob }\OtherTok{\textless{}{-}}\NormalTok{ B }\SpecialCharTok{{-}}\NormalTok{ A}
\NormalTok{ prob}
\end{Highlighting}
\end{Shaded}

\begin{verbatim}
[1] 68.2
\end{verbatim}

\subsection{Calculando probabilidades em uma distribuição
normal}\label{calculando-probabilidades-em-uma-distribuiuxe7uxe3o-normal}

\begin{tcolorbox}[enhanced jigsaw, bottomrule=.15mm, opacitybacktitle=0.6, colframe=quarto-callout-tip-color-frame, arc=.35mm, coltitle=black, toptitle=1mm, colback=white, colbacktitle=quarto-callout-tip-color!10!white, breakable, bottomtitle=1mm, rightrule=.15mm, titlerule=0mm, toprule=.15mm, opacityback=0, leftrule=.75mm, left=2mm, title=\textcolor{quarto-callout-tip-color}{\faLightbulb}\hspace{0.5em}{Exemplo 1}]

A variável \texttt{dados\$altura}, mostrada na
Seção~\ref{sec-aplicafreq}, tem uma média (1.598, mediana de (1.6) e um
coeficiente de variação (CV) muito pequeno de 4.1\%. Isso caracteriza
uma variável com uma distribuição praticamente simétrica. Usando esses
dados (\(X\sim N(\mu=1,598,\sigma=0,065)\)), pode-se calcular
probabilidades, dadas pela área sob a curva.

Baseado nessas informações, qual a probabilidade de se encontrar
mulheres com altura entre 1,47 e 1,73 m?

\end{tcolorbox}

\textbf{Resposta}:

\begin{Shaded}
\begin{Highlighting}[]
\CommentTok{\# Dados}
\NormalTok{ mu }\OtherTok{\textless{}{-}} \FloatTok{1.598}
\NormalTok{ sigma }\OtherTok{\textless{}{-}} \FloatTok{0.065}
\NormalTok{ linf }\OtherTok{\textless{}{-}} \FloatTok{1.47}
\NormalTok{ lsup }\OtherTok{\textless{}{-}} \FloatTok{1.73}
 
 \CommentTok{\# Solução}
\NormalTok{ z1 }\OtherTok{\textless{}{-}}\NormalTok{  (linf }\SpecialCharTok{{-}}\NormalTok{ mu)}\SpecialCharTok{/}\NormalTok{sigma}
\NormalTok{ z2 }\OtherTok{\textless{}{-}}\NormalTok{  (lsup }\SpecialCharTok{{-}}\NormalTok{ mu)}\SpecialCharTok{/}\NormalTok{sigma}

\NormalTok{ p1 }\OtherTok{\textless{}{-}} \FunctionTok{pnorm}\NormalTok{(linf, mu, sigma)}
\NormalTok{ p2 }\OtherTok{\textless{}{-}} \FunctionTok{pnorm}\NormalTok{(lsup, mu, sigma)}
 
\NormalTok{ p2 }\SpecialCharTok{{-}}\NormalTok{ p1}
\end{Highlighting}
\end{Shaded}

\begin{verbatim}
[1] 0.9543975
\end{verbatim}

\begin{figure}

\centering{

\includegraphics[width=0.7\linewidth,height=0.7\textheight]{09-probabilidades_files/figure-pdf/fig-normal954-1.pdf}

}

\caption{\label{fig-normal954}Probabilidade de encontrar mulheres entre
1,47 e 1,725m}

\end{figure}%

A probabilidade de alturas entre 1,47m e 1,73m é igual a 95,4\%, como
mostrado na Figura~\ref{fig-normal954}.

\begin{tcolorbox}[enhanced jigsaw, bottomrule=.15mm, opacitybacktitle=0.6, colframe=quarto-callout-tip-color-frame, arc=.35mm, coltitle=black, toptitle=1mm, colback=white, colbacktitle=quarto-callout-tip-color!10!white, breakable, bottomtitle=1mm, rightrule=.15mm, titlerule=0mm, toprule=.15mm, opacityback=0, leftrule=.75mm, left=2mm, title=\textcolor{quarto-callout-tip-color}{\faLightbulb}\hspace{0.5em}{Exemplo 2}]

Os dados de uma pesquisa mostram informações sobre o tempo de cirurgia
para reconstrução do ligamento cruzado anterior (LCA). A distribuição de
probabilidades se ajusta à distribuição normal com o tempo médio de
cirurgia de 129 minutos com um desvio padrão de 14 minutos.

\begin{enumerate}
\def\labelenumi{\arabic{enumi}.}
\item
  Qual a probabilidade de uma cirurgia de reconstrução do LCA requerer
  um tempo menor do que 100 minutos?
\item
  Se uma cirurgia demorar 160 minutos, o que se conclui em relação a
  essa informação?
\end{enumerate}

\end{tcolorbox}

\textbf{Resposta 1:}

\begin{Shaded}
\begin{Highlighting}[]
\CommentTok{\# Dados}
\NormalTok{ mu }\OtherTok{\textless{}{-}} \DecValTok{129}
\NormalTok{ sigma }\OtherTok{\textless{}{-}} \DecValTok{14}
\NormalTok{ x }\OtherTok{\textless{}{-}} \DecValTok{100}

\CommentTok{\# Solução}
\NormalTok{ z }\OtherTok{\textless{}{-}}\NormalTok{  (x }\SpecialCharTok{{-}}\NormalTok{ mu)}\SpecialCharTok{/}\NormalTok{sigma}

\NormalTok{ p }\OtherTok{\textless{}{-}} \FunctionTok{pnorm}\NormalTok{(x, mu, sigma, }\AttributeTok{lower.tail =} \ConstantTok{TRUE}\NormalTok{)}
\NormalTok{ p}
\end{Highlighting}
\end{Shaded}

\begin{verbatim}
[1] 0.01915938
\end{verbatim}

\begin{figure}

\centering{

\includegraphics[width=0.8\linewidth,height=0.8\textheight]{09-probabilidades_files/figure-pdf/fig-normal192-1.pdf}

}

\caption{\label{fig-normal192}Probabilidade do tempo de cirurgia de LCA
menor ou igual a 100 minutos}

\end{figure}%

De acordo com a distribuição, 1,92\%\% das cirurgias irão demandar
quantidade de tempo menor do que 100 minutos
(Figura~\ref{fig-normal192}).

\textbf{Resposta 2:}

\begin{Shaded}
\begin{Highlighting}[]
\CommentTok{\# Dados}
\NormalTok{ mu }\OtherTok{\textless{}{-}} \DecValTok{129}
\NormalTok{ sigma }\OtherTok{\textless{}{-}} \DecValTok{14}
\NormalTok{ x }\OtherTok{\textless{}{-}} \DecValTok{160}

 \CommentTok{\# Solução}
\NormalTok{ z }\OtherTok{\textless{}{-}}\NormalTok{  (x }\SpecialCharTok{{-}}\NormalTok{ mu)}\SpecialCharTok{/}\NormalTok{sigma}

\NormalTok{ p }\OtherTok{\textless{}{-}} \FunctionTok{pnorm}\NormalTok{(x, mu, sigma, }\AttributeTok{lower.tail =} \ConstantTok{FALSE}\NormalTok{)}
\NormalTok{ p}
\end{Highlighting}
\end{Shaded}

\begin{verbatim}
[1] 0.01340457
\end{verbatim}

\begin{figure}

\centering{

\includegraphics[width=0.8\linewidth,height=0.8\textheight]{09-probabilidades_files/figure-pdf/fig-normal134-1.pdf}

}

\caption{\label{fig-normal134}Probabilidade do tempo de cirurgia de LCA
maior ou igual a 160 minutos}

\end{figure}%

De acordo com a distribuição, 1,34\% das cirurgias irão demandar
quantidade de tempo \(\ge 160\) minutos (Figura~\ref{fig-normal134}). Ou
seja, é uma probabilidade muito pequena!

\begin{tcolorbox}[enhanced jigsaw, bottomrule=.15mm, opacitybacktitle=0.6, colframe=quarto-callout-tip-color-frame, arc=.35mm, coltitle=black, toptitle=1mm, colback=white, colbacktitle=quarto-callout-tip-color!10!white, breakable, bottomtitle=1mm, rightrule=.15mm, titlerule=0mm, toprule=.15mm, opacityback=0, leftrule=.75mm, left=2mm, title=\textcolor{quarto-callout-tip-color}{\faLightbulb}\hspace{0.5em}{Exemplo 3}]

Supondo que em uma determinada ilha hipotética existam duas populações
etnicamente diferentes onde as mulheres têm as seguintes medidas de
altura: população 1 tem μ = 160 cm e σ = 6,6 cm e a população 2 tem μ =
140 cm e σ = 6,6 cm. As alturas de ambas as populações têm distribuição
normal. Essas duas populações têm o mesmo aspecto físico, podendo ser
distinguidas apenas geneticamente.

\begin{enumerate}
\def\labelenumi{\arabic{enumi}.}
\item
  Qual a probabilidade de uma mulher com 150 cm pertencer a população 1?
\item
  Qual a probabilidade de uma mulher com 150 cm pertencer a população 2?
\end{enumerate}

\end{tcolorbox}

\textbf{Resposta 1}:

\begin{Shaded}
\begin{Highlighting}[]
\CommentTok{\# Dados}
\NormalTok{ mu }\OtherTok{\textless{}{-}} \DecValTok{160}
\NormalTok{ sigma }\OtherTok{\textless{}{-}} \FloatTok{6.6}
\NormalTok{ x }\OtherTok{\textless{}{-}} \DecValTok{150}

\CommentTok{\# Solução}
\NormalTok{ z }\OtherTok{\textless{}{-}}\NormalTok{  (x }\SpecialCharTok{{-}}\NormalTok{ mu)}\SpecialCharTok{/}\NormalTok{sigma}

\NormalTok{ p }\OtherTok{\textless{}{-}} \FunctionTok{pnorm}\NormalTok{(x, mu, sigma, }\AttributeTok{lower.tail =} \ConstantTok{TRUE}\NormalTok{)}
\NormalTok{ p}
\end{Highlighting}
\end{Shaded}

\begin{verbatim}
[1] 0.06486702
\end{verbatim}

\begin{figure}

\centering{

\includegraphics[width=0.8\linewidth,height=0.8\textheight]{09-probabilidades_files/figure-pdf/fig-normal649-1.pdf}

}

\caption{\label{fig-normal649}Probabilidade de uma mulher com 150 cmm
pertencer a uma população de média igual a 160 cm}

\end{figure}%

Na população 1, 6.5\% das mulheres tem altura \(\le 1,50\) m
(Figura~\ref{fig-normal649}). Em outras palavras, existe pouca
probabilidade dessa mulher pertencer à população 1.

\textbf{Resposta 2}:

\begin{Shaded}
\begin{Highlighting}[]
\CommentTok{\# Dados}
\NormalTok{ mu }\OtherTok{\textless{}{-}} \DecValTok{140}
\NormalTok{ sigma }\OtherTok{\textless{}{-}} \FloatTok{6.6}
\NormalTok{ x }\OtherTok{\textless{}{-}} \DecValTok{150}

\CommentTok{\# Solução}
\NormalTok{ z }\OtherTok{\textless{}{-}}\NormalTok{  (x }\SpecialCharTok{{-}}\NormalTok{ mu)}\SpecialCharTok{/}\NormalTok{sigma}
 
\NormalTok{ p }\OtherTok{\textless{}{-}} \FunctionTok{pnorm}\NormalTok{(x, mu, sigma, }\AttributeTok{lower.tail =} \ConstantTok{TRUE}\NormalTok{)}
\NormalTok{ p}
\end{Highlighting}
\end{Shaded}

\begin{verbatim}
[1] 0.935133
\end{verbatim}

\begin{figure}

\centering{

\includegraphics[width=0.8\linewidth,height=0.8\textheight]{09-probabilidades_files/figure-pdf/fig-normal935-1.pdf}

}

\caption{\label{fig-normal935}Probabilidade de uma mulher com 150 cmm
pertencer a uma população de média igual a 140 cm}

\end{figure}%

Na população 2, 93,5\% das mulheres tem altura \(\le 1,50\) m
(Figura~\ref{fig-normal935}). Concluindo, ela pode pertencer a qualquer
uma das populações. Pode ser uma mulher alta da população 2 ou uma
``baixinha'' da população 1!

\section{Distribuição Binomial}\label{distribuiuxe7uxe3o-binomial}

A distribuição normal padrão é apenas um dos exemplos de distribuição de
probabilidade. Uma boa parte das situações se ajustam a ela. Entretanto,
diversas situações reais muitas vezes se aproximam de outras
distribuições estocásticas definidas por algumas hipóteses. Daí a
importância de se conhecer e manipular algumas destas distribuições.
Entre elas, a \textbf{distribuição binomial}.

Quando um experimento aleatório resulta em um de dois, mutuamente
exclusivos, desfechos, tais como vivo/morto, positivo/negativo, sim/não,
masculino/feminino é denominado de \emph{Ensaio de Bernoulli}. Recebeu
esta denominação em homenagem ao matemático suíço, Jacob Bernoulli
(1654-1705), considerado fundador do cálculo e da teoria da
probabilidade (92).

A distribuição de frequências que descreve as proporções de um ensaio de
Bernoulli, chama-se \textbf{Distribuição Binomial}. A probabilidade
binomial dá a probabilidade de determinado desfecho ocorrer em
determinado número de ensaios independentes. Uma sequência de ensaios de
Bernoulli forma um \textbf{Processo de Bernoulli}.

A distribuição binomial é importante para variáveis discretas. Existem
poucas condições que precisam ser atendidas para distribuição binomial:

\begin{itemize}
\tightlist
\item
  Cada ensaio resulta em um de dois desfechos, mutuamente exclusivos,
  denominados, arbitrariamente, de sucesso e fracasso;

  \begin{itemize}
  \tightlist
  \item
    A probabilidade de sucesso é fixa, igual a \emph{p}, constante em
    cada ensaio, e a probabilidade de fracasso é igual a \emph{1 -- p};
  \item
    O número de repetições \emph{n} em um ensaio é fixo.
  \end{itemize}
\item
  Os ensaios são independentes
\end{itemize}

A distribuição binomial é na verdade uma família de distribuições, cujos
membros são definidos pelos valores de \emph{n} e \emph{p} (parâmetros
da distribuição binomial).

A probabilidade de sucesso \footnote{\emph{Sucesso}, aqui, não está no
  sentido de vitória, êxito, triunfo, glória e sim com a conotação de
  obter o desfecho esperado. Por exemplo, se uma moeda é lançada e se
  espera obter cara, sucesso significa um resultado igual a cara.}, em
uma distribuição binomial, é dada pela fórmula:

\[
P(X = x)= C \times p^x \times (1 - p)^{n-x}
\]

onde \emph{n} = ensaios, \emph{x} = sucessos, \emph{p} = probabilidade
de um sucesso e \emph{C} representa o número possível de combinações em
um ensaio.

O número de combinações, \emph{C} de \emph{x} sucessos entre \emph{n}
repetições podem ser computado pela fórmula:

\[
C = \frac{n!}{x!(n - x)!}
\]

ou, no R, com a função \texttt{choose\ (n,\ x)}.

O modelo de distribuição binomial trata de encontrar a probabilidade de
sucesso de um evento que tem apenas dois resultados possíveis em uma
série de experimentos. Usando dados de uma distribuição binomial, é
possível calcular os valores esperados de uma variável aleatória
conforme ela passa por tentativas independentes. Em outras palavras, é
possível prever o número exato de caras ou coroas que se deve esperar ao
jogar uma moeda um certo número de vezes.

Também, pode-se usar a probabilidade binomial cumulativa para encontrar
a probabilidade de obter um determinado intervalo de resultados. Por
exemplo, saber a probabilidade do nascimento de até três meninos em 10
nascimentos consecutivos quando a probabilidade de nascer um menino é
0,50.

O \emph{R} tem quatro funções embutidas para gerar distribuição
binomial. Ela são descritas a seguir.

\subsection{Funções da distribuição
binomial}\label{funuxe7uxf5es-da-distribuiuxe7uxe3o-binomial}

\subsubsection{Função pbinom()}\label{funuxe7uxe3o-pbinom}

Esta função retorna o valor da \emph{função de densidade cumulativa}
(cdf) da distribuição binomial dada uma certa variável aleatória
\emph{q}, número de tentativas (\emph{size}) e probabilidade de sucesso
em cada tentativa (\emph{prob}).

\ul{Argumentos}:

\begin{itemize}
\tightlist
\item
  \textbf{q} \(\to\) vetor de quantis\\
\item
  \textbf{size} \(\to\) numero de ensaios\\
\item
  \textbf{prob} \(\to\) probabilidade de sucesso em cada ensaio\\
\item
  \textbf{lower.tail} \(\to\) Se \texttt{TRUE}, as probabilidades são
  (\(P \le x\)), caso contrário \(P(X > x)\)
\end{itemize}

Por exemplo, qual é a probabilidade de nascer até três meninos em cinco
nascimentos, sabendo que a probabiliade de nascer um menino é igual a
0.50?

\begin{Shaded}
\begin{Highlighting}[]
\FunctionTok{pbinom}\NormalTok{ (}\DecValTok{3}\NormalTok{, }\DecValTok{5}\NormalTok{, }\FloatTok{0.50}\NormalTok{)}
\end{Highlighting}
\end{Shaded}

\begin{verbatim}
[1] 0.8125
\end{verbatim}

Isso corresponde a soma das probabilidades de nascer nenhum menino, um
menino, dois meninos e três meninos (Figura~\ref{fig-menino3}). Isto é
calculado pela equação \(P(X = x)\), vista anteriormente.\\
Colocando no \emph{R}:

\begin{Shaded}
\begin{Highlighting}[]
\NormalTok{n }\OtherTok{=} \DecValTok{5}
\NormalTok{p }\OtherTok{=} \FloatTok{0.50}
\NormalTok{x }\OtherTok{\textless{}{-}} \DecValTok{0}\SpecialCharTok{:}\DecValTok{5}
\CommentTok{\# Probabilidades de meninos }
\NormalTok{Fx }\OtherTok{\textless{}{-}}\NormalTok{ (}\FunctionTok{factorial}\NormalTok{(n)}\SpecialCharTok{/}\NormalTok{(}\FunctionTok{factorial}\NormalTok{(x)}\SpecialCharTok{*}\FunctionTok{factorial}\NormalTok{(n}\SpecialCharTok{{-}}\NormalTok{x)))}\SpecialCharTok{*}\NormalTok{ p}\SpecialCharTok{\^{}}\NormalTok{x }\SpecialCharTok{*}\NormalTok{(}\DecValTok{1}\SpecialCharTok{{-}}\NormalTok{p)}\SpecialCharTok{\^{}}\NormalTok{(n}\SpecialCharTok{{-}}\NormalTok{x)}
\NormalTok{Fx}
\end{Highlighting}
\end{Shaded}

\begin{verbatim}
[1] 0.03125 0.15625 0.31250 0.31250 0.15625 0.03125
\end{verbatim}

\begin{verbatim}
[1] 0.8125
\end{verbatim}

\begin{figure}

\centering{

\includegraphics[width=0.7\linewidth,height=0.7\textheight]{09-probabilidades_files/figure-pdf/fig-menino3-1.pdf}

}

\caption{\label{fig-menino3}Distribuição binomial, mostrando a P (x
\textless{} 4) com n = 5 e p = 0.50}

\end{figure}%

\subsubsection{Função qbinom()}\label{funuxe7uxe3o-qbinom}

Esta função retorna o valor da função de densidade cumulativa inversa
(cdf) da distribuição binomial dada uma certa variável aleatória
\emph{q}, número de tentativas (\emph{size}) e probabilidade de sucesso
em cada tentativa (\emph{prob}). Com o uso desta função, podemos
descobrir o quantil da distribuição binomial.

\ul{Argumentos}:

\begin{itemize}
\tightlist
\item
  \textbf{p} \(\to\) probabilidade ou vetor de probabilidades\\
\item
  \textbf{size} \(\to\) numero de ensaios\\
\item
  \textbf{prob} \(\to\) probabilidade de sucesso em cada ensaio\\
\item
  \textbf{lower.tail} \(\to\) Se \texttt{TRUE}, as probabilidades são
  (\(P \le x\)), caso contrário \(P(X > x)\)
\end{itemize}

Por exemplo, quantos meninos nascerão em 5 partos com 81.25\% de
probabilidade cumulativa?

\begin{Shaded}
\begin{Highlighting}[]
\FunctionTok{qbinom}\NormalTok{ (}\FloatTok{0.8125}\NormalTok{, }\AttributeTok{size =} \DecValTok{5}\NormalTok{, }\AttributeTok{prob =} \FloatTok{0.50}\NormalTok{)}
\end{Highlighting}
\end{Shaded}

\begin{verbatim}
[1] 3
\end{verbatim}

\subsubsection{Função rbinom()}\label{funuxe7uxe3o-rbinom}

A função \texttt{rbinom()} permite extrair \emph{n} observações
aleatórias de uma distribuição binomial. Os argumentos da função são
descritos abaixo:

\ul{Argumentos}:

\begin{itemize}
\tightlist
\item
  \textbf{n} \(\to\) número de observações aleatórias a ser gerado\\
\item
  \textbf{size} \(\to\) numero de ensaios\\
\item
  \textbf{prob} \(\to\) probabilidade de sucesso em cada ensaio
\end{itemize}

Para fazer uma simulação de 1000 amostras, aleatoriamente, de tamanho 5
e com probabilidade de nascer menino igual a 0,50, usa-se \footnote{Deve
  ser especificado uma ``semente'' (seed) antes de executar a função,
  senão será obtido um conjunto diferente de observações aleatórias a
  cada execução. Teste para verificar}:

\begin{Shaded}
\begin{Highlighting}[]
\FunctionTok{set.seed}\NormalTok{(}\DecValTok{23}\NormalTok{)}
\NormalTok{menino }\OtherTok{\textless{}{-}} \FunctionTok{rbinom}\NormalTok{(}\AttributeTok{n =} \DecValTok{1000}\NormalTok{, }\AttributeTok{size =} \DecValTok{5}\NormalTok{, }\AttributeTok{prob =} \FloatTok{0.5}\NormalTok{)}
\end{Highlighting}
\end{Shaded}

Cada amostra de n = 5 exibe o número de meninos nascidos. Pode-se fazer
a média que representa o valor esperado do número de sucessos
(nascimento de menino, no exemplo) em um conjunto de ensaios
independentes:

\begin{Shaded}
\begin{Highlighting}[]
\FunctionTok{mean}\NormalTok{(menino)}
\end{Highlighting}
\end{Shaded}

\begin{verbatim}
[1] 2.515
\end{verbatim}

Quanto maior o número de variáveis aleatória criadas, mais próximo a
média do número de sucessos estará do número esperado de sucessos que é
igual ao número de sucessos vezes a probabilidade de sucesso em cada
ensaio (5 x 0,50 = 2,5).

Estranho, não é? Dois meninos e meio, em média por ensaio! É, a média é
assim, uma estimativa, expectativa matemática! Não é real\ldots{}

\subsubsection{Função dbinom()}\label{funuxe7uxe3o-dbinom}

Essa função retorna o valor da função de densidade de probabilidade
(\emph{pdf}) da distribuição binomial dada uma determinada variável
aleatória X, número de tentativas (\emph{size}) e probabilidade de
sucesso em cada tentativa (\emph{prob}). A função tem a seguinte
sintaxe:

\ul{Argumentos}:

\begin{itemize}
\tightlist
\item
  \textbf{x} \(\to\) vetor de números\\
\item
  \textbf{size} \(\to\) numero de ensaios\\
\item
  \textbf{prob} \(\to\) probabilidade de sucesso em cada ensaio
\end{itemize}

A função é usada para encontrar a probabilidade de um determinado valor
para dados que seguem a distribuição binomial, ou seja, encontra
\(P(X=x)\), probabilidade de \emph{x} sucessos em tentativas de tamanho
(\emph{size}) \emph{n} quando a probabilidade (\emph{p}) de sucesso é
\emph{prob}. Obtém o mesmo resultado da fórmula:

\[
P(X = x)= C \times p^x \times (1 - p)^{n-x}
\]

Por exemplo, no nascimento de uma criança, as duas possibilidades,
menino ou menina, são mutuamente excludentes e esses são os únicos
eventos que podem acontecer. A probabilidade de nascimento de menino,
como visto, é 0,50, qual seria a probabilidade de nascerem 4 meninos em
5 partos consecutivos não gemelares (Figura~\ref{fig-menino5})?

\begin{Shaded}
\begin{Highlighting}[]
\FunctionTok{dbinom}\NormalTok{(}\DecValTok{4}\NormalTok{, }\AttributeTok{size =} \DecValTok{5}\NormalTok{, }\AttributeTok{prob =} \FloatTok{0.50}\NormalTok{)}
\end{Highlighting}
\end{Shaded}

\begin{verbatim}
[1] 0.15625
\end{verbatim}

As probabilidades de nascerem meninos em 5 nascimentos são:

\begin{Shaded}
\begin{Highlighting}[]
\NormalTok{Fx }\OtherTok{\textless{}{-}} \FunctionTok{dbinom}\NormalTok{(}\DecValTok{0}\SpecialCharTok{:}\DecValTok{5}\NormalTok{, }\DecValTok{5}\NormalTok{, }\FloatTok{0.50}\NormalTok{)}
\NormalTok{Fx}
\end{Highlighting}
\end{Shaded}

\begin{verbatim}
[1] 0.03125 0.15625 0.31250 0.31250 0.15625 0.03125
\end{verbatim}

\begin{figure}

\centering{

\includegraphics[width=0.7\linewidth,height=0.7\textheight]{09-probabilidades_files/figure-pdf/fig-menino5-1.pdf}

}

\caption{\label{fig-menino5}Distribuição binomial para P (x = 4) com n =
5 e p = 0,50}

\end{figure}%

\subsection{Média e desvio padrão da distribuição
binomial}\label{sec-mudpbinom}

Quando o número de repetições é grande, geralmente há necessidade de
resumir as probabilidades. A distribuição binomial pode ser descrita por
sua \emph{média} e \emph{variância}.

A média é o valor médio da variável aleatória em um longo número de
repetições. É também chamada de \emph{valor esperado} ou
\emph{expectativa}. A expectativa de uma variável aleatória X,
geralmente, é denotada por \(E(X)\) e obtida pela multiplicação do
número de ensaios independentes (\emph{n}) pela probabilidade (\emph{p})
de sucesso em cada ensaio:

\[
\mu = E(X) = n \times p
\]

Portanto, a expectativa (esperança) de nascimento de meninos em 5 partos
é \(E(X)=5 \times 0,50 = 2,5\), como visto na função \texttt{rbinom()}.
Observe que o valor esperado de uma variável aleatória discreta não tem
um valor que a variável aleatória pode realmente assumir.

Por exemplo, para o número médio de meninos em um parto, ou não se tem
menino ou se tem 1 menino, cada uma possibilidade com probabilidade de
0,50 e o valor esperado é (0 × 0,50) + (1 × 0,50) = 0,50. O número de
meninos deve ser 0 ou 1, mas o valor esperado é a metade, a média que se
obteria no longo prazo.

A variância de uma variável aleatória discreta \emph{X} é igual a

\[
\sigma^2=var(X) = n\times p \times (1-p)
\]

Consequentemente, o desvio padrão é igual a

\[
\sigma = \sqrt{var(X)} = \sqrt{n\times p \times (1-p)}
\]

Para o exemplo de 5 nascimentos, a média foi de 2,5 meninos e o desvio
padrão

\[
\sigma =\sqrt{5\times 0.50 \times (1-0.50)}=\sqrt{2.5 \times 0.50}= 1.12
\]

Portanto, se espera que ocorram em média 2,5 (\(\sigma\) = 1,12)
nascimentos de meninos em 5 partos.

\section{Distribuição de Poisson}\label{distribuiuxe7uxe3o-de-poisson}

A distribuição de Poisson é utilizada para descrever a probabilidade do
número de ocorrências em um intervalo contínuo (de tempo ou espaço). No
caso da distribuição binomial, a variável de interesse é o número de
sucessos em um intervalo discreto (\emph{n} ensaios de Bernoulli).

A unidade de medida (tempo ou espaço) é uma variável contínua, mas a
variável aleatória, o número de ocorrências, é discreta. Esta
distribuição segue as mesmas premissas da distribuição binomial:

\begin{itemize}
\tightlist
\item
  as tentativas são independentes;
\item
  a variável aleatória é o número de eventos em cada amostra;
\item
  a probabilidade é constante em cada intervalo
\end{itemize}

Ela é utilizada para modelar eventos discretos que ocorrem com pouca
frequência no tempo ou espaço, por isso é algumas vezes denominada de
\emph{distribuição de eventos raros}. Pode-se usar a distribuição de
Poisson como uma aproximação da distribuição Binomial quando \emph{n}, o
número de tentativas, for grande e \emph{p} ou (1 -- \emph{p}) for
pequeno (eventos raros).

Um bom princípio básico é usar a distribuição de Poisson quando
\(n \ge 20\) e \(n \times p\) ou \(n \times (1- p)\) \textless{} 5\%
(93). Nessas condições, a probabilidade que uma variável aleatória
\emph{X} adote um valor \emph{x} é

\[
P(X = x) = \frac {e^{-\lambda} \times \lambda^x}{x!}
\]

onde \(\lambda\) (lambda) representa o número de ocorrências de um
evento em um intervalo de tempo e é conhecida como parâmetro da
distribuição de Poisson e é igual em média a \(n \times p\).

No \emph{R}, essa probabilidade é dada pela função
\texttt{dpois(x,\ lambda)}.

\ul{Exemplo}:

Suponha que a probabilidade de uma puérpera ter infecção congênita
(rubéola) seja igual a 0,0009. Qual seria a probabilidade, em uma
população de 6000 gestantes, de que 5 estejam infectadas?

\begin{Shaded}
\begin{Highlighting}[]
\NormalTok{p }\OtherTok{\textless{}{-}} \FloatTok{0.0009}
\NormalTok{x }\OtherTok{\textless{}{-}} \DecValTok{5}
\NormalTok{n }\OtherTok{\textless{}{-}} \DecValTok{6000}
\NormalTok{lambda }\OtherTok{\textless{}{-}}\NormalTok{ n }\SpecialCharTok{*}\NormalTok{ p}
\NormalTok{P }\OtherTok{\textless{}{-}} \FunctionTok{dpois}\NormalTok{(x, lambda)}
\FunctionTok{round}\NormalTok{ (P, }\DecValTok{3}\NormalTok{)}
\end{Highlighting}
\end{Shaded}

\begin{verbatim}
[1] 0.173
\end{verbatim}

Portanto, a probabilidade de se encontrar 5 mulheres com infecção
congênita é de aproximadamente 17\%.

\chapter{Assimetria e Curtose}\label{assimetria-e-curtose}

\section{Pacotes necessários neste
capítulo}\label{pacotes-necessuxe1rios-neste-capuxedtulo-2}

\begin{Shaded}
\begin{Highlighting}[]
\NormalTok{pacman}\SpecialCharTok{::}\FunctionTok{p\_load}\NormalTok{(DescTools,}
\NormalTok{               dplyr, }
\NormalTok{               e1071, }
\NormalTok{               flextable,}
\NormalTok{               ggplot2, }
\NormalTok{               ggpubr, }
\NormalTok{               grDevices, }
\NormalTok{               moments, }
\NormalTok{               readxl, }
\NormalTok{               rstatix)}
\end{Highlighting}
\end{Shaded}

\section{Dados usados neste
capítulo}\label{dados-usados-neste-capuxedtulo}

Será usada a mesma variável \texttt{altura} de 1368 mulheres do conjunto
de dados\texttt{dadosMater.xlsx}, já mostrado anteriormente
(Seção~\ref{sec-aplicafreq}).

\subsection{Exploração dos dados}\label{sec-assimetria}

O resumo dos dados pode ser realizado, usando a função
\texttt{summarise()} do pacote \texttt{dplyr}. A moda será calculada
usando com função \texttt{Mode()} do pacote \texttt{DescTools}.

\begin{Shaded}
\begin{Highlighting}[]
\NormalTok{dados }\OtherTok{\textless{}{-}} \FunctionTok{read\_excel}\NormalTok{(}\StringTok{"dados/dadosMater.xlsx"}\NormalTok{) }\SpecialCharTok{\%\textgreater{}\%} 
  \FunctionTok{select}\NormalTok{(altura)}

\NormalTok{resumo }\OtherTok{\textless{}{-}}\NormalTok{ dados }\SpecialCharTok{\%\textgreater{}\%} 
  \FunctionTok{summarise}\NormalTok{(}\AttributeTok{n =} \FunctionTok{n}\NormalTok{(),}
            \AttributeTok{media =} \FunctionTok{mean}\NormalTok{(altura, }\AttributeTok{na.rm =} \ConstantTok{TRUE}\NormalTok{),}
            \AttributeTok{dp =} \FunctionTok{sd}\NormalTok{(altura, }\AttributeTok{na.rm =} \ConstantTok{TRUE}\NormalTok{),}
            \AttributeTok{mediana =} \FunctionTok{median}\NormalTok{(altura, }\AttributeTok{na.rm =} \ConstantTok{TRUE}\NormalTok{),}
            \AttributeTok{moda =} \FunctionTok{Mode}\NormalTok{(altura),}
            \AttributeTok{Q1 =} \FunctionTok{quantile}\NormalTok{ (altura, }\FloatTok{0.25}\NormalTok{),}
            \AttributeTok{Q3 =} \FunctionTok{quantile}\NormalTok{ (altura, }\FloatTok{0.75}\NormalTok{),}
            \AttributeTok{CV =}\NormalTok{ dp}\SpecialCharTok{/}\NormalTok{media)}
\NormalTok{resumo}
\end{Highlighting}
\end{Shaded}

\begin{verbatim}
# A tibble: 1 x 8
      n media     dp mediana  moda    Q1    Q3     CV
  <int> <dbl>  <dbl>   <dbl> <dbl> <dbl> <dbl>  <dbl>
1  1368  1.60 0.0655     1.6   1.6  1.55  1.65 0.0410
\end{verbatim}

Para a exploração visual dos dados, será construído um histograma com um
boxplot sobreposto (Figura~\ref{fig-histbxp}). Para sobrepor o boxplot,
foi usado um y = 350 que está acima da frequência máxima. Ajustamos a
largura do boxplot com a função width = 13 (método dos acerts e erros).
Para colocar o boxplot na horizontal não se usa a função cord\_flip()
(veja Seção~\ref{sec-bxphorizontal}). Aqui o ``truque'' é fixar o valor
de y no boxplot e ajustar a altura com width(). Isto mantém o boxplot
deitado no topo do histograma. É uma apresentação interessante para
visualzar a simetria dos dados.

\begin{Shaded}
\begin{Highlighting}[]
\CommentTok{\# Estruturação do histograma}
\NormalTok{histograma }\OtherTok{\textless{}{-}} \FunctionTok{ggplot}\NormalTok{(dados, }\FunctionTok{aes}\NormalTok{(}\AttributeTok{x =}\NormalTok{ altura)) }\SpecialCharTok{+}
  \FunctionTok{geom\_histogram}\NormalTok{(}\AttributeTok{binwidth =} \FloatTok{0.04}\NormalTok{, }
                 \AttributeTok{fill =} \StringTok{"lightblue"}\NormalTok{, }
                 \AttributeTok{color =} \StringTok{"black"}\NormalTok{)}
\CommentTok{\# Boxplot sobreposto}
\NormalTok{histograma }\SpecialCharTok{+}
  \FunctionTok{geom\_boxplot}\NormalTok{(}\FunctionTok{aes}\NormalTok{(}\AttributeTok{y =} \DecValTok{350}\NormalTok{), }\AttributeTok{width =} \DecValTok{13}\NormalTok{, }\AttributeTok{fill =} \StringTok{"lightblue"}\NormalTok{, }\AttributeTok{color =} \StringTok{"black"}\NormalTok{) }\SpecialCharTok{+}
  \FunctionTok{theme\_classic}\NormalTok{(}\AttributeTok{base\_size =} \DecValTok{13}\NormalTok{) }\SpecialCharTok{+} 
  \FunctionTok{scale\_y\_continuous}\NormalTok{(}\AttributeTok{limits =} \FunctionTok{c}\NormalTok{(}\DecValTok{0}\NormalTok{, }\ConstantTok{NA}\NormalTok{))}\SpecialCharTok{+}
  \FunctionTok{ylab}\NormalTok{(}\StringTok{"Frequência"}\NormalTok{)}\SpecialCharTok{+}
  \FunctionTok{xlab}\NormalTok{(}\StringTok{"Altura (m)"}\NormalTok{)}
\end{Highlighting}
\end{Shaded}

\begin{figure}[H]

\centering{

\includegraphics[width=0.8\linewidth,height=0.8\textheight]{10-assimetria_files/figure-pdf/fig-histbxp-1.pdf}

}

\caption{\label{fig-histbxp}Histograma da altura das gestantes com
boxplot sobreposto.}

\end{figure}%

\section{Assimetria}\label{assimetria}

A assimetria analisa a proximidade ou o afastamento de um conjunto de
dados quantitativos em relação à distribuição normal. Mede o grau de
afastamento de uma distribuição em relação a um eixo central (geralmente
a média).

Quando a curva é simétrica, a média, a mediana e a moda coincidem, num
mesmo ponto, havendo um perfeito equilíbrio na distribuição. Quando o
equilíbrio não acontece, isto é, a média, a mediana e a moda recaem em
pontos diferentes da distribuição esta será assimétrica; enviesada a
direita ou esquerda. podendo-se caracterizar como curvas assimétricas à
direita ou à esquerda. Quando a distribuição é assimétrica à esquerda ou
assimetria negativa, a cauda da curva localiza-se à esquerda, desviando
a média para este lado (Figura~\ref{fig-skew}). Na assimetria positiva,
ocorre o contrário, a cauda está localizada à direita e da mesma forma a
média (94).

\begin{figure}

\centering{

\includegraphics[width=0.8\linewidth,height=\textheight,keepaspectratio]{index_files/mediabag/KRTdQ5v.png}

}

\caption{\label{fig-skew}Assimetria}

\end{figure}%

\subsection{Avaliação da
assimetria}\label{avaliauxe7uxe3o-da-assimetria}

O \emph{R} dispões de diversas maneiras para o cálculo do
\emph{coeficiente de assimetria}. O coeficiente de assimetria é um
método numérico estatístico para medir a assimetria da distribuição ou
conjunto de dados. Ele fala sobre a posição da maioria dos valores de
dados na distribuição em torno do valor central.

\subsubsection{Cálculo do coeficiente de
assimetria}\label{cuxe1lculo-do-coeficiente-de-assimetria}

Várias medidas de coeficientes de assimetria amostrais foram propostas.
O coeficiente de assimetria pode ser calculado no \emph{R}, usando a
função \texttt{skewness()} do pacote \texttt{e1071} (95). Esta função
usa os seguintes argumentos:

\begin{itemize}
\tightlist
\item
  \textbf{x} \(\to\) vetor numérico que contém os valores
\item
  \textbf{na.rm} \(\to\) um valor lógico que indica se os valores NA
  devem ser eliminados antes que o cálculo prossiga.
\item
  \textbf{type} \(\to\) número inteiro entre 1 e 3 selecionando um dos
  algoritmos para calcular assimetria detalhados abaixo.
\end{itemize}

Os três tipos são os seguintes:

\begin{enumerate}
\def\labelenumi{\arabic{enumi}.}
\tightlist
\item
  \textbf{Tipo 1, g1} \(\to\) definição típica usada em muitos livros
  didáticos mais antigos. Dada pela fórmula:
\end{enumerate}

\[
g_1=\frac{m_3}{m_2^\frac{3}{2}}
\]

onde os momentos amostrais para amostras de tamanho \emph{n} são dados
por:

\[
m_r=\frac{\sum(x_i - \overline{x})^r}{n}
\]

Para o momento central amostral de ordem \emph{r} = 3, tem-se:

\[
m_3=\frac{\sum(x_i - \overline{x})^3}{n}
\] Para \emph{r} = 2,

\[
m_2=\frac{\sum(x_i - \overline{x})^2}{n}
\]

Usando o resumo dos \texttt{dados}:

\begin{Shaded}
\begin{Highlighting}[]
\NormalTok{ m3 }\OtherTok{\textless{}{-}}\NormalTok{ (}\FunctionTok{sum}\NormalTok{((dados}\SpecialCharTok{$}\NormalTok{altura }\SpecialCharTok{{-}}\NormalTok{ (}\FunctionTok{mean}\NormalTok{(dados}\SpecialCharTok{$}\NormalTok{altura)))}\SpecialCharTok{\^{}}\DecValTok{3}\NormalTok{))}\SpecialCharTok{/}\NormalTok{resumo}\SpecialCharTok{$}\NormalTok{n}
\NormalTok{ m3}
\end{Highlighting}
\end{Shaded}

\begin{verbatim}
[1] 5.081924e-05
\end{verbatim}

\begin{Shaded}
\begin{Highlighting}[]
\NormalTok{ m2 }\OtherTok{\textless{}{-}}\NormalTok{ (}\FunctionTok{sum}\NormalTok{((dados}\SpecialCharTok{$}\NormalTok{altura }\SpecialCharTok{{-}}\NormalTok{ (}\FunctionTok{mean}\NormalTok{(dados}\SpecialCharTok{$}\NormalTok{altura)))}\SpecialCharTok{\^{}}\DecValTok{2}\NormalTok{))}\SpecialCharTok{/}\NormalTok{resumo}\SpecialCharTok{$}\NormalTok{n}
\NormalTok{ m2}
\end{Highlighting}
\end{Shaded}

\begin{verbatim}
[1] 0.004284321
\end{verbatim}

Colocando os dados na fórmula do \emph{g1} no \emph{R}, chega-se ao
resultado:

\begin{Shaded}
\begin{Highlighting}[]
\NormalTok{ g1 }\OtherTok{\textless{}{-}}\NormalTok{ m3}\SpecialCharTok{/}\NormalTok{(m2)}\SpecialCharTok{\^{}}\NormalTok{(}\DecValTok{3}\SpecialCharTok{/}\DecValTok{2}\NormalTok{)}
\NormalTok{ g1}
\end{Highlighting}
\end{Shaded}

\begin{verbatim}
[1] 0.1812196
\end{verbatim}

Usando a função \texttt{skewness()} do pacote \texttt{e1071}, chega-se
ao mesmo resultado:

\begin{Shaded}
\begin{Highlighting}[]
\NormalTok{e1071}\SpecialCharTok{::}\FunctionTok{skewness}\NormalTok{(dados}\SpecialCharTok{$}\NormalTok{altura, }\AttributeTok{type =} \DecValTok{1}\NormalTok{)}
\end{Highlighting}
\end{Shaded}

\begin{verbatim}
[1] 0.1812196
\end{verbatim}

\begin{enumerate}
\def\labelenumi{\arabic{enumi}.}
\setcounter{enumi}{1}
\tightlist
\item
  \textbf{Tipo 2, G1} \(\to\) Usado em vários pacotes estatísticos. É
  calculado com a seguinte fórmula:
\end{enumerate}

\[
G_1=\frac{g_1 \sqrt{n(n-1)}}{n-2}
\]

Colocando os dados na fórmula na linguagem do \emph{R}, tem-se:

\begin{Shaded}
\begin{Highlighting}[]
\NormalTok{ G1 }\OtherTok{\textless{}{-}}\NormalTok{ (g1}\SpecialCharTok{*}\FunctionTok{sqrt}\NormalTok{((resumo}\SpecialCharTok{$}\NormalTok{n}\SpecialCharTok{*}\NormalTok{(resumo}\SpecialCharTok{$}\NormalTok{n}\DecValTok{{-}1}\NormalTok{))))}\SpecialCharTok{/}\NormalTok{(resumo}\SpecialCharTok{$}\NormalTok{n}\DecValTok{{-}2}\NormalTok{)}
\NormalTok{ G1}
\end{Highlighting}
\end{Shaded}

\begin{verbatim}
[1] 0.1814186
\end{verbatim}

Calculando com a função \texttt{skewness()} do pacote \texttt{e1071}:

\begin{Shaded}
\begin{Highlighting}[]
\NormalTok{e1071}\SpecialCharTok{::}\FunctionTok{skewness}\NormalTok{(dados}\SpecialCharTok{$}\NormalTok{altura, }\AttributeTok{type =} \DecValTok{2}\NormalTok{)}
\end{Highlighting}
\end{Shaded}

\begin{verbatim}
[1] 0.1814186
\end{verbatim}

\begin{enumerate}
\def\labelenumi{\arabic{enumi}.}
\setcounter{enumi}{2}
\tightlist
\item
  \textbf{Tipo 3, b1} \(\to\) É o padrão da função \texttt{skewness()}
  do pacote \texttt{e1071}. Usa-se a seguinte fórmula para o cálculo:
\end{enumerate}

\[
b_1= \frac {m_3}{s^3} 
\]

onde \emph{s} é o desvio padrão da amostra. Na linguagem \emph{R},
tem-se:

\begin{Shaded}
\begin{Highlighting}[]
\NormalTok{ b1 }\OtherTok{\textless{}{-}}\NormalTok{ m3}\SpecialCharTok{/}\NormalTok{(resumo}\SpecialCharTok{$}\NormalTok{dp)}\SpecialCharTok{\^{}}\DecValTok{3}
\NormalTok{ b1}
\end{Highlighting}
\end{Shaded}

\begin{verbatim}
[1] 0.1810209
\end{verbatim}

Usando a função \texttt{skewness()} do pacote \texttt{e1071}:

\begin{Shaded}
\begin{Highlighting}[]
\NormalTok{e1071}\SpecialCharTok{::}\FunctionTok{skewness}\NormalTok{(dados}\SpecialCharTok{$}\NormalTok{altura, }\AttributeTok{type =} \DecValTok{3}\NormalTok{)}
\end{Highlighting}
\end{Shaded}

\begin{verbatim}
[1] 0.1810209
\end{verbatim}

Para amostras grandes, há muito pouca diferença entre as várias medidas
(96). Todas as três medidas de assimetria são imparciais sob
normalidade.

\ul{Interpretação do coeficiente de assimetria}

Quando a \(assimetria = 0\), tem-se uma distribuição simétrica e a
média, a mediana e a moda coincidem; quando a \({assimetria} < {0}\),
\({média} < {mediana} < {moda}\), a distribuição tem \emph{assimetria
negativa} e quando a \({assimetria} > {0}\),
\({média} > {mediana} > {moda}\), a distribuição tem \emph{assimetria
positiva}.

A Tabela~\ref{tbl-assimetria} sugere uma forma de interpretar o
coeficiente de assimetria (97).

\global\setlength{\Oldarrayrulewidth}{\arrayrulewidth}

\global\setlength{\Oldtabcolsep}{\tabcolsep}

\setlength{\tabcolsep}{2pt}

\renewcommand*{\arraystretch}{1.5}



\providecommand{\ascline}[3]{\noalign{\global\arrayrulewidth #1}\arrayrulecolor[HTML]{#2}\cline{#3}}

\begin{longtable}[c]{|p{2.00in}|p{2.00in}}

\caption{\label{tbl-assimetria}Interpretação do Coeficiente de
Assimetria}

\tabularnewline

\ascline{1.5pt}{666666}{1-2}

\multicolumn{1}{>{\raggedright}m{\dimexpr 2in+0\tabcolsep}}{\textcolor[HTML]{000000}{\fontsize{11}{11}\selectfont{\global\setmainfont{Arial}{\textbf{Coeficiente\ de\ assimetria}}}}} & \multicolumn{1}{>{\raggedright}m{\dimexpr 2in+0\tabcolsep}}{\textcolor[HTML]{000000}{\fontsize{11}{11}\selectfont{\global\setmainfont{Arial}{\textbf{Assimetria}}}}} \\

\ascline{1.5pt}{666666}{1-2}\endfirsthead 

\ascline{1.5pt}{666666}{1-2}

\multicolumn{1}{>{\raggedright}m{\dimexpr 2in+0\tabcolsep}}{\textcolor[HTML]{000000}{\fontsize{11}{11}\selectfont{\global\setmainfont{Arial}{\textbf{Coeficiente\ de\ assimetria}}}}} & \multicolumn{1}{>{\raggedright}m{\dimexpr 2in+0\tabcolsep}}{\textcolor[HTML]{000000}{\fontsize{11}{11}\selectfont{\global\setmainfont{Arial}{\textbf{Assimetria}}}}} \\

\ascline{1.5pt}{666666}{1-2}\endhead



\multicolumn{1}{>{\raggedright}m{\dimexpr 2in+0\tabcolsep}}{\textcolor[HTML]{000000}{\fontsize{11}{11}\selectfont{\global\setmainfont{Arial}{-1\ a\ +1}}}} & \multicolumn{1}{>{\raggedright}m{\dimexpr 2in+0\tabcolsep}}{\textcolor[HTML]{000000}{\fontsize{11}{11}\selectfont{\global\setmainfont{Arial}{leve}}}} \\





\multicolumn{1}{>{\raggedright}m{\dimexpr 2in+0\tabcolsep}}{\textcolor[HTML]{000000}{\fontsize{11}{11}\selectfont{\global\setmainfont{Arial}{-1\ a\ -2\ e\ +1\ a\ +2}}}} & \multicolumn{1}{>{\raggedright}m{\dimexpr 2in+0\tabcolsep}}{\textcolor[HTML]{000000}{\fontsize{11}{11}\selectfont{\global\setmainfont{Arial}{moderada}}}} \\





\multicolumn{1}{>{\raggedright}m{\dimexpr 2in+0\tabcolsep}}{\textcolor[HTML]{000000}{\fontsize{11}{11}\selectfont{\global\setmainfont{Arial}{-2\ a\ -3\ e\ +2\ a\ +3}}}} & \multicolumn{1}{>{\raggedright}m{\dimexpr 2in+0\tabcolsep}}{\textcolor[HTML]{000000}{\fontsize{11}{11}\selectfont{\global\setmainfont{Arial}{importante}}}} \\





\multicolumn{1}{>{\raggedright}m{\dimexpr 2in+0\tabcolsep}}{\textcolor[HTML]{000000}{\fontsize{11}{11}\selectfont{\global\setmainfont{Arial}{<\ -3\ ou\ >\ +3}}}} & \multicolumn{1}{>{\raggedright}m{\dimexpr 2in+0\tabcolsep}}{\textcolor[HTML]{000000}{\fontsize{11}{11}\selectfont{\global\setmainfont{Arial}{grave}}}} \\

\ascline{1.5pt}{666666}{1-2}


\end{longtable}

\arrayrulecolor[HTML]{000000}

\global\setlength{\arrayrulewidth}{\Oldarrayrulewidth}

\global\setlength{\tabcolsep}{\Oldtabcolsep}

\renewcommand*{\arraystretch}{1}

Observando o formato da distribuição no histograma e no boxplot, na
Figura~\ref{fig-histbxp}, e no resultado do coeficiente de assimetria,
conclui-se que a variável \texttt{altura} tem uma assimetria positiva
leve, não preocupante. É possível aceitar essa variável como
praticamente simétrica.

\subsubsection{Avaliação da assimetria com o gráfico
QQ}\label{sec-qqplot}

Outra ferramenta gráfica que permite avaliar a simetria dos dados é o
\textbf{gráfico QQ} (gráfico quantil-quantil). Ele permite observar se a
distribuição se ajusta a distribuição normal. O gráfico QQ é um gráfico
de dispersão que compara os quantis \footnote{Sobre os quantis, veja na
  Seção~\ref{sec-quantil}.} da amostra com os quantis teóricos de uma
distribuição de referência. Se os pontos do gráfico QQ formarem uma
reta, isso indica que os dados têm a mesma distribuição da referência.
Se os pontos se afastarem da reta, isso indica que os dados têm uma
distribuição diferente da referência. Para construir um gráfico QQ,
pode-se usar a função \texttt{ggqqplot()}do pacote \texttt{ggpubr}. Ele
apresenta uma linha de referência, acompanhada de uma area sombreada,
correspondente ao Intervalo de Confiança de 95\% (veja o
Capítulo~\ref{sec-estimacao}):

\begin{Shaded}
\begin{Highlighting}[]
\FunctionTok{ggqqplot}\NormalTok{(}\AttributeTok{data =}\NormalTok{ dados, }
         \AttributeTok{x =} \StringTok{"altura"}\NormalTok{,}
         \AttributeTok{conf.int =} \ConstantTok{TRUE}\NormalTok{,}
         \AttributeTok{shape =} \DecValTok{19}\NormalTok{,}
         \AttributeTok{xlab =} \StringTok{"Quantis teóricos"}\NormalTok{,}
         \AttributeTok{ylab =} \StringTok{"Altura (m)"}\NormalTok{,}
         \AttributeTok{color =} \StringTok{"dodgerblue4"}\NormalTok{)}
\end{Highlighting}
\end{Shaded}

\begin{figure}[H]

\centering{

\includegraphics[width=0.7\linewidth,height=0.7\textheight]{10-assimetria_files/figure-pdf/fig-chart-1.pdf}

}

\caption{\label{fig-chart}Gráfico QQ}

\end{figure}%

A Figura~\ref{fig-chart} exibe uma reta com IC95\% que praticamente se
sobrepõe aos pontos. É mais uma informação mostrando que os dados têm
uma distribuição simétrica aceitável.

\subsubsection{Pesquisa de valores
atípicos}\label{pesquisa-de-valores-atuxedpicos}

Os valores atípicos atraem as caudas da dispersão aumentando a
possibilidade de assimetria. No boxplot da Figura~\ref{fig-histbxp},
verifica-se a presença de \emph{outliers} que devem ser avaliados.

Para examinar os \texttt{outliers}, as estatísticas do boxplot são
úteis, pois mostram a quantidade e os respectivos valores. A função
\texttt{boxplot.stats()} do pacote \texttt{grDevices}, entregam as
estatísticas dos 5 números (min, P25, mediana, P75 e max), o total de
observações, o limite inferior e superior do intervalo de confiança de
95\% e os valores atípicos (\emph{outliers})::

\begin{Shaded}
\begin{Highlighting}[]
\FunctionTok{boxplot.stats}\NormalTok{(dados}\SpecialCharTok{$}\NormalTok{altura)}
\end{Highlighting}
\end{Shaded}

\begin{verbatim}
$stats
[1] 1.42 1.55 1.60 1.65 1.78

$n
[1] 1368

$conf
[1] 1.595728 1.604272

$out
[1] 1.40 1.82 1.80 1.40 1.40 1.85 1.80
\end{verbatim}

Outra maneira de identificar os outliers é através da função
\texttt{indentify\_outliers()} do pacote \texttt{rstatix}:

\begin{Shaded}
\begin{Highlighting}[]
\NormalTok{ dados }\SpecialCharTok{\%\textgreater{}\%} 
\NormalTok{   rstatix}\SpecialCharTok{::}\FunctionTok{identify\_outliers}\NormalTok{(altura)}
\end{Highlighting}
\end{Shaded}

\begin{verbatim}
# A tibble: 7 x 3
  altura is.outlier is.extreme
   <dbl> <lgl>      <lgl>     
1   1.4  TRUE       FALSE     
2   1.82 TRUE       FALSE     
3   1.8  TRUE       FALSE     
4   1.4  TRUE       FALSE     
5   1.4  TRUE       FALSE     
6   1.85 TRUE       FALSE     
7   1.8  TRUE       FALSE     
\end{verbatim}

Ambas as funções identificaram 7 valores atípicos (acima ou abaixo 1,5
vezes o intervalo interquartil), mas, como mostra a função
\texttt{identify\_outliers}, eles exercem pouca influência, pois não são
extremos, ou seja, acima de três vezes o intervalo interquartil.

\section{Curtose}\label{curtose}

É o grau de achatamento de uma distribuição, em relação a distribuição
normal. A curtose indica como o pico e as caudas de uma distribuição
diferem da distribuição normal. A assimetria mede essencialmente a
simetria da distribuição, enquanto a curtose determina o peso das caudas
da distribuição. Portanto, é uma medida dos tamanhos combinados das duas
caudas; mede a quantidade de probabilidade nas caudas. A curtose pode
ser de três tipos (Figura~\ref{fig-curtose}):

\begin{itemize}
\tightlist
\item
  \textbf{Mesocúrtica} \(\to\) quando a distribuição é normal;
\item
  \textbf{Leptocúrtica} \(\to\) quando a distribuição é mais pontiaguda
  e concentrada que a normal, mostrando caudas pesadas em ambos os
  lados;
\item
  \textbf{Platicúrtica} \(\to\) quando a distribuição é mais achatada e
  dispersa que a normal, com caudas planas.
\end{itemize}

Uma curtose em excesso é uma medida que compara a curtose de uma
distribuição com a curtose de uma distribuição normal. A curtose de uma
distribuição normal é igual a 3. Portanto, o excesso de curtose é
determinado subtraindo 3 da curtose:

\[
Excesso \space de \space curtose = curtose - 3
\]

A distribuição normal tem uma curtose de zero e é chamada de
\emph{mesocúrtica.} Uma distribuição com curtose maior que zero (ou
três) é mais alta e concentrada que a normal, mostrando caudas pesadas
em ambos os lados, e é chamada de \emph{leptocúrtica.} Uma distribuição
com curtose menor que zero é mais achatada e dispersa que a normal, com
caudas planas, e é chamada de \emph{platicúrtica.}

Os dados que seguem uma \emph{distribuição mesocúrtica} mostram um
excesso de curtose de zero ou próximo de zero. Isso significa que se os
dados seguem uma distribuição normal, eles seguem uma distribuição
mesocúrtica. A \emph{distribuição leptocúrtica} mostra caudas pesadas em
ambos os lados, indicando grandes valores discrepantes. Uma
\emph{distribuição leptocúrtica} manifesta uma curtose excessiva
positiva. Uma \emph{distribuição platicúrtica} mostra uma curtose
excessiva negativa, revela uma distribuição com cauda plana.

\begin{figure}

\centering{

\includegraphics[width=0.6\linewidth,height=\textheight,keepaspectratio]{index_files/mediabag/ckKeN5Z.png}

}

\caption{\label{fig-curtose}Curtose}

\end{figure}%

\subsection{Avaliação da curtose}\label{avaliauxe7uxe3o-da-curtose}

\subsubsection{Cálculo do coeficiente de
curtose}\label{cuxe1lculo-do-coeficiente-de-curtose}

O coeficiente de curtose pode ser calculado no R usando a função
\texttt{kurtosis()} do pacote \texttt{e1071}. Esta função usa os mesmos
argumentos da função \texttt{skewness()}, vista acima. Calcula três
tipos de coeficientes:

\begin{enumerate}
\def\labelenumi{\arabic{enumi}.}
\tightlist
\item
  \textbf{Tipo 1, g2} \(\to\) definição típica usada em muitos livros
  didáticos mais antigos. Dada pela fórmula:
\end{enumerate}

\[
g_2=\frac{m_4}{m_2^2} - 3
\]

onde os momentos amostrais para amostras de tamanho \emph{n} são dados
por:

\[
m_r=\frac{\sum(x_i - \overline{x})^r}{n}
\]

Para o momento central amostral de ordem \emph{r} = 4, tem-se:

\[
m_4=\frac{\sum(x_i - \overline{x})^4}{n}
\]

Para \emph{r} = 2,

\[
m_2=\frac{\sum(x_i - \overline{x})^2}{n}
\]

Usando o resumo dos dados:

\begin{Shaded}
\begin{Highlighting}[]
\NormalTok{ m4 }\OtherTok{\textless{}{-}}\NormalTok{ (}\FunctionTok{sum}\NormalTok{((dados}\SpecialCharTok{$}\NormalTok{altura }\SpecialCharTok{{-}}\NormalTok{ (}\FunctionTok{mean}\NormalTok{(dados}\SpecialCharTok{$}\NormalTok{altura)))}\SpecialCharTok{\^{}}\DecValTok{4}\NormalTok{))}\SpecialCharTok{/}\NormalTok{resumo}\SpecialCharTok{$}\NormalTok{n}
\NormalTok{ m4}
\end{Highlighting}
\end{Shaded}

\begin{verbatim}
[1] 5.734699e-05
\end{verbatim}

\begin{Shaded}
\begin{Highlighting}[]
\NormalTok{ m2 }\OtherTok{\textless{}{-}}\NormalTok{ (}\FunctionTok{sum}\NormalTok{((dados}\SpecialCharTok{$}\NormalTok{altura }\SpecialCharTok{{-}}\NormalTok{ (}\FunctionTok{mean}\NormalTok{(dados}\SpecialCharTok{$}\NormalTok{altura)))}\SpecialCharTok{\^{}}\DecValTok{2}\NormalTok{))}\SpecialCharTok{/}\NormalTok{resumo}\SpecialCharTok{$}\NormalTok{n}
\NormalTok{ m2}
\end{Highlighting}
\end{Shaded}

\begin{verbatim}
[1] 0.004284321
\end{verbatim}

Colocando os dados na fórmula do \emph{g2} no R, chega-se ao resultado:

\begin{Shaded}
\begin{Highlighting}[]
\NormalTok{g2 }\OtherTok{\textless{}{-}}\NormalTok{ (m4}\SpecialCharTok{/}\NormalTok{(m2)}\SpecialCharTok{\^{}}\DecValTok{2}\NormalTok{)}\SpecialCharTok{{-}}\DecValTok{3}
\NormalTok{g2}
\end{Highlighting}
\end{Shaded}

\begin{verbatim}
[1] 0.1242567
\end{verbatim}

Usando a função do pacote \texttt{e1071}, chega-se ao mesmo resultado:

\begin{Shaded}
\begin{Highlighting}[]
\NormalTok{ e1071}\SpecialCharTok{::}\FunctionTok{kurtosis}\NormalTok{(dados}\SpecialCharTok{$}\NormalTok{altura, }\AttributeTok{type =} \DecValTok{1}\NormalTok{)}
\end{Highlighting}
\end{Shaded}

\begin{verbatim}
[1] 0.1242567
\end{verbatim}

\begin{enumerate}
\def\labelenumi{\arabic{enumi}.}
\setcounter{enumi}{1}
\tightlist
\item
  \textbf{Tipo 2, G2} \(\to\) Usado em vários pacotes estatísticos. É
  calculado com a seguinte fórmula:
\end{enumerate}

\[
 G_2=\left (\left (n + 1 \right )g_2 + 6 \right )\frac{\left (n - 1 \right)}{\left ( \left(n-2 \right)\left (n-3 \right) \right )}
\]

Colocando os dados na fórmula na linguagem do \emph{R}, tem-se:

\begin{Shaded}
\begin{Highlighting}[]
\NormalTok{ G2 }\OtherTok{\textless{}{-}}\NormalTok{ ((resumo}\SpecialCharTok{$}\NormalTok{n}\SpecialCharTok{+}\DecValTok{1}\NormalTok{)}\SpecialCharTok{*}\NormalTok{g2 }\SpecialCharTok{+} \DecValTok{6}\NormalTok{)}\SpecialCharTok{*}\NormalTok{(resumo}\SpecialCharTok{$}\NormalTok{n}\DecValTok{{-}1}\NormalTok{)}\SpecialCharTok{/}\NormalTok{((resumo}\SpecialCharTok{$}\NormalTok{n}\DecValTok{{-}2}\NormalTok{)}\SpecialCharTok{*}\NormalTok{(resumo}\SpecialCharTok{$}\NormalTok{n}\DecValTok{{-}3}\NormalTok{))}
\NormalTok{ G2}
\end{Highlighting}
\end{Shaded}

\begin{verbatim}
[1] 0.1291109
\end{verbatim}

Com a função \texttt{kurtosis()} do pacote \texttt{e1071}:

\begin{Shaded}
\begin{Highlighting}[]
\NormalTok{ e1071}\SpecialCharTok{::}\FunctionTok{kurtosis}\NormalTok{(dados}\SpecialCharTok{$}\NormalTok{altura, }\AttributeTok{type =} \DecValTok{2}\NormalTok{)}
\end{Highlighting}
\end{Shaded}

\begin{verbatim}
[1] 0.1291109
\end{verbatim}

\begin{enumerate}
\def\labelenumi{\arabic{enumi}.}
\setcounter{enumi}{2}
\tightlist
\item
  \textbf{Tipo 3, b2} \(\to\) É o padrão da função \texttt{kurtosis()}
  do pacote \texttt{e1071.} Usa-se a seguinte fórmula para o cálculo:
\end{enumerate}

\[
b_2=\frac{m_4}{s^4}-3
\] onde \emph{s} é o desvio padrão da amostra.

Na linguagem \emph{R}, tem-se:

\begin{Shaded}
\begin{Highlighting}[]
\NormalTok{ b2 }\OtherTok{\textless{}{-}}\NormalTok{ m4}\SpecialCharTok{/}\NormalTok{(resumo}\SpecialCharTok{$}\NormalTok{dp)}\SpecialCharTok{\^{}}\DecValTok{4} \SpecialCharTok{{-}} \DecValTok{3}
\NormalTok{ b2}
\end{Highlighting}
\end{Shaded}

\begin{verbatim}
[1] 0.1196907
\end{verbatim}

Com a função \texttt{kurtosis()}:

\begin{Shaded}
\begin{Highlighting}[]
\NormalTok{e1071}\SpecialCharTok{::}\FunctionTok{kurtosis}\NormalTok{(dados}\SpecialCharTok{$}\NormalTok{altura, }\AttributeTok{type =} \DecValTok{3}\NormalTok{)}
\end{Highlighting}
\end{Shaded}

\begin{verbatim}
[1] 0.1196907
\end{verbatim}

Novamente, para amostras grandes, há muito pouca diferença entre as
várias medidas, principalmente entre G2 e b2 (96).

\subsubsection{Interpretação do coeficiente de
curtose}\label{interpretauxe7uxe3o-do-coeficiente-de-curtose}

Os coeficientes calculados pela função do pacote \texttt{e1071} retornam
um resultado equivalente ao excesso de curtose. A curva normal tem um
excesso de curtose próximo a zero e a curva é dita mesocúrtica. Se o
coeficiente for positivo, os dados são \emph{leptocúrticos} e se for
negativo, os dados são \emph{platicúrticos}. O resultado do exemplo
aponta para uma distribuição \emph{leptocúrtica}, pois existe um pequeno
excesso de curtose (g2 = 0.1242567). Os valores que contribuem para a
curtose são aqueles fora da região do pico, ou seja, ou \emph{outliers}.
A curva \emph{mesocúrtica} tem um coeficiente de 3. Portanto, os valores
calculados anteriormente referem-se ao excesso de curtose. O resultado
da g2 = 0,1242567 pode ser escrito como b2 = 3,1242567. Daí o termo
excesso de curtose.

A função \texttt{kurtosis()} do pacote \texttt{moments} retorna um
resultado ao redor de 3, para o coeficiente tipo 1. Para chegar ao mesmo
resultado do coeficiente tipo 1 da função do pacote \texttt{e1071},
deve-se subtrair 3 do resultado.

\begin{Shaded}
\begin{Highlighting}[]
\NormalTok{moments}\SpecialCharTok{::}\FunctionTok{kurtosis}\NormalTok{(dados}\SpecialCharTok{$}\NormalTok{altura)}
\end{Highlighting}
\end{Shaded}

\begin{verbatim}
[1] 3.124257
\end{verbatim}

\begin{tcolorbox}[enhanced jigsaw, bottomrule=.15mm, opacitybacktitle=0.6, colframe=quarto-callout-tip-color-frame, arc=.35mm, coltitle=black, toptitle=1mm, colback=white, colbacktitle=quarto-callout-tip-color!10!white, breakable, bottomtitle=1mm, rightrule=.15mm, titlerule=0mm, toprule=.15mm, opacityback=0, leftrule=.75mm, left=2mm, title=\textcolor{quarto-callout-tip-color}{\faLightbulb}\hspace{0.5em}{Exercício 1}]

Criar um conjunto de dados com distribuição normal com média 0 e desvio
padrão 1 e n = 10000 . Verifique a assimetria e a curtose deste conjunto

\end{tcolorbox}

\ul{Resposta}:

\begin{enumerate}
\def\labelenumi{\arabic{enumi}.}
\tightlist
\item
  Conjunto de dados: \texttt{meusDados}
\end{enumerate}

\begin{Shaded}
\begin{Highlighting}[]
\FunctionTok{set.seed}\NormalTok{(}\DecValTok{1234}\NormalTok{)}
\NormalTok{meusDados }\OtherTok{\textless{}{-}} \FunctionTok{rnorm}\NormalTok{(}\DecValTok{100000}\NormalTok{, }\AttributeTok{mean =} \DecValTok{0}\NormalTok{, }\AttributeTok{sd =} \DecValTok{1}\NormalTok{)}
\end{Highlighting}
\end{Shaded}

\begin{enumerate}
\def\labelenumi{\arabic{enumi}.}
\setcounter{enumi}{1}
\tightlist
\item
  Construa um histograma (Figura~\ref{fig-histNormal}) com curva normal
  sobreposta:
\end{enumerate}

\begin{figure}

\centering{

\includegraphics[width=0.7\linewidth,height=\textheight,keepaspectratio]{10-assimetria_files/figure-pdf/fig-histNormal-1.pdf}

}

\caption{\label{fig-histNormal}Histograma com curva normal}

\end{figure}%

\begin{enumerate}
\def\labelenumi{\arabic{enumi}.}
\setcounter{enumi}{2}
\tightlist
\item
  Observe a skewness e a kurtosis
\end{enumerate}

\begin{Shaded}
\begin{Highlighting}[]
\NormalTok{e1071}\SpecialCharTok{::}\FunctionTok{skewness}\NormalTok{(meusDados)}
\end{Highlighting}
\end{Shaded}

\begin{verbatim}
[1] 0.008609517
\end{verbatim}

\begin{Shaded}
\begin{Highlighting}[]
\NormalTok{e1071}\SpecialCharTok{::}\FunctionTok{kurtosis}\NormalTok{(meusDados)}
\end{Highlighting}
\end{Shaded}

\begin{verbatim}
[1] -0.003450388
\end{verbatim}

Como era de se esperar, usando a \texttt{rnorm()}, a distribuição é um
exemplo de distribuição normal, \(skewness \approx 0\) e
\(kurtosis \approx 0\). Observe que a cada vez que os comandos forem
executados, os resultados serão discretamente diferentes. Para evitar
isso, deve-se usar \texttt{set.seed()}, veja a seção
Seção~\ref{sec-dnp}. Faça o teste!

\chapter{Distribuições Amostrais}\label{sec-cap11}

\section{Pacotes necessários para este
capítulo}\label{pacotes-necessuxe1rios-para-este-capuxedtulo}

\begin{Shaded}
\begin{Highlighting}[]
\NormalTok{pacman}\SpecialCharTok{::}\FunctionTok{p\_load}\NormalTok{(dplyr, }
\NormalTok{               e1071, }
\NormalTok{               ggplot2, }
\NormalTok{               knitr, }
\NormalTok{               readxl)}
\end{Highlighting}
\end{Shaded}

\section{Distribuições populacional e
amostral}\label{distribuiuxe7uxf5es-populacional-e-amostral}

Métricas como a média, a mediana e o desvio padrão são medidas numéricas
de resumo. Quando calculadas a partir de dados de uma amostra são
denominadas \textbf{estatísticas amostrais}. Por outro lado, as mesmas
medidas numéricas de resumo calculadas para dados populacionais são
chamadas de \textbf{parâmetros populacionais}.

Um parâmetro populacional é sempre uma constante, enquanto uma
estatística de amostra é sempre uma variável aleatória. Como cada
variável aleatória deve possuir uma distribuição de probabilidade, cada
estatística de amostra possui uma distribuição de probabilidade. A
distribuição de probabilidade de uma estatística de amostra é mais
comumente chamada de \textbf{distribuição amostral}. Os conceitos
abordados neste capítulo são a base da \textbf{estatística inferencial}.

\subsection{Distribuição
populacional}\label{distribuiuxe7uxe3o-populacional}

A distribuição populacional é a distribuição de probabilidade derivada
das informações sobre todos os elementos de uma \textbf{população}.

Para fins de raciocínio didático, o conjunto de dados de 1368
observações de puérperas e recém-nascidos da Maternidade-escola do
Hospital Geral de Caxias do Sul, RS, será considerado uma população. O
gráfico da Figura~\ref{fig-histbxp}, da Seção~\ref{sec-assimetria},
mostra a distribuição da \texttt{altura} das puérperas dessa
`população'. Os parâmetros (\(\mu\) e \(\sigma\)) dessa
``\emph{população}'' são:

\begin{Shaded}
\begin{Highlighting}[]
\NormalTok{dados }\OtherTok{\textless{}{-}} \FunctionTok{read\_excel}\NormalTok{(}\StringTok{"dados/dadosMater.xlsx"}\NormalTok{) }\SpecialCharTok{\%\textgreater{}\%} 
  \FunctionTok{select}\NormalTok{(altura)}

\NormalTok{ media }\OtherTok{=} \FunctionTok{mean}\NormalTok{(dados}\SpecialCharTok{$}\NormalTok{altura, }\AttributeTok{na.rm =}\ConstantTok{TRUE}\NormalTok{)}
 \FunctionTok{round}\NormalTok{(media, }\DecValTok{3}\NormalTok{)}
\end{Highlighting}
\end{Shaded}

\begin{verbatim}
[1] 1.598
\end{verbatim}

\begin{Shaded}
\begin{Highlighting}[]
\NormalTok{ dp }\OtherTok{=} \FunctionTok{sd}\NormalTok{(dados}\SpecialCharTok{$}\NormalTok{altura, }\AttributeTok{na.rm =}\ConstantTok{TRUE}\NormalTok{)}
 \FunctionTok{round}\NormalTok{(dp, }\DecValTok{3}\NormalTok{)}
\end{Highlighting}
\end{Shaded}

\begin{verbatim}
[1] 0.065
\end{verbatim}

\subsection{Distribuição amostral}\label{sec-dam}

Conforme mencionado no início deste capítulo, o valor de um parâmetro da
população é sempre constante. Por exemplo, para qualquer conjunto de
dados populacionais, há apenas um valor para a média populacional,
\(\mu\).

No entanto, não se pode dizer o mesmo sobre a média amostral. Amostras
diferentes do mesmo tamanho, retiradas da mesma população, produzem
valores diferentes da média amostral, \(\bar{x}\). O valor da média
amostral, para qualquer amostra, dependerá dos elementos incluídos nessa
amostra. Em decorrência, a média amostral é uma variável aleatória.
Portanto, como outras variáveis aleatórias, a média amostral possui uma
distribuição de probabilidade, que é mais comumente chamada de
\textbf{distribuição amostral da média}.

Outras estatísticas de amostra, como mediana, moda e desvio padrão,
também possuem distribuições amostrais. Em geral, a distribuição de
probabilidades de uma amostra é denominada de \textbf{distribuição
amostral}.

Usar a variável \texttt{altura} das puérperas da Maternidade do HGCS
como a população de interesse é apenas uma estratégia didática.
Raramente, na vida real, é possível obter dados da população inteira.
Reunir essa informação costuma ser muito custoso ou impossível. Por essa
razão, a prática é selecionar apenas uma amostra da população e a usar
para compreender as suas características.

A função \texttt{slice\_sample()} do pacote \texttt{dplyr}extrairá uma
amostra \footnote{Para que a cada nova amostragem retorne o mesmo
  conjunto de dados, é usado a função \texttt{set.seed()}(veja
  Seção~\ref{sec-rnorm})} de n = 30 da população. As funções
\texttt{mean()} e \texttt{sd()} calcularão a média e o desvio padrão,
repectivamente:

\begin{Shaded}
\begin{Highlighting}[]
\FunctionTok{set.seed}\NormalTok{(}\DecValTok{234}\NormalTok{)}
\NormalTok{amostra1 }\OtherTok{\textless{}{-}}\NormalTok{ dados }\SpecialCharTok{\%\textgreater{}\%} 
\NormalTok{  dplyr}\SpecialCharTok{::}\FunctionTok{slice\_sample}\NormalTok{(}\AttributeTok{n =} \DecValTok{30}\NormalTok{)}

\NormalTok{media1 }\OtherTok{\textless{}{-}} \FunctionTok{mean}\NormalTok{(amostra1}\SpecialCharTok{$}\NormalTok{altura, }\AttributeTok{na.rm =}\ConstantTok{TRUE}\NormalTok{)}
\NormalTok{dp1 }\OtherTok{\textless{}{-}}  \FunctionTok{sd}\NormalTok{(amostra1}\SpecialCharTok{$}\NormalTok{altura, }\AttributeTok{na.rm =}\ConstantTok{TRUE}\NormalTok{)}
\FunctionTok{print}\NormalTok{(}\FunctionTok{c}\NormalTok{(media1, dp1))}
\end{Highlighting}
\end{Shaded}

\begin{verbatim}
[1] 1.59266667 0.06073875
\end{verbatim}

Se este processo for repetido várias vezes, a cada amostra aleatória
\footnote{Observe que o número da ``semente'' foi modificado para 236.
  Poderia ser qualquer outro número, isto garante que o sorteio seja
  diferente, mas ainda reprodutível.}, serão gerados médias e desvios
padrão diferentes.

\begin{Shaded}
\begin{Highlighting}[]
\FunctionTok{set.seed}\NormalTok{(}\DecValTok{236}\NormalTok{)}
\NormalTok{amostra2 }\OtherTok{\textless{}{-}}\NormalTok{ dados }\SpecialCharTok{\%\textgreater{}\%} 
\NormalTok{  dplyr}\SpecialCharTok{::}\FunctionTok{slice\_sample}\NormalTok{(}\AttributeTok{n =} \DecValTok{30}\NormalTok{)}

\NormalTok{media2 }\OtherTok{\textless{}{-}} \FunctionTok{mean}\NormalTok{(amostra2}\SpecialCharTok{$}\NormalTok{altura, }\AttributeTok{na.rm =}\ConstantTok{TRUE}\NormalTok{)}
\NormalTok{dp2 }\OtherTok{\textless{}{-}}  \FunctionTok{sd}\NormalTok{(amostra2}\SpecialCharTok{$}\NormalTok{altura, }\AttributeTok{na.rm =}\ConstantTok{TRUE}\NormalTok{)}
\FunctionTok{print}\NormalTok{(}\FunctionTok{c}\NormalTok{(media2, dp2))}
\end{Highlighting}
\end{Shaded}

\begin{verbatim}
[1] 1.60633333 0.06960397
\end{verbatim}

À medida que o número de amostras possíveis forem aumentando, elas
constituem uma distribuição cuja média, média das médias,
\(\bar{x}_{\bar{x}}\), é igual a média populacional, \(\mu\). Essa
distribuição, no caso da média, recebe o nome de \textbf{distribuição
amostral das médias}.

Agora, para exemplificar este conceito, serão geradas 5000 amostras e
calculada a média de cada uma das amostras de n = 30 que constituirão a
distribuição, mostrada no gráfico da Figura~\ref{fig-dam}.

\begin{Shaded}
\begin{Highlighting}[]
\CommentTok{\# extraindo 5000 amostras}
\NormalTok{amostras5000 }\OtherTok{\textless{}{-}} \FunctionTok{rep}\NormalTok{ (}\DecValTok{0}\NormalTok{, }\DecValTok{5000}\NormalTok{)}
\ControlFlowTok{for}\NormalTok{ (i }\ControlFlowTok{in} \DecValTok{1}\SpecialCharTok{:}\DecValTok{5000}\NormalTok{) \{}
\NormalTok{  amostra }\OtherTok{\textless{}{-}}\NormalTok{ dados }\SpecialCharTok{\%\textgreater{}\%}\NormalTok{ dplyr}\SpecialCharTok{::}\FunctionTok{slice\_sample}\NormalTok{ (}\AttributeTok{n =} \DecValTok{30}\NormalTok{) }
\NormalTok{  amostras5000 [i] }\OtherTok{\textless{}{-}} \FunctionTok{mean}\NormalTok{(amostra}\SpecialCharTok{$}\NormalTok{altura)}
\NormalTok{\}}
\end{Highlighting}
\end{Shaded}

Media e desvio padrão das 5000 amostras:

\begin{Shaded}
\begin{Highlighting}[]
\NormalTok{mu }\OtherTok{\textless{}{-}} \FunctionTok{round}\NormalTok{ (}\FunctionTok{mean}\NormalTok{ (amostras5000), }\AttributeTok{digits =} \DecValTok{3}\NormalTok{)}
\NormalTok{sigma }\OtherTok{\textless{}{-}} \FunctionTok{round}\NormalTok{ (}\FunctionTok{sd}\NormalTok{ (amostras5000), }\AttributeTok{digits =} \DecValTok{3}\NormalTok{)}
\FunctionTok{print}\NormalTok{(}\FunctionTok{c}\NormalTok{(mu, sigma))}
\end{Highlighting}
\end{Shaded}

\begin{verbatim}
[1] 1.598 0.012
\end{verbatim}

\begin{figure}

\centering{

\includegraphics[width=0.7\linewidth,height=0.7\textheight]{11-distAmostrais_files/figure-pdf/fig-dam-1.pdf}

}

\caption{\label{fig-dam}Distribuição amostral das médias de 5000
amostras de n = 30}

\end{figure}%

Se a média, \(\bar{x}_{\bar{x}}\), dessas 5000 amostras de n = 30, for
comparada com a média populacional, \(\mu\), observa-se que até 3
dígitos decimais não há uma diferença. Entretanto, o desvio padrão é bem
menor (0.012) que o da população (0.065).

\section{Erros amostrais e não
amostrais}\label{erros-amostrais-e-nuxe3o-amostrais}

Amostras diferentes selecionadas da mesma população darão resultados
diferentes porque contêm elementos diferentes. Isso é evidente nas
medias das amostra1 e amostra2, 1.593m e 1.606m, respectivamente,
comparadas com a média da população igual a 1.598m .

\begin{Shaded}
\begin{Highlighting}[]
\NormalTok{erro1 }\OtherTok{\textless{}{-}} \FunctionTok{abs}\NormalTok{(}\FunctionTok{mean}\NormalTok{(amostra1}\SpecialCharTok{$}\NormalTok{altura, }\AttributeTok{na.rm =}\ConstantTok{TRUE}\NormalTok{) }\SpecialCharTok{{-}} \FunctionTok{mean}\NormalTok{(dados}\SpecialCharTok{$}\NormalTok{altura, }\AttributeTok{na.rm =}\ConstantTok{TRUE}\NormalTok{))}
\NormalTok{erro2 }\OtherTok{\textless{}{-}} \FunctionTok{abs}\NormalTok{(}\FunctionTok{mean}\NormalTok{(amostra2}\SpecialCharTok{$}\NormalTok{altura, }\AttributeTok{na.rm =}\ConstantTok{TRUE}\NormalTok{) }\SpecialCharTok{{-}} \FunctionTok{mean}\NormalTok{(dados}\SpecialCharTok{$}\NormalTok{altura, }\AttributeTok{na.rm =}\ConstantTok{TRUE}\NormalTok{))}
\FunctionTok{print}\NormalTok{(}\FunctionTok{c}\NormalTok{(erro1, erro2), }\AttributeTok{digits =} \DecValTok{2}\NormalTok{)}
\end{Highlighting}
\end{Shaded}

\begin{verbatim}
[1] 0.0053 0.0084
\end{verbatim}

Se outras amostras forem extraídas, o resultado obtido de qualquer
amostra geralmente será diferente do resultado obtido da população
correspondente. A diferença entre o valor de uma estatística amostral
obtida de uma amostra e o valor do parâmetro populacional
correspondente, é chamado de \textbf{erro amostral}. Observe que essa
diferença representa o erro amostral apenas se a amostra for aleatória e
não houver nenhum erro não amostral. Caso contrário, apenas uma parte
dessa diferença será devido ao erro amostral.

\[
 erro \quad amostral = \bar{x}_{i} - \mu  
\]

É importante lembrar que o erro amostral ocorre devido ao acaso. Não é
possível evitar o erro amostral. É possível limitar o seu valor através
da seleção de uma amostra adequada. Os erros que ocorrem por outros
motivos, como erros cometidos durante a coleta, registro e tabulação dos
dados, são chamados de \textbf{erros não amostrais}. Esses erros
ocorrem, em geral, por causa de erros humanos e não por acaso.

\section{Média e desvio padrão da
média}\label{muxe9dia-e-desvio-padruxe3o-da-muxe9dia}

A média e o desvio padrão calculados para a distribuição amostral da
média são chamados de \emph{média (}\(\mu_{\bar{x}}\)) e \emph{desvio
padrão (}\(\sigma_{\bar{x}}\)) da média. Na verdade, a média e o desvio
padrão da média são, respectivamente, a média e o desvio padrão das
médias de todas as amostras do mesmo tamanho selecionadas de uma
população. O desvio padrão da média é, comumente, chamado de
\textbf{erro padrão da média} (\(\sigma_{\bar{x}}\)).

A média amostral, \(\bar{x}\), é chamada de \emph{estimador} da média da
população, \(\mu\). Quando o valor esperado (ou média) de uma
estatística amostral é igual ao valor do parâmetro populacional
correspondente, essa estatística amostral é considerada um estimador não
enviesado, consistente.\\
Para a média amostral \(\bar{x}\), \(\mu_{\bar{x}} = \mu\). Logo,
\(\bar{x}\), é um estimador imparcial de \(\mu\). Esta é uma propriedade
muito importante que um estimador deve possuir. No entanto, o desvio
padrão da média, \(\sigma_{\bar{x}}\), não é igual ao desvio padrão,
\(\sigma\), da distribuição populacional (a menos que n = 1). O desvio
padrão da média amostral é igual ao desvio padrão da população dividido
pela raiz quadrada do tamanho amostral:

\[
\sigma_{\bar{x}} = \frac {\sigma}{\sqrt{n}}
\]

A dispersão da distribuição amostral da média é menor do que dispersão
da distribuição populacional correspondente, como mostrado acima. Em
outras palavras, \(\sigma_{\bar{x}} < \sigma\). Isso é visível na
fórmula do \(\sigma_{\bar{x}}\) . Quando \emph{n} é maior que 1, o que
geralmente é verdadeiro, o denominador em \(\frac {\sigma}{\sqrt{n}}\) é
maior que 1. Desta forma, \(\sigma_{\bar{x}}\) é menor que \(\sigma\). O
desvio padrão da distribuição amostral da média diminui à medida que o
tamanho amostral aumenta.

Sempre que o \emph{n} for grande, em geral \textgreater{} 30 (98), pode
ser assumido que a distribuição será uma curva normal e que o desvio
padrão da amostra (\emph{s}) é um estimador não enviesado do desvio
padrão populacional (\(\sigma\)). Então, o erro padrão da média
(\(\sigma_{\bar{x}}\)) pode ser estimado pelo \(EP_{\bar{x}}\):

\[
EP_{\bar{x}} = \frac {s}{\sqrt{n}}
\]

\section{Teorema do Limite Central (TCL)}\label{sec-tcl}

Na maioria das vezes, a população da qual as amostras são extraídas não
é normalmente distribuída. Em tais casos, a forma da distribuição
amostral de X é inferida de um teorema muito importante chamado
\emph{teorema do limite central}. De acordo com este teorema para um
grande tamanho de amostra (\textgreater{} 30), a distribuição amostral
da média é aproximadamente normal, independentemente da forma da
distribuição da população (98). Esta aproximação tornar-se-á mais
acurada à medida que aumenta o tamanho amostral:

\begin{itemize}
\tightlist
\item
  a média da distribuição amostral, \(\mu_{\bar{x}}\), é igual a média
  populacional, \(\mu\);
\item
  desvio padrão da distribuição amostral, \(\sigma_{\bar{x}}\), é igual
  a \(\frac {\sigma}{\sqrt{n}}\);
\item
  o erro padrão da média, \(\sigma_{\bar{x}}\), é sempre menor
  (Figura~\ref{fig-standard}) que o desvio padrão populacional,
  \(\sigma\).
\end{itemize}

\begin{figure}

\centering{

\includegraphics[width=0.7\linewidth,height=\textheight,keepaspectratio]{index_files/mediabag/I2CxksB.png}

}

\caption{\label{fig-standard}Erro padrão versus desvio padrão}

\end{figure}%

\subsection{Exemplo com uma variável
assimétrica}\label{exemplo-com-uma-variuxe1vel-assimuxe9trica}

Como exemplo, será explorada a variável \texttt{renda}, do conjunto de
dados \texttt{dadosMater.xlsx}, que representa a renda familiar em
salários mínimos (\emph{sm}). Como foi feito anteriormente, suponha que
essa variável seja a ``população'' de estudo. Ela tem as seguintes
medidas resumidoras e de assimetria:

\begin{Shaded}
\begin{Highlighting}[]
\NormalTok{dados }\OtherTok{\textless{}{-}}\NormalTok{ readxl}\SpecialCharTok{::}\FunctionTok{read\_excel}\NormalTok{(}\StringTok{"dados/dadosMater.xlsx"}\NormalTok{)}
\NormalTok{resumo }\OtherTok{\textless{}{-}}\NormalTok{ dados }\SpecialCharTok{\%\textgreater{}\%} 
  \FunctionTok{select}\NormalTok{ (renda) }\SpecialCharTok{\%\textgreater{}\%} 
\NormalTok{  dplyr}\SpecialCharTok{::}\FunctionTok{summarise}\NormalTok{ (}\AttributeTok{media.sm =} \FunctionTok{mean}\NormalTok{ (dados}\SpecialCharTok{$}\NormalTok{renda, }\AttributeTok{na.rm =} \ConstantTok{TRUE}\NormalTok{),}
                    \AttributeTok{dp.sm =} \FunctionTok{sd}\NormalTok{(dados}\SpecialCharTok{$}\NormalTok{renda, }\AttributeTok{na.rm =} \ConstantTok{TRUE}\NormalTok{),}
                    \AttributeTok{mediana.sm =} \FunctionTok{median}\NormalTok{(dados}\SpecialCharTok{$}\NormalTok{renda, }\AttributeTok{na.rm =} \ConstantTok{TRUE}\NormalTok{),}
                    \AttributeTok{assimetria =}\NormalTok{ e1071}\SpecialCharTok{::}\FunctionTok{skewness}\NormalTok{(dados}\SpecialCharTok{$}\NormalTok{renda),}
                    \AttributeTok{curtose =}\NormalTok{ e1071}\SpecialCharTok{::}\FunctionTok{kurtosis}\NormalTok{(dados}\SpecialCharTok{$}\NormalTok{renda))}
\NormalTok{resumo}
\end{Highlighting}
\end{Shaded}

\begin{verbatim}
# A tibble: 1 x 5
  media.sm dp.sm mediana.sm assimetria curtose
     <dbl> <dbl>      <dbl>      <dbl>   <dbl>
1     2.22  1.23       1.92       2.22    8.21
\end{verbatim}

O desvio padrão é grande em relação à média, com um coeficiente de
variação de 55.1\% e uma mediana \textless{} média. Estas métricas junto
com os coeficientes de assimetria e curtose apontam para a assimetria
positiva da variável \texttt{renda}. O gráfico da
Figura~\ref{fig-skewpos} confirma esta afirmação:

\begin{figure}

\centering{

\includegraphics[width=0.7\linewidth,height=0.7\textheight]{11-distAmostrais_files/figure-pdf/fig-skewpos-1.pdf}

}

\caption{\label{fig-skewpos}Distribuição assimétrica positiva}

\end{figure}%

Os valores da média e do desvio padrão calculados para a distribuição de
probabilidade dessa população fornecem os valores dos parâmetros
populacionais \(\mu\) e \(\sigma\). Esses valores são \(\mu\) =2.22
\emph{sm} \footnote{Salários mínimos} e \(\sigma\) =1.23 \emph{sm}.\\
Se extrairmos múltiplas amostras dessa população, observa-se a
modificação do formato da distribuição à medida que aumenta o tamanho
amostral, se aproximando progressivamente do modelo normal, com um
número grande de amostras.

\ul{Extração de múltiplas amostras(1000)}

\begin{Shaded}
\begin{Highlighting}[]
\NormalTok{amostras1000 }\OtherTok{\textless{}{-}} \FunctionTok{rep}\NormalTok{ (}\DecValTok{0}\NormalTok{, }\DecValTok{1000}\NormalTok{)}
\ControlFlowTok{for}\NormalTok{ (i }\ControlFlowTok{in} \DecValTok{1}\SpecialCharTok{:}\DecValTok{1000}\NormalTok{) \{}
\NormalTok{  amostra.sm }\OtherTok{\textless{}{-}} \FunctionTok{sample}\NormalTok{ (dados}\SpecialCharTok{$}\NormalTok{renda, }\DecValTok{30}\NormalTok{) }
\NormalTok{  amostras1000 [i] }\OtherTok{\textless{}{-}} \FunctionTok{mean}\NormalTok{(amostra.sm)}
\NormalTok{\}}
\end{Highlighting}
\end{Shaded}

\ul{Media e desvio padrão das 1000 amostras}

\begin{Shaded}
\begin{Highlighting}[]
\NormalTok{mu }\OtherTok{\textless{}{-}} \FunctionTok{round}\NormalTok{ (}\FunctionTok{mean}\NormalTok{ (amostras1000), }\AttributeTok{digits =} \DecValTok{3}\NormalTok{)}
\NormalTok{sigma }\OtherTok{\textless{}{-}} \FunctionTok{round}\NormalTok{ (}\FunctionTok{sd}\NormalTok{ (amostras1000), }\AttributeTok{digits =} \DecValTok{3}\NormalTok{)}
\NormalTok{md }\OtherTok{\textless{}{-}} \FunctionTok{round}\NormalTok{ (}\FunctionTok{median}\NormalTok{(amostras1000), }\AttributeTok{digits =} \DecValTok{3}\NormalTok{)}
\FunctionTok{print}\NormalTok{(}\FunctionTok{c}\NormalTok{(mu, sigma, md))}
\end{Highlighting}
\end{Shaded}

\begin{verbatim}
[1] 2.231 0.236 2.224
\end{verbatim}

\ul{Assimetria e curtose}

\begin{Shaded}
\begin{Highlighting}[]
\NormalTok{b1 }\OtherTok{\textless{}{-}}\NormalTok{ e1071}\SpecialCharTok{::}\FunctionTok{skewness}\NormalTok{(amostras1000)}
\NormalTok{b2 }\OtherTok{\textless{}{-}}\NormalTok{ e1071}\SpecialCharTok{::}\FunctionTok{kurtosis}\NormalTok{(amostras1000)}
\FunctionTok{print}\NormalTok{(}\FunctionTok{c}\NormalTok{(b1, b2))}
\end{Highlighting}
\end{Shaded}

\begin{verbatim}
[1] 0.4357889 0.2241975
\end{verbatim}

\begin{figure}

\centering{

\includegraphics[width=0.7\linewidth,height=0.7\textheight]{11-distAmostrais_files/figure-pdf/fig-semskew-1.pdf}

}

\caption{\label{fig-semskew}Distribuição praticamente normal}

\end{figure}%

Ou seja, extraindo-se 1000 amostras de n = 30 e calculando as mesmas
métricas anteriores, observa-se que, embora a distribuição populacional
original seja assimétrica, a distribuição amostral da média se aproxima
bastante da distribuição gaussiana (Figura~\ref{fig-semskew}).

\begin{tcolorbox}[enhanced jigsaw, bottomrule=.15mm, opacitybacktitle=0.6, colframe=quarto-callout-important-color-frame, arc=.35mm, coltitle=black, toptitle=1mm, colback=white, colbacktitle=quarto-callout-important-color!10!white, breakable, bottomtitle=1mm, rightrule=.15mm, titlerule=0mm, toprule=.15mm, opacityback=0, leftrule=.75mm, left=2mm, title=\textcolor{quarto-callout-important-color}{\faExclamation}\hspace{0.5em}{Importância do Teorema do Limite Central}]

O TCL é um dos pilares da inferência estatística. Entender a sua
importância é como destrancar a porta para aplicar testes, construir
intervalos de confiança e fazer previsões com segurança, mesmo quando os
dados parecem confusos.

O TCL afirma que:

\begin{quote}
A distribuição das médias amostrais tende a ser normal, mesmo que a
população original não seja, desde que o tamanho da amostra seja
suficientemente grande.
\end{quote}

\end{tcolorbox}

\section{Proporções populacional e amostral}\label{sec-popamostra}

O conceito de proporção é o mesmo que o conceito de frequência relativa
e o conceito de probabilidade de sucesso em um experimento binomial,
discutidos anteriormente, na distribuição binomial.

A frequência relativa de uma categoria ou classe dá a proporção da
amostra ou população que pertence a essa categoria ou classe. Da mesma
forma, a probabilidade de sucesso em um experimento binomial representa
a proporção da amostra ou população que possui uma determinada
característica.

A proporção populacional, representada por \emph{p}, é obtida
considerando a razão entre o número de elementos em uma população com
uma característica específica e o número total de elementos na
população. A proporção amostral, denotada por \(\hat{p}\) (pronuncia-se
p-chapéu), fornece uma proporção semelhante para uma amostra.

\[
p = \frac{X}{N} \quad e \quad \hat{p}= \frac{x}{n}
\] onde,

\begin{itemize}
\tightlist
\item
  \emph{N} \(\to\) número total de elementos em uma população
\item
  \emph{n} \(\to\) número total de elementos em uma amostra
\item
  \emph{X} \(\to\) número de elementos na população que possui
  determinada característica
\item
  \emph{x} \(\to\) número de elementos na amostra que possui determinada
  característica
\end{itemize}

Como no caso da média, a diferença entre a proporção amostral e a
proporção populacional correspondente, determina o \emph{erro amostral},
assumindo que a amostra é aleatória e nenhum erro não amostral foi
cometido. Ou seja,

\[
erro \quad amostral = \hat{p} - p
\]

A \textbf{distribuição amostral de uma proporção} é a distribuição das
proporções de todas as amostras possíveis de tamanho \emph{n} retiradas
de uma população.

De acordo com o Teorema Central do Limite, para amostras suficientemente
grandes, a distribuição de \(\hat{p}\) se aproxima de uma distribuição
normal, mesmo que os dados originais sejam binomiais (sucesso/fracasso),
desde que: \(np \geq 5\) e \(n(1 - p) \geq 5\).

Assim,

\[E(\hat{p})=\mu_\hat{p}\]

\[Var(\hat{p})=\sigma^2_\hat{p}=\frac{\hat{p}(1-\hat{p})}{n}\] Logo,

\[E(\hat{p})=\sqrt{\frac{\hat{p}(1-\hat{p})}{n}}\]

Dessa forma, a distribuição amostral de \(\hat{p}\) será:

\[
\hat{p} \sim N(\hat{p}, \frac{\hat{p}(1-\hat{p})}{n})
\]

Quando não conhecemos a proporção populacional \emph{p}, pode-se usar
\(\hat{p}\) como estimativa dessa proporção, desde que as condições
acima sejam satisfeitas.

Dessa forma, pode-se calcular probabilidades aproximadas por uma
distribuição normal com média \(μ = n \times p\) e
\(σ = \sqrt{(n×p(1-p))}\) (veja também Seção~\ref{sec-mudpbinom}).

Considerando a ``população'', usada neste capítulo, o conjunto de dados
\texttt{dadosMater.xlsx}, será verificado a proporção de mulheres
fumantes. Inicialmente, a variável \texttt{fumo}, que está como variável
numérica, será transformada em fator:

\begin{Shaded}
\begin{Highlighting}[]
\NormalTok{dados }\OtherTok{\textless{}{-}} \FunctionTok{read\_excel}\NormalTok{(}\StringTok{"dados/dadosMater.xlsx"}\NormalTok{) }\SpecialCharTok{\%\textgreater{}\%} 
  \FunctionTok{select}\NormalTok{(fumo) }\SpecialCharTok{\%\textgreater{}\%} 
  \FunctionTok{mutate}\NormalTok{(}\AttributeTok{fumo =} \FunctionTok{factor}\NormalTok{(fumo, }
                       \AttributeTok{levels =} \FunctionTok{c}\NormalTok{(}\DecValTok{1}\NormalTok{,}\DecValTok{2}\NormalTok{), }
                       \AttributeTok{labels =} \FunctionTok{c}\NormalTok{(}\StringTok{"Fumante"}\NormalTok{, }\StringTok{"Não fumante"}\NormalTok{)))}
\end{Highlighting}
\end{Shaded}

A proporção de fumantes, frequência relativa (\emph{fr}) é:

\begin{Shaded}
\begin{Highlighting}[]
\NormalTok{fumo }\OtherTok{\textless{}{-}} \FunctionTok{with}\NormalTok{(dados, }\FunctionTok{table}\NormalTok{(fumo))}
\NormalTok{fr.fumo }\OtherTok{\textless{}{-}} \FunctionTok{prop.table}\NormalTok{(fumo)}
\NormalTok{fr.fumo}
\end{Highlighting}
\end{Shaded}

\begin{verbatim}
fumo
    Fumante Não fumante 
  0.2200292   0.7799708 
\end{verbatim}

A saída retorna que a proporção de fumantes entre as mulheres desse
arquivo é 0.22. Esta será considerada a proporção \emph{p} da
`população'. Agora, imagine que esse resultado fosse desconhecido.
Então, para saber a qual a proporção de fumantes dessa `população',
seria necessário extrair uma amostra adequada. Foi selecionada uma
amostra de n = 100 da `população' alvo:

\begin{Shaded}
\begin{Highlighting}[]
 \FunctionTok{set.seed}\NormalTok{(}\DecValTok{134}\NormalTok{)}
\NormalTok{ amostra.fumo }\OtherTok{\textless{}{-}}\NormalTok{ dados }\SpecialCharTok{\%\textgreater{}\%}\NormalTok{ dplyr}\SpecialCharTok{::}\FunctionTok{slice\_sample}\NormalTok{(}\AttributeTok{n =} \DecValTok{100}\NormalTok{)}
\end{Highlighting}
\end{Shaded}

Usando a \texttt{amostra.fumo}, calcula-se a proporção de fumantes:

\begin{Shaded}
\begin{Highlighting}[]
\NormalTok{ tabagismo }\OtherTok{\textless{}{-}} \FunctionTok{with}\NormalTok{(amostra.fumo, }\FunctionTok{table}\NormalTok{(fumo))}
\NormalTok{ fr }\OtherTok{\textless{}{-}} \FunctionTok{prop.table}\NormalTok{(tabagismo)}
\NormalTok{ fp }\OtherTok{\textless{}{-}}\NormalTok{ fr}\SpecialCharTok{*}\DecValTok{100}

\NormalTok{ tab.fumo }\OtherTok{\textless{}{-}} \FunctionTok{cbind}\NormalTok{(}\AttributeTok{n =}\NormalTok{ tabagismo,}
                   \AttributeTok{fr =} \FunctionTok{round}\NormalTok{(fr, }\DecValTok{2}\NormalTok{),}
                   \AttributeTok{fp =} \FunctionTok{round}\NormalTok{(fp, }\DecValTok{2}\NormalTok{))}
\NormalTok{ tab.fumo}
\end{Highlighting}
\end{Shaded}

\begin{verbatim}
             n  fr fp
Fumante     20 0.2 20
Não fumante 80 0.8 80
\end{verbatim}

A proporção de uma amostra é uma variável aleatória: varia de amostra
para amostra de uma forma que não pode ser prevista com certeza. O
Teorema Central do Limite se aplica em proporções. À medida que novas
amostras forem extraídas, o valor da proporção amostral \(\hat{p}\) se
aproxima da proporção populacional \emph{p}. Na ``população'' \emph{p} =
0,22; na amostra de n = 100, \(\hat{p}\) = 0.2. Para amostras grandes, a
proporção amostral tem distribuição aproximadamente normal com as
seguinte características mencionadas acima em relação a \(\mu_\hat{p}\)
e \(\sigma_\hat{p}\).\\
Como verificar se uma amostra é grande?

Uma amostra é grande se o intervalo

\[
[\hat{p}-3 \times \sigma_\hat{p} , \quad \hat{p}-3 \times \sigma_\hat{p}]
\]

estiver totalmente dentro do intervalo {[}0,1{]}.

Na prática, \emph{p} não é conhecido, portanto, \(\sigma_\hat{p}\)
também não é. Nesse caso, para verificar se a amostra é suficientemente
grande, substitui-se o valor de \emph{p} pelo valor conhecido de
\(\hat{p}\). Isso significa verificar se o intervalo

\[
\hat{p}-3\times\sqrt{\frac{\hat{p}(1-\hat{p})}{n}},\quad \hat{p}+3\times\sqrt{\frac{\hat{p}(1-\hat{p})}{n}}
\]

encontra-se totalmente dentro do intervalo {[}0,1{]}.

Transportando os dados da amostra de gestantes, para a fórmula e usando
o \emph{R} para o cálculo, tem-se:

\begin{Shaded}
\begin{Highlighting}[]
\NormalTok{p.chapeu }\OtherTok{\textless{}{-}}\NormalTok{ tab.fumo[}\DecValTok{1}\NormalTok{,}\DecValTok{2}\NormalTok{]}
\NormalTok{n }\OtherTok{\textless{}{-}}\NormalTok{ tab.fumo[}\DecValTok{1}\NormalTok{,}\DecValTok{1}\NormalTok{] }\SpecialCharTok{+}\NormalTok{ tab.fumo[}\DecValTok{2}\NormalTok{,}\DecValTok{1}\NormalTok{]}

\NormalTok{li }\OtherTok{\textless{}{-}}\NormalTok{ p.chapeu }\SpecialCharTok{{-}} \DecValTok{3}\SpecialCharTok{*}\FunctionTok{sqrt}\NormalTok{((p.chapeu}\SpecialCharTok{*}\NormalTok{(}\DecValTok{1}\SpecialCharTok{{-}}\NormalTok{p.chapeu))}\SpecialCharTok{/}\NormalTok{n)}
\NormalTok{ls }\OtherTok{\textless{}{-}}\NormalTok{ p.chapeu }\SpecialCharTok{+} \DecValTok{3}\SpecialCharTok{*}\FunctionTok{sqrt}\NormalTok{((p.chapeu}\SpecialCharTok{*}\NormalTok{(}\DecValTok{1}\SpecialCharTok{{-}}\NormalTok{p.chapeu))}\SpecialCharTok{/}\NormalTok{n)}
\FunctionTok{print}\NormalTok{(}\FunctionTok{c}\NormalTok{(li, ls), }\AttributeTok{digits =} \DecValTok{3}\NormalTok{)}
\end{Highlighting}
\end{Shaded}

\begin{verbatim}
[1] 0.08 0.32
\end{verbatim}

Como os limites ficam no intervalo {[}0, 1{]}, chega-se à conclusão de
que a amostra de \emph{n} = 100 é aceitável para estimar a proporção
populacional.

\begin{tcolorbox}[enhanced jigsaw, bottomrule=.15mm, opacitybacktitle=0.6, colframe=quarto-callout-tip-color-frame, arc=.35mm, coltitle=black, toptitle=1mm, colback=white, colbacktitle=quarto-callout-tip-color!10!white, breakable, bottomtitle=1mm, rightrule=.15mm, titlerule=0mm, toprule=.15mm, opacityback=0, leftrule=.75mm, left=2mm, title=\textcolor{quarto-callout-tip-color}{\faLightbulb}\hspace{0.5em}{Exercício}]

Uma amostra com tamanho n = 40 é suficiente?

\end{tcolorbox}

\ul{Resposta}:

\begin{Shaded}
\begin{Highlighting}[]
\NormalTok{n }\OtherTok{=} \DecValTok{40}
\NormalTok{li }\OtherTok{\textless{}{-}}\NormalTok{ p.chapeu }\SpecialCharTok{{-}} \DecValTok{3}\SpecialCharTok{*}\FunctionTok{sqrt}\NormalTok{((p.chapeu}\SpecialCharTok{*}\NormalTok{(}\DecValTok{1}\SpecialCharTok{{-}}\NormalTok{p.chapeu))}\SpecialCharTok{/}\NormalTok{n) }
\NormalTok{ls }\OtherTok{\textless{}{-}}\NormalTok{ p.chapeu }\SpecialCharTok{+} \DecValTok{3}\SpecialCharTok{*}\FunctionTok{sqrt}\NormalTok{((p.chapeu}\SpecialCharTok{*}\NormalTok{(}\DecValTok{1}\SpecialCharTok{{-}}\NormalTok{p.chapeu))}\SpecialCharTok{/}\NormalTok{n)}
\FunctionTok{print}\NormalTok{(}\FunctionTok{c}\NormalTok{(li, ls), }\AttributeTok{digits =} \DecValTok{3}\NormalTok{)}
\end{Highlighting}
\end{Shaded}

\begin{verbatim}
[1] 0.0103 0.3897
\end{verbatim}

Sim, é aceitável uma amostra de n = 40.

\part{Parte V - Inferência Estatística}

\chapter{Estimação}\label{sec-estimacao}

\section{Pacotes necessários neste
capítulo}\label{pacotes-necessuxe1rios-neste-capuxedtulo-3}

\begin{Shaded}
\begin{Highlighting}[]
\NormalTok{pacman}\SpecialCharTok{::}\FunctionTok{p\_load}\NormalTok{(DescTools, }
\NormalTok{               dplyr,}
\NormalTok{               ggplot2, }
\NormalTok{               flextable,}
\NormalTok{               knitr,}
\NormalTok{               readxl, }
\NormalTok{               Rmisc, }
\NormalTok{               tidyr)}
\end{Highlighting}
\end{Shaded}

\section{Dados usados neste capítulo}\label{sec-dadoscap12}

Os dados deste capítulo são provenientes do conjunto de dados
\texttt{dadosMater.xlsx} (Seção~\ref{sec-dadosMater}), considerando
apenas as variáveis \texttt{altura} e \texttt{pesoRN}:

\begin{Shaded}
\begin{Highlighting}[]
\NormalTok{dados }\OtherTok{\textless{}{-}} \FunctionTok{read\_excel}\NormalTok{(}\StringTok{"dados/dadosMater.xlsx"}\NormalTok{) }\SpecialCharTok{\%\textgreater{}\%} 
  \FunctionTok{select}\NormalTok{(altura, fumo, ig, pesoRN) }\SpecialCharTok{\%\textgreater{}\%} 
  \FunctionTok{mutate}\NormalTok{(}\AttributeTok{fumo =} \FunctionTok{factor}\NormalTok{(fumo, }
                       \AttributeTok{levels =} \FunctionTok{c}\NormalTok{(}\DecValTok{1}\NormalTok{,}\DecValTok{2}\NormalTok{), }
                       \AttributeTok{labels =} \FunctionTok{c}\NormalTok{(}\StringTok{"Fumante"}\NormalTok{, }\StringTok{"Não fumante"}\NormalTok{)))}

\NormalTok{dadosRNT }\OtherTok{\textless{}{-}}\NormalTok{ dados }\SpecialCharTok{\%\textgreater{}\%} 
  \FunctionTok{filter}\NormalTok{(ig}\SpecialCharTok{\textgreater{}=}\DecValTok{37} \SpecialCharTok{\&}\NormalTok{ ig}\SpecialCharTok{\textless{}}\DecValTok{42}\NormalTok{) }\SpecialCharTok{\%\textgreater{}\%} 
  \FunctionTok{select}\NormalTok{(pesoRN)}
\end{Highlighting}
\end{Shaded}

\section{Introdução}\label{introduuxe7uxe3o-3}

A estatística inferencial é a parte da estatística que usa os resultados
da amostra para tomar decisões e tirar conclusões sobre a população de
onde a amostra foi retirada. A estimação e o teste de hipóteses, tomados
em conjunto, constituem a \textbf{inferência estatística}.

\emph{Estimação} é um procedimento pelo qual um valor ou valores
numéricos são atribuídos a um parâmetro populacional com base nas
informações de uma amostra. Na estatística inferencial, \(\mu\) é
chamada de média populacional e \emph{p} é chamada de proporção
populacional. Existem muitos outros parâmetros populacionais, como
mediana, moda, variância e desvio padrão, como observado na
Seção~\ref{sec-dam}.

Se houvesse possibilidade de realizar um \emph{censo} (pesquisa
incluindo toda a população de interesse), não haveria necessidade dos
procedimentos de estimação. Seria equivalente ao que ocorre em uma
eleição, basta contar os votos, para declarar os vencedores da eleição.
No entanto, em saúde, realizar censo é um procedimento caro, demorado ou
virtualmente impossível. Portanto, geralmente é utilizada uma amostra da
população e calculada o valor das estatísticas da amostra apropriada.
Baseado nessas estatísticas, é atribuído valores ao parâmetro.

A estatística usada para estimar um parâmetro é chamada de estimador.
Assim, a média da amostra, \(\bar{x}\), é um estimador da média da
população, \(\mu\); e a proporção da amostra, \(\hat{p}\), é um
estimador da proporção da população, \emph{p}. Estimativa é um valor que
a função estimador assume.

\section{Estimativa Pontual e Intervalo de
Confiança}\label{estimativa-pontual-e-intervalo-de-confianuxe7a}

A partir do \emph{dataframe} \texttt{dados}, serão calculados a média e
o desvio padrão da variável \texttt{pesoRN} (peso dos recém-nascidos em
g) que, para fins didáticos, serão considerados os parâmetros dessa
``população'':

\begin{Shaded}
\begin{Highlighting}[]
\NormalTok{ mu }\OtherTok{\textless{}{-}} \FunctionTok{round}\NormalTok{(}\FunctionTok{mean}\NormalTok{(dadosRNT}\SpecialCharTok{$}\NormalTok{pesoRN, }\AttributeTok{na.rm =} \ConstantTok{TRUE}\NormalTok{))}
\NormalTok{ sigma }\OtherTok{\textless{}{-}} \FunctionTok{round}\NormalTok{(}\FunctionTok{sd}\NormalTok{(dadosRNT}\SpecialCharTok{$}\NormalTok{pesoRN, }\AttributeTok{na.rm =} \ConstantTok{TRUE}\NormalTok{))}
 \FunctionTok{print}\NormalTok{(}\AttributeTok{x =} \FunctionTok{c}\NormalTok{(mu, sigma))}
\end{Highlighting}
\end{Shaded}

\begin{verbatim}
[1] 3216  462
\end{verbatim}

A seguir, será extraída, dessa população, uma amostra de \emph{n} = 30
\footnote{Repetindo, é importante lembrar que toda vez que for extraída
  uma nova amostra, o resultado será um conjunto de números diferentes
  e, em consequência, a média será diferente. Por isso, se for
  importante repetir o mesmo resultado, deve-se usar a função
  \texttt{set.seed()}. Consulte a Seção~\ref{sec-rnorm}.} e calculado os
mesmas medidas resumidoras, que se constituirão nas estimativas da
\texttt{amostra}:

\begin{Shaded}
\begin{Highlighting}[]
\FunctionTok{set.seed}\NormalTok{ (}\DecValTok{1234}\NormalTok{)}
\NormalTok{amostra }\OtherTok{\textless{}{-}}\NormalTok{ dadosRNT }\SpecialCharTok{\%\textgreater{}\%} \FunctionTok{slice\_sample}\NormalTok{(}\AttributeTok{n =} \DecValTok{30}\NormalTok{)}

\CommentTok{\# Média amostral}
\NormalTok{x\_barra }\OtherTok{\textless{}{-}} \FunctionTok{round}\NormalTok{(}\FunctionTok{mean}\NormalTok{(amostra}\SpecialCharTok{$}\NormalTok{pesoRN, }\AttributeTok{na.rm =} \ConstantTok{TRUE}\NormalTok{))}

\CommentTok{\# Desvio padrão amostral  }
\NormalTok{s }\OtherTok{\textless{}{-}} \FunctionTok{round}\NormalTok{(}\FunctionTok{sd}\NormalTok{(amostra}\SpecialCharTok{$}\NormalTok{pesoRN, }\AttributeTok{na.rm =} \ConstantTok{TRUE}\NormalTok{))}

\FunctionTok{print}\NormalTok{(}\FunctionTok{c}\NormalTok{(x\_barra, s))}
\end{Highlighting}
\end{Shaded}

\begin{verbatim}
[1] 3222  407
\end{verbatim}

O valor de 3222g é a média amostral, \(\bar{x}\), usado como um
estimativa da \(\mu\), é denominado de \textbf{estimativa pontual}. Como
já mencionado anteriormente, espera-se que cada amostra selecionada
produza um valor diferente da estatística amostral. Assim, o valor
atribuído a uma média populacional, \(\mu\), com base em uma estimativa
pontual depende de qual das amostras está sendo usada. Consequentemente,
a estimativa pontual atribui um valor a \(\mu\) que quase sempre difere
da mesma.

Para melhorar a precisão, usa-se uma estimativa de intervalo. Em vez de
atribuir um único valor para o parâmetro populacional, é construído um
intervalo, acrescentando ou subtraindo um valor, chamado de
\textbf{margem de erro}, à estimativa pontual.

Este procedimento é conhecido como \emph{estimação por intervalo} e o
intervalo construído, estabelecendo um limite inferior e um limite
superior em torno da estimativa amostral, é denominado de
\textbf{intervalo de confiança}. Desta forma, é possível afirmar que o
intervalo de confiança, provavelmente, contém o parâmetro populacional
correspondente (Figura~\ref{fig-intconf}).

\begin{figure}

\centering{

\includegraphics[width=0.6\linewidth,height=\textheight,keepaspectratio]{index_files/mediabag/oAGF5Xs.png}

}

\caption{\label{fig-intconf}Intervalo de Confiança}

\end{figure}%

A construção do intervalo de confiança depende da obtenção da margem de
erro. Este processo necessita de dois fatores:

\begin{itemize}
\tightlist
\item
  do \emph{desvio padrão da distribuição amostral},
  \(\sigma_{\bar{x}}=\frac{\sigma }{\sqrt{n}}\), que em decorrência do
  Teorema do Limite Central, pode ser escrito
  \(EP_{\bar{x}}=\frac{s}{\sqrt{n}}\);
\item
  do \emph{nível de confiança} (NC) atribuído ao intervalo.
\end{itemize}

Primeiro, quanto maior for o desvio padrão de \(\bar{x}\), maior será a
margem de erro subtraída e adicionada à estimativa pontual.
Consequentemente, o intervalo de confiança se modifica de acordo com a
margem de erro. Quanto maior a margem de erro mais amplo o intervalo de
confiança.

Em segundo lugar, a quantidade subtraída e adicionada à estimativa se
modifica de acordo o \emph{nível de confiança}. Para ter uma maior
confiança, deve-se aumentar a margem de erro, de acordo com a
probabilidade declarada. Quanto maior o nível de confiança, maior a
probabilidade. O nível de confiança é mostrado como
\((1 - \alpha) \times 100\)\%, onde \(\alpha\) é o \textbf{nível de
significância}. Tradicionalmente, o valor de \(\alpha\) é igual a 0,05,
mas qualquer outro valor pode ser usado.

\section{\texorpdfstring{Estimação da média populacional: \(\sigma\)
conhecido}{Estimação da média populacional: \textbackslash sigma conhecido}}\label{estimauxe7uxe3o-da-muxe9dia-populacional-sigma-conhecido}

A \emph{margem de erro} para a estimativa da média populacional,
\(\mu\), quando se conhece o desvio padrão populacional,\(\sigma\), e
\(n \ge 30\) ou, mesmo que \(n < 30\), mas a população de onde amostra
foi selecionada tem distribuição normal, é a quantidade que é subtraída
ou adicionada ao valor da média da amostra, \(\bar{x}\), para obter o
intervalo de confiança para \(\mu\). Desta forma, a margem de erro é
igual a:

\[
margem \quad de\quad erro\quad(me)= z_{(1-\frac{\alpha}{2})} \times \sigma_{\bar{x}}
\]

Ou,

\[
me = z_{(1-\frac{\alpha}{2})} \times \frac{\sigma }{\sqrt{n}}
\]

Logo, o intervalo de confiança para a média populacional, \(\mu\), para
um nível de confiança (1 - \(\alpha \times 100\))\%, é igual a:

\[
IC_{(1-\alpha)}(\mu) \rightarrow  \bar{x} \pm  me
\]

Se objetivo é construir um intervalo de confiança de 95\%, a última
equação passa a ser:

\[
IC_{(1-\alpha)}(\mu) \rightarrow  \bar{x} \pm  z_{(0,975)} \times me
\]

Onde \emph{z} é o valor crítico para o nível de confiança escolhido,
obtido da tabela de distribuição normal padrão, e \emph{me} é a margem
de erro (\(z_{0,975} \times erro \quad padrao\)). Um intervalo de
confiança de 95\% significa que a área total sob a curva normal entre
dois pontos em torno da média populacional, \(\mu\), é igual a 95\%, ou
0,95. A área das caudas é \(\alpha\), ou seja, cada cauda á igual a
\(\frac{\alpha}{2}\) (Figura~\ref{fig-twotailed})).

\begin{figure}

\centering{

\includegraphics[width=0.7\linewidth,height=\textheight,keepaspectratio]{index_files/mediabag/q6Duexo.png}

}

\caption{\label{fig-twotailed}Intervalo de Confiança de 95\%}

\end{figure}%

Para encontrar o valor de \emph{z} para um nível de confiança de 95\%,
primeiro encontram-se as áreas à esquerda desses dois pontos, \(z_1\) e
\(z_2\). Esses dois valores de \emph{z} serão iguais, mas com sinais
opostos. A área total sob a curva é igual a 1. A área entre \(z_1\) e
\(z_2\) é igual a \(1 - \alpha = 0,95\).

A área a esquerda de \(z_1\) é igual a 0,025 e a área a esquerda de
\(z_2\) é igual a 1 -- 0,025 = 0,975. No \emph{R}, os valores \(z_1\) e
\(z_2\) podem facilmente ser obtidos com a função \texttt{qnorm()}:

\begin{Shaded}
\begin{Highlighting}[]
\FunctionTok{print}\NormalTok{(}\FunctionTok{c}\NormalTok{(}\FunctionTok{qnorm}\NormalTok{(}\FloatTok{0.025}\NormalTok{),}\FunctionTok{qnorm}\NormalTok{(}\FloatTok{0.975}\NormalTok{)), }\DecValTok{3}\NormalTok{)}
\end{Highlighting}
\end{Shaded}

\begin{verbatim}
[1] -1.96  1.96
\end{verbatim}

Dessa maneira, para uma confiança de 95\%, é usado um \(z = 1.96\),
onde:

\[
p(-1,96 \le z \le 1,96) = 0,95
\] Logo,

\[
IC_{95\%}(\mu) \rightarrow  \bar{x} \pm  (1.96 \times \sigma_{\bar{x}})
\] ou

\[
IC_{95\%}(\mu) \rightarrow  \bar{x} \pm  (1.96 \times \frac{\sigma}{\sqrt{n}})
\]

\subsection{\texorpdfstring{Cálculo do intervalo de confiança com
\(\sigma\)
conhecido}{Cálculo do intervalo de confiança com \textbackslash sigma conhecido}}\label{cuxe1lculo-do-intervalo-de-confianuxe7a-com-sigma-conhecido}

Usando a média dos pesos dos recém-nascidos da amostra (\emph{n} = 30),
\(\bar{x}\)= 3222 g, e o desvio padrão populacional conhecido,
\(\sigma\)= 462 g, tem-se que o intervalo de confiança de 95\% (IC95\%),
para o peso dos recém-nascidos a termo na `população' de onde esta
amostra é proveniente:

\ul{Dados do exemplo para o cálculo}

\begin{Shaded}
\begin{Highlighting}[]
\NormalTok{ n }\OtherTok{\textless{}{-}} \DecValTok{30}
\NormalTok{ x\_barra }\OtherTok{\textless{}{-}} \DecValTok{3222}
\NormalTok{ sigma }\OtherTok{\textless{}{-}} \DecValTok{462}
\end{Highlighting}
\end{Shaded}

Com 95\% de confiança a margem de erro é igual a 1,96 vezes o erro
padrão da média:

\begin{Shaded}
\begin{Highlighting}[]
\NormalTok{ n }\OtherTok{\textless{}{-}} \DecValTok{30}
\NormalTok{ me }\OtherTok{\textless{}{-}} \FloatTok{1.96} \SpecialCharTok{*}\NormalTok{ sigma}\SpecialCharTok{/}\FunctionTok{sqrt}\NormalTok{(n)}
 \FunctionTok{round}\NormalTok{(me,}\DecValTok{2}\NormalTok{)}
\end{Highlighting}
\end{Shaded}

\begin{verbatim}
[1] 165.32
\end{verbatim}

Basta, agora, adicionar e subtrair a margem de erro da média:

\begin{Shaded}
\begin{Highlighting}[]
\NormalTok{ lim\_inf }\OtherTok{\textless{}{-}}\NormalTok{ x\_barra }\SpecialCharTok{{-}}\NormalTok{ me}
\NormalTok{ lim\_sup }\OtherTok{\textless{}{-}}\NormalTok{ x\_barra }\SpecialCharTok{+}\NormalTok{ me}
\NormalTok{ ic95 }\OtherTok{\textless{}{-}} \FunctionTok{c}\NormalTok{(lim\_inf, lim\_sup)}
 \FunctionTok{round}\NormalTok{(ic95, }\DecValTok{1}\NormalTok{)}
\end{Highlighting}
\end{Shaded}

\begin{verbatim}
[1] 3056.7 3387.3
\end{verbatim}

Assim, tem-se uma confiança de 95\% de que a verdadeira média, esteja
incluída no intervalo. O nome para isso é \emph{intervalo de confiança
de 95\% para a média populacional}.

\subsection{\texorpdfstring{Função para calcular IC com \(\sigma\)
conhecido}{Função para calcular IC com \textbackslash sigma conhecido}}\label{funuxe7uxe3o-para-calcular-ic-com-sigma-conhecido}

O cálculo manual é simples, mas enfadonho, nos tempos dos computadores.
Em decorrência, como o R não tem uma função para encontrar os intervalos
de confiança para a média de dados com distribuição normal quando o
desvio padrão da população é conhecido, foi criada uma função para
cumprir essa ação. Ela necessita dos seguintes argumentos:

\begin{itemize}
\tightlist
\item
  \textbf{x} \(\to\) conjunto de números da amostra
\item
  \textbf{s} \(\to\) desvio padrão populacional
\item
  \textbf{nc} \(\to\) nível de confiança. Padrão: nc = 0.95
\end{itemize}

\begin{Shaded}
\begin{Highlighting}[]
\NormalTok{IC\_z }\OtherTok{\textless{}{-}} \ControlFlowTok{function}\NormalTok{ (x, s, }\AttributeTok{nc =} \FloatTok{0.975}\NormalTok{)}
\NormalTok{\{}
  \StringTok{\textasciigrave{}}\AttributeTok{\%\textgreater{}\%}\StringTok{\textasciigrave{}} \OtherTok{\textless{}{-}}\NormalTok{ dplyr}\SpecialCharTok{::}\StringTok{\textasciigrave{}}\AttributeTok{\%\textgreater{}\%}\StringTok{\textasciigrave{}}
\NormalTok{   n }\OtherTok{\textless{}{-}} \FunctionTok{length}\NormalTok{(x)}
\NormalTok{   me }\OtherTok{\textless{}{-}} \FunctionTok{abs}\NormalTok{(}\FunctionTok{qnorm}\NormalTok{((}\DecValTok{1}\SpecialCharTok{{-}}\NormalTok{nc)}\SpecialCharTok{/}\DecValTok{2}\NormalTok{))}\SpecialCharTok{*}\NormalTok{ sigma}\SpecialCharTok{/}\FunctionTok{sqrt}\NormalTok{(n)}
\NormalTok{   df\_out }\OtherTok{\textless{}{-}} \FunctionTok{data.frame}\NormalTok{( }\AttributeTok{tamanho\_amostral =}\NormalTok{ n, }
                         \AttributeTok{media\_amostral =} \FunctionTok{mean}\NormalTok{(x), }
                         \AttributeTok{margem\_erro =}\NormalTok{ me,}
                         \StringTok{\textquotesingle{}IC limite inferior\textquotesingle{}}\OtherTok{=}\NormalTok{(}\FunctionTok{mean}\NormalTok{(x) }\SpecialCharTok{{-}}\NormalTok{ me),}
                         \StringTok{\textquotesingle{}IC limite superior\textquotesingle{}}\OtherTok{=}\NormalTok{(}\FunctionTok{mean}\NormalTok{(x) }\SpecialCharTok{+}\NormalTok{ me)) }\SpecialCharTok{\%\textgreater{}\%}
\NormalTok{    tidyr}\SpecialCharTok{::}\FunctionTok{pivot\_longer}\NormalTok{(}\AttributeTok{names\_to =} \StringTok{"Medidas"}\NormalTok{, }\AttributeTok{values\_to =}\StringTok{"valores"}\NormalTok{, }\DecValTok{1}\SpecialCharTok{:}\DecValTok{5}\NormalTok{ )}
  \FunctionTok{return}\NormalTok{(df\_out)}
\NormalTok{\}}
\end{Highlighting}
\end{Shaded}

\begin{Shaded}
\begin{Highlighting}[]
\FunctionTok{IC\_z}\NormalTok{(}\AttributeTok{x =}\NormalTok{ amostra}\SpecialCharTok{$}\NormalTok{pesoRN, }\AttributeTok{s =}\NormalTok{ sigma, }\AttributeTok{nc =} \FloatTok{0.95}\NormalTok{)}
\end{Highlighting}
\end{Shaded}

\begin{verbatim}
# A tibble: 5 x 2
  Medidas            valores
  <chr>                <dbl>
1 tamanho_amostral       30 
2 media_amostral       3222.
3 margem_erro           165.
4 IC.limite.inferior   3056.
5 IC.limite.superior   3387.
\end{verbatim}

Essa função pode ser salva no seu diretório e, quando necessária, pode
ser ativada com a função \texttt{source()}, como visto na
Seção~\ref{sec-funcpropria}. Com essa função fica fácil alterar o nível
de confiança, por exemplo, para 99\%. Isso mudará o \emph{Z} crítico
para:

\begin{Shaded}
\begin{Highlighting}[]
\NormalTok{ alpha }\OtherTok{\textless{}{-}} \FloatTok{0.01}
\NormalTok{ p }\OtherTok{\textless{}{-}} \DecValTok{1}\SpecialCharTok{{-}}\NormalTok{(alpha}\SpecialCharTok{/}\DecValTok{2}\NormalTok{)}
\NormalTok{ p}
\end{Highlighting}
\end{Shaded}

\begin{verbatim}
[1] 0.995
\end{verbatim}

\begin{Shaded}
\begin{Highlighting}[]
\NormalTok{ z\_critico }\OtherTok{\textless{}{-}} \FunctionTok{qnorm}\NormalTok{(p)}
 \FunctionTok{round}\NormalTok{(z\_critico, }\DecValTok{2}\NormalTok{)}
\end{Highlighting}
\end{Shaded}

\begin{verbatim}
[1] 2.58
\end{verbatim}

Com a função \texttt{IC\_z()}:

\begin{Shaded}
\begin{Highlighting}[]
\FunctionTok{IC\_z}\NormalTok{(}\AttributeTok{x =}\NormalTok{ amostra}\SpecialCharTok{$}\NormalTok{pesoRN, }\AttributeTok{s =}\NormalTok{ sigma, }\AttributeTok{nc =} \FloatTok{0.995}\NormalTok{)}
\end{Highlighting}
\end{Shaded}

\begin{verbatim}
# A tibble: 5 x 2
  Medidas            valores
  <chr>                <dbl>
1 tamanho_amostral       30 
2 media_amostral       3222.
3 margem_erro           237.
4 IC.limite.inferior   2985.
5 IC.limite.superior   3458.
\end{verbatim}

Observando o IC95\% e o IC99\%, verifica-se que a amplitude do intervalo
aumentou com o crescimento da confiança de 95\% para 99\%, porque houve
um aumento na margem de erro (Figura~\ref{fig-IC9599}).

\begin{figure}

\centering{

\includegraphics[width=0.6\linewidth,height=\textheight,keepaspectratio]{index_files/mediabag/vaPrkze.png}

}

\caption{\label{fig-IC9599}Comparação entre IC95\% e IC99\%}

\end{figure}%

\subsection{Interpretação do intervalo de
confiança}\label{interpretauxe7uxe3o-do-intervalo-de-confianuxe7a}

Se fossem extraídas todas as possíveis amostras de \emph{n} = 30 da
população de recém-nascidos a termo e construído para cada uma delas um
intervalo de confiança de 95\% em torno de cada média amostral,
espera-se que 95\% desses intervalos incluirão a média populacional e
5\% não incluirão.\\
O IC95\% informa sobre a \emph{precisão} com que a média amostral estima
a média populacional desconhecida \footnote{Anteriormente, mostrou-se a
  media populacional por uma questão didática. A regra é não se conhecer
  a média populacional, razão da importância do intervalo de confiança}.

Na Figura~\ref{fig-vinte}, são mostradas 20 amostras diferentes de
tamanho \emph{n} = 30, dessa população. Junto aparecem os intervalos de
confiança de 95\% construídos em torno dessas amostras. Observa-se que
apenas uma amostra (em vermelho) não inclui a média populacional (linha
tracejada vertical em azul). Pode-se afirmar com 95\% de confiança que
se forem extraídas muitas amostras do mesmo tamanho de uma população e
construído intervalos de confiança de 95\% em torno das médias dessas
amostras, 95\% desses intervalos de confiança incluirão a média
populacional.

\begin{figure}

\centering{

\includegraphics[width=0.6\linewidth,height=0.7\textheight]{12-estimacao_files/figure-pdf/fig-vinte-1.pdf}

}

\caption{\label{fig-vinte}Intervalos de confiança de 95\% que mostra 20
replicações simuladas de amostras de n = 30 do peso do recém-nascido.
Apenas um intervalo (em vermelho) não inclui a média populacional (linha
vertical azul)}

\end{figure}%

\section{\texorpdfstring{Estimação da média populacional: \(\sigma\)
desconhecido}{Estimação da média populacional: \textbackslash sigma desconhecido}}\label{estimauxe7uxe3o-da-muxe9dia-populacional-sigma-desconhecido}

Com amostras pequenas, usar o modelo normal para construir intervalos de
confiança, pode gerar um erro, pois os pressupostos do teorema do limite
central não são respeitados. Quando o desvio padrão populacional,
\(\sigma\), é desconhecido e o tamanho amostral é pequeno (\textless{}
30), a estimação da média populacional é feita usando a
\textbf{distribuição t}.

\subsection{\texorpdfstring{Distribuição
\emph{t}}{Distribuição t}}\label{distribuiuxe7uxe3o-t}

A distribuição \emph{t}, desenvolvida por \emph{William Sealy Gosset},
em 1908, é semelhante à distribuição normal. Como a curva de
distribuição normal, a curva de distribuição \emph{t} é unimodal,
simétrica (em forma de sino) em torno da média e nunca encontra o eixo
horizontal. A área total sob uma curva de distribuição \emph{t} é 1 ou
100\%. A curva da distribuição \emph{t} é mais plana do que a curva de
distribuição normal padrão. Em outras palavras, ela é mais achatada e
mais espalhada. No entanto, conforme o tamanho da amostra aumenta, a
distribuição \emph{t} aproxima-se da distribuição normal padrão.

O formato de uma curva de distribuição \emph{t} particular depende do
número de graus de liberdade. O número de graus de liberdade
(\emph{gl})\footnote{Veja também a \textbf{?@sec-gl}.} para uma
distribuição \emph{t} é igual ao tamanho da amostra menos um, ou seja,
\(gl=n-1\), veja Seção~\ref{sec-variancia}.

O número de graus de liberdade é o único parâmetro da distribuição
\emph{t}. Há uma diferente distribuição \emph{t} para cada número de
graus de liberdade, portanto, a distribuição \emph{t} se constitui em
uma família de distribuições (Figura~\ref{fig-familia}).

\begin{figure}

\centering{

\includegraphics[width=0.8\linewidth,height=0.8\textheight]{12-estimacao_files/figure-pdf/fig-familia-1.pdf}

}

\caption{\label{fig-familia}Curvas de distribuição t conforme o grau de
liberdade comparadas à distribuição normal}

\end{figure}%

Da mesma maneira que a distribuição normal padrão, a média da
distribuição padrão \emph{t} é 0. Entretanto, ao contrário da
distribuição normal padrão, cujo desvio padrão é 1, o desvio padrão de
uma distribuição \emph{t} é \(\sqrt{\frac{gl}{gl-2}}\) , para \emph{gl}
\textgreater{} 2, sempre é maior do que 1. Assim, o desvio padrão de uma
distribuição \emph{t} é maior do que o desvio padrão da distribuição
normal padrão.\\
Os valores de \({t}_{crítico}\) podem ser obtidos usando a função
\texttt{qt()} que usa os seguintes argumentos:

\begin{itemize}
\tightlist
\item
  \textbf{p} \(\to\) probabilidade, igual a \(1 - \frac{\alpha}{2}\),
  considerando-se bicaudal e \(1 - \alpha\) quando unicaudal;
\item
  \textbf{df} \(\to\) graus de liberdade;
\item
  \textbf{lower.tail} \(\to\) lógico; se TRUE, informa a probabilidade
  da cauda inferior. O padrão é TRUE.
\end{itemize}

Assim, o valor do \({t}_{crítico}\) para \(gl=10\) é:

\begin{Shaded}
\begin{Highlighting}[]
\NormalTok{alpha  }\OtherTok{\textless{}{-}}  \FloatTok{0.05}
\NormalTok{p }\OtherTok{\textless{}{-}} \DecValTok{1} \SpecialCharTok{{-}}\NormalTok{ (alpha}\SpecialCharTok{/}\DecValTok{2}\NormalTok{)}
\NormalTok{gl }\OtherTok{=} \DecValTok{10}
\NormalTok{t }\OtherTok{\textless{}{-}} \FunctionTok{qt}\NormalTok{(}\AttributeTok{p =}\NormalTok{ p, }\AttributeTok{df =} \DecValTok{10}\NormalTok{, }\AttributeTok{lower.tail =} \ConstantTok{TRUE}\NormalTok{)}
\FunctionTok{round}\NormalTok{(t, }\AttributeTok{digits =} \DecValTok{2}\NormalTok{)}
\end{Highlighting}
\end{Shaded}

\begin{verbatim}
[1] 2.23
\end{verbatim}

A área compreendida entre \(\pm\) 2.23\$ é igual a 95\%
(Figura~\ref{fig-bilateral}):

\[
p(-2,23\le t\le 2,23)=0,95
\]

\begin{figure}

\centering{

\includegraphics[width=0.7\linewidth,height=0.7\textheight]{12-estimacao_files/figure-pdf/fig-bilateral-1.pdf}

}

\caption{\label{fig-bilateral}Distribuição t com gl = 10, bilateral}

\end{figure}%

Quando se considera apenas uma das caudas (unicaudal ou unilateral), o
valor do \({t}_{crítico}\) para \(gl=10\) é

\begin{Shaded}
\begin{Highlighting}[]
\NormalTok{t1 }\OtherTok{\textless{}{-}} \FunctionTok{qt}\NormalTok{(}\AttributeTok{p =} \FloatTok{0.95}\NormalTok{, }\AttributeTok{df =} \DecValTok{10}\NormalTok{, }\AttributeTok{lower.tail =} \ConstantTok{TRUE}\NormalTok{)}
\FunctionTok{round}\NormalTok{(t1, }\AttributeTok{digits =} \DecValTok{2}\NormalTok{)}
\end{Highlighting}
\end{Shaded}

\begin{verbatim}
[1] 1.81
\end{verbatim}

Assim, a área abaixo de 1.81 é igual a 95\%
(Figura~\ref{fig-unilateral}).

\[
p(t\le 1,81)=0,95
\]

\begin{figure}

\centering{

\includegraphics[width=0.7\linewidth,height=0.7\textheight]{12-estimacao_files/figure-pdf/fig-unilateral-1.pdf}

}

\caption{\label{fig-unilateral}Distribuição t com gl = 10, unilateral}

\end{figure}%

\subsection{\texorpdfstring{Cálculo do intervalo de confiança com
\(\sigma\)
desconhecido}{Cálculo do intervalo de confiança com \textbackslash sigma desconhecido}}\label{cuxe1lculo-do-intervalo-de-confianuxe7a-com-sigma-desconhecido}

Serão utilizados nesta seção, os dados da altura de mulheres, obtidos na
Seção~\ref{sec-dadoscap12}, atribuídos ao objeto \texttt{dados}.
Suponha-se que os parâmetros sejam desconhecidos. Para estimar esses
parâmetros, selecionou-se uma amostra de n = 30 desse conjunto dados.
Tomando essa amostra, calcula-se a sua média e o seu desvio padrão:

\begin{Shaded}
\begin{Highlighting}[]
\FunctionTok{set.seed}\NormalTok{(}\DecValTok{2345}\NormalTok{)}
\NormalTok{amostra1 }\OtherTok{\textless{}{-}}\NormalTok{ dados }\SpecialCharTok{\%\textgreater{}\%}
  \FunctionTok{slice\_sample}\NormalTok{(}\AttributeTok{n =} \DecValTok{30}\NormalTok{)}

\NormalTok{x\_barra1 }\OtherTok{\textless{}{-}} \FunctionTok{mean}\NormalTok{(amostra1}\SpecialCharTok{$}\NormalTok{altura, }\AttributeTok{na.rm =} \ConstantTok{TRUE}\NormalTok{)}
\NormalTok{s1 }\OtherTok{\textless{}{-}} \FunctionTok{sd}\NormalTok{(amostra1}\SpecialCharTok{$}\NormalTok{altura, }\AttributeTok{na.rm =} \ConstantTok{TRUE}\NormalTok{)}
\FunctionTok{print}\NormalTok{(}\FunctionTok{round}\NormalTok{(}\FunctionTok{c}\NormalTok{(x\_barra1, s1),}\DecValTok{3}\NormalTok{))}
\end{Highlighting}
\end{Shaded}

\begin{verbatim}
[1] 1.599 0.051
\end{verbatim}

A maneira mais intuitiva de estimar a média da população com base na
amostra, é, simplesmente, calcular a média e o desvio padrão.
Entretanto, para uma maior precisão, é sempre importante calcular o
intervalo de confiança.

\subsubsection{Cálculo manual do IC}\label{cuxe1lculo-manual-do-ic}

Quando o desvio padrão da população (\(\sigma\)) não é conhecido,
pode-se usar o seu estimador que é o desvio padrão da amostra
(\emph{s}), respeitando os pressupostos (99). Então, o erro padrão da
média (\(\sigma_{\bar{x}}\)) pode ser estimado pelo \(EP_{\bar{x}}\).

\[
EP_{\bar{x}}=\frac{s}{\sqrt{n}}
\]

O intervalo de confiança para a \(\mu\) para um nível de confiança
(\emph{NC}) de \((1 – \alpha) \times100\)\% é igual a:

\[
IC_{NC}(\mu)\rightarrow x\pm (t_{({1-\frac{alpha}{2})} } \times \frac {s}{\sqrt{n}})
\]

Quando o tamanho amostral é grande, o valor de \emph{t} se aproxima do
valor de \emph{z}, portanto, em situações em que não se conhece o desvio
padrão populacional, não há muita diferença se houver uma aproximação de
\emph{t} para \emph{z} (Tabela~\ref{tbl-tabtz}).

\global\setlength{\Oldarrayrulewidth}{\arrayrulewidth}

\global\setlength{\Oldtabcolsep}{\tabcolsep}

\setlength{\tabcolsep}{2pt}

\renewcommand*{\arraystretch}{1.5}



\providecommand{\ascline}[3]{\noalign{\global\arrayrulewidth #1}\arrayrulecolor[HTML]{#2}\cline{#3}}

\begin{longtable}[c]{|p{1.50in}|p{1.50in}|p{1.50in}|p{1.50in}}

\caption{\label{tbl-tabtz}Comparação dos valores z e t(gl)}

\tabularnewline

\ascline{1.5pt}{666666}{1-4}

\multicolumn{1}{>{\centering}m{\dimexpr 1.5in+0\tabcolsep}}{\textcolor[HTML]{000000}{\fontsize{11}{11}\selectfont{\global\setmainfont{Arial}{\textbf{n}}}}} & \multicolumn{1}{>{\centering}m{\dimexpr 1.5in+0\tabcolsep}}{\textcolor[HTML]{000000}{\fontsize{11}{11}\selectfont{\global\setmainfont{Arial}{\textbf{gl}}}}} & \multicolumn{1}{>{\centering}m{\dimexpr 1.5in+0\tabcolsep}}{\textcolor[HTML]{000000}{\fontsize{11}{11}\selectfont{\global\setmainfont{Arial}{\textbf{z}}}}} & \multicolumn{1}{>{\centering}m{\dimexpr 1.5in+0\tabcolsep}}{\textcolor[HTML]{000000}{\fontsize{11}{11}\selectfont{\global\setmainfont{Arial}{\textbf{t}}}}} \\

\ascline{1.5pt}{666666}{1-4}\endfirsthead 

\ascline{1.5pt}{666666}{1-4}

\multicolumn{1}{>{\centering}m{\dimexpr 1.5in+0\tabcolsep}}{\textcolor[HTML]{000000}{\fontsize{11}{11}\selectfont{\global\setmainfont{Arial}{\textbf{n}}}}} & \multicolumn{1}{>{\centering}m{\dimexpr 1.5in+0\tabcolsep}}{\textcolor[HTML]{000000}{\fontsize{11}{11}\selectfont{\global\setmainfont{Arial}{\textbf{gl}}}}} & \multicolumn{1}{>{\centering}m{\dimexpr 1.5in+0\tabcolsep}}{\textcolor[HTML]{000000}{\fontsize{11}{11}\selectfont{\global\setmainfont{Arial}{\textbf{z}}}}} & \multicolumn{1}{>{\centering}m{\dimexpr 1.5in+0\tabcolsep}}{\textcolor[HTML]{000000}{\fontsize{11}{11}\selectfont{\global\setmainfont{Arial}{\textbf{t}}}}} \\

\ascline{1.5pt}{666666}{1-4}\endhead



\multicolumn{1}{>{\centering}m{\dimexpr 1.5in+0\tabcolsep}}{\textcolor[HTML]{000000}{\fontsize{11}{11}\selectfont{\global\setmainfont{Arial}{5}}}} & \multicolumn{1}{>{\centering}m{\dimexpr 1.5in+0\tabcolsep}}{\textcolor[HTML]{000000}{\fontsize{11}{11}\selectfont{\global\setmainfont{Arial}{4}}}} & \multicolumn{1}{>{\centering}m{\dimexpr 1.5in+0\tabcolsep}}{\textcolor[HTML]{000000}{\fontsize{11}{11}\selectfont{\global\setmainfont{Arial}{1.96}}}} & \multicolumn{1}{>{\centering}m{\dimexpr 1.5in+0\tabcolsep}}{\textcolor[HTML]{000000}{\fontsize{11}{11}\selectfont{\global\setmainfont{Arial}{2.57}}}} \\





\multicolumn{1}{>{\centering}m{\dimexpr 1.5in+0\tabcolsep}}{\textcolor[HTML]{000000}{\fontsize{11}{11}\selectfont{\global\setmainfont{Arial}{10}}}} & \multicolumn{1}{>{\centering}m{\dimexpr 1.5in+0\tabcolsep}}{\textcolor[HTML]{000000}{\fontsize{11}{11}\selectfont{\global\setmainfont{Arial}{9}}}} & \multicolumn{1}{>{\centering}m{\dimexpr 1.5in+0\tabcolsep}}{\textcolor[HTML]{000000}{\fontsize{11}{11}\selectfont{\global\setmainfont{Arial}{1.96}}}} & \multicolumn{1}{>{\centering}m{\dimexpr 1.5in+0\tabcolsep}}{\textcolor[HTML]{000000}{\fontsize{11}{11}\selectfont{\global\setmainfont{Arial}{2.23}}}} \\





\multicolumn{1}{>{\centering}m{\dimexpr 1.5in+0\tabcolsep}}{\textcolor[HTML]{000000}{\fontsize{11}{11}\selectfont{\global\setmainfont{Arial}{30}}}} & \multicolumn{1}{>{\centering}m{\dimexpr 1.5in+0\tabcolsep}}{\textcolor[HTML]{000000}{\fontsize{11}{11}\selectfont{\global\setmainfont{Arial}{29}}}} & \multicolumn{1}{>{\centering}m{\dimexpr 1.5in+0\tabcolsep}}{\textcolor[HTML]{000000}{\fontsize{11}{11}\selectfont{\global\setmainfont{Arial}{1.96}}}} & \multicolumn{1}{>{\centering}m{\dimexpr 1.5in+0\tabcolsep}}{\textcolor[HTML]{000000}{\fontsize{11}{11}\selectfont{\global\setmainfont{Arial}{2.04}}}} \\





\multicolumn{1}{>{\centering}m{\dimexpr 1.5in+0\tabcolsep}}{\textcolor[HTML]{000000}{\fontsize{11}{11}\selectfont{\global\setmainfont{Arial}{50}}}} & \multicolumn{1}{>{\centering}m{\dimexpr 1.5in+0\tabcolsep}}{\textcolor[HTML]{000000}{\fontsize{11}{11}\selectfont{\global\setmainfont{Arial}{49}}}} & \multicolumn{1}{>{\centering}m{\dimexpr 1.5in+0\tabcolsep}}{\textcolor[HTML]{000000}{\fontsize{11}{11}\selectfont{\global\setmainfont{Arial}{1.96}}}} & \multicolumn{1}{>{\centering}m{\dimexpr 1.5in+0\tabcolsep}}{\textcolor[HTML]{000000}{\fontsize{11}{11}\selectfont{\global\setmainfont{Arial}{2.01}}}} \\





\multicolumn{1}{>{\centering}m{\dimexpr 1.5in+0\tabcolsep}}{\textcolor[HTML]{000000}{\fontsize{11}{11}\selectfont{\global\setmainfont{Arial}{100}}}} & \multicolumn{1}{>{\centering}m{\dimexpr 1.5in+0\tabcolsep}}{\textcolor[HTML]{000000}{\fontsize{11}{11}\selectfont{\global\setmainfont{Arial}{99}}}} & \multicolumn{1}{>{\centering}m{\dimexpr 1.5in+0\tabcolsep}}{\textcolor[HTML]{000000}{\fontsize{11}{11}\selectfont{\global\setmainfont{Arial}{1.96}}}} & \multicolumn{1}{>{\centering}m{\dimexpr 1.5in+0\tabcolsep}}{\textcolor[HTML]{000000}{\fontsize{11}{11}\selectfont{\global\setmainfont{Arial}{1.98}}}} \\





\multicolumn{1}{>{\centering}m{\dimexpr 1.5in+0\tabcolsep}}{\textcolor[HTML]{000000}{\fontsize{11}{11}\selectfont{\global\setmainfont{Arial}{200}}}} & \multicolumn{1}{>{\centering}m{\dimexpr 1.5in+0\tabcolsep}}{\textcolor[HTML]{000000}{\fontsize{11}{11}\selectfont{\global\setmainfont{Arial}{199}}}} & \multicolumn{1}{>{\centering}m{\dimexpr 1.5in+0\tabcolsep}}{\textcolor[HTML]{000000}{\fontsize{11}{11}\selectfont{\global\setmainfont{Arial}{1.96}}}} & \multicolumn{1}{>{\centering}m{\dimexpr 1.5in+0\tabcolsep}}{\textcolor[HTML]{000000}{\fontsize{11}{11}\selectfont{\global\setmainfont{Arial}{1.97}}}} \\





\multicolumn{1}{>{\centering}m{\dimexpr 1.5in+0\tabcolsep}}{\textcolor[HTML]{000000}{\fontsize{11}{11}\selectfont{\global\setmainfont{Arial}{500}}}} & \multicolumn{1}{>{\centering}m{\dimexpr 1.5in+0\tabcolsep}}{\textcolor[HTML]{000000}{\fontsize{11}{11}\selectfont{\global\setmainfont{Arial}{499}}}} & \multicolumn{1}{>{\centering}m{\dimexpr 1.5in+0\tabcolsep}}{\textcolor[HTML]{000000}{\fontsize{11}{11}\selectfont{\global\setmainfont{Arial}{1.96}}}} & \multicolumn{1}{>{\centering}m{\dimexpr 1.5in+0\tabcolsep}}{\textcolor[HTML]{000000}{\fontsize{11}{11}\selectfont{\global\setmainfont{Arial}{1.96}}}} \\





\multicolumn{1}{>{\centering}m{\dimexpr 1.5in+0\tabcolsep}}{\textcolor[HTML]{000000}{\fontsize{11}{11}\selectfont{\global\setmainfont{Arial}{1,000}}}} & \multicolumn{1}{>{\centering}m{\dimexpr 1.5in+0\tabcolsep}}{\textcolor[HTML]{000000}{\fontsize{11}{11}\selectfont{\global\setmainfont{Arial}{999}}}} & \multicolumn{1}{>{\centering}m{\dimexpr 1.5in+0\tabcolsep}}{\textcolor[HTML]{000000}{\fontsize{11}{11}\selectfont{\global\setmainfont{Arial}{1.96}}}} & \multicolumn{1}{>{\centering}m{\dimexpr 1.5in+0\tabcolsep}}{\textcolor[HTML]{000000}{\fontsize{11}{11}\selectfont{\global\setmainfont{Arial}{1.96}}}} \\

\ascline{1.5pt}{666666}{1-4}


\end{longtable}

\arrayrulecolor[HTML]{000000}

\global\setlength{\arrayrulewidth}{\Oldarrayrulewidth}

\global\setlength{\tabcolsep}{\Oldtabcolsep}

\renewcommand*{\arraystretch}{1}

A \texttt{amostra1} de \emph{n} = 30, \(\overline x\) = 1.599m e \(s\) =
0.051m. Essas estimativas servirão para o cálculo do intervalo de
confiança, usando uma distribuição t bicaudal e um nível de
significância \(\alpha = 0,05\).

\begin{Shaded}
\begin{Highlighting}[]
\NormalTok{n1 }\OtherTok{\textless{}{-}}  \FunctionTok{length}\NormalTok{(amostra1}\SpecialCharTok{$}\NormalTok{altura)}
\NormalTok{alpha }\OtherTok{\textless{}{-}} \FloatTok{0.05}
\NormalTok{p }\OtherTok{\textless{}{-}} \DecValTok{1} \SpecialCharTok{{-}}\NormalTok{ alpha}\SpecialCharTok{/}\DecValTok{2}
\CommentTok{\# Graus de liberdade  }
\NormalTok{gl }\OtherTok{\textless{}{-}}\NormalTok{ n1 }\SpecialCharTok{{-}} \DecValTok{1}
\CommentTok{\# Valor t crítico  }
\NormalTok{tc }\OtherTok{\textless{}{-}}  \FunctionTok{qt}\NormalTok{(p, gl, }\AttributeTok{lower.tail =} \ConstantTok{TRUE}\NormalTok{)}
\CommentTok{\# Erro padrão}
\NormalTok{EP1 }\OtherTok{\textless{}{-}} \FunctionTok{round}\NormalTok{(s1}\SpecialCharTok{/}\FunctionTok{sqrt}\NormalTok{(n1),}\DecValTok{3}\NormalTok{)}
\FunctionTok{print}\NormalTok{(}\FunctionTok{round}\NormalTok{(}\FunctionTok{c}\NormalTok{(tc, EP1),}\DecValTok{3}\NormalTok{))}
\end{Highlighting}
\end{Shaded}

\begin{verbatim}
[1] 2.045 0.009
\end{verbatim}

Com esses dados, calcula-se o intervalo de confiança de 95\%:

\begin{Shaded}
\begin{Highlighting}[]
\NormalTok{me1 }\OtherTok{\textless{}{-}}\NormalTok{ tc}\SpecialCharTok{*}\NormalTok{EP1}
\NormalTok{lim\_inf }\OtherTok{\textless{}{-}}\NormalTok{ x\_barra1 }\SpecialCharTok{{-}}\NormalTok{ me1}
\NormalTok{lim\_sup }\OtherTok{\textless{}{-}}\NormalTok{ x\_barra1 }\SpecialCharTok{+}\NormalTok{ me1}
\NormalTok{ic95 }\OtherTok{\textless{}{-}} \FunctionTok{c}\NormalTok{(lim\_inf, lim\_sup)}
\FunctionTok{round}\NormalTok{(ic95, }\DecValTok{2}\NormalTok{)}
\end{Highlighting}
\end{Shaded}

\begin{verbatim}
[1] 1.58 1.62
\end{verbatim}

\subsubsection{Cálculo usando uma função do
R}\label{cuxe1lculo-usando-uma-funuxe7uxe3o-do-r}

O R possui algumas funções que calculam o intervalo de confiança para
variáveis numéricas, baseadas na distribuição \emph{t}. Entre elas, a
função \texttt{CI()}, incluída no pacote \texttt{Rmisc}. Esta função tem
dois argumentos:

\begin{itemize}
\tightlist
\item
  \textbf{x} ⟶ vetor de dados;
\item
  \textbf{ci} ⟶ intervalo de confiança a ser calculado
\end{itemize}

\begin{Shaded}
\begin{Highlighting}[]
\NormalTok{IC95 }\OtherTok{\textless{}{-}} \FunctionTok{CI}\NormalTok{(amostra1}\SpecialCharTok{$}\NormalTok{altura, }\AttributeTok{ci =} \FloatTok{0.95}\NormalTok{)}
\FunctionTok{round}\NormalTok{(IC95, }\DecValTok{2}\NormalTok{)}
\end{Highlighting}
\end{Shaded}

\begin{verbatim}
upper  mean lower 
 1.62  1.60  1.58 
\end{verbatim}

\section{Intervalo de Confiança para uma proporção
populacional}\label{intervalo-de-confianuxe7a-para-uma-proporuxe7uxe3o-populacional}

\subsection{Dados para estimar a proporção
populacional}\label{dados-para-estimar-a-proporuxe7uxe3o-populacional}

Aqui, será utilizada uma amostra aleatória de \emph{n} = 60 do conjunto
de dados \texttt{dados} para estimar a proporção de mulheres fumantes.

\begin{Shaded}
\begin{Highlighting}[]
\FunctionTok{set.seed}\NormalTok{(}\DecValTok{2346}\NormalTok{)}
\NormalTok{dados60 }\OtherTok{\textless{}{-}}\NormalTok{  dados }\SpecialCharTok{\%\textgreater{}\%} 
  \FunctionTok{select}\NormalTok{ (fumo) }\SpecialCharTok{\%\textgreater{}\%} 
  \FunctionTok{slice\_sample}\NormalTok{(}\AttributeTok{n =} \DecValTok{60}\NormalTok{)}

\FunctionTok{str}\NormalTok{(dados60)}
\end{Highlighting}
\end{Shaded}

\begin{verbatim}
tibble [60 x 1] (S3: tbl_df/tbl/data.frame)
 $ fumo: Factor w/ 2 levels "Fumante","Não fumante": 2 2 1 2 2 2 2 2 2 1 ...
\end{verbatim}

\subsection{Cálculo da estimativa pontual da
proporção}\label{cuxe1lculo-da-estimativa-pontual-da-proporuxe7uxe3o}

Nessa amostra, a proporção de fumantes é:

\begin{Shaded}
\begin{Highlighting}[]
\NormalTok{tab }\OtherTok{\textless{}{-}} \FunctionTok{table}\NormalTok{(dados60}\SpecialCharTok{$}\NormalTok{fumo)}
\NormalTok{tab}
\end{Highlighting}
\end{Shaded}

\begin{verbatim}

    Fumante Não fumante 
         14          46 
\end{verbatim}

\begin{Shaded}
\begin{Highlighting}[]
\NormalTok{tabFumo }\OtherTok{\textless{}{-}} \FunctionTok{round}\NormalTok{ (}\FunctionTok{prop.table}\NormalTok{ (tab), }\DecValTok{3}\NormalTok{)}
\NormalTok{tabFumo}
\end{Highlighting}
\end{Shaded}

\begin{verbatim}

    Fumante Não fumante 
      0.233       0.767 
\end{verbatim}

\subsection{Cálculo do intervalo de confiança para a
proporção}\label{sec-ICproporcao}

\textbf{Cálculo manual com aproximação normal}

\emph{1ª etapa}: verificar a premissa de que quando a proporção
populacional é desconhecida a proporção pontual (\(\hat p\)) e o seu
complemento (\(\hat q = 1 - \hat p\)) multiplicados, cada um, por \(n\),
devem ser maior do que 5.

\begin{Shaded}
\begin{Highlighting}[]
\NormalTok{n }\OtherTok{\textless{}{-}} \FunctionTok{length}\NormalTok{(dados60}\SpecialCharTok{$}\NormalTok{fumo)}
\NormalTok{(tabFumo) }\SpecialCharTok{*}\NormalTok{ n}
\end{Highlighting}
\end{Shaded}

\begin{verbatim}

    Fumante Não fumante 
      13.98       46.02 
\end{verbatim}

Como se observa, ambos os valores são maiores do que 5.

\emph{2ª Etapa}: O intervalo pode ser estimado pela distribuição normal
e é necessário calcular o \emph{z\_crítico}:

\begin{Shaded}
\begin{Highlighting}[]
\NormalTok{alpha }\OtherTok{\textless{}{-}} \FloatTok{0.05}
\NormalTok{p }\OtherTok{\textless{}{-}}  \DecValTok{1} \SpecialCharTok{{-}}\NormalTok{ alpha}\SpecialCharTok{/}\DecValTok{2}
\NormalTok{zc }\OtherTok{\textless{}{-}} \FunctionTok{qnorm}\NormalTok{ (p, }\AttributeTok{mean =} \DecValTok{0}\NormalTok{, }\AttributeTok{sd =} \DecValTok{1}\NormalTok{)}
\FunctionTok{round}\NormalTok{(zc, }\DecValTok{2}\NormalTok{)}
\end{Highlighting}
\end{Shaded}

\begin{verbatim}
[1] 1.96
\end{verbatim}

\emph{3ª Etapa}: Cálculo do erro padrão da proporção
(\(\sqrt \frac {\hat p \times \hat q}{n}\)) e da margem de erro (veja
também a Seção~\ref{sec-popamostra}):

\begin{Shaded}
\begin{Highlighting}[]
\CommentTok{\# Extração da proporção amostral do tabFumo}
\NormalTok{prop }\OtherTok{\textless{}{-}}\NormalTok{ tabFumo [}\DecValTok{1}\NormalTok{]}

\CommentTok{\# Cálculo do EP amostral}
\NormalTok{EP }\OtherTok{\textless{}{-}} \FunctionTok{sqrt}\NormalTok{((prop }\SpecialCharTok{*}\NormalTok{ (}\DecValTok{1} \SpecialCharTok{{-}}\NormalTok{ prop))}\SpecialCharTok{/}\NormalTok{n)}

\CommentTok{\# Cálculo da margem de erro(me)}
\NormalTok{me }\OtherTok{\textless{}{-}}\NormalTok{ zc }\SpecialCharTok{*}\NormalTok{ EP}

\CommentTok{\# dados necessários para o cálculo do IC95\%}
\FunctionTok{print}\NormalTok{(}\FunctionTok{c}\NormalTok{(prop, me), }\AttributeTok{digits =} \DecValTok{3}\NormalTok{)}
\end{Highlighting}
\end{Shaded}

\begin{verbatim}
Fumante Fumante 
  0.233   0.107 
\end{verbatim}

\emph{4ª Etapa}: Intervalo de confiança

\begin{Shaded}
\begin{Highlighting}[]
\NormalTok{ic\_prop }\OtherTok{\textless{}{-}} \FunctionTok{c}\NormalTok{((prop }\SpecialCharTok{{-}}\NormalTok{ me), (prop }\SpecialCharTok{+}\NormalTok{ me))}
\FunctionTok{round}\NormalTok{(ic\_prop, }\DecValTok{3}\NormalTok{)}
\end{Highlighting}
\end{Shaded}

\begin{verbatim}
Fumante Fumante 
  0.126   0.340 
\end{verbatim}

\textbf{Cálculo usando uma função}

O chamado \emph{Intervalo de Confiança Exato} corrigem as deficiências
da aproximação normal. O R tem uma função para este cálculo:
\texttt{BinomCI()} do pacote \texttt{DescTools}(100). É preferível usar
o método de Clopper e Pearson que fornece o IC exato.

Os argumentos da função \texttt{BinomCI()} são:

\begin{itemize}
\tightlist
\item
  \textbf{x} \(\to\) é o número de desfechos, sucessos;
\item
  \textbf{n} \(\to\) é o tamanho da amostra, número de ensaios;
\item
  \textbf{p} \(\to\) probabilidade, hipótese nula; se ignorada o padrão
  é 0,50;
\item
  \textbf{conf.level} \(\to\) nível de confiança, o padrão é 0.95;
\item
  \textbf{method} \(\to\) possui vários métodos para calcular intervalos
  de confiança para uma proporção binomial como: ``clopper-pearson''
  (exact interval), ``wilson'', ``wald'', ``agresti-coull'',
  ``jeffreys'', ``modified wilson'', ``modified jeffreys'', ``arcsine'',
  ``logit'', ``witting'', ``pratt''. O método padrão é o de ``wilson''.
  Qualquer outro método, há necessidade de solicitar;
\item
  \textbf{sides} \(\to\) hipótese alternativa padrão ``two.sided''
  (bilateral), mas pode ser ``right'' ou ``left'' (unilateral a direita
  ou a esquerda, respectivamente).
\end{itemize}

\begin{Shaded}
\begin{Highlighting}[]
\NormalTok{x }\OtherTok{\textless{}{-}}\NormalTok{  tab[}\DecValTok{1}\NormalTok{]}
\NormalTok{IC }\OtherTok{\textless{}{-}} \FunctionTok{BinomCI}\NormalTok{ (x, }
\NormalTok{               n, }
               \AttributeTok{conf.level =} \FloatTok{0.95}\NormalTok{, }
               \AttributeTok{method =} \StringTok{"clopper{-}pearson"}\NormalTok{)}
\FunctionTok{round}\NormalTok{(IC, }\DecValTok{3}\NormalTok{)}
\end{Highlighting}
\end{Shaded}

\begin{verbatim}
       est lwr.ci upr.ci
[1,] 0.233  0.134   0.36
\end{verbatim}

Observe que existe uma pequena diferença entre os valores da aproximação
normal e o exato, com o método de ``clopper-pearson''

\chapter{Teste de Hipóteses}\label{teste-de-hipuxf3teses}

\section{Pacotes necessários neste
capítulo}\label{pacotes-necessuxe1rios-neste-capuxedtulo-4}

\begin{Shaded}
\begin{Highlighting}[]
\NormalTok{pacman}\SpecialCharTok{::}\FunctionTok{p\_load}\NormalTok{(dplyr,}
\NormalTok{               lsr,}
\NormalTok{               pwr,}
\NormalTok{               readxl,}
\NormalTok{               rstatix)}
\end{Highlighting}
\end{Shaded}

\section{Dados para o exemplo}\label{sec-dadosth}

Considere o mesmo arquivo \texttt{dadosMater.xlsx}, usado várias vezes
neste livro e disponível para consulta na Seção~\ref{sec-dadosMater}.
Após a leitura do arquivo com a função \texttt{read\_excel()} do pacote
\texttt{readxl}, serão filtrados as gestações a termo (37 a 42 semanas
de gestação) e selecionadas as varáveis \texttt{sexo} e \texttt{pesoRN}.
Considerando esses dados como uma ``população'' para fins didáticos,
será extraída uma amostra de 200 observações e atribuido o resultado ao
objeto \texttt{dados}.

\begin{tcolorbox}[enhanced jigsaw, bottomrule=.15mm, opacitybacktitle=0.6, colframe=quarto-callout-tip-color-frame, arc=.35mm, coltitle=black, toptitle=1mm, colback=white, colbacktitle=quarto-callout-tip-color!10!white, breakable, bottomtitle=1mm, rightrule=.15mm, titlerule=0mm, toprule=.15mm, opacityback=0, leftrule=.75mm, left=2mm, title=\textcolor{quarto-callout-tip-color}{\faLightbulb}\hspace{0.5em}{Pergunta motivadora}]

Existe uma diferença estatisticamente significativa nos pesos dos
recém-nascidos de acordo com o sexo?

\end{tcolorbox}

\begin{Shaded}
\begin{Highlighting}[]
\NormalTok{dados }\OtherTok{\textless{}{-}}\NormalTok{ readxl}\SpecialCharTok{::}\FunctionTok{read\_excel}\NormalTok{(}\StringTok{"dados/dadosMater.xlsx"}\NormalTok{) }\SpecialCharTok{\%\textgreater{}\%} 
\NormalTok{  dplyr}\SpecialCharTok{::}\FunctionTok{filter}\NormalTok{(ig}\SpecialCharTok{\textgreater{}=}\DecValTok{37} \SpecialCharTok{\&}\NormalTok{ ig}\SpecialCharTok{\textless{}}\DecValTok{42}\NormalTok{) }\SpecialCharTok{\%\textgreater{}\%} 
  \FunctionTok{select}\NormalTok{(sexo, pesoRN) }\SpecialCharTok{\%\textgreater{}\%}
  \FunctionTok{mutate}\NormalTok{(}\AttributeTok{sexo =} \FunctionTok{factor}\NormalTok{(sexo, }
                       \AttributeTok{levels =} \FunctionTok{c}\NormalTok{(}\DecValTok{1}\NormalTok{,}\DecValTok{2}\NormalTok{), }
                       \AttributeTok{labels =} \FunctionTok{c}\NormalTok{(}\StringTok{"Masculino"}\NormalTok{, }\StringTok{"Feminino"}\NormalTok{))) }\SpecialCharTok{\%\textgreater{}\%} 
  \FunctionTok{slice\_sample}\NormalTok{(}\AttributeTok{n =} \DecValTok{200}\NormalTok{)}
\end{Highlighting}
\end{Shaded}

\subsection{Exploração e transformação dos
dados}\label{explorauxe7uxe3o-e-transformauxe7uxe3o-dos-dados}

Inicialmente, para ter uma visão da estrutura dos dados, usa-se:

\begin{Shaded}
\begin{Highlighting}[]
\FunctionTok{str}\NormalTok{(dados)}
\end{Highlighting}
\end{Shaded}

\begin{verbatim}
tibble [200 x 2] (S3: tbl_df/tbl/data.frame)
 $ sexo  : Factor w/ 2 levels "Masculino","Feminino": 2 1 1 1 2 1 1 1 2 1 ...
 $ pesoRN: num [1:200] 3040 3110 3410 3490 3250 ...
\end{verbatim}

Este conjunto de dados fica restrito, portanto, a 200 casos, contendo
duas variáveis \texttt{sexo} e \texttt{pesoRN}, necessárias neste
capítulo e assim resumidas:

\begin{Shaded}
\begin{Highlighting}[]
\NormalTok{resumo }\OtherTok{\textless{}{-}}\NormalTok{ dados }\SpecialCharTok{\%\textgreater{}\%} 
\NormalTok{  dplyr}\SpecialCharTok{::}\FunctionTok{group\_by}\NormalTok{(sexo) }\SpecialCharTok{\%\textgreater{}\%} 
\NormalTok{  dplyr}\SpecialCharTok{::}\FunctionTok{summarise}\NormalTok{ (}\AttributeTok{n =} \FunctionTok{n}\NormalTok{(),}
                    \AttributeTok{media =} \FunctionTok{mean}\NormalTok{(pesoRN, }\AttributeTok{na.rm =} \ConstantTok{TRUE}\NormalTok{),}
                    \AttributeTok{dp =} \FunctionTok{sd}\NormalTok{(pesoRN, }\AttributeTok{na.rm =} \ConstantTok{TRUE}\NormalTok{))}
\NormalTok{resumo}
\end{Highlighting}
\end{Shaded}

\begin{verbatim}
# A tibble: 2 x 4
  sexo          n media    dp
  <fct>     <int> <dbl> <dbl>
1 Masculino   117 3263.  440.
2 Feminino     83 3125.  433.
\end{verbatim}

Esta amostra de 117 meninos e 83 meninas, informa que os meninos têm, em
média, 3263 g ao nascer e as meninas 3125 g. Esta diferença de peso
entre os sexos pode ter ocorrido devido ao acaso. Portanto, há
necessidade de realizar um teste de hipóteses para tomar uma decisão
sobre o parâmetro populacional. Esta diferença é grande o suficiente
para rejeitar a hipótese de igualdade entre os pesos e concluir que
existe uma diferença real entre eles?

\section{Introdução}\label{introduuxe7uxe3o-4}

No capítulo anterior, foi discutido aspectos relacionados à estimação,
que se constitui, junto com o teste de hipótese, em procedimentos
básicos da \emph{estatística inferencial}. Em um teste de hipóteses,
testa-se uma teoria ou crença sobre um parâmetro populacional (101). Na
maioria das vezes, como mencionado anteriormente, obtém-se informações a
partir de uma amostra em função da impossibilidade ou dificuldade de se
conseguir essas informações a partir da população. Portanto, extrapolar
ou estender os resultados, obtidos de uma amostra, para a população,
significa aceitá-los como representações adequadas da mesma.

Sabe-se que as estimativas amostrais diferem dos valores reais
(populacionais) e o objetivo dos testes de hipóteses é estabelecer a
probabilidade de essa diferença ser explicada pelo acaso. O teste de
hipóteses fornece um sistema referencial para a tomada de decisão sobre
a adequação ou não dos dados amostrais serem representativos de uma
população. Este sistema referencial é a distribuição de probabilidade do
evento observado (102).

Inicialmente é importante fazer uma distinção entre \textbf{hipótese de
pesquisa} e \textbf{hipótese estatística}. Uma hipótese de pesquisa é
uma afirmação que expressa a relação esperada entre as variáveis de um
estudo científico. Ela é baseada em uma pergunta de pesquisa e serve
para orientar a coleta e análise dos dados. Uma hipótese de pesquisa
pode ser confirmada ou refutada pelos resultados do estudo. Um exemplo
de hipótese de pesquisa é: ``O tabagismo durante a gestação interfere
sobre o peso dos conceptos''. Uma hipótese de pesquisa corresponde
àquilo que se quer acreditar sobre o mundo. Uma hipótese estatística é
uma afirmação relacionada aos parâmetros de uma população. Baseia-se em
uma hipótese de pesquisa e serve para testar a validade da mesma usando
técnicas estatísticas. Uma hipótese estatística pode ser aceita ou
rejeitada com um certo nível de confiança. A hipótese estatística deve
ter uma relação clara com as hipóteses de pesquisa Por exemplo: ``A
média de peso dos recém-nascidos de mães fumantes é menor do que o das
não fumantes''; ``A média de peso dos recém-nascidos masculinos é igual
ao peso dos recém-nascidos femininos'', ou ainda, ``A média de peso dos
recém-nascidos masculinos é diferente do peso dos recém-nascidos
femininos''. Todos esses exemplos são legítimos de uma hipótese
estatística porque são afirmações sobre um parâmetro populacional e
estão significativamente relacionados à hipótese de pesquisa.

\section{Hipótese nula e
alternativa}\label{hipuxf3tese-nula-e-alternativa}

Em função da hipótese de pesquisa, mencionada anteriormente, foram
gerados os dados do exemplo. A hipótese de pesquisa corresponde ao que
se quer acreditar, ``o sexo interfere no peso dos neonatos''. Para
refutar ou não essa afirmação constrói-se um teste de hipótese para
verificar se ela é compatível ou não com os dados disponíveis (103).

No teste de hipóteses (TH), existem dois tipos de hipóteses, definidas
como:

\textbf{Hipótese nula}(\(H_{0}\)): hipótese que afirma a não existência
de diferença entre os grupos e, portanto, a diferença observada é
atribuível ao acaso. É a hipótese a ser testada, aquela que se busca
afastar, demonstrando que é, provavelmente \footnote{Ter em mente que
  nunca se pode saber com total certeza se existe um efeito na
  população.}, falsa, não válida. É denotada como:

\[
H_{0}: \mu_{1}= \mu_{2} \quad ou \quad \mu_{1} - \mu_{2}=0
\]

\textbf{Hipótese alternativa} (\(H_{1} \quad ou \quad H_{a}\)): é a
hipótese contrária, como o nome diz, alternativa à \(H_{0}\). Representa
a posição de uma nova perspectiva, a conclusão que será apoiada se
\(H_{0}\) for rejeitada. Ela supõe que realmente exista uma diferença
entre os grupos. É a hipótese que o pesquisador pretende comprovar. É
denotada, em geral, simplesmente como havendo uma diferença entre os
grupos, sem indicar uma direção, \emph{hipótese bilateral} ou
\emph{bicaudal}:

\[
H_{1}: \mu_{1} \neq  \mu_{2} \quad ou \quad \mu_{1} - \mu_{2} \neq  0
\]

Ou, se houver uma suspeita, através de um conhecimento prévio, apontar
uma direção para a diferença, ou seja, usar uma \emph{hipótese
unilateral} ou \emph{monocaudal}. Neste caso existe duas possibilidade:

\begin{enumerate}
\def\labelenumi{\arabic{enumi})}
\tightlist
\item
  Unilateral à direita:
\end{enumerate}

\[
H_{1}: \mu_{1} > \mu_{2} \quad ou \quad \mu_{1} - \mu_{2} > 0 
\]

Consequentemente,

\[
H_{0}: \mu_{1} \le \mu_{2} \quad ou \quad \mu_{1}- \mu_{2} \le 0
\] 2)

\begin{enumerate}
\def\labelenumi{\arabic{enumi})}
\setcounter{enumi}{1}
\tightlist
\item
  Unilateral à esquerda:
\end{enumerate}

\[
H_{1}: \mu_{1} < \mu_{2} \quad ou \quad \mu_{1} - \mu_{2} < 0   
\]

Consequentemente,

\[
H_{0}: \mu_{1} \ge \mu_{2} \quad ou \quad \mu_{1}- \mu_{2} \ge 0
\]

A \(H_{0}\) e \(H_{1}\) são opostas e mutuamente exclusivas. No teste de
hipótese calcula-se a probabilidade de obter os resultados encontrados
caso não haja efeito na população, ou seja, caso a \(H_{0}\) seja
verdadeira. Portanto, o TH é um teste de significância para a \(H_{0}\).

\subsection{Exemplo}\label{sec-exeth}

Voltando à hipótese de pesquisa, usando os dados da
Seção~\ref{sec-dadosth}, as hipóteses estatísticas seriam escritas da
seguinte maneira, considerando uma hipótese alternativa bilateral.

\[
H_{0}: \mu_{peso_{masc}} = \mu_{peso_{fem}} \quad ou \quad \mu_{peso_{masc}} - \mu_{peso_{fem}}=0
\]

\[
H_{1}: \mu_{peso_{masc}} \neq \mu_{peso_{fem}} \quad ou \quad \mu_{peso_{masc}} - \mu_{peso_{fem}} \neq 0
\]

\section{Escolha do teste estatítico e regra de
decisão}\label{escolha-do-teste-estatuxedtico-e-regra-de-decisuxe3o}

\subsection{Teste estatístico}\label{teste-estatuxedstico}

Usa-se um teste estatístico para testar as hipóteses estabelecidas. Este
depende do tipo de distribuição da variável, por exemplo, teste
\emph{z}, teste \emph{t}, teste \emph{F}, qui-quadrado (\(\chi^2\)).
Cada teste fornece um valor para dirigir a decisão de rejeitar ou não a
hipótese nula. Essa decisão depende da magnitude do teste valor. O nome
para esse indicador, calculado para orientar a escolha, é
\textbf{estatística de teste}. Para fazer isso, há necessidade de
determinar qual seria a distribuição amostral da estatística de teste se
a hipótese nula fosse realmente verdadeira. Depois de analisar esse
valor, decide-se se a hipótese nula está correta ou, caso contrário, ela
é rejeitada em favor da alternativa.\\
É fundamental lembrar que cada teste estatístico tem suas
características e seus pressupostos que devem ser analisados para
garantir a validade das estatísticas de teste. Para uma boa parte deles,
por exemplo, deve-se verificar se os dados se ajustam à distribuição
normal (normalidade), a igualdade das variâncias (homocedasticidade),
independência entre os grupos, tipo de correlação, etc.

\subsection{Regra de decisão}\label{regra-de-decisuxe3o}

Realizado o teste estaístico, para rejeitar ou não rejeitar a \(H_{0}\),
partindo do pressuposto de que ela é verdadeira, há necessidade de
determinar uma \emph{regra de decisão} que permita uma declaração
fundamentada. Essa regra de decisão cria duas regiões, uma
\textbf{região de rejeição} e uma \textbf{região de não rejeição} da
\(H_{0}\), demarcadas por um \textbf{valor crítico}.

Este valor de referência é determinado pelo \textbf{nível de
significância}, \(\alpha\), e deve ser explicitamente mencionado
\emph{antes} de se iniciar a pesquisa, pois é baseado nele que se
fundamentam as conclusões da mesma. O nível de significância corresponde
a probabilidade de rejeitar uma hipótese nula verdadeira. Quando a
hipótese alternativa não tem uma direção definida, a área de rejeição,
\(\alpha\), é colocada nas duas caudas (Figura~\ref{fig-rejeicao}),
superior), dividindo a probabilidade (\(\frac {\alpha}{2}\)); quando
houver indicação prévia de um sentido, a área de rejeição ficará a
direita (Figura~\ref{fig-rejeicao}), inferior) ou a esquerda dependendo
da direção escolhida.

\begin{figure}

\centering{

\includegraphics[width=0.6\linewidth,height=\textheight,keepaspectratio]{index_files/mediabag/ttCoLro.png.png}

}

\caption{\label{fig-rejeicao}Regiões bicaudais (acima) e monocaudal à
direita (abaixo) de rejeição e não rejeição da hipótese nula}

\end{figure}%

Quais valores exatos da estatística de teste deve-se associar à hipótese
nula e quais valores exatos devem ser associados à hipótese alternativa?
Para encontrar a região de rejeição, deve-se levar em consideração:

\begin{itemize}
\tightlist
\item
  A estatística do teste deve ser muito grande ou muito pequena para que
  a hipótese nula seja rejeitada;\\
\item
  Distribuição da variável de teste, que depende da distribuição da
  população em estudo e do tamanho da amostra;\\
\item
  Nível se significância adotado, em geral, usa-se um \(\alpha\) = 0,05,
  o que equivale a dizer que a região de rejeição abrange 5\% da
  distribuição.
\end{itemize}

É importante entender bem este último ponto. A região de rejeição
corresponde aos valores da estatística de teste para os quais se rejeita
a hipótese nula e a distribuição amostral em questão descreve a
probabilidade de obtermos um determinado valor da estatística de teste
se a hipótese nula for efetivamente verdadeira.

\begin{tcolorbox}[enhanced jigsaw, bottomrule=.15mm, opacitybacktitle=0.6, colframe=quarto-callout-important-color-frame, arc=.35mm, coltitle=black, toptitle=1mm, colback=white, colbacktitle=quarto-callout-important-color!10!white, breakable, bottomtitle=1mm, rightrule=.15mm, titlerule=0mm, toprule=.15mm, opacityback=0, leftrule=.75mm, left=2mm, title=\textcolor{quarto-callout-important-color}{\faExclamation}\hspace{0.5em}{Importante}]

Suponha-se que foi escolhido uma região de rejeição que cobre 10\% da
distribuição amostral e que a hipótese nula é realmente verdadeira. Qual
seria a probabilidade de rejeitar incorretamente a hipótese nula?

Obviamente, a resposta é 10\%! E o teste usado teria um nível \(\alpha\)
= 0,10. Ou seja, se a hipótese nula é verdadeira e for rejeitada, foi
cometido um erro.

\end{tcolorbox}

\subsubsection{Erros de decisão}\label{sec-erros}

Como se observa, ao se tomar uma decisão existe a possibilidade de se
cometer \emph{erros}. O primeiro erro é denominado de \textbf{erro tipo
I} e ocorre quando, baseado na regra de decisão escolhida, uma hipótese
nula verdadeira é rejeitada. Nesse caso, tem-se um resultado \emph{falso
positivo}. Há uma conclusão de que existe um efeito quando na verdade
ele não existe. A probabilidade de cometer esse tipo de erro é
\(\alpha\), o mesmo usado como nível de significância no estabelecimento
da regra de decisão.

\[
P(rejeitar \quad H_{0}|H_{0} \quad verdadeira) = \alpha
\]

Qual o valor de \(\alpha\) que pode representar forte evidencia contra
\(H_{0}\), reduzindo a possibilidade de erro tipo I?

O valor de \(\alpha\) escolhido, apesar de arbitrário, deve corresponder
a importância do que se pretende demonstrar, quanto mais importante,
menor deve ser o valor de \(\alpha\). Nesses casos, não se quer rejeitar
incorretamente \(H_{0}\) mais de 5\% das vezes. Isso corresponde ao
nível de significância mais usado de 0,05 (\(\alpha = 0,05\)). Em
algumas situações também são utilizados 0,01 e 0,10. Como mencionado, o
valor de \(\alpha\) deve ser escolhido antes de iniciar o estudo.

Existe uma outra possibilidade de erro, denominado de \textbf{erro tipo
II}, que ocorre quando a hipótese nula é realmente falsa, mas com base
na regra de decisão escolhida, não se rejeita essa hipótese nula. Nesse
caso, o resultado é um \emph{falso negativo}; não se conseguiu encontrar
um efeito que realmente existe. A probabilidade de cometer esse tipo de
erro é chamada de \(\beta\).

\[
P(não \quad rejeitar \quad H_{0}|H_{0} \quad falsa) = \beta
\]

Na construção de um teste de hipótese, o erro tipo II é considerado
menos grave que o erro tipo I. Entretanto, ele é bastante importante.
Tradicionalmente, adota-se o limite de 0,10 a 0,20 para o erro tipo II.

Na Figura~\ref{fig-erros} estão resumidas as possíveis consequências na
tomada de decisão em um teste de hipótese (104).

\begin{figure}

\centering{

\includegraphics[width=0.8\linewidth,height=\textheight,keepaspectratio]{index_files/mediabag/wL61R9C.png}

}

\caption{\label{fig-erros}Tomada de decisão e erros}

\end{figure}%

\subsection{Exemplo (continuação)}\label{sec-exeth1}

Continuando com o exemplo da Seção~\ref{sec-exeth}, aceita-se que os
pesos dos recém-nascidos de ambas as amostras tenham distribuição normal
e que as variâncias são semelhantes. Apesar de o desvio padrão
(\(\sigma\)) da \texttt{população-alvo} ser conhecido (440, 433g), será
suposto que ele é desconhecido \footnote{Esta foi uma suposição inicial!
  Fingiu-se que os dados do arquivo \texttt{dadosMater.xlsx} com o
  filtro para as gestações a termo é a ``população''}. Portanto, o teste
\emph{t} de amostras independentes será o teste escolhido como o teste
estatístico. A hipótese alternativa é bilateral e o \(\alpha\) = 0,05.\\
A distribuição \emph{t} é dependente dos grau de liberdade, que para
duas amostras independentes é igual \(gl=n_1+n_2-2\). Para os dados em
uso, tem-se:

\begin{Shaded}
\begin{Highlighting}[]
\NormalTok{ n1 }\OtherTok{\textless{}{-}}\NormalTok{ resumo}\SpecialCharTok{$}\NormalTok{n[}\DecValTok{1}\NormalTok{]}
\NormalTok{ n2 }\OtherTok{\textless{}{-}}\NormalTok{ resumo}\SpecialCharTok{$}\NormalTok{n[}\DecValTok{2}\NormalTok{]}
\NormalTok{ gl }\OtherTok{\textless{}{-}}\NormalTok{ n1 }\SpecialCharTok{+}\NormalTok{ n2 }\SpecialCharTok{{-}} \DecValTok{2}
\NormalTok{ gl}
\end{Highlighting}
\end{Shaded}

\begin{verbatim}
[1] 198
\end{verbatim}

Para o nível de significância escolhido, o valor crítico de \emph{t}
para \emph{gl} = 198 e uma hipótese alternativa bilateral pode ser
obtido da seguinte maneira:

\begin{Shaded}
\begin{Highlighting}[]
\NormalTok{alpha }\OtherTok{\textless{}{-}} \FloatTok{0.05}
\NormalTok{p }\OtherTok{\textless{}{-}} \DecValTok{1} \SpecialCharTok{{-}}\NormalTok{ alpha}\SpecialCharTok{/}\DecValTok{2}
\NormalTok{tc }\OtherTok{\textless{}{-}} \FunctionTok{round}\NormalTok{(}\FunctionTok{qt}\NormalTok{(p, gl),}\DecValTok{3}\NormalTok{)}
\NormalTok{tc}
\end{Highlighting}
\end{Shaded}

\begin{verbatim}
[1] 1.972
\end{verbatim}

A partir do cálculo do valor crítico de \emph{t}, podemos estabelecer a
regra de decisão para as hipóteses estatísticas:

\[
|t_{calculado}| < |t_{crítico}|  \to não \quad se \quad rejeita \quad H_{0}
\]

\[
|t_{calculado}| \ge |t_{crítico}| \to rejeita-se \quad H_{0}
\]

O teste \emph{t} pode ser calculado no \emph{R}, usando a função
\texttt{t\_teste()} do pacote \texttt{rstatix}. Esta função usa, entre
outros, os seguintes argumentos:

\begin{itemize}
\tightlist
\item
  \textbf{data} \(\to\) dataframe contendo as variáveis da formula;
\item
  \textbf{formula} \(\to\) uma fórmula da forma
  \texttt{x\ \textasciitilde{}\ grupo} onde \texttt{x} é uma variável
  numérica que fornece os valores dos dados e grupo é um fator;
\item
  \textbf{paired} \(\to\) lógico; indicando se o teste é pareado. Padrão
  é \texttt{FALSE};
\item
  \textbf{var.equal} \(\to\) lógico: se \texttt{TRUE}, uma variância
  combinada é usada; caso contrário, a aproximação de Welch dos graus de
  liberdade é usada
\item
  \textbf{alternative} \(\to\) \texttt{two.sided} (padrão) ou
  \texttt{greater} ou \texttt{less}.
\end{itemize}

\begin{Shaded}
\begin{Highlighting}[]
\NormalTok{teste }\OtherTok{\textless{}{-}}\NormalTok{ rstatix}\SpecialCharTok{::}\FunctionTok{t\_test}\NormalTok{(}\AttributeTok{data =}\NormalTok{ dados, }
                         \AttributeTok{formula =}\NormalTok{ pesoRN}\SpecialCharTok{\textasciitilde{}}\NormalTok{sexo, }
                         \AttributeTok{alternative =} \StringTok{"two.sided"}\NormalTok{,}
                         \AttributeTok{detailed =} \ConstantTok{TRUE}\NormalTok{)}
\NormalTok{teste}
\end{Highlighting}
\end{Shaded}

\begin{verbatim}
# A tibble: 1 x 15
  estimate estimate1 estimate2 .y.    group1 group2    n1    n2 statistic      p
*    <dbl>     <dbl>     <dbl> <chr>  <chr>  <chr>  <int> <int>     <dbl>  <dbl>
1     138.     3263.     3125. pesoRN Mascu~ Femin~   117    83      2.21 0.0281
# i 5 more variables: df <dbl>, conf.low <dbl>, conf.high <dbl>, method <chr>,
#   alternative <chr>
\end{verbatim}

A saída do teste mostra uma estatística de teste \footnote{Para ver
  todas as estatísticas do teste, basta escrever \texttt{teste\$} e
  apertar a tecla TAB do teclado e surgirá um menu para escolha.} igual
a 2.213. Esta é maior do que o \emph{t\_crítico} = 1.972,
consequentemente, rejeita-se a hipótese nula e conclui-se, com uma
confiança de 95\%, que existe uma diferença estatisticamente
significativa no peso dos recém-nascidos entre os sexos. Esta diferença
é em média igual a 138 g (IC95\%: 15, 262),
\(peso_{meninos} > peso_{meninas}\).

\section{\texorpdfstring{Valor \emph{p} do
teste}{Valor p do teste}}\label{sec-valorp}

Nas seções anteriores, foi discutido um procedimento onde se encontrou o
valor de probabilidade tal que uma dada hipótese nula é rejeitada ou não
é rejeitada, de acordo com o nível de significância, \(\alpha\), fixado,
pelo pesquisador, no início da pesquisa.

Essa abordagem do valor de probabilidade, mais comumente chamada de
abordagem do valor \emph{p}, fornece esse valor. Uma vez realizada a
pesquisa, o pesquisador calcula a \emph{probabilidade de obter um
resultado tão ou mais extremo que o observado, uma vez que a hipótese
nula é verdadeira}. O valor \emph{p} também é conhecido como \emph{nível
descritivo do teste} (105).

O objetivo de um teste estatístico é transformar em probabilidade a
magnitude do desvio verificado em relação ao valor esperado, fornecendo
o valor \emph{p}. A partir daí pode-se, também, definir a regra de
decisão, usando esse valor \emph{p}. Toma-se o valor predeterminado (em
geral, 0,05) de \(\alpha\) e, então, compara-se o valor \emph{p} com
\(\alpha\) e toma-se a decisão. Usando essa abordagem, rejeita-se a
\(H_{0}\) se o valor \emph{p} \textless{} \(\alpha\) e não se rejeita se
o valor \emph{p} \textgreater{} \(\alpha\). Costuma-se dizer que se o
valor \emph{p} \textless{} \(\alpha\), o resultado é significativo e não
significativo quando \emph{p} \textgreater{} \(\alpha\).

Uma boa parte dos pesquisadores, principalmente no início da carreira,
ficam empolgados pelo conhecimento do valor \emph{p}. Entretanto, deve
ser sempre lembrado que encontrar o valor \emph{p} não é o único foco da
pesquisa. O foco deve estar dirigido ao \emph{tamanho do efeito}
(\emph{effect size}). O valor \emph{p} obtido pelo teste estatístico,
vai informar apenas sobre a probabilidade de se cometer erro ao rejeitar
ou não rejeitar a hipóteses nula.

\subsection{Exemplo (continuação)}\label{exemplo-continuauxe7uxe3o}

O teste realizado, \texttt{t\_test()}, fornece o valor \emph{p} =
0.0281. Este valor é menor do que \(\alpha\) e leva as mesmas conclusões
da Seção~\ref{sec-exeth1}.

\section{Poder do teste}\label{poder-do-teste}

O poder do teste estatístico é a probabilidade de que um teste de
hipótese rejeite corretamente a hipótese nula quando uma hipótese
alternativa específica é verdadeira. É denotado comumente por
\(1 - \beta\) e representa a capacidade de um teste para detectar um
efeito, se esse efeito realmente existir. O poder varia de 0 a 1 e, à
medida que o poder do teste aumenta, a probabilidade \(\beta\) de
cometer um erro tipo II diminui.

\[
Poder \quad do \quad teste = P(rejeitar \quad H_{0}|H_{0} \quad falsa)
\]\\

Na Figura Figura~\ref{fig-power}, visualiza-se o poder em verde mais
escuro. Em um teste de hipótese, o valor \(\alpha\) sempre é
estabelecido com antecedência, que geralmente é definido como 0,05, de
modo que a taxa de erro do Tipo I é definida antes mesmo de se iniciar o
teste. Em seguida, pode-se calcular o valor crítico mínimo necessário
para rejeitar \(H_0\). É possível traçar uma linha da distribuição da
hipótese nula até a distribuição da hipótese alternativa e separar a
área sob a curva em duas partes. Se o valor \emph{t} calculado cair à
esquerda da linha tracejada, não se consegue rejeitar \(H_0\) quando
\(H_1\) for verdadeira e é cometido um erro do Tipo II. Se o valor
calculado cair à direita, rejeita-se \(H_0\) quando \(H_1\) é verdadeira
e a decisão é correta. Portanto, a área à direita da curva é o poder.

\begin{figure}

\centering{

\includegraphics[width=0.8\linewidth,height=\textheight,keepaspectratio]{index_files/mediabag/GsC4d6D.png}

}

\caption{\label{fig-power}Nível de significância, probabilidade de erro
tipo II, poder e nível de confiança em um teste de hipótese e a região
de rejeição da hipótese nula (à direita da linha vertical tracejada).}

\end{figure}%

O poder do teste depende de vários fatores, como:

\begin{itemize}
\tightlist
\item
  O nível de significância do teste, que é a probabilidade de rejeitar a
  hipótese nula quando ela é verdadeira (erro tipo I).
\item
  A magnitude do efeito, que é a diferença entre o valor real do
  parâmetro e o valor considerado na hipótese nula.
\item
  A variabilidade da população, que é medida pelo desvio padrão ou pela
  variância dos dados.
\item
  O tamanho da amostra, que é o número de observações coletadas para o
  teste.
\end{itemize}

Em geral, pode-se dizer:

\begin{itemize}
\tightlist
\item
  quanto maior o nível de significância, maior o poder do teste;\\
\item
  quanto maior a magnitude do efeito, maior o poder do teste;\\
\item
  quanto menor a variabilidade da população, maior o poder do teste;\\
\item
  quanto maior o tamanho da amostra, maior o poder do teste.\\
  Existem diferentes métodos para calcular o poder do teste, dependendo
  do tipo de teste e da distribuição dos dados. Por exemplo, para um
  teste de uma média com variância desconhecida, usa-se a distribuição
  \emph{t} de Student com \(n - 1\) graus de liberdade. Para um teste de
  duas proporções, usa-se a distribuição normal aproximada. A análise de
  poder é uma ferramenta útil para planejar um estudo e determinar o
  tamanho da amostra necessário para obter um poder desejado. Ela também
  pode ser usada para avaliar a qualidade de um estudo realizado e
  verificar se o teste foi capaz de detectar um efeito relevante.
\end{itemize}

\subsection{Exemplo (continuação)}\label{exemplo-continuauxe7uxe3o-1}

O teste \emph{t} retornou um resultado significativo, com valor de t =
2.213 \textgreater{} 1.972, com \emph{p} = 0.0281. Um resultado
significativo não informa sobre a magnitude do efeito. Para isso,
lançamos mão do teste \emph{d} de Cohen que pode ser calculado, usando a
função \texttt{cohensD()} do pacote \texttt{lsr}:

\begin{Shaded}
\begin{Highlighting}[]
\NormalTok{d }\OtherTok{\textless{}{-}}\NormalTok{ lsr}\SpecialCharTok{::}\FunctionTok{cohensD}\NormalTok{ (}\AttributeTok{data =}\NormalTok{ dados, }\AttributeTok{formula =}\NormalTok{ pesoRN }\SpecialCharTok{\textasciitilde{}}\NormalTok{ sexo)}
\NormalTok{d}
\end{Highlighting}
\end{Shaded}

\begin{verbatim}
[1] 0.31676
\end{verbatim}

Na Seção~\ref{sec-cohen}, se entrará em maiores detalhes, por enquanto,
será assumido que a magnitude do efeito é pequena.

De posse do valor do \emph{d} de Cohen, é possível calcular, através da
função \texttt{pwr.t.test()} do pacote \texttt{pwr}, o poder do teste
estatístico. Os argumentos dessa função são:

\begin{itemize}
\tightlist
\item
  \textbf{n} \(\to\) número de observações por amostra;
\item
  \textbf{d} \(\to\) magnitude do efeito, d de Cohen;
\item
  \textbf{sig.level} \(\to\) nível de significância (padrão = 0.05);
\item
  \textbf{power} \(\to\) poder do teste;
\item
  \textbf{type} \(\to\) tipo de teste (one- , two- ou paired-samples);
\item
  \textbf{alternative} \(\to\) hipótese alternativa, deve ser
  ``one-sided'' ou ``two-sided (padrão).
\end{itemize}

O parâmetro que se quer calcular deve ser passado como \texttt{NULL}.
Assim, o poder do teste estatístico do exemplo é:

\begin{Shaded}
\begin{Highlighting}[]
\NormalTok{poder }\OtherTok{\textless{}{-}}\NormalTok{ pwr}\SpecialCharTok{::}\FunctionTok{pwr.t.test}\NormalTok{(}\AttributeTok{n =} \DecValTok{150}\NormalTok{,}
                         \AttributeTok{d =}\NormalTok{ d,}
                         \AttributeTok{sig.level =} \FloatTok{0.05}\NormalTok{, }
                         \AttributeTok{power =} \ConstantTok{NULL}\NormalTok{,}
                         \AttributeTok{type =} \StringTok{"two.sample"}\NormalTok{,}
                         \AttributeTok{alternative =} \StringTok{"two.sided"}\NormalTok{)}
\NormalTok{poder}
\end{Highlighting}
\end{Shaded}

\begin{verbatim}

     Two-sample t test power calculation 

              n = 150
              d = 0.31676
      sig.level = 0.05
          power = 0.7806564
    alternative = two.sided

NOTE: n is number in *each* group
\end{verbatim}

A saída mostra que no lugar do \texttt{NULL}, aparece o poder do teste
estatístico. Ou seja, o poder foi de 0.781 , consequentemente, como
\(\beta = 1 – Poder\), então, \(\beta\) = 0.219.

O poder geralmente é definido em 0,80 (ou 0,90). Isto significa que se
existirem efeitos verdadeiros a serem encontrados em 100 estudos
diferentes com 80\% de poder, apenas 80 em 100 testes estatísticos irão
realmente detectá-los. Se não for garantido poder suficiente, é possível
que nenhum efeito seja detectado, por isso, deve-se calcular o tamanho
amostral necessário, antes de iniciar qualquer estudo, para garantir o
poder pretendido.

\part{Parte VI - Testes Paramétricos}

\chapter{Comparação entre duas médias}\label{sec-testet}

\section{Pacotes necessários para este
capítulo}\label{pacotes-necessuxe1rios-para-este-capuxedtulo-1}

\begin{Shaded}
\begin{Highlighting}[]
\NormalTok{pacman}\SpecialCharTok{::}\FunctionTok{p\_load}\NormalTok{(car, }
\NormalTok{               effectsize, }
\NormalTok{               flextable,}
\NormalTok{               ggpubr, }
\NormalTok{               ggsci, }
\NormalTok{               knitr,}
\NormalTok{               readxl, }
\NormalTok{               rstatix, }
\NormalTok{               tidyverse)}
\end{Highlighting}
\end{Shaded}

\section{\texorpdfstring{Teste \emph{t} para amostras
independentes}{Teste t para amostras independentes}}\label{teste-t-para-amostras-independentes}

O teste \emph{t} de amostras independentes é usado para comparar duas
médias de amostras de grupos não relacionados. Isso significa que há
pessoas diferentes fornecendo escores para cada grupo. O objetivo desse
teste é determinar se as amostras são diferentes uma da outra.

\subsection{Dados usados neste
capítulo}\label{dados-usados-neste-capuxedtulo-1}

Suponha que. em uma determinada Universidade, tenham sido coletadas as
notas de Bioestatística de uma turma de 40 alunos. Estes dados estão
\href{https://github.com/petronioliveira/Arquivos/blob/main/dadosNotas.xlsx}{\textbf{aqui}}.
Salve o mesmo no seu diretório de trabalho para a leitura dos dados.

\subsubsection{Leitura dos dados}\label{leitura-dos-dados-1}

Para a leitura dos dados, será usada a função \texttt{read\_excel()}
incluída no pacote \texttt{readxl}, que precisa ser instalado e
carregado. Os dados serão recebidos por um objeto que será denominado de
\texttt{dados}:

\begin{Shaded}
\begin{Highlighting}[]
\NormalTok{dados }\OtherTok{\textless{}{-}}\NormalTok{ readxl}\SpecialCharTok{::}\FunctionTok{read\_excel}\NormalTok{(}\StringTok{"dados/dadosNotas.xlsx"}\NormalTok{) }\SpecialCharTok{\%\textgreater{}\%} 
  \FunctionTok{mutate}\NormalTok{(}\AttributeTok{sexo =} \FunctionTok{factor}\NormalTok{(sexo,}
                       \AttributeTok{levels =} \FunctionTok{c}\NormalTok{(}\DecValTok{1}\NormalTok{, }\DecValTok{2}\NormalTok{),}
                       \AttributeTok{labels =} \FunctionTok{c}\NormalTok{(}\StringTok{"Masculino"}\NormalTok{, }\StringTok{"Feminino"}\NormalTok{)))}

\FunctionTok{str}\NormalTok{(dados)}
\end{Highlighting}
\end{Shaded}

\begin{verbatim}
tibble [40 x 2] (S3: tbl_df/tbl/data.frame)
 $ notas: num [1:40] 63.1 76.3 57.7 66.9 73.1 70.3 63.6 75.7 73.5 83 ...
 $ sexo : Factor w/ 2 levels "Masculino","Feminino": 2 2 2 2 2 2 2 2 2 2 ...
\end{verbatim}

Observa-se que existem 40 alunos, sendo 20 mulheres e 0 homens. A
variável \texttt{notas} é uma variável numérica que corresponde a a nota
centesimal e \texttt{sexo} é uma variável categórica.

\subsubsection{Exploração e resumo dos
dados}\label{explorauxe7uxe3o-e-resumo-dos-dados}

Inicialmente, calcular a média e o desvio padrão da variável
\texttt{notas} de acordo com \texttt{sexo}, usando a função
\texttt{group\_by\ ()} e \texttt{summarise} do pacote \texttt{dplyr}

\begin{Shaded}
\begin{Highlighting}[]
\NormalTok{resumo }\OtherTok{\textless{}{-}}\NormalTok{ dados }\SpecialCharTok{\%\textgreater{}\%} 
\NormalTok{  dplyr}\SpecialCharTok{::}\FunctionTok{group\_by}\NormalTok{(sexo) }\SpecialCharTok{\%\textgreater{}\%} 
\NormalTok{  dplyr}\SpecialCharTok{::} \FunctionTok{summarise}\NormalTok{(}\AttributeTok{n =} \FunctionTok{n}\NormalTok{(),}
                    \AttributeTok{media =} \FunctionTok{mean}\NormalTok{(notas, }\AttributeTok{na.rm =} \ConstantTok{TRUE}\NormalTok{),}
                    \AttributeTok{dp =} \FunctionTok{sd}\NormalTok{(notas, }\AttributeTok{na.rm =} \ConstantTok{TRUE}\NormalTok{),}
                    \AttributeTok{mediana =} \FunctionTok{median}\NormalTok{(notas, }\AttributeTok{na.rm =} \ConstantTok{TRUE}\NormalTok{),}
                    \AttributeTok{Q1 =} \FunctionTok{quantile}\NormalTok{(notas,}\FloatTok{0.25}\NormalTok{, }\AttributeTok{na.rm =} \ConstantTok{TRUE}\NormalTok{),    }
                    \AttributeTok{Q3 =} \FunctionTok{quantile}\NormalTok{(notas, }\FloatTok{0.75}\NormalTok{, }\AttributeTok{na.rm =} \ConstantTok{TRUE}\NormalTok{),}
                    \AttributeTok{me =} \FloatTok{1.96} \SpecialCharTok{*}\NormalTok{ dp}\SpecialCharTok{/}\FunctionTok{sqrt}\NormalTok{(n)) }
\NormalTok{resumo}
\end{Highlighting}
\end{Shaded}

\begin{verbatim}
# A tibble: 2 x 8
  sexo          n media    dp mediana    Q1    Q3    me
  <fct>     <int> <dbl> <dbl>   <dbl> <dbl> <dbl> <dbl>
1 Masculino    20  59.9  7.34    59.2  54.3  63.5  3.22
2 Feminino     20  68.4  7.79    68.6  63.2  73.8  3.41
\end{verbatim}

A saída informa que a média das notas das mulheres é 59.9, bem acima das
notas dos homens, mostrando uma diferença nos escores de -8.5. Parece
que o desempenho das mulheres em Bioestatística é melhor do que o dos
homens!

Além do resumo numérico, é interessante construir um gráfico do tipo
boxplot (Figura~\ref{fig-bxpnotas}), usando o pacote \texttt{ggplot2}
(veja Seção~\ref{sec-ggplot2}) para observar a distribuição dos dados:

\begin{Shaded}
\begin{Highlighting}[]
\NormalTok{ggplot2}\SpecialCharTok{::}\FunctionTok{ggplot}\NormalTok{(}\AttributeTok{data =}\NormalTok{ dados, }\FunctionTok{aes}\NormalTok{(}\AttributeTok{x =}\NormalTok{ sexo, }
                                  \AttributeTok{y =}\NormalTok{ notas, }
                                  \AttributeTok{fill =}\NormalTok{ sexo)) }\SpecialCharTok{+} 
  \FunctionTok{geom\_errorbar}\NormalTok{(}\AttributeTok{stat =} \StringTok{"boxplot"}\NormalTok{, }\AttributeTok{width =} \FloatTok{0.1}\NormalTok{) }\SpecialCharTok{+}
  \FunctionTok{geom\_boxplot}\NormalTok{() }\SpecialCharTok{+}
  \FunctionTok{geom\_jitter}\NormalTok{(}\AttributeTok{width =} \FloatTok{0.05}\NormalTok{) }\SpecialCharTok{+}
  \FunctionTok{scale\_fill\_manual}\NormalTok{(}\AttributeTok{values =} \FunctionTok{c}\NormalTok{(}\StringTok{"cyan"}\NormalTok{,}\StringTok{"pink2"}\NormalTok{)) }\SpecialCharTok{+}
  \FunctionTok{labs}\NormalTok{ (}\AttributeTok{x =} \StringTok{"Sexo"}\NormalTok{, }
        \AttributeTok{y =} \StringTok{"Notas"}\NormalTok{) }\SpecialCharTok{+} 
  \FunctionTok{theme\_classic}\NormalTok{(}\AttributeTok{base\_size =} \DecValTok{13}\NormalTok{) }\SpecialCharTok{+} 
  \FunctionTok{theme}\NormalTok{(}\AttributeTok{legend.position=}\StringTok{"none"}\NormalTok{)}
\end{Highlighting}
\end{Shaded}

\begin{figure}[H]

\centering{

\includegraphics[width=0.7\linewidth,height=0.7\textheight]{14-teste-t_files/figure-pdf/fig-bxpnotas-1.pdf}

}

\caption{\label{fig-bxpnotas}Boxplot dos dados}

\end{figure}%

Os boxplot sugerem que as notas dos alunos diferem, de acordo o sexo.

\subsection{Definição das hipóteses
estatísticas}\label{definiuxe7uxe3o-das-hipuxf3teses-estatuxedsticas}

As hipóteses comparam as médias dos dois grupos. Para um teste bicaudal,
as hipóteses são escritas como:

\[
H_{0}: \mu_{F} = \mu_{M}
\]

\[
H_{1}: \mu_{F} \neq \mu_{M}
\]

\subsection{Definição da regra de
decisão}\label{definiuxe7uxe3o-da-regra-de-decisuxe3o}

O nível significância, \(\alpha\), escolhido é igual a \texttt{0.05}. A
distribuição \emph{t} é dependente dos graus de liberdade, dados por:

No exemplo,

\begin{Shaded}
\begin{Highlighting}[]
\NormalTok{n1 }\OtherTok{\textless{}{-}}\NormalTok{ resumo}\SpecialCharTok{$}\NormalTok{n[}\DecValTok{1}\NormalTok{]}
\NormalTok{n2 }\OtherTok{\textless{}{-}}\NormalTok{ resumo}\SpecialCharTok{$}\NormalTok{n[}\DecValTok{2}\NormalTok{]}
\NormalTok{gl }\OtherTok{\textless{}{-}}\NormalTok{ n1 }\SpecialCharTok{+}\NormalTok{ n2 }\SpecialCharTok{{-}} \DecValTok{2}
\NormalTok{gl}
\end{Highlighting}
\end{Shaded}

\begin{verbatim}
[1] 38
\end{verbatim}

Para um \(\alpha = 0,05\), o valor crítico de \emph{t} para gl =38 para
uma hipótese alternativa bicaudal é obtido com a função
\texttt{qt\ (p,\ df)}, onde \(df = gl\) e \(p = 1 - \alpha/2\)

\begin{Shaded}
\begin{Highlighting}[]
\NormalTok{alpha }\OtherTok{\textless{}{-}} \FloatTok{0.05}
\NormalTok{p }\OtherTok{\textless{}{-}} \DecValTok{1} \SpecialCharTok{{-}}\NormalTok{ alpha}\SpecialCharTok{/}\DecValTok{2}
\NormalTok{tc }\OtherTok{\textless{}{-}} \FunctionTok{round}\NormalTok{ (}\FunctionTok{qt}\NormalTok{((}\DecValTok{1}\SpecialCharTok{{-}}\NormalTok{alpha}\SpecialCharTok{/}\DecValTok{2}\NormalTok{), gl), }\DecValTok{3}\NormalTok{)}
\NormalTok{tc}
\end{Highlighting}
\end{Shaded}

\begin{verbatim}
[1] 2.024
\end{verbatim}

Portanto, se

\[
|t_{calculado}| < |t_{crítico}|  \to não \quad se \quad rejeita \quad H_{0}
\]

\[
t_{calculado}| \ge t_{crítico}| \to rejeita-se \quad H_{0}
\]

\subsection{Teste estatístico}\label{teste-estatuxedstico-1}

Para determinar se existe uma diferença estatisticamente significativa
entre as médias das notas de dois grupos independentes, será usado o
teste \emph{t} para duas amostras independentes, também conhecido como
teste \emph{t} de Student, baseado na distribuição de mesmo nome.

\subsubsection{\texorpdfstring{Lógica do teste
\emph{t}}{Lógica do teste t}}\label{luxf3gica-do-teste-t}

O teste \emph{t} compara as médias de duas amostras independentes,
usando o erro padrão como métrica da diferença entre essas médias.
Quanto maior o valor de \emph{t} , maior a probabilidade de que as
amostras pertençam a grupos diferentes, ocorrendo nessas circunstâncias
a rejeição da hipótese nula (106).

Calcula-se o teste \emph{t} com a seguinte equação:

\[
t = \frac{(\bar{x}_1 - \bar{x}_2) - (\mu_1 - \mu_2)}{EP_{d}}
\]

Onde \(EP_d\) é o erro padrão da diferença entre a médias
\(\bar{x}_1 - \bar{x}_2\). Se a hipótese nula for verdadeira, as
amostras foram retiradas da mesma população e, portanto,
\(\mu_1 - \mu_2 = 0\). Assim, a equação fica:

\[
t = \frac{(\bar{x}_1 - \bar{x}_2)}{EP_d}
\]

O erro padrão da diferença \(\bar{x}_1 - \bar{x}_2\) é calculado de
maneiras diferentes:

\begin{enumerate}
\def\labelenumi{\arabic{enumi})}
\tightlist
\item
  Se a variâncias nos dois grupos forem iguais, usa-se:
\end{enumerate}

\[
EP_d = \sqrt{s_o^2(\frac{1}{n_1}+\frac{1}{n_2})}
\]

Onde \(s_o^2\) é a variância combinada ou conjugada que é, simplesmente,
a média ponderada das variância dos grupos:

\[
s_0^2 = \frac{(n_1 - 1)s_1^2 + (n_2 -1)s_2^2}{(n_1 -1)+ (n_2-1)}
\] Quando os grupos têm o mesmo tamanho (\(n_1 = n_2\)), \(s_o^2\) é
simplesmente a média aritmética da variância dos grupos:

\[
s_0^2 = \frac {s_1^2 + s_2^2}{2}
\]

\[
EP_d = \sqrt{\frac{2 s_o^2}{n}}
\]

\begin{enumerate}
\def\labelenumi{\arabic{enumi})}
\setcounter{enumi}{1}
\tightlist
\item
  Se as variâncias dos dois grupos forem diferentes:
\end{enumerate}

\[
EP_d = \sqrt{\frac{s_1^2}{n_1}+\frac{s_2^2}{n_2}}
\]

Esta explicação da lógica e dedução da estatística de teste serve para
uma melhor compreensão de como o teste funciona, mas para executar um
teste \emph{t} não há necessidade disso, basta saber como encaminhar ao
R e como interpretar o resultado fornecido por ele.

\subsubsection{\texorpdfstring{Pressupostos do teste
\emph{t}}{Pressupostos do teste t}}\label{sec-homovar}

O teste \emph{t} assume que:

\begin{enumerate}
\def\labelenumi{\arabic{enumi}.}
\tightlist
\item
  As amostras são independentes;
\item
  Deve haver distribuição normal. Entretanto, quando as amostras são
  grandes (teorema do limite central), isso não é muito importante;
\item
  Exista homocedasticidade, ou seja, as variâncias dos grupos devem ser
  iguais.
\end{enumerate}

Violar o pressuposto de número 3 tem importância se os tamanhos dos
grupos forem diferentes. Se os grupos tiverem o mesmo tamanho e a
amostra for grande, este pressuposto torna-se menos importante, não
preocupando muito se essa hipótese foi violada (107). O pressuposto tem
mais importância em grupos pequenos e desiguais. Existe um teste,
denominado \emph{teste t de Welch} que corrige essa violação. É possível
portanto, esquecer esse pressuposto e fazer o teste de Welch sempre.

\ul{Avaliação da normalidade}

Uma boa parte dos procedimentos estatísticos são testes paramétricos
\footnote{Teste paramétricos são testes estatísticos que se baseiam nos
  padrões da distribuição populacional da variável em estudo, por
  exemplo, a distribuição normal é descrita por dois parâmetros -- média
  e desvio padrão -- que são suficientes para se conhecer as
  probabilidades. Os testes que não requerem a especificação da forma de
  distribuição da população, ou seja, têm \textbf{distribuição livre},
  são denominados de não paramétricos.} com base na distribuição normal.
Ou seja, se assume que a distribuição dos dados segue o modelo da
distribuição normal. Se essa suposição não for atendida, a lógica por
trás do teste de hipóteses pode ser violada.

Pode-se verificar a normalidade de maneira visual, observando o
comportamento dos dados através de gráficos como o próprio boxplot
(Figura~\ref{fig-bxpnotas}), onde se observa que as medianas dos
boxplots se encontra praticamente no centro das caixas. Outra forma, é o
\emph{gráfico Q-Q} (Figura~\ref{fig-qqnotas}). O \emph{gráfico QQ} (ou
gráfico quantil-quantil) desenha a correlação entre uma determinada
amostra e a distribuição normal. Uma linha de referência de 45 graus
também é plotada. Um gráfico Q-Q é um gráfico de dispersão criado
plotando dois conjuntos de quantis um contra o outro. Se ambos os
conjuntos de quantis vierem da mesma distribuição, observa-se os pontos
formando uma linha aproximadamente reta. Se os valores caírem na
diagonal do gráfico, a variável é normalmente distribuída. Os desvios da
diagonal mostram desvios da normalidade. Para desenhar um gráfico Q-Q
pode ser usado a função \texttt{ggqqplot\ ()}\footnote{Veja também a
  Seção~\ref{sec-qqplot}.} do pacote \texttt{ggpubr} que produz um
gráfico QQ normal com uma linha de referência, acompanhada de area
sombreada, correspondente ao IC95\%.

\begin{Shaded}
\begin{Highlighting}[]
\FunctionTok{ggqqplot}\NormalTok{(dados, }\AttributeTok{x =} \StringTok{"notas"}\NormalTok{, }\AttributeTok{color =} \StringTok{"sexo"}\NormalTok{) }\SpecialCharTok{+}
  \FunctionTok{labs}\NormalTok{(}\AttributeTok{y =} \StringTok{"Notas"}\NormalTok{,}
       \AttributeTok{x =} \StringTok{"Quantis teóricos"}\NormalTok{)}
\end{Highlighting}
\end{Shaded}

\begin{figure}[H]

\centering{

\includegraphics[width=0.7\linewidth,height=0.7\textheight]{14-teste-t_files/figure-pdf/fig-qqnotas-1.pdf}

}

\caption{\label{fig-qqnotas}Gráficos Q-Q}

\end{figure}%

Observando os gráficos, verifica-se que a variável \texttt{notas} tem
uma distribuição visualmente normal aceitável em ambas populações, pois
o histograma se ajusta à curva normal e os gráficos Q-Q mostram que os
dados seguem aproximadamente a linha diagonal.

Outra maneira de analisar a normalidade é verificar se a distribuição
como um todo se desvia de uma distribuição normal comparável. Para isso,
usam-se \emph{testes estatísticos de normalidade}. Os dois principais
são o \emph{teste de Shapiro-Wilk} e o \emph{teste de Kolmogorov-Smirnov
(K-S)}.

Esses testes comparam os dados da amostra com um conjunto de valores
normalmente distribuídos com a mesma média e desvio padrão. Se o teste
não for significativo (\emph{P} \textgreater{} 0,05), informa-se que a
distribuição da amostra não é significativamente diferente de uma
distribuição normal. Se, no entanto, o teste for significativo (\emph{P}
\(\le\) 0,05), a distribuição em questão será significativamente
diferente de uma distribuição normal.

O método de Shapiro-Wilk é amplamente recomendado para teste de
normalidade (108), (109), (110).

\begin{Shaded}
\begin{Highlighting}[]
\NormalTok{sw }\OtherTok{\textless{}{-}}\NormalTok{ dados }\SpecialCharTok{\%\textgreater{}\%} 
\NormalTok{  dplyr}\SpecialCharTok{::}\FunctionTok{group\_by}\NormalTok{(sexo) }\SpecialCharTok{\%\textgreater{}\%}
\NormalTok{  rstatix}\SpecialCharTok{::}\FunctionTok{shapiro\_test}\NormalTok{(notas)}
\NormalTok{sw}
\end{Highlighting}
\end{Shaded}

\begin{verbatim}
# A tibble: 2 x 4
  sexo      variable statistic     p
  <fct>     <chr>        <dbl> <dbl>
1 Masculino notas        0.972 0.802
2 Feminino  notas        0.973 0.812
\end{verbatim}

A saída mostra que ambos valores \emph{P} do teste, 0.802 e 0.812, estão
acima de 0,05, corroborando com a não rejeição da normalidade dos dados.

\ul{Homogeneidade da Variância}

Na visualização da Figura~\ref{fig-bxpnotas}, nos dois grupos de alunos,
observa-se que há, entre os limites inferior e superior, uma dispersão
das notas em torno da região central. Esta dispersão parece ser
semelhante nos grupos. Isto sugere que haja \emph{homogeneidade das
variâncias}.

Portanto, homogeneidade da variância é o pressuposto de que a dispersão
das medidas é aproximadamente igual em diferentes grupos de casos, ou
que a dispersão dos valores são aproximadamente iguais em pontos
diferentes da variável preditora.

Além do aspecto visual, a homogeneidade da variância pode ser testada
com o \emph{teste de Levene}. Neste teste, a \(H_{0}\) é todas as
variâncias são iguais. No \emph{R}, a função que calcula o teste é
\texttt{leveneTest()} do pacote \texttt{car} (111). Os argumentos são:

\begin{itemize}
\tightlist
\item
  \textbf{y} \(\to\) variável de resposta para o método padrão ou um
  objeto \texttt{lm} ou \texttt{fórmula}. Se \texttt{y} for um objeto de
  modelo linear ou uma fórmula, as variáveis do lado direito do modelo
  devem ser todas fatores e devem ser completamente cruzadas;
\item
  \textbf{group} \(\to\) fator que define os grupos;
\item
  \textbf{center} \(\to\) O nome de uma função para calcular o centro de
  cada grupo; \texttt{mean} fornece o teste de Levene original; o
  padrão, \texttt{median}, fornece um teste mais robusto;
\item
  \textbf{data} \(\to\) conjunto de dados para avaliar a
  \texttt{formula}.
\end{itemize}

\begin{Shaded}
\begin{Highlighting}[]
\NormalTok{levene }\OtherTok{\textless{}{-}}\NormalTok{ car}\SpecialCharTok{::}\FunctionTok{leveneTest}\NormalTok{(notas}\SpecialCharTok{\textasciitilde{}}\NormalTok{sexo, }
                          \AttributeTok{center =}\NormalTok{ mean, }
                          \AttributeTok{data =}\NormalTok{ dados)}
\NormalTok{levene}
\end{Highlighting}
\end{Shaded}

\begin{verbatim}
Levene's Test for Homogeneity of Variance (center = mean)
      Df F value Pr(>F)
group  1  0.4575 0.5029
      38               
\end{verbatim}

A saída do teste de Levene retorna um valor \emph{p} \textgreater{}
0,05, confirma a impressão visual dos boxplots de que os grupos têm
homogeneidade das variâncias, portanto a hipótese nula de igualdade das
variâncias não pode ser rejeitada.

Um outro teste que compara duas variância poderia ser usado. É o teste F
que pode ser calculado com a função \texttt{var.test()} do pacote
\texttt{stats}, incluído no R base. Seus argumentos pode ser consultados
na ajuda do R.

\begin{Shaded}
\begin{Highlighting}[]
\NormalTok{teste.Var }\OtherTok{\textless{}{-}} \FunctionTok{var.test}\NormalTok{(notas}\SpecialCharTok{\textasciitilde{}}\NormalTok{sexo, }\AttributeTok{alternative =} \StringTok{"two.sided"}\NormalTok{ , }\AttributeTok{data =}\NormalTok{ dados)}
\NormalTok{teste.Var}
\end{Highlighting}
\end{Shaded}

\begin{verbatim}

    F test to compare two variances

data:  notas by sexo
F = 0.88701, num df = 19, denom df = 19, p-value = 0.7966
alternative hypothesis: true ratio of variances is not equal to 1
95 percent confidence interval:
 0.3510912 2.2409994
sample estimates:
ratio of variances 
         0.8870148 
\end{verbatim}

A saída do teste permite uma conclusão igual ao teste de Levene, pois o
valor \emph{p} = 0.7966.

\subsubsection{\texorpdfstring{Execução do teste \emph{t} de
Student}{Execução do teste t de Student}}\label{execuuxe7uxe3o-do-teste-t-de-student}

Os pressupostos do teste não foram violados, portanto ele pode ser
realizado com confiança. Será utilizado a função \texttt{t\_test()} do
pacote \texttt{rstatix} (112) para calcular o teste \emph{t} para
amostras independentes. Ele fornece uma estrutura compatível com
operador pipe \%\textgreater\% (\emph{pipe-friendly}) para executar
testes \emph{t} de uma e duas amostras. Para consultar os argumentos,
consulte a Seção~\ref{sec-exeth1} ou a ajuda do \emph{RStudio}.

\begin{Shaded}
\begin{Highlighting}[]
\NormalTok{ teste }\OtherTok{\textless{}{-}}\NormalTok{ dados }\SpecialCharTok{\%\textgreater{}\%}\NormalTok{ rstatix}\SpecialCharTok{::}\FunctionTok{t\_test}\NormalTok{(}\AttributeTok{formula =}\NormalTok{ notas }\SpecialCharTok{\textasciitilde{}}\NormalTok{ sexo,}
                                    \AttributeTok{detailed =} \ConstantTok{TRUE}\NormalTok{,}
                                    \AttributeTok{var.equal =} \ConstantTok{TRUE}\NormalTok{)}
\NormalTok{ teste}
\end{Highlighting}
\end{Shaded}

\begin{verbatim}
# A tibble: 1 x 15
  estimate estimate1 estimate2 .y.   group1 group2    n1    n2 statistic       p
*    <dbl>     <dbl>     <dbl> <chr> <chr>  <chr>  <int> <int>     <dbl>   <dbl>
1    -8.51      59.9      68.4 notas Mascu~ Femin~    20    20     -3.56 0.00102
# i 5 more variables: df <dbl>, conf.low <dbl>, conf.high <dbl>, method <chr>,
#   alternative <chr>
\end{verbatim}

A saída retorna a estimativa da diferença média (-8.515), as estimativas
das médias dos grupos (arredondadas), a estatística do teste
(-3.5586328) o valor P (0.00102), graus de liberdade (38)\footnote{Se as
  variâncias forem diferentes (\texttt{var.equal\ =\ FALSE}), o teste
  calcula os graus de liberdade pela fórmula de Welch, bem mais
  complicada.} e outras métricas.

Também é possível ver os resultados do teste \emph{t} , usando o objeto
\texttt{teste} que os recebeu. Por exemplo, os limites inferior
(\texttt{conf.low}) e superior (\texttt{conf.high}) do intervalo de
confiança de 95\% da estimativa da diferença entre as médias.

\begin{Shaded}
\begin{Highlighting}[]
\NormalTok{IC95 }\OtherTok{\textless{}{-}} \FunctionTok{round}\NormalTok{(}\FunctionTok{c}\NormalTok{(teste}\SpecialCharTok{$}\NormalTok{conf.low, teste}\SpecialCharTok{$}\NormalTok{conf.high),}\DecValTok{3}\NormalTok{)}
\NormalTok{IC95}
\end{Highlighting}
\end{Shaded}

\begin{verbatim}
[1] -13.359  -3.671
\end{verbatim}

\subsection{Conclusão}\label{conclusuxe3o}

Como \(|t_{calculado}|\) = -3.559 \textgreater{} \(|t_{0,05;58}|\) =
2.024, rejeita-se \(H_{0}\). Observa-se que o valor \emph{P} é muito
pequeno (0.00102) e, portanto, a diferença observada nas médias dos dois
grupos deve ser assumida como significativa.

Assim, pode-se admitir que as médias das notas são diferentes, com
probabilidade de erro extremamente pequena. A estimativa da diferença
média (\(\mu_1 - \mu_2\)) é fornecida pelo intervalo de confiança de
95\% (-13.359, -3.671). Observe que o valor zero não está contido no
intervalo e isto confirma a não significância estatística da diferença.

Concluindo, as notas de Bioestatística das mulheres e as notas dos
homens são diferentes, a diferença (\(\mu_1 - \mu_2\)) encontrada é
estatisticamente significativa (\emph{t} = -3.559, gl = 38, \emph{P} =
0.00102), com uma confiança de 95\%.

Esta conclusão pode ser visualizada em um gráfico
(Figura~\ref{fig-bxpt}) que exibirá a saída do teste \emph{t}:

\begin{enumerate}
\def\labelenumi{\arabic{enumi})}
\tightlist
\item
  Construir dois boxplots, usando o \texttt{ggplot2} com cores do
  \emph{New England Journal of Medicine} (NEJM), do pacote
  \texttt{ggsci} . Atribuir a um objeto \texttt{bp}:
\end{enumerate}

\begin{Shaded}
\begin{Highlighting}[]
\NormalTok{bp }\OtherTok{\textless{}{-}} \FunctionTok{ggplot}\NormalTok{(dados, }\FunctionTok{aes}\NormalTok{(}\AttributeTok{x=}\NormalTok{sexo, }\AttributeTok{y=}\NormalTok{notas)) }\SpecialCharTok{+}
    \FunctionTok{geom\_errorbar}\NormalTok{(}\AttributeTok{stat =} \StringTok{"boxplot"}\NormalTok{, }\AttributeTok{width =} \FloatTok{0.1}\NormalTok{)}\SpecialCharTok{+}
    \FunctionTok{geom\_boxplot}\NormalTok{(}\FunctionTok{aes}\NormalTok{(}\AttributeTok{fill =}\NormalTok{ sexo),}
                 \AttributeTok{color =} \StringTok{"black"}\NormalTok{)}\SpecialCharTok{+}
    \FunctionTok{scale\_color\_nejm}\NormalTok{() }\SpecialCharTok{+}
    \FunctionTok{theme\_classic}\NormalTok{(}\AttributeTok{base\_size =} \DecValTok{13}\NormalTok{) }\SpecialCharTok{+}
    \FunctionTok{theme}\NormalTok{(}\AttributeTok{legend.position=}\StringTok{"none"}\NormalTok{)}
\end{Highlighting}
\end{Shaded}

\begin{enumerate}
\def\labelenumi{\arabic{enumi})}
\setcounter{enumi}{1}
\tightlist
\item
  Adicionar ao boxplot novos rótulos e os testes realizados:
\end{enumerate}

\begin{Shaded}
\begin{Highlighting}[]
\NormalTok{bp }\SpecialCharTok{+}
 \FunctionTok{labs}\NormalTok{(}\AttributeTok{x =} \StringTok{"Sexo"}\NormalTok{, }
      \AttributeTok{y =} \StringTok{"Notas"}\NormalTok{, }
      \AttributeTok{title =} \StringTok{"Notas de Bioestatística"}\NormalTok{,}
      \AttributeTok{subtitle =}\NormalTok{ rstatix}\SpecialCharTok{::}\FunctionTok{get\_test\_label}\NormalTok{(}\AttributeTok{stat.test =}\NormalTok{ teste,}
                                           \AttributeTok{correction =} \StringTok{"none"}\NormalTok{,}
                                           \AttributeTok{detailed =} \ConstantTok{TRUE}\NormalTok{,}
                                           \AttributeTok{type =} \StringTok{"expression"}\NormalTok{,}
                                           \AttributeTok{p.col =} \StringTok{"p"}\NormalTok{))}
\end{Highlighting}
\end{Shaded}

\begin{figure}[H]

\centering{

\includegraphics[width=0.7\linewidth,height=0.7\textheight]{14-teste-t_files/figure-pdf/fig-bxpt-1.pdf}

}

\caption{\label{fig-bxpt}Boxplots comparando os dois grupos}

\end{figure}%

\subsection{Tamanho do Efeito}\label{tamanho-do-efeito}

A significância estatística deve ter uma atenção relativa do
pesquisador, pois ela apenas mede a probabilidade de rejeitar uma
hipótese nula, uma vez que ela seja verdadeira. Ajudam a determinar, em
uma pesquisa, a significância dos resultados encontrados em relação à
hipótese nula, mas não informam nada em relação a magnitude do efeito.
Por exemplo, mostra se determinado tratamento afeta as pessoas, mas não
dizem quanto isso as afeta.

O tamanho do efeito (\emph{effect size}) é uma medida quantitativa da
magnitude do efeito. Quanto maior o tamanho do efeito, mais forte é a
relação entre duas variáveis. É possível observar o tamanho do efeito ao
comparar dois grupos quaisquer para ver quão substancialmente diferentes
eles são.

Normalmente, em ensaios clínicos tem-se um grupo de tratamento e um
grupo de controle. O grupo de tratamento é uma intervenção que se espera
efetue um resultado específico. O valor do tamanho do efeito mostrará se
a terapia teve um efeito pequeno, médio ou grande. Isso tem mais
relevância do que simplesmente informar o tamanho do valor \emph{P}.

\subsubsection{\texorpdfstring{\emph{d} de
Cohen}{d de Cohen}}\label{sec-cohen}

Também conhecida como \emph{diferença média padronizada}, o \emph{d} de
Cohen (113) (114) é uma medida adequada e bastante popular para
encontrar a magnitude do efeito na comparação entre duas médias.

Para calcular a diferença média padronizada se verifica a diferença
entre as médias dos dois grupos e se divide pelo desvio padrão
conjugado:

\[
d = \frac{(\bar{x}_1 - \bar{x}_2)}{s_{o}}
\]

Onde,

\[
s_o =\sqrt \frac{(n_1 - 1)s_1^2 + (n_2 -1)s_2^2}{n_1 + n_2 - 2}
\]

Voltando ao exemplo das notas dos alunos de Bioestatística, o \emph{d}
de Cohen é calculado, usando a função \texttt{cohensD()} do pacote
\texttt{lsr} que usa os seguintes argumentos:

\begin{itemize}
\tightlist
\item
  \textbf{x} \(\to\) um vetor numérico de valores de dados, variável
  preditora;
\item
  \textbf{y} \(\to\) um vetor numérico de valores de dados, variável
  resposta;
\item
  \textbf{formula} \(\to\) Fórmula na forma variável \emph{resposta
  \textasciitilde{} grupo};
\item
  \textbf{data} \(\to\) dataframe ou matriz;
\item
  \textbf{method} \(\to\) Qual versão da estatística d devemos calcular?
  Os valores possíveis são \emph{pooled}(padrão), \emph{x.sd},
  \emph{y.sd}, \emph{corrected}, \emph{raw}, \emph{paired} e
  \emph{unequal}.;
\item
  \textbf{mu} \(\to\) O valor ``nulo'' contra o qual o tamanho do efeito
  deve ser medido. Quase sempre é 0 (padrão); raramente especificado.
\end{itemize}

Assim, o \emph{d} de Cohen pode ser obtido da seguinte forma:

\begin{Shaded}
\begin{Highlighting}[]
\NormalTok{d }\OtherTok{\textless{}{-}}\NormalTok{ lsr}\SpecialCharTok{::}\FunctionTok{cohensD}\NormalTok{ (notas }\SpecialCharTok{\textasciitilde{}}\NormalTok{ sexo, }\AttributeTok{data =}\NormalTok{ dados)}
\NormalTok{d}
\end{Highlighting}
\end{Shaded}

\begin{verbatim}
[1] 1.125339
\end{verbatim}

Bastante simples! Agora, como interpretar este resultado de d = 1,3
(arredondado)? Sua interpretação não é intuitiva, recomenda-se usar a
Tabela~\ref{tbl-effect} para interpretar (113).

\begin{Shaded}
\begin{Highlighting}[]
\NormalTok{df }\OtherTok{\textless{}{-}} \FunctionTok{data.frame}\NormalTok{(}\AttributeTok{d =} \FunctionTok{c}\NormalTok{(}\StringTok{"\textless{} 0,2"}\NormalTok{, }\StringTok{"0,2 \textless{} 0.5"}\NormalTok{, }
                       \StringTok{"0.5 \textless{} 0.8"}\NormalTok{, }\StringTok{"\textgreater{}= 0,8 "}\NormalTok{),}
                 \AttributeTok{sig =} \FunctionTok{c}\NormalTok{(}\StringTok{"insignificante"}\NormalTok{, }\StringTok{"pequeno"}\NormalTok{, }
                         \StringTok{"médio"}\NormalTok{, }\StringTok{"grande"}\NormalTok{))}

\NormalTok{minha\_tab }\OtherTok{\textless{}{-}} \FunctionTok{flextable}\NormalTok{(df) }\SpecialCharTok{\%\textgreater{}\%}
  \FunctionTok{set\_header\_labels}\NormalTok{(}
    \AttributeTok{d =} \StringTok{"d de Cohen"}\NormalTok{,}
    \AttributeTok{sig =} \StringTok{"Interpretação"}\NormalTok{) }\SpecialCharTok{\%\textgreater{}\%}
  \FunctionTok{autofit}\NormalTok{() }\SpecialCharTok{\%\textgreater{}\%}
  \FunctionTok{theme\_booktabs}\NormalTok{() }\SpecialCharTok{\%\textgreater{}\%}
  \FunctionTok{width}\NormalTok{(}\AttributeTok{j =} \DecValTok{1}\SpecialCharTok{:}\DecValTok{2}\NormalTok{, }\AttributeTok{width =} \FloatTok{1.5}\NormalTok{) }\SpecialCharTok{\%\textgreater{}\%}
  \FunctionTok{align}\NormalTok{(}\AttributeTok{align =} \StringTok{"left"}\NormalTok{, }\AttributeTok{part =} \StringTok{"header"}\NormalTok{) }\SpecialCharTok{\%\textgreater{}\%}
  \FunctionTok{align}\NormalTok{(}\AttributeTok{align =} \StringTok{"left"}\NormalTok{, }\AttributeTok{part =} \StringTok{"body"}\NormalTok{) }\SpecialCharTok{\%\textgreater{}\%}
  \FunctionTok{bold}\NormalTok{(}\AttributeTok{part =} \StringTok{"header"}\NormalTok{) }

\NormalTok{minha\_tab}
\end{Highlighting}
\end{Shaded}

\global\setlength{\Oldarrayrulewidth}{\arrayrulewidth}

\global\setlength{\Oldtabcolsep}{\tabcolsep}

\setlength{\tabcolsep}{2pt}

\renewcommand*{\arraystretch}{1.5}



\providecommand{\ascline}[3]{\noalign{\global\arrayrulewidth #1}\arrayrulecolor[HTML]{#2}\cline{#3}}

\begin{longtable}[c]{|p{1.50in}|p{1.50in}}

\caption{\label{tbl-effect}Tamanho do Efeito}

\tabularnewline

\ascline{1.5pt}{666666}{1-2}

\multicolumn{1}{>{\raggedright}m{\dimexpr 1.5in+0\tabcolsep}}{\textcolor[HTML]{000000}{\fontsize{11}{11}\selectfont{\global\setmainfont{Arial}{\textbf{d\ de\ Cohen}}}}} & \multicolumn{1}{>{\raggedright}m{\dimexpr 1.5in+0\tabcolsep}}{\textcolor[HTML]{000000}{\fontsize{11}{11}\selectfont{\global\setmainfont{Arial}{\textbf{Interpretação}}}}} \\

\ascline{1.5pt}{666666}{1-2}\endfirsthead 

\ascline{1.5pt}{666666}{1-2}

\multicolumn{1}{>{\raggedright}m{\dimexpr 1.5in+0\tabcolsep}}{\textcolor[HTML]{000000}{\fontsize{11}{11}\selectfont{\global\setmainfont{Arial}{\textbf{d\ de\ Cohen}}}}} & \multicolumn{1}{>{\raggedright}m{\dimexpr 1.5in+0\tabcolsep}}{\textcolor[HTML]{000000}{\fontsize{11}{11}\selectfont{\global\setmainfont{Arial}{\textbf{Interpretação}}}}} \\

\ascline{1.5pt}{666666}{1-2}\endhead



\multicolumn{1}{>{\raggedright}m{\dimexpr 1.5in+0\tabcolsep}}{\textcolor[HTML]{000000}{\fontsize{11}{11}\selectfont{\global\setmainfont{Arial}{<\ 0,2}}}} & \multicolumn{1}{>{\raggedright}m{\dimexpr 1.5in+0\tabcolsep}}{\textcolor[HTML]{000000}{\fontsize{11}{11}\selectfont{\global\setmainfont{Arial}{insignificante}}}} \\





\multicolumn{1}{>{\raggedright}m{\dimexpr 1.5in+0\tabcolsep}}{\textcolor[HTML]{000000}{\fontsize{11}{11}\selectfont{\global\setmainfont{Arial}{0,2\ <\ 0.5}}}} & \multicolumn{1}{>{\raggedright}m{\dimexpr 1.5in+0\tabcolsep}}{\textcolor[HTML]{000000}{\fontsize{11}{11}\selectfont{\global\setmainfont{Arial}{pequeno}}}} \\





\multicolumn{1}{>{\raggedright}m{\dimexpr 1.5in+0\tabcolsep}}{\textcolor[HTML]{000000}{\fontsize{11}{11}\selectfont{\global\setmainfont{Arial}{0.5\ <\ 0.8}}}} & \multicolumn{1}{>{\raggedright}m{\dimexpr 1.5in+0\tabcolsep}}{\textcolor[HTML]{000000}{\fontsize{11}{11}\selectfont{\global\setmainfont{Arial}{médio}}}} \\





\multicolumn{1}{>{\raggedright}m{\dimexpr 1.5in+0\tabcolsep}}{\textcolor[HTML]{000000}{\fontsize{11}{11}\selectfont{\global\setmainfont{Arial}{>=\ 0,8\ }}}} & \multicolumn{1}{>{\raggedright}m{\dimexpr 1.5in+0\tabcolsep}}{\textcolor[HTML]{000000}{\fontsize{11}{11}\selectfont{\global\setmainfont{Arial}{grande}}}} \\

\ascline{1.5pt}{666666}{1-2}


\end{longtable}

\arrayrulecolor[HTML]{000000}

\global\setlength{\arrayrulewidth}{\Oldarrayrulewidth}

\global\setlength{\tabcolsep}{\Oldtabcolsep}

\renewcommand*{\arraystretch}{1}

Assim, as notas dos alunos diferem significativamente (\emph{P}
\textless{} 0,0001) de acordo com a sexo, sendo que as mulheres têm
notas mais altas do que os homens e a magnitude dessa diferença é grande
(d = 1.13).

\section{\texorpdfstring{Teste \emph{t} para grupos
pareados}{Teste t para grupos pareados}}\label{teste-t-para-grupos-pareados}

Um \emph{teste t pareado} é usado para estimar se as médias de duas
medidas relacionadas são significativamente diferentes uma da outra.
Esse teste é usado quando duas variáveis contínuas são relacionadas
porque são coletadas do mesmo participante em momentos diferentes (antes
e depois), de locais diferentes na mesma pessoa ao mesmo tempo ou de
casos e seus controles correspondentes.

\subsection{Dados usados nesta
seção}\label{dados-usados-nesta-seuxe7uxe3o}

O banco de dados é constituído por uma amostra de 15 escolares
portadores de asma não controlada. Fizeram avaliação da sua função
pulmonar no início do uso de um novo corticoide inalatório. Após 60
dias, repetiram a avaliação da função pulmonar\footnote{O autor entende
  e recomenda que esta metodologia ``ante e depois'' deve ser evitada,
  pois traz consigo importante vieses como regressão à média, efeito
  placebo ou nocebo, viés temporal, viés de mensuração, etc. Ver
  Seção~\ref{sec-trials} . O uso aqui tem objetivo didático de mostrar a
  lógica estatística.}.

Para baixar o banco de dados, clique
\href{https://github.com/petronioliveira/Arquivos/blob/main/dadosPar.xlsx}{\textbf{aqui}}.
Faça o downloado para o seu diretório de trabalho.

\subsubsection{Leitura e transformação dos dados}\label{sec-pivot}

Leia o arquivo \texttt{dadosPar.xlsx} a partir do diretório de trabalho,
usando a função \texttt{read\_excel()} do pacote \texttt{readxl}.
Atribuir os dados a um objeto com o nome \texttt{dados}.

\begin{Shaded}
\begin{Highlighting}[]
\NormalTok{dados }\OtherTok{\textless{}{-}}\NormalTok{ readxl}\SpecialCharTok{::}\FunctionTok{read\_excel}\NormalTok{(}\StringTok{"dados/dadosPar.xlsx"}\NormalTok{)}
\end{Highlighting}
\end{Shaded}

A estrutura dos dados podem ser visualizada, usando a função
\texttt{str()}:

\begin{Shaded}
\begin{Highlighting}[]
\FunctionTok{str}\NormalTok{(dados)}
\end{Highlighting}
\end{Shaded}

\begin{verbatim}
tibble [15 x 3] (S3: tbl_df/tbl/data.frame)
 $ id   : num [1:15] 1 2 3 4 5 6 7 8 9 10 ...
 $ basal: num [1:15] 1.3 1.47 2.06 1.95 1.47 1.13 1.48 0.94 1.05 0.87 ...
 $ final: num [1:15] 1.53 1.63 2.35 2.7 2.01 1.53 1.66 1.59 1.5 1.61 ...
\end{verbatim}

O dataframe \texttt{dados} encontra-se no formato amplo (\emph{wide}),
ou seja, com as colunas basal e final colocadas lado a lado como se
fossem duas variáveis distintas, quando, na realidade, constituem-se em
apenas uma variável contendo as medidas de VEF1 (Volume Expiratório
Forçado no primeiro segundo).

A função \texttt{pivot\_longer()} do pacote \texttt{tidyr} fará a
transformação do formato amplo para o longo (\emph{long}). Este processo
não é obrigatório, mas será realizado para fins de treinamento. O novo
banco de dados será atribuído ao objeto \texttt{dadosL}. A função
\texttt{pivot\_longer()} necessita dos seguintes argumentos:

\begin{itemize}
\tightlist
\item
  \textbf{dados} \(\to\) dataframe a ser pivotado, tranformado;
\item
  \textbf{cols} \(\to\) colunas a serem transformadas no formato longo;
\item
  \textbf{names\_to} \(\to\) Especifica o nome da coluna a ser criada a
  partir dos dados armazenados nos nomes das colunas de dados;
\item
  \textbf{values\_to} \(\to\) Especifica o nome da coluna a ser criada a
  partir dos dados armazenados nos valores das células;
\item
  \textbf{\ldots{}} \(\to\) possui outros argumento. Ver ajuda.
\end{itemize}

\begin{Shaded}
\begin{Highlighting}[]
\NormalTok{dadosL }\OtherTok{\textless{}{-}}\NormalTok{ dados }\SpecialCharTok{\%\textgreater{}\%} 
\NormalTok{  tidyr}\SpecialCharTok{::}\FunctionTok{pivot\_longer}\NormalTok{(}\FunctionTok{c}\NormalTok{(basal, final), }
                      \AttributeTok{names\_to =} \StringTok{"momento"}\NormalTok{,}
                      \AttributeTok{values\_to =} \StringTok{"medidas"}\NormalTok{)}
\FunctionTok{str}\NormalTok{(dadosL)}
\end{Highlighting}
\end{Shaded}

\begin{verbatim}
tibble [30 x 3] (S3: tbl_df/tbl/data.frame)
 $ id     : num [1:30] 1 1 2 2 3 3 4 4 5 5 ...
 $ momento: chr [1:30] "basal" "final" "basal" "final" ...
 $ medidas: num [1:30] 1.3 1.53 1.47 1.63 2.06 2.35 1.95 2.7 1.47 2.01 ...
\end{verbatim}

\subsubsection{Medidas Resumidoras}\label{medidas-resumidoras}

Para resumir as variáveis, serão usadas as funções \texttt{group\_by()}
e \texttt{summarise()} do pacote \texttt{dplyr}, aplicadas ao formato
longo \texttt{dadosL}:

\begin{Shaded}
\begin{Highlighting}[]
\NormalTok{resumo }\OtherTok{\textless{}{-}}\NormalTok{ dadosL }\SpecialCharTok{\%\textgreater{}\%} 
\NormalTok{  dplyr}\SpecialCharTok{::}\FunctionTok{group\_by}\NormalTok{(momento) }\SpecialCharTok{\%\textgreater{}\%} 
\NormalTok{  dplyr}\SpecialCharTok{::}\FunctionTok{summarise}\NormalTok{(}\AttributeTok{n =} \FunctionTok{n}\NormalTok{ (),}
                   \AttributeTok{media =} \FunctionTok{mean}\NormalTok{(medidas, }\AttributeTok{na.rm =} \ConstantTok{TRUE}\NormalTok{),}
                   \AttributeTok{dp =} \FunctionTok{sd}\NormalTok{ (medidas, }\AttributeTok{na.rm =} \ConstantTok{TRUE}\NormalTok{),}
                   \AttributeTok{mediana =} \FunctionTok{median}\NormalTok{ (medidas, }\AttributeTok{na.rm =} \ConstantTok{TRUE}\NormalTok{),}
                   \AttributeTok{IIQ =} \FunctionTok{IQR}\NormalTok{ (medidas, }\AttributeTok{na.rm =}\ConstantTok{TRUE}\NormalTok{),}
                   \AttributeTok{ep =}\NormalTok{ dp}\SpecialCharTok{/}\FunctionTok{sqrt}\NormalTok{(n),}
                   \AttributeTok{me =}\NormalTok{ ep }\SpecialCharTok{*} \FunctionTok{qt}\NormalTok{(}\DecValTok{1} \SpecialCharTok{{-}}\NormalTok{ (}\FloatTok{0.05}\SpecialCharTok{/}\DecValTok{2}\NormalTok{), n }\SpecialCharTok{{-}} \DecValTok{1}\NormalTok{)) }
\NormalTok{resumo}
\end{Highlighting}
\end{Shaded}

\begin{verbatim}
# A tibble: 2 x 8
  momento     n media    dp mediana   IIQ    ep    me
  <chr>   <int> <dbl> <dbl>   <dbl> <dbl> <dbl> <dbl>
1 basal      15  1.31 0.427    1.26  0.48 0.110 0.236
2 final      15  1.69 0.471    1.59  0.38 0.122 0.261
\end{verbatim}

\subsubsection{Visualização dos
dados}\label{visualizauxe7uxe3o-dos-dados}

\begin{enumerate}
\def\labelenumi{\arabic{enumi})}
\tightlist
\item
  \textbf{Gráficos}
\end{enumerate}

Apenas, por uma questão didática, serão apresentadas três maneiras de
mostrar os dados visualmente. Podem ser usados qualquer um dos tipos a
seguir, pois todos dão, praticamente, a mesma informação.

\ul{Gráfico de barra de erro}

\begin{figure}

\centering{

\includegraphics[width=0.7\linewidth,height=0.7\textheight]{14-teste-t_files/figure-pdf/fig-be-1.pdf}

}

\caption{\label{fig-be}}

\end{figure}%

Nesse gráfico (Figura~\ref{fig-be}), a altura da barra representa a
média do \emph{Volume Forçado em 1 seg} (VEF1) nos diferentes momentos
(basal e final). O erro corresponde a margem de erro (me) a partir do
ponto (média), ou seja, é o intervalo de confiança de 95\%. O limite
inferior do IC95\% foi suprimido.

\ul{Boxplot}

\begin{figure}

\centering{

\includegraphics[width=0.7\linewidth,height=0.7\textheight]{14-teste-t_files/figure-pdf/fig-bp-1.pdf}

}

\caption{\label{fig-bp}}

\end{figure}%

A altura da caixa dos boxplots (Figura~\ref{fig-bp})) é o intervalo
interquartil (IIQ) e corresponde a 50\% dos dados. A linha que corta
horizontalmente a caixa é a mediana. Os bigodes da caixa
(\emph{whiskers}) em suas extremidades são os limites inferior e
superior dos dados, excluindo os valores atípicos (\emph{outliers}),
representado no boxplot final por um ponto vermelho, acima do limite
superior. Os pontos em vermelho (dentro das caixas) representam as
médias.

\ul{Gráfico de linha}

\begin{figure}

\centering{

\includegraphics[width=0.7\linewidth,height=0.7\textheight]{14-teste-t_files/figure-pdf/fig-lin-1.pdf}

}

\caption{\label{fig-lin}}

\end{figure}%

Este gráfico de linha (Figura~\ref{fig-lin}) com representação da margem
de erro tem a mesma interpretação do gráfico de barra de erro. A escolha
do tipo de gráfico depende da ênfase do autor sobre os dados.

\subsubsection{Criação de uma variável que represente a diferença entre
as
médias}\label{criauxe7uxe3o-de-uma-variuxe1vel-que-represente-a-diferenuxe7a-entre-as-muxe9dias}

A diferença entre as média basal e final será atribuída ao nome
\texttt{D}. Esta ação será realizada, utilizando o banco de dados amplo
(\texttt{dados}):

\begin{Shaded}
\begin{Highlighting}[]
\CommentTok{\#|message: false }
\CommentTok{\#|warning: false}

\NormalTok{dados}\SpecialCharTok{$}\NormalTok{D }\OtherTok{\textless{}{-}}\NormalTok{ dados}\SpecialCharTok{$}\NormalTok{basal }\SpecialCharTok{{-}}\NormalTok{ dados}\SpecialCharTok{$}\NormalTok{final}

\FunctionTok{head}\NormalTok{ (dados)}
\end{Highlighting}
\end{Shaded}

\begin{verbatim}
# A tibble: 6 x 4
     id basal final     D
  <dbl> <dbl> <dbl> <dbl>
1     1  1.3   1.53 -0.23
2     2  1.47  1.63 -0.16
3     3  2.06  2.35 -0.29
4     4  1.95  2.7  -0.75
5     5  1.47  2.01 -0.54
6     6  1.13  1.53 -0.4 
\end{verbatim}

\begin{tcolorbox}[enhanced jigsaw, bottomrule=.15mm, opacitybacktitle=0.6, colframe=quarto-callout-caution-color-frame, arc=.35mm, coltitle=black, toptitle=1mm, colback=white, colbacktitle=quarto-callout-caution-color!10!white, breakable, bottomtitle=1mm, rightrule=.15mm, titlerule=0mm, toprule=.15mm, opacityback=0, leftrule=.75mm, left=2mm, title=\textcolor{quarto-callout-caution-color}{\faFire}\hspace{0.5em}{Atenção}]

O banco de dados, agora, apresenta uma nova variável D, pois o foco do
teste \emph{t} pareado é essa diferença entre as médias, basal e final,
a média das diferenças.

\end{tcolorbox}

\textbf{Resumo da variável \texttt{D}}

Ao resumo será atribuído ao nome \texttt{sumario} (sem acento):

\begin{Shaded}
\begin{Highlighting}[]
\NormalTok{resumoD }\OtherTok{\textless{}{-}}\NormalTok{ dados }\SpecialCharTok{\%\textgreater{}\%} 
\NormalTok{  dplyr}\SpecialCharTok{::}\FunctionTok{summarise}\NormalTok{(}\AttributeTok{media =} \FunctionTok{mean}\NormalTok{ (D),}
                   \AttributeTok{dp =} \FunctionTok{sd}\NormalTok{ (D),}
                   \AttributeTok{mediana =} \FunctionTok{median}\NormalTok{ (D),}
                   \AttributeTok{IIQ =} \FunctionTok{IQR}\NormalTok{ (D),}
                   \AttributeTok{min =} \FunctionTok{min}\NormalTok{ (D),}
                   \AttributeTok{max =} \FunctionTok{max}\NormalTok{ (D))}
\NormalTok{resumoD}
\end{Highlighting}
\end{Shaded}

\begin{verbatim}
# A tibble: 1 x 6
   media    dp mediana   IIQ   min     max
   <dbl> <dbl>   <dbl> <dbl> <dbl>   <dbl>
1 -0.377 0.218   -0.38 0.285 -0.75 -0.0400
\end{verbatim}

Existe uma diferença de 0.38L entre o VEF1 basal e o final. A pergunta
que se faz é: Esta diferença tem significância estatística? Os gráficos
sugerem que sim!

\subsection{Definição das hipóteses
estatísticas}\label{definiuxe7uxe3o-das-hipuxf3teses-estatuxedsticas-1}

Será usado um teste bicaudal. Se a intervenção não produz efeito, então:

\[
H_0: \mu_D = 0  
\] Se a intervenção produz efeito, então:

\[
H_1: \mu_D \neq 0
\]

\subsection{Regra de decisão}\label{regra-de-decisuxe3o-1}

O nível significância, \(\alpha\), escolhido é igual a 0,05. A
distribuição da estatística do teste, sob a \(H_{0}\), é a distribuição
\emph{t} que é dependente dos graus de liberdade. O número de graus de
liberdade á igual ao número de observações menos 1, neste caso são o
número de pares menos 1.

\begin{Shaded}
\begin{Highlighting}[]
\NormalTok{n }\OtherTok{\textless{}{-}} \FunctionTok{length}\NormalTok{(dados}\SpecialCharTok{$}\NormalTok{D)}
\NormalTok{gl }\OtherTok{\textless{}{-}}\NormalTok{ n }\SpecialCharTok{{-}} \DecValTok{1}
\NormalTok{gl}
\end{Highlighting}
\end{Shaded}

\begin{verbatim}
[1] 14
\end{verbatim}

Para um \(\alpha = 0,05\), o valor crítico de \emph{t} para gl = 14 para
uma hipótese alternativa bicaudal:

\begin{Shaded}
\begin{Highlighting}[]
\NormalTok{alpha }\OtherTok{\textless{}{-}} \FloatTok{0.05}
\NormalTok{p }\OtherTok{\textless{}{-}} \DecValTok{1} \SpecialCharTok{{-}}\NormalTok{ alpha}\SpecialCharTok{/}\DecValTok{2}
\FunctionTok{round}\NormalTok{(}\FunctionTok{qt}\NormalTok{(p, }\DecValTok{14}\NormalTok{), }\DecValTok{3}\NormalTok{)}
\end{Highlighting}
\end{Shaded}

\begin{verbatim}
[1] 2.145
\end{verbatim}

Portanto, se

\[
\mid t_{calculado}\mid < \mid t_{crítico}\mid -> não \quad rejeitar \quad H_{0} \\ \mid t_{calculado}\mid > \mid t_{crítico}\mid -> rejeitar \quad H_{0}
\]

\subsection{Teste estatístico}\label{teste-estatuxedstico-2}

\subsubsection{Lógica do teste}\label{luxf3gica-do-teste}

A estatística do teste \emph{t} dependente é a mesma do teste \emph{t}
independente r dada por:

\[
T = \frac{\bar{D} - \mu_{D}}{EP_{D}}
\]

Como na equação do teste \emph{t} para amostras independentes, sob a
hipótese nula igual a zero, \(\mu_{D} = 0\), assim, a equação fica:

\[
T = \frac{\bar{D}}{EP_{D}}
\]

A estimativa do erro padrão das diferenças é dada por:

\[
EP_{D}=\frac{s_{D}}{\sqrt{n}}
\]

O desvio padrão das diferenças, \(s_{D}\) , é dado por:

\[
s_{D}=\sqrt\frac{\Sigma(D_{i} - \bar{D})^2}{n - 1}
\]

Onde \(D_{i}\) são as diferença individuais
(\(x_1 - y_1, x_2 - y_2, ..., x_n - y_n\)).

Da mesma maneira que no teste \emph{t} para grupos independentes, essa
demonstração serve para uma melhor compreensão de como o teste funciona,
mas para executar este teste \emph{t} não há necessidade disso, basta
saber como encaminhar ao R, como será visto adiante.

\subsubsection{Pressupostos do teste}\label{pressupostos-do-teste}

O teste \emph{t} pareado assume que os seguintes pressupostos devem ser
atendidos:

\begin{enumerate}
\def\labelenumi{(\arabic{enumi})}
\tightlist
\item
  Os dados devem ser dependentes;
\item
  A variável desfecho deve estar em uma escala contínua;
\item
  As diferenças entre os pares devem ter distribuição normal.
\end{enumerate}

Ao usar um teste \emph{t} pareado, a variação entre os pares de medidas
é a estatística mais importante e a variação entre os participantes,
como no teste t de duas amostras independentes, é de pouco interesse,
não havendo necessidade de se verificar se as variâncias dos grupos são
iguais.

Para testar o pressuposto de \emph{normalidade} das diferenças, usa-se a
variável criada da diferença entre os pares, \emph{D}. Verifica-se a
normalidade dessa variável com o teste Shapiro-Wilk, usando a função
\texttt{shapiro\_test()} do pacote \texttt{rstatix}, já usada no teste
\emph{t} de amostras independentes.

\begin{Shaded}
\begin{Highlighting}[]
\NormalTok{shapiro }\OtherTok{\textless{}{-}}\NormalTok{ dados }\SpecialCharTok{\%\textgreater{}\%} 
\NormalTok{   rstatix}\SpecialCharTok{::}\FunctionTok{shapiro\_test}\NormalTok{(D)}
\NormalTok{ shapiro}
\end{Highlighting}
\end{Shaded}

\begin{verbatim}
# A tibble: 1 x 3
  variable statistic     p
  <chr>        <dbl> <dbl>
1 D            0.942 0.410
\end{verbatim}

O teste de Shapiro-Wilk retorna um valor P \textgreater{} 0,05,
mostrando que a variável D que não se pode rejeitar a hipóteses nula de
sua normalidade.

Além disso, um gráfico Q-Q (Figura~\ref{fig-qq}) pode ser usado para
avaliar a normalidade, com a função \texttt{ggqqplot()} do pacote
\texttt{ggpubr} (115) que produz um gráfico QQ normal com uma linha de
referência, acompanhada de area sombreada, correspondente ao IC95\%

\begin{figure}

\centering{

\includegraphics[width=0.6\linewidth,height=0.6\textheight]{14-teste-t_files/figure-pdf/fig-qq-1.pdf}

}

\caption{\label{fig-qq}}

\end{figure}%

Os resultados do teste de Shapiro-Wilk e ográfico QQ, mostram que a
\(H_{0}\) de normalidade da variável \emph{D} não é rejeitada, apesar de
haver uma pequena assimetria à esquerda que não impede o prosseguimento
da análise.

\subsubsection{Execução do teste
estatístico}\label{execuuxe7uxe3o-do-teste-estatuxedstico}

O cálculo do teste \emph{t} pareado pode usar a mesma função do teste
\emph{t} para amostras independentes, \texttt{t\_test()}, do pacote
\texttt{rstatix}, mudando o argumento \texttt{paired\ =FALSE}(padrão)
por \texttt{paired\ =TRUE}. Assim:

\begin{Shaded}
\begin{Highlighting}[]
\NormalTok{teste\_par }\OtherTok{\textless{}{-}}\NormalTok{ dadosL }\SpecialCharTok{\%\textgreater{}\%} 
\NormalTok{  rstatix}\SpecialCharTok{::} \FunctionTok{t\_test}\NormalTok{(}\AttributeTok{formula =}\NormalTok{ medidas }\SpecialCharTok{\textasciitilde{}}\NormalTok{ momento,}
                   \AttributeTok{paired =} \ConstantTok{TRUE}\NormalTok{,}
                   \AttributeTok{detailed =} \ConstantTok{TRUE}\NormalTok{) }
\NormalTok{teste\_par}
\end{Highlighting}
\end{Shaded}

\begin{verbatim}
# A tibble: 1 x 13
  estimate .y.     group1 group2    n1    n2 statistic         p    df conf.low
*    <dbl> <chr>   <chr>  <chr>  <int> <int>     <dbl>     <dbl> <dbl>    <dbl>
1   -0.377 medidas basal  final     15    15     -6.70 0.0000102    14   -0.497
# i 3 more variables: conf.high <dbl>, method <chr>, alternative <chr>
\end{verbatim}

Observe que foi usado o conjunto de dados de formato longo
(\texttt{dadosL}) para usar a fórmula
(\texttt{x\ \textasciitilde{}\ grupo}). Da mesma maneira do que o teste
\emph{t} para amostras independentes, é possível ver os resultados do
teste \emph{t} , usando o objeto \texttt{teste\_par} que os recebeu.

Por exemplo, os limites inferior (\texttt{conf.low}) e superior
(\texttt{conf.high}) do intervalo de confiança de 95\% da estimativa de
diferença (\emph{D}) entre as médias

\begin{Shaded}
\begin{Highlighting}[]
\NormalTok{IC95 }\OtherTok{\textless{}{-}} \FunctionTok{round}\NormalTok{(}\FunctionTok{c}\NormalTok{(teste\_par}\SpecialCharTok{$}\NormalTok{conf.low, teste\_par}\SpecialCharTok{$}\NormalTok{conf.high),}\DecValTok{3}\NormalTok{)}
\NormalTok{IC95}
\end{Highlighting}
\end{Shaded}

\begin{verbatim}
[1] -0.497 -0.256
\end{verbatim}

\subsection{Conclusão}\label{conclusuxe3o-1}

Conclui-se que o VEF1 dos escolares asmáticos se modificou
significativamente entre o início e após 60 dias do uso de um novo
medicamento com uma confiança de 95\%. A diferença
(\(\mu_{basal} - \mu_{final}\)) encontrada é estatisticamente
significativa (t = -6.6969, gl = 14, \emph{P} =
\ensuremath{1.02\times 10^{-5}}), com uma confiança de 95\%.

Observe que o intervalo de confiança de 95\% da diferença de -0.38 está
todo abaixo de zero (-0.497, -0.256), confirmando a significância.

\subsection{Tamanho do Efeito}\label{tamanho-do-efeito-1}

O tamanho do efeito pode ser determinado, também, com o teste \emph{d}
de Cohen, usando a função \texttt{cohensD()} do pacote \texttt{lsr}:

\begin{Shaded}
\begin{Highlighting}[]
\NormalTok{d\_par }\OtherTok{\textless{}{-}}\NormalTok{ lsr}\SpecialCharTok{::}\FunctionTok{cohensD}\NormalTok{ (dados}\SpecialCharTok{$}\NormalTok{basal, dados}\SpecialCharTok{$}\NormalTok{final)}
\NormalTok{d\_par}
\end{Highlighting}
\end{Shaded}

\begin{verbatim}
[1] 0.8379499
\end{verbatim}

Dessa forma, o uso do novo corticoide inalatório modificou
significativamente o VEF1 dos escolares asmáticos com o uso de um novo
corticoide inalatório (\emph{P} = \ensuremath{1.02\times 10^{-5}}),
mostrando um aumento deste e que a magnitude dessa diferença é grande
(\emph{d} = 0.84).

Os resultados podem ser apresentados usando um gráfico de linha
(Figura~\ref{fig-fim})), aproveitando o resultado da função
\texttt{t\_test()}.

\begin{Shaded}
\begin{Highlighting}[]
\NormalTok{resumo }\SpecialCharTok{\%\textgreater{}\%} 
\NormalTok{    ggplot2}\SpecialCharTok{::}\FunctionTok{ggplot}\NormalTok{(}\FunctionTok{aes}\NormalTok{(}\AttributeTok{x=}\NormalTok{momento, }\AttributeTok{y=}\NormalTok{media, }\AttributeTok{group=}\DecValTok{1}\NormalTok{)) }\SpecialCharTok{+}
    \FunctionTok{geom\_line}\NormalTok{(}\AttributeTok{linetype =}\StringTok{\textquotesingle{}dashed\textquotesingle{}}\NormalTok{) }\SpecialCharTok{+}
    \FunctionTok{geom\_errorbar}\NormalTok{(}\FunctionTok{aes}\NormalTok{(}\AttributeTok{ymin=}\NormalTok{media }\SpecialCharTok{{-}}\NormalTok{ me, }
                      \AttributeTok{ymax=}\NormalTok{media }\SpecialCharTok{+}\NormalTok{ me), }
                  \AttributeTok{width=}\FloatTok{0.1}\NormalTok{,}
                  \AttributeTok{size =} \DecValTok{1}\NormalTok{,}
                  \AttributeTok{col =} \FunctionTok{c}\NormalTok{(}\StringTok{"cyan4"}\NormalTok{,}\StringTok{"cyan3"}\NormalTok{)) }\SpecialCharTok{+}
    \FunctionTok{geom\_point}\NormalTok{(}\AttributeTok{size =} \DecValTok{2}\NormalTok{) }\SpecialCharTok{+}
   \FunctionTok{labs}\NormalTok{(}\AttributeTok{title=}\StringTok{"Avaliação do Uso de Corticosteroide Inalatório"}\NormalTok{,}
       \AttributeTok{subtitle =}\NormalTok{ rstatix}\SpecialCharTok{::}\FunctionTok{get\_test\_label}\NormalTok{(}\AttributeTok{stat.test =}\NormalTok{ teste\_par,}
                                          \AttributeTok{correction =} \StringTok{"none"}\NormalTok{,}
                                          \AttributeTok{detailed =} \ConstantTok{TRUE}\NormalTok{,}
                                          \AttributeTok{type =} \StringTok{"expression"}\NormalTok{),}
       \AttributeTok{x=}\StringTok{"Momento"}\NormalTok{, }
       \AttributeTok{y =} \StringTok{"Volume Expiratório Forçado em 1 seg (L)"}\NormalTok{,}
       \AttributeTok{caption =} \StringTok{"d Cohen = 0,84"}\NormalTok{)}\SpecialCharTok{+}
   \FunctionTok{theme\_bw}\NormalTok{() }\SpecialCharTok{+} 
   \FunctionTok{theme}\NormalTok{(}\AttributeTok{legend.position=}\StringTok{"none"}\NormalTok{)}
\end{Highlighting}
\end{Shaded}

\begin{figure}[H]

\centering{

\includegraphics[width=0.7\linewidth,height=0.7\textheight]{14-teste-t_files/figure-pdf/fig-fim-1.pdf}

}

\caption{\label{fig-fim}Gráfico de linha mostrando a diferença na
resposta ao corticoide inalatório}

\end{figure}%

\chapter{Análise de Variância}\label{anuxe1lise-de-variuxe2ncia}

\section{Pacotes necessários para este
capítulo}\label{pacotes-necessuxe1rios-para-este-capuxedtulo-2}

\begin{Shaded}
\begin{Highlighting}[]
\NormalTok{pacman}\SpecialCharTok{::}\FunctionTok{p\_load}\NormalTok{(car,}
\NormalTok{               dplyr,}
\NormalTok{               effectsize,}
\NormalTok{               emmeans,}
\NormalTok{               flextable,}
\NormalTok{               ggplot2,}
\NormalTok{               ggpubr,}
\NormalTok{               ggsci,}
\NormalTok{               knitr,}
\NormalTok{               readxl,}
\NormalTok{               rstatix)}
\end{Highlighting}
\end{Shaded}

\section{Por que realizar uma ANOVA?}\label{por-que-realizar-uma-anova}

Para analisar três ou mais grupos, uma tendência intuitiva seria fazer
comparações por pares, usando um teste \emph{t} de amostras
independentes. Com quatro grupos, por exemplo, é possível compará-los
realizando seis testes, grupo 1 versus grupo 2, grupo 1 versus grupo 3,
grupo 1 versus grupo 4, grupo 2 versus grupo 3, grupo 2 versus grupo 4 e
grupo 3 versus grupo 4. Com 5 grupos o número de testes necessários
seria igual a 10. Generalizando, a expressão combinatória, usada para
calcular o número de combinações de \emph{n} elementos distintos de um
conjunto de \emph{k} grupos, sem se importar com a ordem é igual a:

\[
\frac {k!}{n!(k-n)!}
\]

Ou, usando uma função do R , \texttt{choose(k,\ n)}, facilmente se chega
ao resultado. Portanto, pa quatro grupos:

\begin{Shaded}
\begin{Highlighting}[]
\FunctionTok{choose}\NormalTok{(}\DecValTok{4}\NormalTok{, }\DecValTok{2}\NormalTok{)}
\end{Highlighting}
\end{Shaded}

\begin{verbatim}
[1] 6
\end{verbatim}

Foi visto na Seção~\ref{sec-erros} que ao realizar um teste de hipótese
podem ocorrer erros. Em geral, tolera-se aceitar um taxa de falsos
positivos (erro tipo I) de até 5\% (\(\alpha\) = 0,05).
Consequentemente, a probabilidade de um erro do tipo I não ocorrer para
cada teste \emph{t} é de 0,95 (isto é, 1 -- 0,05). Como os três testes
são independentes, a probabilidade de um erro do tipo I não ocorrer nos
seis testes é de \((0,95)^6 = 0,735\). Dessa maneira, a probabilidade de
ocorrer pelo menos um erro do tipo I nos seis testes \emph{t} de duas
amostras é de 1 -- 0,735 ou 0,265 (26,5\%), o que é mais alto do que o
nível de significância definido de 0,05 (116). Ou seja, realizar
múltiplos testes \emph{t}, infla o erro tipo I.

Por essa razão, a ANOVA de um fator é usada para verificar as diferenças
entre vários grupos dentro de um fator, reduzindo assim o número de
comparações em pares e a probabilidade de ocorrer um erro tipo I.

\subsection{Lógica do Modelo da ANOVA}\label{sec-logicamodel}

O procedimento de ANOVA é utilizado para testar a hipótese nula de que
as médias de três \footnote{Pode ser usada também para comparar a média
  de duas populações e o resultado será o mesmo de um teste \emph{t}
  para amostras independentes. undefined} ou mais populações são as
mesmas contra hipótese alternativa de que nem todas as médias são
iguais.

Na Seção~\ref{sec-homovar}, foram comparadas duas variâncias, usando um
teste, denominado de \emph{teste F}. Este teste, é uma razão entre duas
variâncias e recebeu este nome em homenagem a \emph{Sir Ronald Aylmer
Fisher} (veja a Seção~\ref{sec-historia}). A variância é uma medida de
dispersão que mensura como os dados estão espalhados em torno da média
(veja Seção~\ref{sec-variancia} ) . Quanto maior o seu valor, maior a
dispersão.

Considere a Figura~\ref{fig-logica}, onde está representada a
distribuição de uma variável \emph{X} em três grupos independentes.
Pode-se, claramente, distinguir observações provenientes dessas
distribuições, pois a sobreposição delas é pequena. Cada uma dela se
dispersa pouco em torno da média.

\begin{figure}

\centering{

\includegraphics[width=0.7\linewidth,height=0.7\textheight]{15-anova_files/figure-pdf/fig-logica-1.pdf}

}

\caption{\label{fig-logica}Três distribuições diferentes}

\end{figure}%

Agora, observe o Figura~\ref{fig-logicb}, onde a distribuição da
variável \emph{X} é mostrada, mantendo as mesmas médias, mas com
variâncias maiores. Isto torna claro que se o objetivo é distinguir
observações provenientes desses grupos não basta avaliar suas médias, há
necessidade de comparar a variação entre os grupos com a variação dentro
de cada grupo (117).

\begin{figure}

\centering{

\includegraphics[width=0.7\linewidth,height=0.7\textheight]{15-anova_files/figure-pdf/fig-logicb-1.pdf}

}

\caption{\label{fig-logicb}Distribuições com mesmas médias da figura
anterior, mas variâncias maiores}

\end{figure}%

Se a variação entre os grupos for grande quando comparada à variação
dentro de cada grupo, aumenta a probabilidade de reconhecer a
proveniência das observações (Figura~\ref{fig-logica}). Entretanto, se a
variação entre os grupos for pequena comparada à variação dentro do
grupo, torna difícil a distinção de observações provenientes dos grupos
(Figura~\ref{fig-logicb}).

Portanto, usar o teste \emph{F} para determinar se as médias de grupo
são iguais é apenas uma questão de incluir as variâncias corretas na
razão. Na ANOVA com um fator, por exemplo, a estatística \emph{F} é a
razão dos estimadores das \textbf{variâncias entre e dentro dos grupos}.

\[
F = \frac{variância \quad ENTRE \quad os \quad grupos}{variância \quad DENTRO \quad dos \quad grupos}
\]

Quando o valor de \emph{F} fica próximo de 1, significa que as
variâncias são muito próximas; quando \emph{F} é significativamente
maior do que 1, é possível distinguir os indivíduos de diferentes
grupos. Ou seja, se o objetivo for mostrar que as médias são diferentes,
será bom que a variância dentro dos grupos seja baixa. Pode-se pensar na
variância dentro do grupo como o ruído que pode obscurecer a diferença
entre os sons (as médias). No gráfico da Figura~\ref{fig-logica}, o
valor de \emph{F} seria alto; no da Figura~\ref{fig-logicb}, seria
baixo.

Como saber se o valor de \emph{F} é alto o suficiente? Um único valor
\emph{F} é difícil de interpretar sozinho. Há necessidade de colocá-lo
em um contexto maior antes que seja possível interpretá-lo. Para fazer
isso, usa-se a distribuição \emph{F} para calcular as probabilidades.

\subsection{\texorpdfstring{Distribuição
\emph{F}}{Distribuição F}}\label{distribuiuxe7uxe3o-f}

A \textbf{distribuição F}, também conhecida como distribuição de
Fisher-Snedecor, é uma distribuição de probabilidade contínua
fundamental na estatística inferencial. Sua principal aplicação é para
comparar variâncias de duas ou mais populações normais.

A razão entre a \textbf{variabilidade entre os grupos} e a
\textbf{variabilidade dentro do grupo} segue uma distribuição \emph{F}
quando a hipótese nula é verdadeira. Quando se realiza uma ANOVA com um
fator obtém-se um valor \emph{F}. No entanto, se forem extraídas várias
amostras aleatórias do mesmo tamanho da mesma população e fosse repetida
a mesma análise, o resultado seriam muitos valores \emph{F} diferentes,
constituindo uma distribuição amostral, denominada de
\textbf{distribuição F}.

A distribuição \emph{F} assumindo que a hipótese nula é verdadeira, é
possível colocar o resultado de qualquer valor \emph{F} e determinar
quão consistente ele é com a hipótese nula e calcular a probabilidade. A
probabilidade que se quer calcular é a probabilidade de observar uma
estatística \emph{F} que é pelo menos tão alta quanto o valor que o
estudo obteve. Essa probabilidade permite determinar quão comum ou raro
é o valor \emph{F}, sob a suposição de que a hipótese nula é verdadeira.
Se a probabilidade for pequena o suficiente, pode-se concluir que dados
são inconsistentes com a hipótese nula. Como já foi mostrado em outros
momentos, essa probabilidade é o valor \emph{p}
(Seção~\ref{sec-valorp}).

A forma da curva de distribuição \emph{F} varia de acordo com
\textbf{dois graus de iberdade} - um para o numerador (variância entre)
e outro para o denominador (variância dentro). Cada combinação de graus
de liberdade fornece uma curva de distribuição \emph{F} diferente. As
unidades de uma distribuição \emph{F} são denotadas por \emph{F}, que
possui características importantes:

\begin{enumerate}
\def\labelenumi{\arabic{enumi}.}
\item
  \textbf{Distribuição contínua}: Como as distribuições normal, \emph{t}
  e qui-quadrado (veja \textbf{?@sec-qui}), a distribuição \emph{F} é
  uma distribuição contínua;
\item
  \textbf{Valores positivos}: A variável F só pode assumir valores
  maiores ou iguais a zero; A distribuição é limitada à esquerda por
  zero e não tem limite superior, estendendo-se indefinidamente à
  direita, o que reflete o fato de que as variâncias são sempre não
  negativas.
\item
  \textbf{Assimetria positiva}: A distribuição F é assimetrica à
  direita, ou seja, tem uma cauda longa para o lado positivo, mas a
  assimetria diminui à medida que o número de graus de liberdade
  aumenta, conforme observado na Figura~\ref{fig-distf}.
\item
  \textbf{Parâmetros}: A distribuição F é definida por seus graus de
  liberdade, o do numerador (gl1) e o do denominador (gl2)\footnote{Quando
    se referir a abreviação \texttt{df} de \texttt{degree\ of\ freedom}
    em algumas funções, optou-se por manter \texttt{df} ,sem a tradução
    para \texttt{gl} (graus de liberdade), como em geral se usa no
    texto.}.
\end{enumerate}

\begin{figure}

\centering{

\includegraphics[width=0.9\linewidth,height=0.9\textheight]{15-anova_files/figure-pdf/fig-distf-1.pdf}

}

\caption{\label{fig-distf}Distribuições F}

\end{figure}%

\subsubsection{Distribuição F e os graus de
liberdade}\label{distribuiuxe7uxe3o-f-e-os-graus-de-liberdade}

Na Figura~\ref{fig-distf}, observa-se que à medida que os graus de
liberdade aumentam, a distribuição F realmente começa a se parecer com a
distribuição normal, especialmente em termos de simetria e concentração
em torno da média. Entretanto, há pormenores importantes sobre como isso
acontece com o grau de liberdade do numerador (gl\textsubscript{1}) e
com o grau de liberdade do denominador (gl\textsubscript{2}).

\begin{Shaded}
\begin{Highlighting}[]
\NormalTok{df }\OtherTok{\textless{}{-}} \FunctionTok{data.frame}\NormalTok{(}
\AttributeTok{gl =} \FunctionTok{c}\NormalTok{(}\StringTok{"gl1 (numerador)"}\NormalTok{, }\StringTok{"gl2 (denominador)"}\NormalTok{),}
\AttributeTok{papel =} \FunctionTok{c}\NormalTok{(}\StringTok{"Relacionado ao número de grupos ou variáveis"}\NormalTok{,}
          \StringTok{"Relacionado ao tamanho da amostra"}\NormalTok{),}
\AttributeTok{aumento =} \FunctionTok{c}\NormalTok{(}\StringTok{"A curva fica menos assimétrica, mas ainda com cauda à direita"}\NormalTok{,}
            \StringTok{"A curva se aproxima mais rapidamente da normal"}\NormalTok{))}

\NormalTok{tab }\OtherTok{\textless{}{-}} \FunctionTok{flextable}\NormalTok{(df) }\SpecialCharTok{\%\textgreater{}\%}
  \FunctionTok{set\_header\_labels}\NormalTok{(}
    \AttributeTok{gl =} \StringTok{"Graus de Liberdade"}\NormalTok{,}
    \AttributeTok{papel =} \StringTok{"Papel na Distribuição F"}\NormalTok{,}
    \AttributeTok{aumento =} \StringTok{"Efeito ao aumentar"}\NormalTok{)  }\SpecialCharTok{\%\textgreater{}\%}
  \FunctionTok{theme\_booktabs}\NormalTok{() }\SpecialCharTok{\%\textgreater{}\%}
  \FunctionTok{width}\NormalTok{(}\AttributeTok{j =} \DecValTok{1}\SpecialCharTok{:}\DecValTok{3}\NormalTok{, }\AttributeTok{width =} \FunctionTok{c}\NormalTok{(}\DecValTok{2}\NormalTok{,}\FloatTok{2.5}\NormalTok{,}\FloatTok{2.5}\NormalTok{)) }\SpecialCharTok{\%\textgreater{}\%}
  \FunctionTok{align}\NormalTok{(}\AttributeTok{align =} \StringTok{"left"}\NormalTok{, }\AttributeTok{part =} \StringTok{"header"}\NormalTok{) }\SpecialCharTok{\%\textgreater{}\%}
  \FunctionTok{align}\NormalTok{(}\AttributeTok{align =} \StringTok{"left"}\NormalTok{, }\AttributeTok{part =} \StringTok{"body"}\NormalTok{) }\SpecialCharTok{\%\textgreater{}\%}
  \FunctionTok{bold}\NormalTok{(}\AttributeTok{part =} \StringTok{"header"}\NormalTok{) }

\NormalTok{tab}
\end{Highlighting}
\end{Shaded}

\global\setlength{\Oldarrayrulewidth}{\arrayrulewidth}

\global\setlength{\Oldtabcolsep}{\tabcolsep}

\setlength{\tabcolsep}{2pt}

\renewcommand*{\arraystretch}{1.5}



\providecommand{\ascline}[3]{\noalign{\global\arrayrulewidth #1}\arrayrulecolor[HTML]{#2}\cline{#3}}

\begin{longtable}[c]{|p{2.00in}|p{2.50in}|p{2.50in}}

\caption{\label{tbl-gl1gl2}Importância dos graus de liberdade}

\tabularnewline

\ascline{1.5pt}{666666}{1-3}

\multicolumn{1}{>{\raggedright}m{\dimexpr 2in+0\tabcolsep}}{\textcolor[HTML]{000000}{\fontsize{11}{11}\selectfont{\global\setmainfont{Arial}{\textbf{Graus\ de\ Liberdade}}}}} & \multicolumn{1}{>{\raggedright}m{\dimexpr 2.5in+0\tabcolsep}}{\textcolor[HTML]{000000}{\fontsize{11}{11}\selectfont{\global\setmainfont{Arial}{\textbf{Papel\ na\ Distribuição\ F}}}}} & \multicolumn{1}{>{\raggedright}m{\dimexpr 2.5in+0\tabcolsep}}{\textcolor[HTML]{000000}{\fontsize{11}{11}\selectfont{\global\setmainfont{Arial}{\textbf{Efeito\ ao\ aumentar}}}}} \\

\ascline{1.5pt}{666666}{1-3}\endfirsthead 

\ascline{1.5pt}{666666}{1-3}

\multicolumn{1}{>{\raggedright}m{\dimexpr 2in+0\tabcolsep}}{\textcolor[HTML]{000000}{\fontsize{11}{11}\selectfont{\global\setmainfont{Arial}{\textbf{Graus\ de\ Liberdade}}}}} & \multicolumn{1}{>{\raggedright}m{\dimexpr 2.5in+0\tabcolsep}}{\textcolor[HTML]{000000}{\fontsize{11}{11}\selectfont{\global\setmainfont{Arial}{\textbf{Papel\ na\ Distribuição\ F}}}}} & \multicolumn{1}{>{\raggedright}m{\dimexpr 2.5in+0\tabcolsep}}{\textcolor[HTML]{000000}{\fontsize{11}{11}\selectfont{\global\setmainfont{Arial}{\textbf{Efeito\ ao\ aumentar}}}}} \\

\ascline{1.5pt}{666666}{1-3}\endhead



\multicolumn{1}{>{\raggedright}m{\dimexpr 2in+0\tabcolsep}}{\textcolor[HTML]{000000}{\fontsize{11}{11}\selectfont{\global\setmainfont{Arial}{gl1\ (numerador)}}}} & \multicolumn{1}{>{\raggedright}m{\dimexpr 2.5in+0\tabcolsep}}{\textcolor[HTML]{000000}{\fontsize{11}{11}\selectfont{\global\setmainfont{Arial}{Relacionado\ ao\ número\ de\ grupos\ ou\ variáveis}}}} & \multicolumn{1}{>{\raggedright}m{\dimexpr 2.5in+0\tabcolsep}}{\textcolor[HTML]{000000}{\fontsize{11}{11}\selectfont{\global\setmainfont{Arial}{A\ curva\ fica\ menos\ assimétrica,\ mas\ ainda\ com\ cauda\ à\ direita}}}} \\





\multicolumn{1}{>{\raggedright}m{\dimexpr 2in+0\tabcolsep}}{\textcolor[HTML]{000000}{\fontsize{11}{11}\selectfont{\global\setmainfont{Arial}{gl2\ (denominador)}}}} & \multicolumn{1}{>{\raggedright}m{\dimexpr 2.5in+0\tabcolsep}}{\textcolor[HTML]{000000}{\fontsize{11}{11}\selectfont{\global\setmainfont{Arial}{Relacionado\ ao\ tamanho\ da\ amostra}}}} & \multicolumn{1}{>{\raggedright}m{\dimexpr 2.5in+0\tabcolsep}}{\textcolor[HTML]{000000}{\fontsize{11}{11}\selectfont{\global\setmainfont{Arial}{A\ curva\ se\ aproxima\ mais\ rapidamente\ da\ normal}}}} \\

\ascline{1.5pt}{666666}{1-3}


\end{longtable}

\arrayrulecolor[HTML]{000000}

\global\setlength{\arrayrulewidth}{\Oldarrayrulewidth}

\global\setlength{\tabcolsep}{\Oldtabcolsep}

\renewcommand*{\arraystretch}{1}

\ul{Por que que isso acontece} ?

Bom, é um pouco mais complicado!

Vamos lá, procurando tornar o mais simples possível\footnote{Na
  realidade, bem simplista!}, acima (Seção~\ref{sec-logicamodel}) foi
mostrado que a razão F na distribuição F é definida como a razão entre
duas variâncias estimadas e variância (Seção~\ref{sec-variancia}) é a
razão da soma dos quadrados (\emph{SQ}) dividido pelos graus de
liberdade, logo

\[
F = \frac{(SQ_1 / gl_1)}{(SQ_2 / gl_2)}
\]

onde (SQ\textsubscript{1}) e (SQ\textsubscript{2}) são variáveis
independentes com distribuições qui-quadrado (veja \textbf{?@sec-qui})
com graus de liberdade (gl\textsubscript{1}) e (gl\textsubscript{2}),
respectivamente.

Como a distribuição qui-quadrado se aproxima da normal quando os graus
de liberdade aumentam, a razão entre elas (a F) também começa a se
comportar como uma normal --- especialmente quando gl\textsubscript{2} é
grande (Tabela~\ref{tbl-gl1gl2}) (118).\\
Esse comportamento é útil na prática porque permite usar testes que
assumem normalidade quando os graus de liberdade são grandes, mesmo
quando a distribuição original é mais complexa.

\subsubsection{Funções do R para trabalhar com a distribuição
F}\label{funuxe7uxf5es-do-r-para-trabalhar-com-a-distribuiuxe7uxe3o-f}

No \emph{R}, existem quatro funções principaisque são ferramentas
essenciais para trabalhar com a distribuição F. Elas seguem um padrão
comum a muitas distribuições de probabilidade, e cada uma tem uma
finalidade específica.

\paragraph{Função de Densidade de Probabilidade
(FDP)}\label{funuxe7uxe3o-de-densidade-de-probabilidade-fdp}

A função \texttt{df()} calcula a densidade de probabilidade de um valor
específico em uma distribuição F. A densidade de probabilidade não é a
probabilidade em si, mas sim a altura da curva da distribuição F em um
determinado ponto.\\
É usada principalmente para visualizar a forma da distribuição F ou para
cálculos mais avançados que exigem a densidade em um ponto. Ela ajuda a
entender a ``concentração'' de probabilidade em diferentes valores. Usa
os seguintes argumentos:

\begin{itemize}
\tightlist
\item
  \texttt{x} : representa o valor na escala F, que é o valor que se
  obtém ao calcular a estatística de um teste F
  (F\textsubscript{observado}) em uma análise, como um teste ANOVA, por
  exemplo;\\
\item
  \texttt{df1}: são os graus de liberdade do numerador;\\
\item
  \texttt{df2}; são os graus de liberdade do denominador.
\end{itemize}

\begin{Shaded}
\begin{Highlighting}[]
\CommentTok{\# Exemplo: PDF de 2, com df1 = 4 e df2 = 50}

\FunctionTok{df}\NormalTok{(}\AttributeTok{x =} \DecValTok{2}\NormalTok{, }\AttributeTok{df1 =} \DecValTok{4}\NormalTok{, }\AttributeTok{df2 =} \DecValTok{50}\NormalTok{)}
\end{Highlighting}
\end{Shaded}

\begin{verbatim}
[1] 0.1512717
\end{verbatim}

\paragraph{Função de Distribuição Cumulativa
(CDF)}\label{funuxe7uxe3o-de-distribuiuxe7uxe3o-cumulativa-cdf}

A função \texttt{pf()} calcula a probabilidade acumulada de uma
distribuição F. Ela retorna a probabilidade de que uma variável
aleatória F seja menor ou igual a um valor \texttt{q} (quantil).\\
É a função mais usada para obter o valor \emph{p} de um teste F. Dado um
valor F\textsubscript{observado} (estatística de teste), você usa
\texttt{pf()} para encontrar a probabilidade de obter um valor F igual
ou mais extremo, o que é fundamental para tomar decisões sobre rejeitar
ou não a hipótese nula. Usa os seguintes argumentos:

\begin{itemize}
\tightlist
\item
  \texttt{q} : O valor quantil para o qual se quer calcular a
  probabilidade acumulada;
\item
  \texttt{df1}: são os graus de liberdade do numerador;
\item
  \texttt{df2}; são os graus de liberdade do denominador;
\item
  \texttt{lower.tail\ =\ TRUE}: (Padrão) Calcula a probabilidade da
  cauda inferior (\(P(F \le q)\)). Usar \texttt{FALSE} para a cauda
  superior (\(P(F \gt q)\)), que é o que geralmente se busca em testes
  de hipótese.
\end{itemize}

\begin{Shaded}
\begin{Highlighting}[]
\CommentTok{\# Exemplo: Probabilidade de F ser menor que 2, com df = 4 e df2 = 50}
\FunctionTok{pf}\NormalTok{(}\AttributeTok{q =} \DecValTok{2}\NormalTok{, }\AttributeTok{df1 =} \DecValTok{4}\NormalTok{, }\AttributeTok{df2 =} \DecValTok{50}\NormalTok{)}
\end{Highlighting}
\end{Shaded}

\begin{verbatim}
[1] 0.8911717
\end{verbatim}

\begin{Shaded}
\begin{Highlighting}[]
\CommentTok{\# Exemplo: Probabilidade da cauda superior de F ser maior que 2 (para valor p)}
\FunctionTok{pf}\NormalTok{(}\AttributeTok{q =} \DecValTok{2}\NormalTok{, }\AttributeTok{df1 =} \DecValTok{4}\NormalTok{, }\AttributeTok{df2 =} \DecValTok{50}\NormalTok{, }\AttributeTok{lower.tail =} \ConstantTok{FALSE}\NormalTok{)}
\end{Highlighting}
\end{Shaded}

\begin{verbatim}
[1] 0.1088283
\end{verbatim}

\paragraph{Função Quantil}\label{funuxe7uxe3o-quantil}

A função \texttt{qf()} é o inverso de \texttt{pf()}. Ela recebe uma
probabilidade \emph{p} e retorna o valor quantil correspondente, ou
seja, o valor \emph{q} que tem uma probabilidade acumulada \emph{p}.

É usada para encontrar o valor crítico de uma distribuição F para um
certo nível de significância (\(\alpha\)). Esse valor crítico é a linha
de corte que se compara com sua estatística de teste
F\textsubscript{observada} para decidir sobre o resultado do teste. Usa
os seguintes argumentos:

\begin{itemize}
\item
  \texttt{p} : A probabilidade acumulada (geralmente, \(1 − \alpha\)
  para testes de cauda superior);
\item
  \texttt{df1}: são os graus de liberdade do numerador;
\item
  \texttt{df2}; são os graus de liberdade do denominador;
\item
  \texttt{lower.tail\ =\ TRUE}: (Padrão) Encontra o valor \texttt{q} tal
  que (\(P(F \le q) = p\)).
\end{itemize}

\begin{Shaded}
\begin{Highlighting}[]
\CommentTok{\# Exemplo: Encontrar o valor crítico de F para um nível de significância de 5\% (0.05) com df1 = 4 e df2 = 50.}
\FunctionTok{qf}\NormalTok{(}\AttributeTok{p =} \FloatTok{0.95}\NormalTok{, }\AttributeTok{df1 =} \DecValTok{4}\NormalTok{, }\AttributeTok{df2 =} \DecValTok{50}\NormalTok{) }
\end{Highlighting}
\end{Shaded}

\begin{verbatim}
[1] 2.557179
\end{verbatim}

\begin{Shaded}
\begin{Highlighting}[]
\CommentTok{\# ou }
\FunctionTok{qf}\NormalTok{(}\FloatTok{0.05}\NormalTok{, }\DecValTok{4}\NormalTok{, }\DecValTok{50}\NormalTok{, }\AttributeTok{lower.tail =} \ConstantTok{FALSE}\NormalTok{)}
\end{Highlighting}
\end{Shaded}

\begin{verbatim}
[1] 2.557179
\end{verbatim}

\paragraph{Geração de Números
Aleatórios}\label{gerauxe7uxe3o-de-nuxfameros-aleatuxf3rios}

A função \texttt{rf()} gera números aleatórios que seguem a distribuição
F.\\
É útil para simulações, modelagem estatística, testes de Monte
Carlo\footnote{Testes de Monte Carlo, também conhecidos como Simulações
  de Monte Carlo, são uma classe de métodos computacionais que dependem
  de \textbf{amostragem aleatória} repetida para obter resultados
  numéricos. O nome ``Monte Carlo'' vem do famoso cassino em Mônaco,
  pois o método se baseia na aleatoriedade de um jogo de roleta.} ou
para entender a forma da distribuição F gerando amostras aleatórias.
Seus argumentos são:\\
- \texttt{n} : O número de observações aleatórias que se deseja gerar.\\
- \texttt{df1}: são os graus de liberdade do numerador;

\begin{itemize}
\tightlist
\item
  \texttt{df2}; são os graus de liberdade do denominador.
\end{itemize}

\begin{Shaded}
\begin{Highlighting}[]
\CommentTok{\# Exemplo: Gerar 50 números aleatórios de uma distribuição F com df1 = 4 e df2 = 50}
\NormalTok{amostra\_aleatoria }\OtherTok{\textless{}{-}} \FunctionTok{rf}\NormalTok{(}\AttributeTok{n =} \DecValTok{50}\NormalTok{, }\AttributeTok{df1 =} \DecValTok{4}\NormalTok{, }\AttributeTok{df2 =} \DecValTok{50}\NormalTok{)}
\NormalTok{amostra\_aleatoria}
\end{Highlighting}
\end{Shaded}

\begin{verbatim}
 [1] 0.48281051 0.60032443 0.41896506 1.27822138 0.71491387 0.53099728
 [7] 0.35209998 1.29059115 1.97401475 0.82273377 1.27224975 0.12101887
[13] 1.33288942 0.76878466 0.79091036 0.52988767 1.60221982 1.00208504
[19] 1.48720984 1.03410162 0.65314890 1.17389347 1.68077049 0.42515926
[25] 1.47976491 0.72884849 4.55757524 1.54613639 2.76931127 0.77745058
[31] 0.48409823 1.19256785 0.64445595 1.91902136 0.95871735 2.26549874
[37] 0.71945395 0.46833898 1.00915309 1.25785262 1.58570262 0.62383735
[43] 3.08186312 2.04296174 0.43128821 0.17244711 0.04079461 0.37222089
[49] 1.45357046 0.45361949
\end{verbatim}

\begin{tcolorbox}[enhanced jigsaw, bottomrule=.15mm, opacitybacktitle=0.6, colframe=quarto-callout-tip-color-frame, arc=.35mm, coltitle=black, toptitle=1mm, colback=white, colbacktitle=quarto-callout-tip-color!10!white, breakable, bottomtitle=1mm, rightrule=.15mm, titlerule=0mm, toprule=.15mm, opacityback=0, leftrule=.75mm, left=2mm, title=\textcolor{quarto-callout-tip-color}{\faLightbulb}\hspace{0.5em}{Exercício 1}]

Essas funções são úteis para resolver problemas de probabilidade
envolvendo a distribuição \emph{F}. Por exemplo, qual é a probabilidade
de uma variável aleatória \emph{F} com 4 e 50 graus de liberdade no
numerador e no denominador, respectivamente, ser menor que 1? Qual a
função deve ser usada?

\end{tcolorbox}

\ul{Resposta}:

Pode-se usar a função \texttt{pf()} :

\begin{Shaded}
\begin{Highlighting}[]
\NormalTok{q }\OtherTok{\textless{}{-}} \DecValTok{1}
\FunctionTok{pf}\NormalTok{(q, }\AttributeTok{df1=}\DecValTok{4}\NormalTok{, }\AttributeTok{df2=}\DecValTok{50}\NormalTok{)}
\end{Highlighting}
\end{Shaded}

\begin{verbatim}
[1] 0.5835786
\end{verbatim}

Ou seja, ao se observar a curva da Figura~\ref{fig-distf} de cor verde
(gl1 = 4 e gl2 = 50), a probabilidade abaixo de x = 1 é igual a 58,4\%,
arrendondando.

\begin{tcolorbox}[enhanced jigsaw, bottomrule=.15mm, opacitybacktitle=0.6, colframe=quarto-callout-tip-color-frame, arc=.35mm, coltitle=black, toptitle=1mm, colback=white, colbacktitle=quarto-callout-tip-color!10!white, breakable, bottomtitle=1mm, rightrule=.15mm, titlerule=0mm, toprule=.15mm, opacityback=0, leftrule=.75mm, left=2mm, title=\textcolor{quarto-callout-tip-color}{\faLightbulb}\hspace{0.5em}{Exercício 2}]

Para saber altura (densidade de probabilidade) da curva de cor verde
quando \emph{x} = 1, basta olhar na Figura~\ref{fig-distf}, ou seja, ao
redor de 0.50. Entretanto, é difícil saber o valor exato. O que fazer
para obter este valor?

\end{tcolorbox}

\ul{Resposta}:

Calcular a densidade de probabilidade com a função \texttt{df()}

\begin{Shaded}
\begin{Highlighting}[]
\NormalTok{x }\OtherTok{\textless{}{-}} \DecValTok{1}
\FunctionTok{df}\NormalTok{(x, }\AttributeTok{df1=}\DecValTok{4}\NormalTok{, }\AttributeTok{df2=}\DecValTok{50}\NormalTok{)}
\end{Highlighting}
\end{Shaded}

\begin{verbatim}
[1] 0.5207772
\end{verbatim}

\begin{tcolorbox}[enhanced jigsaw, bottomrule=.15mm, opacitybacktitle=0.6, colframe=quarto-callout-tip-color-frame, arc=.35mm, coltitle=black, toptitle=1mm, colback=white, colbacktitle=quarto-callout-tip-color!10!white, breakable, bottomtitle=1mm, rightrule=.15mm, titlerule=0mm, toprule=.15mm, opacityback=0, leftrule=.75mm, left=2mm, title=\textcolor{quarto-callout-tip-color}{\faLightbulb}\hspace{0.5em}{Exercício 3}]

Qual o valor do nível crítico de F que deixa 50\% da área da curva à
esquerda, supondo-se os mesmos graus de liberdade anteriores?

\end{tcolorbox}

\ul{Resposta}:

Calcular a com a função \texttt{qf()}

\begin{Shaded}
\begin{Highlighting}[]
\NormalTok{p }\OtherTok{\textless{}{-}} \FloatTok{0.50} 
\NormalTok{Fcrit }\OtherTok{\textless{}{-}} \FunctionTok{qf}\NormalTok{(p, }\AttributeTok{df1 =} \DecValTok{4}\NormalTok{, }\AttributeTok{df2 =} \DecValTok{50}\NormalTok{)}
\NormalTok{Fcrit}
\end{Highlighting}
\end{Shaded}

\begin{verbatim}
[1] 0.8506612
\end{verbatim}

Para representar, graficamente, esse resultado, foi construido o gráfico
da Figura~\ref{fig-distf450} com a função \texttt{ggolot2()}.
Verifica-se que a área sob a curva abaixo de 0,85 é igual a 50\%.

\begin{figure}

\centering{

\includegraphics[width=0.7\linewidth,height=0.7\textheight]{15-anova_files/figure-pdf/fig-distf450-1.pdf}

}

\caption{\label{fig-distf450}Área da curva da distribuição F (4,50)
abaixo de x = 0,85 é igual a 50\%}

\end{figure}%

\begin{tcolorbox}[enhanced jigsaw, bottomrule=.15mm, opacitybacktitle=0.6, colframe=quarto-callout-tip-color-frame, arc=.35mm, coltitle=black, toptitle=1mm, colback=white, colbacktitle=quarto-callout-tip-color!10!white, breakable, bottomtitle=1mm, rightrule=.15mm, titlerule=0mm, toprule=.15mm, opacityback=0, leftrule=.75mm, left=2mm, title=\textcolor{quarto-callout-tip-color}{\faLightbulb}\hspace{0.5em}{Exercício 4}]

Gerar 10.000 valores aleatórios de uma distribuição F (20, 100) e após
plotar um histograma com curva da distribuição F sobreposta.

\end{tcolorbox}

\ul{Resposta}:

\begin{Shaded}
\begin{Highlighting}[]
\CommentTok{\# Parâmetros da distribuição F}
\NormalTok{df1 }\OtherTok{\textless{}{-}} \DecValTok{20}
\NormalTok{df2 }\OtherTok{\textless{}{-}} \DecValTok{100}

\CommentTok{\# Gerar 100.000 valores aleatórios da distribuição F}
\FunctionTok{set.seed}\NormalTok{(}\DecValTok{123}\NormalTok{)  }\CommentTok{\# para reprodutibilidade}
\NormalTok{valores }\OtherTok{\textless{}{-}} \FunctionTok{rf}\NormalTok{(}\DecValTok{10000}\NormalTok{, df1, df2)}


\CommentTok{\# Criar histograma com densidade}
\NormalTok{df\_dados }\OtherTok{\textless{}{-}} \FunctionTok{data.frame}\NormalTok{(}\AttributeTok{F\_valores =}\NormalTok{ valores)}

\CommentTok{\# Curva teórica da distribuição F}
\NormalTok{x\_teorico }\OtherTok{\textless{}{-}} \FunctionTok{seq}\NormalTok{(}\DecValTok{0}\NormalTok{, }\FunctionTok{max}\NormalTok{(valores), }\AttributeTok{length.out =} \DecValTok{500}\NormalTok{)}
\NormalTok{y\_teorico }\OtherTok{\textless{}{-}} \FunctionTok{df}\NormalTok{(x\_teorico, df1, df2)}
\NormalTok{df\_teorico }\OtherTok{\textless{}{-}} \FunctionTok{data.frame}\NormalTok{(}\AttributeTok{x =}\NormalTok{ x\_teorico, }\AttributeTok{y =}\NormalTok{ y\_teorico)}

\CommentTok{\# Gráfico com histograma e curva teórica}
\FunctionTok{ggplot}\NormalTok{(df\_dados, }\FunctionTok{aes}\NormalTok{(}\AttributeTok{x =}\NormalTok{ F\_valores)) }\SpecialCharTok{+}
  \FunctionTok{geom\_histogram}\NormalTok{(}\FunctionTok{aes}\NormalTok{(}\AttributeTok{y =} \FunctionTok{after\_stat}\NormalTok{(density)), }\AttributeTok{bins =} \DecValTok{50}\NormalTok{, }\AttributeTok{fill =} \StringTok{"lightblue"}\NormalTok{, }\AttributeTok{color =} \StringTok{"black"}\NormalTok{, }\AttributeTok{alpha =} \FloatTok{0.6}\NormalTok{) }\SpecialCharTok{+}
  \FunctionTok{geom\_line}\NormalTok{(}\AttributeTok{data =}\NormalTok{ df\_teorico, }\FunctionTok{aes}\NormalTok{(}\AttributeTok{x =}\NormalTok{ x, }\AttributeTok{y =}\NormalTok{ y), }\AttributeTok{color =} \StringTok{"red"}\NormalTok{, }\AttributeTok{size =} \FloatTok{1.2}\NormalTok{) }\SpecialCharTok{+}
  \FunctionTok{labs}\NormalTok{(}
    \AttributeTok{title =} \StringTok{"Histograma de valores simulados da distribuição F(20, 100)"}\NormalTok{,}
    \AttributeTok{subtitle =} \StringTok{"Com curva teórica sobreposta"}\NormalTok{,}
    \AttributeTok{x =} \StringTok{"Valor de F"}\NormalTok{,}
    \AttributeTok{y =} \StringTok{"Densidade"}
\NormalTok{  ) }\SpecialCharTok{+}
  \FunctionTok{theme\_minimal}\NormalTok{(}\AttributeTok{base\_size =} \DecValTok{13}\NormalTok{)}
\end{Highlighting}
\end{Shaded}

\begin{figure}[H]

\centering{

\includegraphics[width=0.7\linewidth,height=0.7\textheight]{15-anova_files/figure-pdf/fig-distfcurve-1.pdf}

}

\caption{\label{fig-distfcurve}Histograma com curva sobreposta de uma
distribuição F (20,100)}

\end{figure}%

Observando o gráfico da Figura~\ref{fig-distfcurve}, verifica-se que a
curva se assemelha muito com a curva normal (Seção~\ref{sec-normal}).

\section{ANOVA de um fator}\label{anova-de-um-fator}

A \textbf{análise de variância (ANOVA) de um fator}, também conhecida
como ANOVA de uma via, é uma extensão do teste \emph{t} independente
para comparar duas médias em uma situação em que há mais de dois grupos.
Dito de outra forma, o teste \emph{t} para uso com duas amostras
independentes é um caso especial da análise de variância de uma via. A
ANOVA de um fator compara o efeito de uma variável preditora (variável
independente, fator) sobre uma variável contínua (desfecho). Por
exemplo, verificar se a intensidade do tabagismo na gestação afeta o
peso dos recém-nascidos Figura~\ref{fig-bxpfumo}.

\subsection{Dados do exemplo}\label{dados-do-exemplo}

Para testar a hipótese de que a intensidade do tabagismo materno afeta o
peso do recém-nascido, foram selecionadas as variáveis
\texttt{quantFumo} (número de cigarros por dia) e \texttt{pesoRN} (peso
ao nascer, em gramas) do banco de dados \texttt{dadosMater.xlsx}
(Seção~\ref{sec-dadosMater}). A variável \texttt{quantFumo} foi
categorizada em três níveis, formando a variável \texttt{tabagismo}, por
meio das funções \texttt{mutate()} e \texttt{case\_when()}:

\begin{itemize}
\tightlist
\item
  \textbf{Não fumantes}: gestantes que relataram nunca ter fumado;\\
\item
  \textbf{Fumantes leves}: gestantes que fumavam até 10 cigarros por
  dia;\\
\item
  \textbf{Fumantes intensas}: gestantes que fumavam mais de 10 cigarros
  por dia.
\end{itemize}

Em seguida, foram filtrados os recém-nascidos a termo\footnote{Filtrar
  recém-nascidos a termo (≥37 e \textless42 semanas) foi uma escolha
  para reduzir variabilidade.} (entre 37 e 41 semanas completas de
gestação), e o resultado foi atribuído ao objeto \texttt{dados}. Como os
grupos apresentavam tamanhos amostrais bastante
desbalanceados\footnote{O desbalanceamento pode introduzir vieses que
  podem afetar a validade das conclusões, pois grupos pequenos podem não
  ser representativos; os grupos maiores podem ter uma média mais
  estável, enquanto que os menores podem ter flutuações aleatórias ou
  outliers.}, foi realizada uma subamostragem aleatória\footnote{A
  subamostragem foi realizada com controle de aleatoriedade (set.seed)
  para garantir reprodutibilidade, e os grupos foram praticamente
  igualados , permitindo aplicação da ANOVA de uma via com maior
  robustez e comparabilidade entre os níveis de exposição ao tabagismo.}
de 120 gestantes não fumantes, de modo a equilibrar os três grupos para
a análise. O conjunto final de dados balanceados foi atribuído ao objeto
\texttt{dados\_balanceados}.

\begin{Shaded}
\begin{Highlighting}[]
\NormalTok{dados }\OtherTok{\textless{}{-}}\NormalTok{ readxl}\SpecialCharTok{::}\FunctionTok{read\_excel}\NormalTok{(}\StringTok{"dados/dadosmater.xlsx"}\NormalTok{) }\SpecialCharTok{\%\textgreater{}\%} 
\NormalTok{  dplyr}\SpecialCharTok{::}\FunctionTok{select}\NormalTok{(quantFumo, pesoRN, ig) }\SpecialCharTok{\%\textgreater{}\%}   
\NormalTok{  dplyr}\SpecialCharTok{::}\FunctionTok{mutate}\NormalTok{(}\AttributeTok{tabagismo =} \FunctionTok{case\_when}\NormalTok{(}
\NormalTok{                  quantFumo }\SpecialCharTok{==} \DecValTok{0} \SpecialCharTok{\textasciitilde{}} \StringTok{"Não"}\NormalTok{,}
\NormalTok{                  quantFumo }\SpecialCharTok{\textgreater{}}\DecValTok{0} \SpecialCharTok{\&}\NormalTok{ quantFumo }\SpecialCharTok{\textless{}=} \DecValTok{10} \SpecialCharTok{\textasciitilde{}} \StringTok{"Leve"}\NormalTok{,}
\NormalTok{                  quantFumo }\SpecialCharTok{\textgreater{}} \DecValTok{10}  \SpecialCharTok{\textasciitilde{}} \StringTok{"Intenso"}\NormalTok{),}
                \AttributeTok{tabagismo =} \FunctionTok{factor}\NormalTok{(tabagismo,}
                                   \AttributeTok{levels =} \FunctionTok{c}\NormalTok{(}\StringTok{"Não"}\NormalTok{, }\StringTok{"Leve"}\NormalTok{, }\StringTok{"Intenso"}\NormalTok{))) }\SpecialCharTok{\%\textgreater{}\%} 
\NormalTok{  dplyr}\SpecialCharTok{::}\FunctionTok{filter}\NormalTok{(ig}\SpecialCharTok{\textgreater{}=}\DecValTok{37} \SpecialCharTok{\&}\NormalTok{ ig }\SpecialCharTok{\textless{}} \DecValTok{42}\NormalTok{)}

\FunctionTok{str}\NormalTok{(dados)}
\end{Highlighting}
\end{Shaded}

\begin{verbatim}
tibble [1,085 x 4] (S3: tbl_df/tbl/data.frame)
 $ quantFumo: num [1:1085] 0 0 20 0 20 20 0 0 0 0 ...
 $ pesoRN   : num [1:1085] 3285 3100 3100 2800 3270 ...
 $ ig       : num [1:1085] 37 37 37 38 39 39 39 39 39 39 ...
 $ tabagismo: Factor w/ 3 levels "Não","Leve","Intenso": 1 1 3 1 3 3 1 1 1 1 ...
\end{verbatim}

\begin{Shaded}
\begin{Highlighting}[]
\FunctionTok{table}\NormalTok{(dados}\SpecialCharTok{$}\NormalTok{tabagismo)}
\end{Highlighting}
\end{Shaded}

\begin{verbatim}

    Não    Leve Intenso 
    853     121     111 
\end{verbatim}

\begin{Shaded}
\begin{Highlighting}[]
\CommentTok{\# Subamostragem dos não fumantes}
\FunctionTok{set.seed}\NormalTok{(}\DecValTok{123}\NormalTok{)  }\CommentTok{\# para reprodutibilidade}

\NormalTok{nao\_fumantes }\OtherTok{\textless{}{-}}\NormalTok{ dados }\SpecialCharTok{\%\textgreater{}\%}
  \FunctionTok{filter}\NormalTok{(tabagismo }\SpecialCharTok{==} \StringTok{"Não"}\NormalTok{) }\SpecialCharTok{\%\textgreater{}\%}
  \FunctionTok{slice\_sample}\NormalTok{(}\AttributeTok{n=}\DecValTok{120}\NormalTok{)}

\CommentTok{\# Manter os outros grupos como estão}
\NormalTok{leves }\OtherTok{\textless{}{-}}\NormalTok{ dados }\SpecialCharTok{\%\textgreater{}\%} \FunctionTok{filter}\NormalTok{(tabagismo }\SpecialCharTok{==} \StringTok{"Leve"}\NormalTok{)}
\NormalTok{intensos }\OtherTok{\textless{}{-}}\NormalTok{ dados }\SpecialCharTok{\%\textgreater{}\%} \FunctionTok{filter}\NormalTok{(tabagismo }\SpecialCharTok{==} \StringTok{"Intenso"}\NormalTok{)}

\CommentTok{\# Unir os três grupos balanceados}
\NormalTok{dados\_balanceados }\OtherTok{\textless{}{-}} \FunctionTok{bind\_rows}\NormalTok{(nao\_fumantes, leves, intensos)}

\FunctionTok{str}\NormalTok{(dados\_balanceados)}
\end{Highlighting}
\end{Shaded}

\begin{verbatim}
tibble [352 x 4] (S3: tbl_df/tbl/data.frame)
 $ quantFumo: num [1:352] 0 0 0 0 0 0 0 0 0 0 ...
 $ pesoRN   : num [1:352] 3110 2580 2965 3430 3060 ...
 $ ig       : num [1:352] 40 41 39 37 39 41 37 39 39 39 ...
 $ tabagismo: Factor w/ 3 levels "Não","Leve","Intenso": 1 1 1 1 1 1 1 1 1 1 ...
\end{verbatim}

\begin{Shaded}
\begin{Highlighting}[]
\FunctionTok{table}\NormalTok{(dados\_balanceados}\SpecialCharTok{$}\NormalTok{tabagismo)}
\end{Highlighting}
\end{Shaded}

\begin{verbatim}

    Não    Leve Intenso 
    120     121     111 
\end{verbatim}

\begin{tcolorbox}[enhanced jigsaw, bottomrule=.15mm, opacitybacktitle=0.6, colframe=quarto-callout-caution-color-frame, arc=.35mm, coltitle=black, toptitle=1mm, colback=white, colbacktitle=quarto-callout-caution-color!10!white, breakable, bottomtitle=1mm, rightrule=.15mm, titlerule=0mm, toprule=.15mm, opacityback=0, leftrule=.75mm, left=2mm, title=\textcolor{quarto-callout-caution-color}{\faFire}\hspace{0.5em}{CUIDADO! Perda do Poder estatístico}]

Ao reduzir o grupo de \texttt{não\ fumantes} de 853 para 120, perde-se
informação. E isso pode diminuir a precisão da estimativa da média desse
grupo. Aumenta o risco de erro tipo II (veja Seção~\ref{sec-erros}).

Entretanto, como o objetivo é comparar grupos de forma justa, esse
sacrifício é aceitável --- especialmente em análises exploratórias ou
didáticas.

\end{tcolorbox}

\subsubsection{Exploração e resumo dos
dados}\label{explorauxe7uxe3o-e-resumo-dos-dados-1}

As medidas resumidoras serão obtidas, usando as funções
\texttt{group\_by\ ()} e \texttt{summarise\ ()} do pacote
\texttt{dplyr}.

\begin{Shaded}
\begin{Highlighting}[]
\NormalTok{alpha }\OtherTok{=} \FloatTok{0.05}
\NormalTok{resumo }\OtherTok{\textless{}{-}}\NormalTok{ dados\_balanceados }\SpecialCharTok{\%\textgreater{}\%}
\NormalTok{  dplyr}\SpecialCharTok{::}\FunctionTok{group\_by}\NormalTok{(tabagismo) }\SpecialCharTok{\%\textgreater{}\%}
\NormalTok{  dplyr}\SpecialCharTok{::}\FunctionTok{summarise}\NormalTok{(}\AttributeTok{n =} \FunctionTok{n}\NormalTok{(),}
                   \AttributeTok{media =} \FunctionTok{mean}\NormalTok{(pesoRN, }\AttributeTok{na.rm =} \ConstantTok{TRUE}\NormalTok{),}
                   \AttributeTok{dp =} \FunctionTok{sd}\NormalTok{ (pesoRN, }\AttributeTok{na.rm =} \ConstantTok{TRUE}\NormalTok{),}
                   \AttributeTok{ep =}\NormalTok{ dp}\SpecialCharTok{/}\FunctionTok{sqrt}\NormalTok{(n),}
                   \AttributeTok{me =} \FunctionTok{qt}\NormalTok{ ((}\DecValTok{1}\SpecialCharTok{{-}}\NormalTok{alpha}\SpecialCharTok{/}\DecValTok{2}\NormalTok{), n}\DecValTok{{-}1}\NormalTok{)}\SpecialCharTok{*}\NormalTok{ep,}
                   \AttributeTok{IC\_Inf =}\NormalTok{ media }\SpecialCharTok{{-}}\NormalTok{ me,}
                   \AttributeTok{IC\_sup =}\NormalTok{ media }\SpecialCharTok{+}\NormalTok{ me)}
\NormalTok{resumo}
\end{Highlighting}
\end{Shaded}

\begin{verbatim}
# A tibble: 3 x 8
  tabagismo     n media    dp    ep    me IC_Inf IC_sup
  <fct>     <int> <dbl> <dbl> <dbl> <dbl>  <dbl>  <dbl>
1 Não         120 3235.  461.  42.1  83.4  3152.  3319.
2 Leve        121 3121.  444.  40.4  80.0  3041.  3201.
3 Intenso     111 3045.  533.  50.6 100.   2944.  3145.
\end{verbatim}

\begin{Shaded}
\begin{Highlighting}[]
\NormalTok{media\_geral }\OtherTok{\textless{}{-}} \FunctionTok{mean}\NormalTok{(dados\_balanceados}\SpecialCharTok{$}\NormalTok{pesoRN)}
\FunctionTok{round}\NormalTok{(media\_geral, }\DecValTok{0}\NormalTok{)}
\end{Highlighting}
\end{Shaded}

\begin{verbatim}
[1] 3136
\end{verbatim}

\subsubsection{Visualização gráfica dos
dados}\label{visualizauxe7uxe3o-gruxe1fica-dos-dados}

Os boxplots (Figura~\ref{fig-bxpfumo}) são uma maneira interessante de
visualizar os dados, principalmente com o pacote
\texttt{ggplot2}\footnote{Volte à Seção~\ref{sec-ggplot2} para mais
  informações sobre o como fazer gráficos no \texttt{ggplot2.}} :

\begin{figure}

\centering{

\includegraphics[width=0.7\linewidth,height=0.7\textheight]{15-anova_files/figure-pdf/fig-bxpfumo-1.pdf}

}

\caption{\label{fig-bxpfumo}Boxplots do impacto do tabagismo materno no
peso ao nascer}

\end{figure}%

Observa-se que há uma tendência de o peso ao nascer diminuir à medida
que quantidade de cigarros fumados aumenta. Entretanto, esta diferença
pode por acaso.

\subsection{Definição das hipóteses
estatísticas}\label{definiuxe7uxe3o-das-hipuxf3teses-estatuxedsticas-2}

Para testar a igualdade entre as médias,
\(H_{0}: \mu_{1} = \mu_{2} =  \mu_{3} =  \mu_{4}\), supondo
homocedasticidade, isto é, as variâncias \(σ_1^2=σ_2^2=σ_3^3=σ_4^2\).

A hipótese alternativa, \(H_1\), diz que, pelo menos, uma das médias é
diferente das demais. Ela não é unilateral ou bilateral, é
\emph{multifacetada} porque permite qualquer relação que não seja
\emph{todas as médias iguais}. Por exemplo, a \(H_1\) inclui o caso em
que \(μ_1=μ_2=μ_3\), mas \(μ_4\) tem um valor diferente.

\subsection{Definição da regra de
decisão}\label{definiuxe7uxe3o-da-regra-de-decisuxe3o-1}

O nível significância, \(\alpha\), geralmente escolhido é igual a 0,05.
A distribuição da estatística do teste, sob a \(H_{0}\), é a
distribuição \emph{F}. O número de graus de liberdade total \((n – 1)\)
é dividido em dois componentes:

\begin{itemize}
\tightlist
\item
  Grau de liberdade do numerador (ENTRE) é dado por \(gl_{E} = k - 1\),
  onde \emph{k} é o número de grupos.
\item
  Grau de liberdade do denominador (DENTRO ou residual) é dado por
  \(gl_{D} = n - k\), onde, \(n = \sum n_{i}\).
\end{itemize}

No exemplo, para um \(\alpha = 0,05\), tem-se:

\begin{Shaded}
\begin{Highlighting}[]
\NormalTok{alpha }\OtherTok{\textless{}{-}} \FloatTok{0.05}
\NormalTok{k }\OtherTok{\textless{}{-}}  \FunctionTok{length}\NormalTok{(resumo}\SpecialCharTok{$}\NormalTok{media)}
\NormalTok{n }\OtherTok{\textless{}{-}} \FunctionTok{nrow}\NormalTok{(dados\_balanceados)}
\NormalTok{glE }\OtherTok{\textless{}{-}}\NormalTok{  k }\SpecialCharTok{{-}} \DecValTok{1}
\NormalTok{glE}
\end{Highlighting}
\end{Shaded}

\begin{verbatim}
[1] 2
\end{verbatim}

\begin{Shaded}
\begin{Highlighting}[]
\NormalTok{glD }\OtherTok{\textless{}{-}}\NormalTok{ n }\SpecialCharTok{{-}}\NormalTok{ k}
\NormalTok{glD}
\end{Highlighting}
\end{Shaded}

\begin{verbatim}
[1] 349
\end{verbatim}

Com esses dados, usando a a função \texttt{qf()}calcula-se o valor
crítico de \emph{F} (Figura~\ref{fig-fc2349}) que é igual:

\begin{Shaded}
\begin{Highlighting}[]
\NormalTok{Fcrit }\OtherTok{\textless{}{-}} \FunctionTok{qf}\NormalTok{(}\DecValTok{1} \SpecialCharTok{{-}}\NormalTok{ alpha, glE, glD)}
\FunctionTok{round}\NormalTok{(Fcrit, }\DecValTok{2}\NormalTok{)}
\end{Highlighting}
\end{Shaded}

\begin{verbatim}
[1] 3.02
\end{verbatim}

Portanto, se

\[
\begin{array}{l}
|F_{calculado}| < |F_{crítico}| \Rightarrow \text{não se rejeita } H_0 \\
|F_{calculado}| \ge |F_{crítico}| \Rightarrow \text{rejeita-se } H_0
\end{array}
\]

\begin{figure}

\centering{

\includegraphics[width=0.8\linewidth,height=0.8\textheight]{15-anova_files/figure-pdf/fig-fc2349-1.pdf}

}

\caption{\label{fig-fc2349}Curva da Distribuição F 3,196 = 2,65}

\end{figure}%

\subsection{Teste Estatístico}\label{teste-estatuxedstico-3}

A estatística de teste é obtida calculando duas estimativas da variância
populacional, \(\sigma^2\): a \emph{variância entre os grupos}
(\(s_{E}^2\)) e a \emph{variância dentro dos grupos} (\(s_{D}^2\)).

A variância entre os grupos também é chamada de \emph{quadrado médio
entre os grupos} (\(QM_{E}\)) e é igual a soma dos quadrados entre
(\(SQ_{E}\)) ou do fator dividida pelos graus de liberdade entre:

\[
QM_{E} = \frac{SQ_{E}}{gl_{E}}
\]

A variância dentro dos grupos é também denominada de \emph{quadrado
médio dentro dos grupos} ou residual (\(QM_{D}\)) e é igual a soma dos
quadrados dentro dividida pelos graus de liberdade dentro:

\[
QM_{D} = \frac {SQ_{D}}{gl_{D}}
\]

A variância entre os grupos, \(QM_{E}\), dá uma estimativa de
\(\sigma^2\) com base na variação entre as médias das amostras extraídas
de diferentes populações. Para o exemplo das três categorias de
tabagismo durante a gestação, o \(QM_{E}\) será baseado nos valores das
médias dos pesos dos recém-nascidos nos três grupos diferentes. Se as
médias de todas as populações em consideração forem iguais, as médias
das respectivas amostras ainda serão diferentes, mas a variação entre
elas deverá ser pequena e, consequentemente, espera-se que o valor do
\(QM_{E}\) seja pequeno. No entanto, se as médias das populações
consideradas não são todas iguais, espera-se que a variação entre as
médias das respectivas amostras seja grande e, consequentemente, o valor
de \(QM_{E}\) seja grande.

A variância dentro das amostras, \(QM_{D}\), dá uma estimativa de
\(\sigma^2\) com base na variação dos dados de diferentes amostras. Para
o exemplo das tr\textasciitilde es categorias de tabagismo durante a
gestação, o \(QM_{D}\) será baseado nas médias individuais dos pesos dos
recém-nascidos incluídos nas três amostras retiradas de três populações.
O conceito de \(QM_{D}\) é semelhante ao conceito de desvio padrão
conjugado ou agrupado, \(s_{o}\), para duas amostras.

A estatística de teste é, como visto, a \textbf{razão das variâncias
entre e dentro do grupo}. Dessa maneira,

\[
F = \frac {s_{E}^2}{s_{D}^2} = \frac {\frac {SQ_{E}}{gl_{E}}}{\frac {SQ_{D}}{gl_{D}}} = \frac {QM_{E}}{QM_{D}}
\]

\subsubsection{Avaliação dos pressupostos do
teste}\label{avaliauxe7uxe3o-dos-pressupostos-do-teste}

Ao realizar um teste de ANOVA de um fator deve-se assumir que:

\begin{enumerate}
\def\labelenumi{\arabic{enumi}.}
\tightlist
\item
  As populações das quais as amostras são retiradas são normalmente
  distribuídas;
\item
  As populações das quais as amostras são retiradas têm a mesma
  variância (homocedasticidade);
\item
  Amostras aleatórias e independentes;
\item
  Todos os grupos devem ter tamanho amostral adequado. Grupos com menos
  de 10 participantes são problemáticos por reduzirem a precisão da
  média. Na prática, deve-se evitar menos de 30 participantes. A relação
  entre os grupos não deve ser maior do que 1:4 (119);
\item
  Não devem existir valores atípicos (\emph{outliers});
\item
  A mensuração dos dados deve ser em nível intervalar ou de razão.
\end{enumerate}

Portanto, antes iniciar com o teste de hipótese, verifica-se se as
suposições mencionadas para o teste de hipótese ANOVA unidirecional
foram atendidas.

As amostras são amostras aleatórias e independentes. Isto já é um bom
começo!

\ul{Avaliação da normalidade}

Verifica-se a premissa de normalidade, usando o teste de Shapiro-Wilk
para os múltiplos grupos e desenhando um gráfico de probabilidade normal
(\emph{gráficos Q-Q}) para cada grupo.

\begin{Shaded}
\begin{Highlighting}[]
\NormalTok{ dados\_balanceados }\SpecialCharTok{\%\textgreater{}\%} 
\NormalTok{  dplyr}\SpecialCharTok{::}\FunctionTok{group\_by}\NormalTok{(tabagismo) }\SpecialCharTok{\%\textgreater{}\%} 
  \FunctionTok{shapiro\_test}\NormalTok{(pesoRN)}
\end{Highlighting}
\end{Shaded}

\begin{verbatim}
# A tibble: 3 x 4
  tabagismo variable statistic     p
  <fct>     <chr>        <dbl> <dbl>
1 Não       pesoRN       0.991 0.582
2 Leve      pesoRN       0.984 0.152
3 Intenso   pesoRN       0.984 0.188
\end{verbatim}

Para o gráfico Q-Q (Figura~\ref{fig-qqfumo}), pode ser usado a função
\texttt{ggqqplot\ ()} do pacote \texttt{ggpubr} que produz um gráfico QQ
normal com uma linha de referência, acompanhada de area sombreada,
correspondente ao IC95\%.

\begin{Shaded}
\begin{Highlighting}[]
\NormalTok{ggpubr}\SpecialCharTok{::}\FunctionTok{ggqqplot}\NormalTok{(}
\NormalTok{  dados\_balanceados,}
  \AttributeTok{x =} \StringTok{"pesoRN"}\NormalTok{,}
  \AttributeTok{facet.by =} \StringTok{"tabagismo"}\NormalTok{,}
  \AttributeTok{color =} \StringTok{"tabagismo"}\NormalTok{,               }
  \AttributeTok{palette =} \FunctionTok{c}\NormalTok{(}\StringTok{"forestgreen"}\NormalTok{, }\StringTok{"darkgoldenrod2"}\NormalTok{, }\StringTok{"red1"}\NormalTok{),                     ) }\SpecialCharTok{+}
  \FunctionTok{labs}\NormalTok{(}\AttributeTok{y =} \StringTok{"Peso ao nascer (g) (m)"}\NormalTok{,}
       \AttributeTok{x =} \StringTok{"Quantis teóricos"}\NormalTok{) }\SpecialCharTok{+}
  \FunctionTok{theme}\NormalTok{(}\AttributeTok{legend.position =} \StringTok{"none"}\NormalTok{)}
\end{Highlighting}
\end{Shaded}

\begin{figure}[H]

\centering{

\includegraphics[width=0.9\linewidth,height=0.9\textheight]{15-anova_files/figure-pdf/fig-qqfumo-1.pdf}

}

\caption{\label{fig-qqfumo}Gráficos Q-Q}

\end{figure}%

O resultado do teste de Shapiro-Wilk entregou todos os resultados com
valor \emph{P} acima de 0.05 e os gráficos Q-Q, não são perfeitos, mas
pode-se assumir que os dados para cada grupo caem aproximadamente em uma
linha reta.

\ul{Avaliação da homogeneidade das variâncias}

Em seguida, testa-se a suposição de que as variâncias são iguais, usando
o Teste de Levene através da função \texttt{leveneTest\ ()} do pacote
\texttt{car}.

\begin{Shaded}
\begin{Highlighting}[]
\NormalTok{car}\SpecialCharTok{::}\FunctionTok{leveneTest}\NormalTok{(pesoRN}\SpecialCharTok{\textasciitilde{}}\NormalTok{tabagismo, }
                \AttributeTok{center =}\NormalTok{ mean, }
                \AttributeTok{data =}\NormalTok{ dados\_balanceados)}
\end{Highlighting}
\end{Shaded}

\begin{verbatim}
Levene's Test for Homogeneity of Variance (center = mean)
       Df F value Pr(>F)
group   2  1.7955 0.1676
      349               
\end{verbatim}

O teste de Levene exibe como resultado um valor \emph{p} \textgreater{}
0,05, mostrando que não é possível rejeitar a \(H_0\) de igualdade das
variâncias.

O teste de Levene calcula a \textbf{distância de cada observação ao
centro do grupo} (média ou mediana), e testa se essas distâncias têm
variâncias semelhantes entre os grupos. Quando os dados são
aproximadamente normais usa-se \texttt{center\ =\ mean}. É mais
sensível, mas menos robusto. Quando há outliers ou dados assimétricos,
recomenda-se \texttt{center\ =\ median}, porque é mais
robusto\footnote{Um método é mais \textbf{robusto} quando ele
  \textbf{continua funcionando bem mesmo que algumas suposições sejam
  violadas}, como presença de outliers ou dados não perfeitamente
  normais.}.

Observe que , neste exemplo, tanto faz usar um ou outro método, pois os
dados não violam o a pressuposição de noemalidade:

\begin{Shaded}
\begin{Highlighting}[]
\NormalTok{car}\SpecialCharTok{::}\FunctionTok{leveneTest}\NormalTok{ (pesoRN}\SpecialCharTok{\textasciitilde{}}\NormalTok{tabagismo, }
                 \AttributeTok{center =}\NormalTok{ median, }
                 \AttributeTok{data =}\NormalTok{ dados\_balanceados)}
\end{Highlighting}
\end{Shaded}

\begin{verbatim}
Levene's Test for Homogeneity of Variance (center = median)
       Df F value Pr(>F)
group   2  1.7984 0.1671
      349               
\end{verbatim}

\ul{Verificação da presença de outliers}

Pode-se aqui, além de verificar nos boxplots, usar a função
\texttt{by\_group()} do pacote \texttt{dplyr} junto com a função
\texttt{identify\_outliers()} do pacote \texttt{rstatix} (112) :

\begin{Shaded}
\begin{Highlighting}[]
\NormalTok{dados\_balanceados }\SpecialCharTok{\%\textgreater{}\%} 
\NormalTok{  dplyr}\SpecialCharTok{::}\FunctionTok{group\_by}\NormalTok{(tabagismo) }\SpecialCharTok{\%\textgreater{}\%} 
\NormalTok{  rstatix}\SpecialCharTok{::}\FunctionTok{identify\_outliers}\NormalTok{(pesoRN)}
\end{Highlighting}
\end{Shaded}

\begin{verbatim}
# A tibble: 11 x 6
   tabagismo quantFumo pesoRN    ig is.outlier is.extreme
   <fct>         <dbl>  <dbl> <dbl> <lgl>      <lgl>     
 1 Não               0   2051    39 TRUE       FALSE     
 2 Não               0   4305    39 TRUE       FALSE     
 3 Não               0   4300    40 TRUE       FALSE     
 4 Não               0   4315    39 TRUE       FALSE     
 5 Leve              5   4350    37 TRUE       FALSE     
 6 Leve              8   4205    39 TRUE       FALSE     
 7 Leve             10   1715    37 TRUE       FALSE     
 8 Leve             10   4620    40 TRUE       FALSE     
 9 Intenso          20   1440    39 TRUE       FALSE     
10 Intenso          20   4410    40 TRUE       FALSE     
11 Intenso          20   4390    40 TRUE       FALSE     
\end{verbatim}

Como mostrado nos boxplots, existem vários valores atípicos, ou seja,
valores que estão fora de \(\pm\) 1,5 IIQ. Entretanto, nenhum é extremo
(fora de \(\pm\) 3 IIQ).

Da mesma maneira que no teste \emph{t}, os pressupostos têm mais
importância em grupos pequenos e desiguais. Para o exemplo em análise,
os pressupostos foram verificados e pode-se assumir que os grupos são
independentes e as médias têm distribuição normal e existe
homocedasticidade, além disso, os grupos têm o tamanhos semelhantes.
Portanto, a análise pode ser continuada.

\ul{O que fazer se os pressupostos são violados?}

Se a homogeneidade da variância é o problema, um teste possível de ser
implementado no \emph{R} é o \emph{F de Welch}, aplicando a
função\texttt{welch.test()}, incluída no pacote \texttt{onewaytests}
(120). Existem também testes não paramétricos, como o \emph{Teste de
Kruskal-Wallis}, que será visto mais adiante
(\textbf{?@sec-kruskalwallis}).

\subsubsection{Execução do teste
estatístico}\label{execuuxe7uxe3o-do-teste-estatuxedstico-1}

Para realizar um teste de hipótese ANOVA unidirecional, aplica-se a
função \texttt{aov()} do R base\footnote{O teste estatístico também pode
  ser realizado com a função \texttt{anova\_test()} do pacote
  \texttt{rstatix} que será usado na construção da
  Figura~\ref{fig-view}.}. Esta função espera a chamada notação de
\texttt{fórmula}, portanto, os dados são incluídos separando as duas
variáveis de interesse separadas por \(\sim\) (til) e os dados
(\texttt{dados\_balanceados}), onde as variáveis especificadas na
fórmula, são encontradas.

\begin{Shaded}
\begin{Highlighting}[]
\NormalTok{modelo.aov }\OtherTok{\textless{}{-}} \FunctionTok{aov}\NormalTok{(pesoRN }\SpecialCharTok{\textasciitilde{}}\NormalTok{ tabagismo, dados\_balanceados)}

\FunctionTok{summary}\NormalTok{(modelo.aov)}
\end{Highlighting}
\end{Shaded}

\begin{verbatim}
             Df   Sum Sq Mean Sq F value Pr(>F)  
tabagismo     2  2136213 1068107   4.647 0.0102 *
Residuals   349 80209539  229827                 
---
Signif. codes:  0 '***' 0.001 '**' 0.01 '*' 0.05 '.' 0.1 ' ' 1
\end{verbatim}

A saída, liberada pela função \texttt{summary()}, é bem reduzida,
relatando as informações específicas da \emph{Tabela da ANOVA}, a
estatística \emph{F} junto com o valor \emph{p} e os graus de liberdade,
soma dos quadrados (\emph{Sum Sq}) e quadrados médios (\emph{Mean Sq}),
que com frequência se necessita para o relatório do modelo.

A variância entre os grupos também é chamada de \textbf{quadrado médio
entre os grupos} e é igual à soma dos quadrados entre ou do fator
dividida pelos graus de liberdade entre. A variância dentro dos grupos é
também denominada de \textbf{quadrado médio dentro dos grupos ou
residual} e é igual à soma dos quadrados dentro dividida pelos graus de
liberdade dentro.

A ANOVA detectou um efeito significativo do fator, que neste caso é o
\texttt{tabagismo}, o valor
\(F_{calculado} = 4,647 > F_{crítico} = 3.02\) e o valor \emph{p} =
0,01102 (\textless{} 0,05).

Pode-se simplesmente relatar isso e encerrar, mas é provável que se
queira saber quais grupos diferem uns dos outros. Lembre-se de que não
se pode apenas inferir isso a partir de uma visão dos dados, existem
testes estatísticos para ajudar a entender as diferenças dos grupos.

\subsection{\texorpdfstring{Testes
\emph{post-hoc}}{Testes post-hoc}}\label{testes-post-hoc}

Os testes de comparações múltiplas constituem-se em uma análise após a
realização da ANOVA. Se houve uma diferença, indicada pela ANOVA, os
testes de comparações múltiplas ou também conhecidos como \emph{teste
post hoc}, ajudam a quantificar as diferenças entre os grupos para
determinar quais grupos diferem significativamente uns dos outros.

Aqui será usado o \emph{HSD de Tukey}, que é conservador. \emph{HSD} vem
da expressão em inglês - \emph{Honest Significant Difference
(\textbf{*damasio2021posthoc?}). Este teste requer um objeto
\texttt{aov} no qual executa seu procedimento, que chamaremos de
\texttt{pwc}\footnote{Foi escolhido o nome \texttt{pwc} com o objtivo de
  lembrar que uma comparação de pares (\textbf{P}air\textbf{w}ise
  \textbf{C}omparison).}. O procedimento de Tukey HSD executará uma
comparação de pares de todas as combinações possíveis dos grupos e
testará esses pares para diferenças significativas entre suas médias,
tudo enquanto ajusta o valor }p* a um limite superior de significância
para compensar o fato de que muitos testes estatísticos estão sendo
realizados e a probabilidade de um falso positivo aumenta com o aumento
do número de testes. A função a ser usada é a \texttt{tukey\_hsd()}, do
pacote \texttt{rstatix}.

\begin{Shaded}
\begin{Highlighting}[]
\NormalTok{pwc }\OtherTok{\textless{}{-}}\NormalTok{ rstatix}\SpecialCharTok{::}\FunctionTok{tukey\_hsd}\NormalTok{ (modelo.aov)}
\NormalTok{pwc}
\end{Highlighting}
\end{Shaded}

\begin{verbatim}
# A tibble: 3 x 9
  term      group1 group2  null.value estimate conf.low conf.high   p.adj
* <chr>     <chr>  <chr>        <dbl>    <dbl>    <dbl>     <dbl>   <dbl>
1 tabagismo Não    Leve             0   -114.     -259.      31.6 0.158  
2 tabagismo Não    Intenso          0   -191.     -339.     -42.1 0.00761
3 tabagismo Leve   Intenso          0    -77.0    -225.      71.3 0.441  
# i 1 more variable: p.adj.signif <chr>
\end{verbatim}

Com base nos valores \emph{p} \textless{} 0,05 tem-se três combinações
de grupos: Não-Leve, Não-Intenso e Leve-Intenso Isto mostra que uma
diferença significativa (p = 0,0076) entre mães não fumantes e as
fumates intensas. Os demais grupos não apresentaram uma difirença
significativa.

Pode-se visualizar isso na Figura~\ref{fig-tukeyhsd} obtida , usando os
resultados da função \texttt{tukey\_hsd()} Esta função gera o teste de
Tukey com as diferença entre os pares e os intervalos de confiança que
permitem a construção do gráfico, com o código abaixo:

\begin{Shaded}
\begin{Highlighting}[]
\CommentTok{\# Criar uma coluna com os pares comparados}
\NormalTok{pwc}\SpecialCharTok{$}\NormalTok{comparacao }\OtherTok{\textless{}{-}} \FunctionTok{paste}\NormalTok{(pwc}\SpecialCharTok{$}\NormalTok{group1, }\StringTok{"vs"}\NormalTok{, pwc}\SpecialCharTok{$}\NormalTok{group2)}

\CommentTok{\# Reordenar os pares para o eixo Y}
\NormalTok{pwc}\SpecialCharTok{$}\NormalTok{comparacao }\OtherTok{\textless{}{-}} \FunctionTok{factor}\NormalTok{(pwc}\SpecialCharTok{$}\NormalTok{comparacao, }\AttributeTok{levels =} \FunctionTok{rev}\NormalTok{(pwc}\SpecialCharTok{$}\NormalTok{comparacao))}

\CommentTok{\# Gráfico horizontal}
\FunctionTok{ggplot}\NormalTok{(pwc, }\FunctionTok{aes}\NormalTok{(}\AttributeTok{x =}\NormalTok{ estimate, }\AttributeTok{y =}\NormalTok{ comparacao, }\AttributeTok{color =}\NormalTok{ p.adj.signif)) }\SpecialCharTok{+}
  \FunctionTok{geom\_point}\NormalTok{(}\AttributeTok{size =} \DecValTok{3}\NormalTok{) }\SpecialCharTok{+}
  \FunctionTok{geom\_errorbarh}\NormalTok{(}\FunctionTok{aes}\NormalTok{(}\AttributeTok{xmin =}\NormalTok{ conf.low, }\AttributeTok{xmax =}\NormalTok{ conf.high), }\AttributeTok{height =} \FloatTok{0.2}\NormalTok{, }\AttributeTok{size=}\FloatTok{1.2}\NormalTok{) }\SpecialCharTok{+}
  \FunctionTok{geom\_vline}\NormalTok{(}\AttributeTok{xintercept =} \DecValTok{0}\NormalTok{, }\AttributeTok{linetype =} \StringTok{"dashed"}\NormalTok{, }\AttributeTok{color =} \StringTok{"gray40"}\NormalTok{) }\SpecialCharTok{+}
  \FunctionTok{labs}\NormalTok{(}
    \AttributeTok{title =} \StringTok{"Teste de Tukey: Diferença entre grupos de tabagismo"}\NormalTok{,}
    \AttributeTok{x =} \StringTok{"Diferença estimada no peso ao nascer (g)"}\NormalTok{,}
    \AttributeTok{y =} \StringTok{"Comparações entre grupos"}\NormalTok{,}
    \AttributeTok{color =} \StringTok{"Significância"}
\NormalTok{  ) }\SpecialCharTok{+}
  \FunctionTok{theme\_bw}\NormalTok{(}\AttributeTok{base\_size =} \DecValTok{13}\NormalTok{)}
\end{Highlighting}
\end{Shaded}

\begin{figure}[H]

\centering{

\includegraphics[width=0.7\linewidth,height=0.7\textheight]{15-anova_files/figure-pdf/fig-tukeyhsd-1.pdf}

}

\caption{\label{fig-tukeyhsd}Gráficos do Teste de Tukey}

\end{figure}%

\subsection{Tamanho do efeito}\label{tamanho-do-efeito-2}

Uma das medidas de tamanho de efeito mais comumente relatadas para a
ANOVA é o \textbf{eta ao quadrado} (\(\eta^2\)), que é um índice da
força da associação entre um fator e uma variável dependente. Eta ao
quadrado é a proporção da variação total atribuível ao fator. É
calculado como a razão da variância do fator para a variância total e os
valores variam de 0 a 1.

Esta medida pode ser obtida com o pacote \texttt{effectsize} (121),
usando a função \texttt{eta\_squared()}com um objeto da classe tipo
\texttt{modelo.aov}.

\begin{Shaded}
\begin{Highlighting}[]
\NormalTok{effectsize}\SpecialCharTok{::}\FunctionTok{eta\_squared}\NormalTok{ (modelo.aov, }\AttributeTok{partial =} \ConstantTok{FALSE}\NormalTok{)}
\end{Highlighting}
\end{Shaded}

\begin{verbatim}
# Effect Size for ANOVA (Type I)

Parameter | Eta2 |       95% CI
-------------------------------
tabagismo | 0.03 | [0.00, 1.00]

- One-sided CIs: upper bound fixed at [1.00].
\end{verbatim}

O \emph{eta quadrado} é uma estimativa tendenciosa da força da
associação, na medida em que superestima os efeitos, especialmente para
amostras pequenas. Uma outra medida do tamanho do efeito menos
tendenciosa é o \emph{ômega ao quadrado} (\(\omega^2\)). O ômega ao
quadrado é uma medida corrigida, menos enviesada e menos inflacionada.
Ela pode ser calculada com a função \texttt{omega\_squared()}, também do
pacote \texttt{effectsize}:

\begin{Shaded}
\begin{Highlighting}[]
\NormalTok{effectsize}\SpecialCharTok{::}\FunctionTok{omega\_squared}\NormalTok{ (modelo.aov, }\AttributeTok{partial =} \ConstantTok{FALSE}\NormalTok{)}
\end{Highlighting}
\end{Shaded}

\begin{verbatim}
# Effect Size for ANOVA (Type I)

Parameter | Omega2 |       95% CI
---------------------------------
tabagismo |   0.02 | [0.00, 1.00]

- One-sided CIs: upper bound fixed at [1.00].
\end{verbatim}

Apesar de ser controverso, pode-se seguir a orientação da
Tabela~\ref{tbl-effectsize}), para a interpretação (122):

\global\setlength{\Oldarrayrulewidth}{\arrayrulewidth}

\global\setlength{\Oldtabcolsep}{\tabcolsep}

\setlength{\tabcolsep}{2pt}

\renewcommand*{\arraystretch}{1.5}



\providecommand{\ascline}[3]{\noalign{\global\arrayrulewidth #1}\arrayrulecolor[HTML]{#2}\cline{#3}}

\begin{longtable}[c]{|p{1.50in}|p{2.00in}}

\caption{\label{tbl-effectsize}Interpretação do Tamanho do Efeito}

\tabularnewline

\ascline{1.5pt}{666666}{1-2}

\multicolumn{1}{>{\raggedright}m{\dimexpr 1.5in+0\tabcolsep}}{\textcolor[HTML]{000000}{\fontsize{11}{11}\selectfont{\global\setmainfont{Arial}{\textbf{Resultado}}}}} & \multicolumn{1}{>{\raggedright}m{\dimexpr 2in+0\tabcolsep}}{\textcolor[HTML]{000000}{\fontsize{11}{11}\selectfont{\global\setmainfont{Arial}{\textbf{Tamanho\ do\ Efeito}}}}} \\

\ascline{1.5pt}{666666}{1-2}\endfirsthead 

\ascline{1.5pt}{666666}{1-2}

\multicolumn{1}{>{\raggedright}m{\dimexpr 1.5in+0\tabcolsep}}{\textcolor[HTML]{000000}{\fontsize{11}{11}\selectfont{\global\setmainfont{Arial}{\textbf{Resultado}}}}} & \multicolumn{1}{>{\raggedright}m{\dimexpr 2in+0\tabcolsep}}{\textcolor[HTML]{000000}{\fontsize{11}{11}\selectfont{\global\setmainfont{Arial}{\textbf{Tamanho\ do\ Efeito}}}}} \\

\ascline{1.5pt}{666666}{1-2}\endhead



\multicolumn{1}{>{\raggedright}m{\dimexpr 1.5in+0\tabcolsep}}{\textcolor[HTML]{000000}{\fontsize{11}{11}\selectfont{\global\setmainfont{Arial}{0.01}}}} & \multicolumn{1}{>{\raggedright}m{\dimexpr 2in+0\tabcolsep}}{\textcolor[HTML]{000000}{\fontsize{11}{11}\selectfont{\global\setmainfont{Arial}{pequeno}}}} \\





\multicolumn{1}{>{\raggedright}m{\dimexpr 1.5in+0\tabcolsep}}{\textcolor[HTML]{000000}{\fontsize{11}{11}\selectfont{\global\setmainfont{Arial}{0,06}}}} & \multicolumn{1}{>{\raggedright}m{\dimexpr 2in+0\tabcolsep}}{\textcolor[HTML]{000000}{\fontsize{11}{11}\selectfont{\global\setmainfont{Arial}{médio}}}} \\





\multicolumn{1}{>{\raggedright}m{\dimexpr 1.5in+0\tabcolsep}}{\textcolor[HTML]{000000}{\fontsize{11}{11}\selectfont{\global\setmainfont{Arial}{0,14}}}} & \multicolumn{1}{>{\raggedright}m{\dimexpr 2in+0\tabcolsep}}{\textcolor[HTML]{000000}{\fontsize{11}{11}\selectfont{\global\setmainfont{Arial}{grande}}}} \\

\ascline{1.5pt}{666666}{1-2}


\end{longtable}

\arrayrulecolor[HTML]{000000}

\global\setlength{\arrayrulewidth}{\Oldarrayrulewidth}

\global\setlength{\tabcolsep}{\Oldtabcolsep}

\renewcommand*{\arraystretch}{1}

\subsection{Conclusão}\label{conclusuxe3o-2}

O peso dos recém-nascidos foi estatisticamente diferente entre os
diferentes grupos, \emph{F}(2, 349) = 4.65, \emph{P} = 0.0102,
\(\eta^2\) = 0,03.

As análises \emph{post-hoc} de Tukey revelaram que o peso dos
recém-nascidos a termo no grupo das gestantes não fumantes apresentou
uma diferença estatisticamente significativa do grupo de tabagismo
intenso (-191g, IC95\%: -339 a -42.1 g; \emph{P} = 0,0076). Nos demais
grupos não houve diferença significativa.

\subsubsection{Apresentação dos
resultados}\label{apresentauxe7uxe3o-dos-resultados}

Serão apresentados boxplots (Figura~\ref{fig-view})), com
\texttt{ggboxplot()}, do pacote \texttt{ggpubr}, utilizando, para cores,
a \texttt{pallete\ =\ "jama"}, do pacote \texttt{ggsci}. Para adicionar
teste estatístico, usou-se a função \texttt{get\_test\_label()} e para o
teste \emph{post hoc}, a função \texttt{get\_pwc\_label()}, ambas do
pacote \texttt{rstatix}.

\begin{Shaded}
\begin{Highlighting}[]
\NormalTok{tab.aov }\OtherTok{\textless{}{-}} \FunctionTok{anova\_test}\NormalTok{(dados\_balanceados, }
\NormalTok{                      pesoRN }\SpecialCharTok{\textasciitilde{}}\NormalTok{ tabagismo, }
                      \AttributeTok{type =} \DecValTok{2}\NormalTok{)}

\NormalTok{pwc }\OtherTok{\textless{}{-}} \FunctionTok{tukey\_hsd}\NormalTok{(dados\_balanceados,pesoRN}\SpecialCharTok{\textasciitilde{}}\NormalTok{tabagismo)}
\NormalTok{pwc }\OtherTok{\textless{}{-}}\NormalTok{ pwc }\SpecialCharTok{\%\textgreater{}\%} \FunctionTok{add\_xy\_position}\NormalTok{ (}\AttributeTok{x =} \StringTok{"tabagismo"}\NormalTok{)}
\NormalTok{bxp }\OtherTok{\textless{}{-}}\NormalTok{ ggplot2}\SpecialCharTok{::}\FunctionTok{ggplot}\NormalTok{(dados\_balanceados, }\FunctionTok{aes}\NormalTok{(}\AttributeTok{x=}\NormalTok{tabagismo, }\AttributeTok{y=}\NormalTok{pesoRN)) }\SpecialCharTok{+}
  \FunctionTok{stat\_boxplot}\NormalTok{(}\AttributeTok{geom =} \StringTok{"errorbar"}\NormalTok{, }
               \AttributeTok{width =} \FloatTok{0.1}\NormalTok{) }\SpecialCharTok{+} 
  \FunctionTok{geom\_boxplot}\NormalTok{(}\FunctionTok{aes}\NormalTok{(}\AttributeTok{color =}\NormalTok{ tabagismo), }\AttributeTok{size =} \FloatTok{0.8}\NormalTok{) }\SpecialCharTok{+}
  \FunctionTok{scale\_color\_nejm}\NormalTok{() }\SpecialCharTok{+}
  \FunctionTok{labs}\NormalTok{(}\AttributeTok{x =} \StringTok{"Tabagismo"}\NormalTok{, }
       \AttributeTok{y =} \StringTok{"Peso ao nascer (g)"}\NormalTok{,}
       \AttributeTok{subtitle =} \FunctionTok{get\_test\_label}\NormalTok{ (tab.aov, }\AttributeTok{detailed =} \ConstantTok{TRUE}\NormalTok{),}
       \AttributeTok{caption =} \FunctionTok{get\_pwc\_label}\NormalTok{(pwc)) }\SpecialCharTok{+}
  \FunctionTok{stat\_pvalue\_manual}\NormalTok{ (pwc,}
                      \AttributeTok{label =} \StringTok{"p.adj.signif"}\NormalTok{,}
                      \AttributeTok{label.size =} \FloatTok{3.5}\NormalTok{,}
                      \AttributeTok{hide.ns =} \ConstantTok{TRUE}\NormalTok{) }\SpecialCharTok{+} 
  \FunctionTok{theme}\NormalTok{ (}\AttributeTok{text =} \FunctionTok{element\_text}\NormalTok{ (}\AttributeTok{size =} \DecValTok{12}\NormalTok{)) }\SpecialCharTok{+}
  \FunctionTok{theme\_classic}\NormalTok{(}\AttributeTok{base\_size =} \DecValTok{13}\NormalTok{) }\SpecialCharTok{+}
  \FunctionTok{theme}\NormalTok{(}\AttributeTok{legend.position =} \StringTok{"none"}\NormalTok{) }
\FunctionTok{print}\NormalTok{(bxp) }
\end{Highlighting}
\end{Shaded}

\begin{figure}[H]

\centering{

\includegraphics[width=0.8\linewidth,height=0.8\textheight]{15-anova_files/figure-pdf/fig-view-1.pdf}

}

\caption{\label{fig-view}Efeito do tabagismo na gestação sobre o peso do
recém-nascido.({[}**{]}: \emph{p} entre 0,001 e 0,01).}

\end{figure}%

\section{ANOVA de dois fatores}\label{anova-de-dois-fatores}

A \textbf{ANOVA de dois fatores} é uma extensão da ANOVA de um fator.
Neste tipo de ANOVA, ao invés de observar o efeito de um fator sobre a
variável desfecho contínua, é analisado simultaneamente o efeito de duas
variáveis de agrupamento. Outros sinônimos para a ANOVA de dois fatores
são: \emph{ANOVA fatorial} ou \emph{ANOVA de duas vias}. Quando se tem
dois ou mais fatores, além de observar o efeito desses fatores sobre a
variável desfecho, há necessidade de verificar se eles não interagem
entre si. Portanto, é um objetivo importante da ANOVA fatorial avaliar
se há um efeito de \emph{interação} estatisticamente significativo entre
os fatores.

\subsection{Dados do exemplo}\label{dados-do-exemplo-1}

O conjunto de dados \texttt{dadosMemoria.xlsx} que contém informações de
um teste de memória realizado em homens e mulheres, após o consumo de
álcool, categorizado em três grupos (nenhum, 3 latas\footnote{Três latas
  dessa cerveja equivalem a aproximadamente 4,5 unidades (12,5g) de
  álcool e elevam, após 3 h, a concentração sanguínea a níveis ao redor
  de 50 mg/dL (pessoas de 60 , 70 kg, considerado uma intoxicação leve.}
e 6 latas de cerveja tipo \emph{pilsen} com 4,5\% de álcool). O grupo
sem consumo de álcool (cerveja sem álcool) serve como controle. Após o
consumo de álcool, foi avaliada a memória para a realização de uma
tarefa cognitiva.

Neste exemplo, modificado de Andy Field (123), o efeito do álcool sobre
a memória do indivíduo é a variável focal, a principal preocupação.
Acredita-se que o efeito de álcool depende de outro fator, sexo, que são
chamados de variáveis moderadoras.

Para baixar o banco de dados, clique
\href{https://github.com/petronioliveira/Arquivos/blob/main/dadosMemoria.xlsx}{\textbf{aqui}}.
Salve o mesmo no seu diretório de trabalho.

\subsubsection{Leitura dos dados}\label{leitura-dos-dados-2}

A leitura será feita com a função \texttt{read\_excel()} do pacote
\texttt{readxl} e serão atribuídos a um objeto de nome \texttt{dados} e
verificada a sua estrutura com a função \texttt{str()}.

\begin{Shaded}
\begin{Highlighting}[]
\NormalTok{dados }\OtherTok{\textless{}{-}}\NormalTok{ readxl}\SpecialCharTok{::}\FunctionTok{read\_excel}\NormalTok{(}\StringTok{"dados/dadosMemoria.xlsx"}\NormalTok{) }\SpecialCharTok{\%\textgreater{}\%} 
\NormalTok{  dplyr}\SpecialCharTok{::}\FunctionTok{mutate}\NormalTok{(}\AttributeTok{sexo =} \FunctionTok{as.factor}\NormalTok{(sexo),}
                \AttributeTok{alcool =} \FunctionTok{factor}\NormalTok{(alcool, }
                                \AttributeTok{levels =} \FunctionTok{c}\NormalTok{(}\StringTok{"nenhum"}\NormalTok{,}
                                           \StringTok{"3 latas"}\NormalTok{,}
                                           \StringTok{"6 latas"}\NormalTok{)))}
\end{Highlighting}
\end{Shaded}

\subsubsection{Exploração e sumarização dos
dados}\label{explorauxe7uxe3o-e-sumarizauxe7uxe3o-dos-dados}

Na saída da função \texttt{str()}, verifica-se que as variáveis
\texttt{alcool} está como \texttt{factor} com 3 níveis colocados de
consumo de álcool em uma ordem lógica (nenhum consumo, três latas e 6
latas). A variável \texttt{sexo} estão como \texttt{fator} em dois
níveis: Feminino e Masculino em ordem alfabética (padrão do R) porque
não tem uma ordem lógica. As demais variáveis, \texttt{id}
(identificação) e \texttt{escore} (escore de memória) podem permanecer
com \texttt{num} (numérica).

A sumarização dos dados será feita com as funções \texttt{group\_by()} e
\texttt{summarise()} do pacote \texttt{dplyr} para a variável
\texttt{escore} por grupos, \texttt{sexo} e \texttt{alcool}.

\begin{Shaded}
\begin{Highlighting}[]
\NormalTok{alpha }\OtherTok{\textless{}{-}} \FloatTok{0.05}
\NormalTok{resumo }\OtherTok{\textless{}{-}}\NormalTok{ dados }\SpecialCharTok{\%\textgreater{}\%} 
\NormalTok{  dplyr}\SpecialCharTok{::}\FunctionTok{group\_by}\NormalTok{(sexo, alcool) }\SpecialCharTok{\%\textgreater{}\%} 
\NormalTok{  dplyr}\SpecialCharTok{::}\FunctionTok{summarise}\NormalTok{(}\AttributeTok{n =} \FunctionTok{n}\NormalTok{(),}
            \AttributeTok{media =} \FunctionTok{mean}\NormalTok{(escore, }\AttributeTok{na.rm=}\ConstantTok{TRUE}\NormalTok{),}
            \AttributeTok{dp =} \FunctionTok{sd}\NormalTok{(escore, }\AttributeTok{na.rm=}\ConstantTok{TRUE}\NormalTok{),}
            \AttributeTok{ep =}\NormalTok{ dp}\SpecialCharTok{/}\FunctionTok{sqrt}\NormalTok{(n),}
            \AttributeTok{me =} \FunctionTok{qt}\NormalTok{((}\DecValTok{1} \SpecialCharTok{{-}}\NormalTok{ alpha}\SpecialCharTok{/}\DecValTok{2}\NormalTok{),n}\DecValTok{{-}1}\NormalTok{)}\SpecialCharTok{*}\NormalTok{ep,}
            \AttributeTok{linf =}\NormalTok{ media }\SpecialCharTok{{-}}\NormalTok{ me,}
            \AttributeTok{lsup =}\NormalTok{ media }\SpecialCharTok{+}\NormalTok{ me)}
\NormalTok{resumo}
\end{Highlighting}
\end{Shaded}

\begin{verbatim}
# A tibble: 6 x 9
# Groups:   sexo [2]
  sexo      alcool      n media    dp    ep    me  linf  lsup
  <fct>     <fct>   <int> <dbl> <dbl> <dbl> <dbl> <dbl> <dbl>
1 Feminino  nenhum      8  60.6  4.96  1.75  4.14  56.5  64.8
2 Feminino  3 latas     8  62.5  6.55  2.31  5.47  57.0  68.0
3 Feminino  6 latas     8  57.5  7.07  2.5   5.91  51.6  63.4
4 Masculino nenhum      8  66.9 10.3   3.65  8.64  58.2  75.5
5 Masculino 3 latas     8  66.9 12.5   4.43 10.5   56.4  77.3
6 Masculino 6 latas     8  35.6 10.8   3.83  9.06  26.6  44.7
\end{verbatim}

Os dados estão estruturados com um desenho onde as células tem um
formato 2 x 3 (Tabela~\ref{tbl-tab2x3}) com os fatores \texttt{sexo} e
\texttt{alcool} e 8 indivíduos em cada célula. O fator \texttt{sexo} tem
dois níveis (feminino e masculino) e o fator \texttt{alcool} tem três
níveis (nenhum, 3 latas e 6 latas). Observe que o desenho é
\emph{balanceado}, pois todas as células têm o mesmo número de
indivíduos. Esta estrutura é o caso mais simples; desenhos não
balanceados são mais complexos.

\global\setlength{\Oldarrayrulewidth}{\arrayrulewidth}

\global\setlength{\Oldtabcolsep}{\tabcolsep}

\setlength{\tabcolsep}{2pt}

\renewcommand*{\arraystretch}{1.5}



\providecommand{\ascline}[3]{\noalign{\global\arrayrulewidth #1}\arrayrulecolor[HTML]{#2}\cline{#3}}

\begin{longtable}[c]{|p{1.50in}|p{1.50in}|p{1.50in}|p{1.50in}}

\caption{\label{tbl-tab2x3}Número de indivíduos por sexo}

\tabularnewline

\ascline{1.5pt}{666666}{1-4}

\multicolumn{1}{>{\centering}m{\dimexpr 1.5in+0\tabcolsep}}{\textcolor[HTML]{000000}{\fontsize{11}{11}\selectfont{\global\setmainfont{Arial}{\textbf{Sexo}}}}} & \multicolumn{1}{>{\centering}m{\dimexpr 1.5in+0\tabcolsep}}{\textcolor[HTML]{000000}{\fontsize{11}{11}\selectfont{\global\setmainfont{Arial}{\textbf{Nenhum}}}}} & \multicolumn{1}{>{\centering}m{\dimexpr 1.5in+0\tabcolsep}}{\textcolor[HTML]{000000}{\fontsize{11}{11}\selectfont{\global\setmainfont{Arial}{\textbf{Três\ latas*}}}}} & \multicolumn{1}{>{\centering}m{\dimexpr 1.5in+0\tabcolsep}}{\textcolor[HTML]{000000}{\fontsize{11}{11}\selectfont{\global\setmainfont{Arial}{\textbf{Seis\ latas*}}}}} \\

\ascline{1.5pt}{666666}{1-4}\endfirsthead 

\ascline{1.5pt}{666666}{1-4}

\multicolumn{1}{>{\centering}m{\dimexpr 1.5in+0\tabcolsep}}{\textcolor[HTML]{000000}{\fontsize{11}{11}\selectfont{\global\setmainfont{Arial}{\textbf{Sexo}}}}} & \multicolumn{1}{>{\centering}m{\dimexpr 1.5in+0\tabcolsep}}{\textcolor[HTML]{000000}{\fontsize{11}{11}\selectfont{\global\setmainfont{Arial}{\textbf{Nenhum}}}}} & \multicolumn{1}{>{\centering}m{\dimexpr 1.5in+0\tabcolsep}}{\textcolor[HTML]{000000}{\fontsize{11}{11}\selectfont{\global\setmainfont{Arial}{\textbf{Três\ latas*}}}}} & \multicolumn{1}{>{\centering}m{\dimexpr 1.5in+0\tabcolsep}}{\textcolor[HTML]{000000}{\fontsize{11}{11}\selectfont{\global\setmainfont{Arial}{\textbf{Seis\ latas*}}}}} \\

\ascline{1.5pt}{666666}{1-4}\endhead



\multicolumn{1}{>{\raggedright}m{\dimexpr 1.5in+0\tabcolsep}}{\textcolor[HTML]{000000}{\fontsize{11}{11}\selectfont{\global\setmainfont{Arial}{Feminino}}}} & \multicolumn{1}{>{\centering}m{\dimexpr 1.5in+0\tabcolsep}}{\textcolor[HTML]{000000}{\fontsize{11}{11}\selectfont{\global\setmainfont{Arial}{8}}}} & \multicolumn{1}{>{\centering}m{\dimexpr 1.5in+0\tabcolsep}}{\textcolor[HTML]{000000}{\fontsize{11}{11}\selectfont{\global\setmainfont{Arial}{8}}}} & \multicolumn{1}{>{\centering}m{\dimexpr 1.5in+0\tabcolsep}}{\textcolor[HTML]{000000}{\fontsize{11}{11}\selectfont{\global\setmainfont{Arial}{8}}}} \\





\multicolumn{1}{>{\raggedright}m{\dimexpr 1.5in+0\tabcolsep}}{\textcolor[HTML]{000000}{\fontsize{11}{11}\selectfont{\global\setmainfont{Arial}{Masculino}}}} & \multicolumn{1}{>{\centering}m{\dimexpr 1.5in+0\tabcolsep}}{\textcolor[HTML]{000000}{\fontsize{11}{11}\selectfont{\global\setmainfont{Arial}{8}}}} & \multicolumn{1}{>{\centering}m{\dimexpr 1.5in+0\tabcolsep}}{\textcolor[HTML]{000000}{\fontsize{11}{11}\selectfont{\global\setmainfont{Arial}{8}}}} & \multicolumn{1}{>{\centering}m{\dimexpr 1.5in+0\tabcolsep}}{\textcolor[HTML]{000000}{\fontsize{11}{11}\selectfont{\global\setmainfont{Arial}{8}}}} \\

\ascline{1.5pt}{666666}{1-4}



\multicolumn{4}{>{\raggedright}m{\dimexpr 6in+6\tabcolsep}}{\textcolor[HTML]{000000}{\fontsize{11}{11}\selectfont{\global\setmainfont{Arial}{*\ Tamanho\ médio\ da\ lata\ =\ 350\ ml\ ou\ 12,5\ g\ de\ álcool}}}} \\




\end{longtable}

\arrayrulecolor[HTML]{000000}

\global\setlength{\arrayrulewidth}{\Oldarrayrulewidth}

\global\setlength{\tabcolsep}{\Oldtabcolsep}

\renewcommand*{\arraystretch}{1}

\subsubsection{Visualização gráfica dos
dados}\label{visualizauxe7uxe3o-gruxe1fica-dos-dados-1}

Para visualizar os dados, será construido um gráfico com boxplots
(Figura~\ref{fig-alcool}), usando o pacote \texttt{ggpubr} (115), com a
função \texttt{ggboxplot()}, que fornece algumas funções fáceis de usar
para criar e personalizar gráficos prontos para publicação baseados em
`ggplot2'. O boxplot irá plotar os dados agrupados pelas combinações dos
níveis dos dois fatores.

\begin{Shaded}
\begin{Highlighting}[]
\NormalTok{ggpubr}\SpecialCharTok{::}\FunctionTok{ggboxplot}\NormalTok{ (}\AttributeTok{data =}\NormalTok{ dados,}
                   \AttributeTok{bxp.errorbar =} \ConstantTok{TRUE}\NormalTok{,}
                   \AttributeTok{bxp.errorbar.width =} \FloatTok{0.2}\NormalTok{,}
                   \AttributeTok{x =} \StringTok{"alcool"}\NormalTok{, }
                   \AttributeTok{y =} \StringTok{"escore"}\NormalTok{, }
                   \AttributeTok{color =} \StringTok{"black"}\NormalTok{,}
                   \AttributeTok{fill =} \StringTok{"sexo"}\NormalTok{,}
                   \AttributeTok{palette =} \StringTok{"bmj"}\NormalTok{,}
                   \AttributeTok{ylab =} \StringTok{"Escore da Memória"}\NormalTok{,}
                   \AttributeTok{xlab =} \StringTok{""}\NormalTok{,}
                   \AttributeTok{legend.title =} \StringTok{""}\NormalTok{,}
                   \AttributeTok{legend =} \StringTok{"top"}\NormalTok{) }\SpecialCharTok{+}
  \FunctionTok{theme}\NormalTok{ (}\AttributeTok{text =} \FunctionTok{element\_text}\NormalTok{ (}\AttributeTok{size =} \DecValTok{12}\NormalTok{))}
\end{Highlighting}
\end{Shaded}

\begin{figure}[H]

\centering{

\includegraphics[width=0.85\linewidth,height=0.85\textheight]{15-anova_files/figure-pdf/fig-alcool-1.pdf}

}

\caption{\label{fig-alcool}Efeito do álcool na memória de acordo com o
sexo.}

\end{figure}%

Além dos boxplot, é interessante desenhar um gráfico de linhas
(Figura~\ref{fig-alcool2})) que plota a média (ou outro resumo) da
variável \texttt{escore} (resposta) para combinações bidirecionais de
fatores, ilustrando assim possíveis interações. Aqui, pode-se usar a
função \texttt{ggline()}, também pertencente ao interessante pacote
\texttt{ggpubr}.

\begin{Shaded}
\begin{Highlighting}[]
\NormalTok{ggpubr}\SpecialCharTok{::}\FunctionTok{ggline}\NormalTok{(}\AttributeTok{data =}\NormalTok{ dados, }
               \AttributeTok{x =} \StringTok{"alcool"}\NormalTok{, }
               \AttributeTok{y =} \StringTok{"escore"}\NormalTok{, }
               \AttributeTok{color =} \StringTok{"sexo"}\NormalTok{,}
               \AttributeTok{linewidth =}  \FloatTok{0.9}\NormalTok{,}
               \AttributeTok{linetype =} \StringTok{"dashed"}\NormalTok{,}
               \AttributeTok{legend.title =} \StringTok{""}\NormalTok{,}
               \AttributeTok{position =} \FunctionTok{position\_dodge}\NormalTok{(}\AttributeTok{width =} \FloatTok{0.2}\NormalTok{),}
               \AttributeTok{add =} \StringTok{"mean\_ci"}\NormalTok{,}
               \AttributeTok{palette =} \StringTok{"bmj"}\NormalTok{)}
\end{Highlighting}
\end{Shaded}

\begin{figure}[H]

\centering{

\includegraphics[width=0.85\linewidth,height=0.85\textheight]{15-anova_files/figure-pdf/fig-alcool2-1.pdf}

}

\caption{\label{fig-alcool2}Efeito do álcool na memória de acordo com o
sexo.}

\end{figure}%

O gráfico sugere um possível efeito do álcool sobre a memória, bem como
uma interação entre os sexos.

\subsection{Hipóteses estatísticas}\label{hipuxf3teses-estatuxedsticas}

Como mencionado acima, uma ANOVA de duas vias é usada para avaliar
simultaneamente o efeito de duas variáveis categóricas em uma variável
quantitativa contínua. Ela é chamada de ANOVA de duas vias porque
compara grupos formados por duas variáveis categóricas independentes.

No exemplo, o objetivo é saber se a memória depende da álcool e/ou do
sexo. Em particular, estamos interessados em:

\begin{enumerate}
\def\labelenumi{\arabic{enumi}.}
\tightlist
\item
  medir e testar a relação entre a alcool e a memória,
\item
  medir e testar a relação entre sexo e memória, e
\item
  possivelmente verificar se a relação entre álcool e memória é
  diferente para mulheres e homens (o que é equivalente a verificar se a
  relação entre sexo e memória depende da álcool)
\end{enumerate}

As duas primeiras relações são chamadas de \emph{efeitos principais},
enquanto o item 3 é conhecido como \emph{efeito de interação}.

Os efeitos principais testam se pelo menos um grupo é diferente de outro
(durante o controle da outra variável independente). Por outro lado, o
efeito de interação tem como objetivo testar se a relação entre duas
variáveis difere dependendo do nível de uma terceira variável. Em outras
palavras, se a variação entre a resposta e a primeira variável
categórica não depender das modalidades da segunda variável categórica,
então não há interação entre as duas variáveis. Se, ao contrário, houver
uma modificação dessa variação, seja por um aumento no efeito da
primeira variável, seja por uma diminuição, então há uma interação.

Voltando ao exemplo, tem-se os seguintes testes de hipótese:

\ul{Efeito principal do sexo no escore de memória:}

\(H_{0}\): o escore de memória médio é igual entre mulheres e homens.

\(H_{1}\): o escore de memória médio é diferente entre mulheres e
homens.

\ul{Efeito principal do álcool no escore de memória:}

\(H_{0}\): o escore de memória médio é igual entre as categorias de
ingesta de álcool.\\
\(H_{1}\): o escore de memória médio é diferente entre as categorias de
ingesta de. álcool

\ul{Interação entre sexo e álcool:}

\(H_{0}\): não há interação entre sexo e álcool, o que significa que a
relação entre álcool e memória é a mesma para mulheres e homens (da
mesma forma, a relação entre sexo e memória é a mesma para todas as três
categorias de ingesta de álcool).\\
\(H_{1}\): há interação entre sexo e álcool, o que significa que a
relação entre álcool e memória é diferente para mulheres e homens (da
mesma forma, a relação entre sexo e memória depende da ingesta de
álcool).

\subsection{Pressupostos do modelo}\label{pressupostos-do-modelo}

Para usar uma ANOVA de duas vias, os dados devem atender a certos
pressupostos. A ANOVA de duas vias faz todas as suposições usuais de um
teste paramétrico de diferença:

\begin{enumerate}
\def\labelenumi{\arabic{enumi}.}
\tightlist
\item
  \ul{Independência de observações}
\end{enumerate}

As variáveis respostas não devem ser dependentes umas das outras (ou
seja, uma não deve causar a outra). Isso é impossível de testar com
variáveis categóricas - só pode ser garantido por um bom projeto
experimental.

Além disso, a variável dependente deve representar observações únicas -
não devem ser agrupadas em locais ou indivíduos. Se esta premissa for
violada, você pode incluir uma variável de bloqueio e/ou usar uma ANOVA
de medidas repetidas.

\begin{enumerate}
\def\labelenumi{\arabic{enumi}.}
\setcounter{enumi}{1}
\tightlist
\item
  \ul{Normalidade}
\end{enumerate}

Variável desfecho normalmente distribuída em todos os grupos.

\begin{enumerate}
\def\labelenumi{\arabic{enumi}.}
\setcounter{enumi}{2}
\tightlist
\item
  \ul{Ausência de valores atípicos (\emph{outliers})}
\end{enumerate}

Um valor aberrante ou valor atípico, é uma observação que apresenta um
grande afastamento das demais da série, \(\pm 1,5\) o intervalo
interquartil (IIQ) e extremo se estiver \(\pm 3\) IIQ. A existência de
outliers implica, tipicamente, em prejuízos à interpretação dos
resultados.

\begin{enumerate}
\def\labelenumi{\arabic{enumi}.}
\setcounter{enumi}{3}
\tightlist
\item
  \ul{Homogeneidade de variância (homocedasticidade)}
\end{enumerate}

A variação em torno da média para cada grupo sendo comparado deve ser
semelhante entre todos os grupos. Se os dados não atenderem a essa
suposição, é possível usar uma alternativa não paramétrica, como o teste
de Kruskal-Wallis (\textbf{?@sec-kruskalwallis}).

\subsubsection{Verificação dos pressupostos nos dados
brutos}\label{verificauxe7uxe3o-dos-pressupostos-nos-dados-brutos}

Existe uma discussão se os pressupostos devem ser avaliados nos dados
brutos ou apenas nos resíduos. Aqui serão realizadas as duas abordagens
que frequentemente resultam no mesmo resultado.

\paragraph{Normalidade}\label{normalidade}

A variável dependente (\texttt{escore}) deve apresentar distribuição
aproximadamente normal dentro de cada grupo. Os grupos aqui serão
formados pela combinação das duas variáveis independentes (\texttt{sexo}
e \texttt{alcool}). A normalidade será avaliada pelo
\texttt{teste\ de\ Shapiro-Wilk}, com a função \texttt{shapiro\_test()}
do pacote \texttt{rstatix} (112), separando os grupos com a função
\texttt{group\_by()} do pacote \texttt{dplyr}, encadeadas com o operador
\texttt{pipe} (\texttt{\%\textgreater{}\%}):

\begin{Shaded}
\begin{Highlighting}[]
\NormalTok{dados }\SpecialCharTok{\%\textgreater{}\%} 
\NormalTok{     dplyr}\SpecialCharTok{::}\FunctionTok{group\_by}\NormalTok{ (sexo, alcool) }\SpecialCharTok{\%\textgreater{}\%} 
\NormalTok{     rstatix}\SpecialCharTok{::}\FunctionTok{shapiro\_test}\NormalTok{ (escore)}
\end{Highlighting}
\end{Shaded}

\begin{verbatim}
# A tibble: 6 x 5
  sexo      alcool  variable statistic     p
  <fct>     <fct>   <chr>        <dbl> <dbl>
1 Feminino  nenhum  escore       0.872 0.156
2 Feminino  3 latas escore       0.899 0.283
3 Feminino  6 latas escore       0.897 0.273
4 Masculino nenhum  escore       0.941 0.622
5 Masculino 3 latas escore       0.967 0.870
6 Masculino 6 latas escore       0.951 0.720
\end{verbatim}

Os resultados suportam a conclusão de não rejeição da hipótese nula de
que os dados se ajustam a distribuição normal.

\paragraph{Pesquisa de valores
atípicos}\label{pesquisa-de-valores-atuxedpicos-1}

A forma mais simples de verificar a presença de um valor atípico é
observar o boxplot, mostrado anteriormente. Se observa a presença de
valores atípicos entre as mulheres que não ingeriram álcool e nas que
ingeriram 3 latas de cerveja. Agora, para confirmar esse achado, será
usado a função \texttt{identify\_outliers\ ()}, do pacote
\texttt{rstatix}:

\begin{Shaded}
\begin{Highlighting}[]
\NormalTok{dados }\SpecialCharTok{\%\textgreater{}\%} 
\NormalTok{      dplyr}\SpecialCharTok{::}\FunctionTok{group\_by}\NormalTok{ (sexo, alcool) }\SpecialCharTok{\%\textgreater{}\%} 
\NormalTok{      rstatix}\SpecialCharTok{::}\FunctionTok{identify\_outliers}\NormalTok{(escore)}
\end{Highlighting}
\end{Shaded}

\begin{verbatim}
# A tibble: 2 x 5
  sexo     alcool  escore is.outlier is.extreme
  <fct>    <fct>    <dbl> <lgl>      <lgl>     
1 Feminino nenhum      70 TRUE       TRUE      
2 Feminino 3 latas     50 TRUE       FALSE     
\end{verbatim}

A saída do teste confirma a existência dos dois valores atípicos, sendo
um deles extremo, entretanto como estes valores são possíveis e,
relativamente, próximos da média do sexo feminino, portanto, causam
pouca preocupação, principalmente porque o teste de ANOVA é bastante
robusto.

\paragraph{Verificação da homogeneidade das
variâncias}\label{verificauxe7uxe3o-da-homogeneidade-das-variuxe2ncias}

Para verificar a homocedasticidade, como os dados têm distribuição
normal, é possível usar o teste de Levene, o \texttt{leveneTest()} do
pacote \texttt{car} (111).

\begin{Shaded}
\begin{Highlighting}[]
\NormalTok{car}\SpecialCharTok{::}\FunctionTok{leveneTest}\NormalTok{ (escore }\SpecialCharTok{\textasciitilde{}}\NormalTok{ sexo}\SpecialCharTok{*}\NormalTok{alcool, }
                 \AttributeTok{data =}\NormalTok{ dados, }
                 \AttributeTok{center =}\NormalTok{ mean)}
\end{Highlighting}
\end{Shaded}

\begin{verbatim}
Levene's Test for Homogeneity of Variance (center = mean)
      Df F value Pr(>F)
group  5  1.5268 0.2021
      42               
\end{verbatim}

\subsubsection{Verificação dos pressupostos nos
resíduos}\label{sec-preresiduos}

O modelo da ANOVA pode ser considerado como um modelo de regressão.
Desta forma, este modelo de regressão vai usar os dados brutos para
criar um modelo de previsão para esses dados. Este modelo de regressão
não é perfeito, existe uma diferença entre os valores previstos e os
valores observados, são os resíduos. Faz sentido, então, preocupar-se
com os resíduos quando se analisa fatores tentando explicar uma variável
dependente contínua, como na ANOVA, pensando em uma regressão linear
simples.

A ANOVA prevê que todos os valores do grupo sejam iguais a média do
grupo. Ou seja, um homem que ingere 3 latas de cerveja tem um valor de
seu escore de memória igual ao deste grupo. Por este motivo, fazer a
análise dos resíduos é praticamente o mesmo que a análise dos valores
brutos.

Para analisar os resíduos (diferença entre os valores observados e o
previsto pelo modelo), em primeiro lugar se constrói o modelo da ANOVA
com efeito da interação, usando a função \texttt{lm()} do pacote
\texttt{stats}, incluído no R base:

\begin{Shaded}
\begin{Highlighting}[]
\NormalTok{mod.int.lm }\OtherTok{\textless{}{-}} \FunctionTok{lm}\NormalTok{(}\AttributeTok{formula =}\NormalTok{ escore }\SpecialCharTok{\textasciitilde{}}\NormalTok{ alcool }\SpecialCharTok{*}\NormalTok{ sexo,}
              \AttributeTok{data =}\NormalTok{ dados)}
\end{Highlighting}
\end{Shaded}

Ao se executar o comando, tem-se a impressão que nada ocorreu,
entretanto foi criado o \emph{modelo da ANOVA} com uma série de
variáveis:

\begin{itemize}
\tightlist
\item
  Resíduos
\item
  Coeficientes estimados (intercepto e inclinações)
\item
  Erros padrão
\item
  Estatísticas t e valor P
\item
  R2 e R2 ajustado
\item
  Estatísticas F
\end{itemize}

Os resíduos (\texttt{resdiduals}) geralmente têm média zero ou muito
próxima de zero. Isto acontece porque quando o modelo linear é ajustado
com mínimos quadrados ordinários (OLS), ele encontra os coeficientes que
minimizam a soma dos quadrados dos resíduos. Uma das propriedades
fundamentais dessa técnica é que a soma dos resíduos é zero., o que
implica que a média dos resíduos também é zero. Matemáticamente:

\[
\sum_{i=1}^{n} e_i = 0 \quad \Rightarrow \quad \bar{e} = 0
\]

\begin{Shaded}
\begin{Highlighting}[]
\FunctionTok{sum}\NormalTok{(mod.int.lm}\SpecialCharTok{$}\NormalTok{residuals)}
\end{Highlighting}
\end{Shaded}

\begin{verbatim}
[1] 1.976197e-14
\end{verbatim}

\begin{Shaded}
\begin{Highlighting}[]
\FunctionTok{mean}\NormalTok{(mod.int.lm}\SpecialCharTok{$}\NormalTok{residuals)}
\end{Highlighting}
\end{Shaded}

\begin{verbatim}
[1] 4.116445e-16
\end{verbatim}

Se obtém resultados muito próximo de zero. A função \texttt{summary()}
fornece um resumo estatístico dos resíduos:

\begin{Shaded}
\begin{Highlighting}[]
\FunctionTok{summary}\NormalTok{(mod.int.lm}\SpecialCharTok{$}\NormalTok{residuals)}
\end{Highlighting}
\end{Shaded}

\begin{verbatim}
   Min. 1st Qu.  Median    Mean 3rd Qu.    Max. 
-21.875  -5.625  -0.625   0.000   5.156  19.375 
\end{verbatim}

\paragraph{Avaliação da normalidade dos
resíduos}\label{avaliauxe7uxe3o-da-normalidade-dos-resuxedduos}

Uma das suposições de uma ANOVA é que os resíduos são normalmente
distribuídos. A normalidade dos resíduos, inicialmente, será verificada,
usando o teste de Shapiro-Wilk com a função \texttt{shapiro.test()},
também pertencente ao pacote \texttt{stats}.

\begin{Shaded}
\begin{Highlighting}[]
\FunctionTok{shapiro\_test}\NormalTok{ (mod.int.lm}\SpecialCharTok{$}\NormalTok{residuals)}
\end{Highlighting}
\end{Shaded}

\begin{verbatim}
# A tibble: 1 x 3
  variable             statistic p.value
  <chr>                    <dbl>   <dbl>
1 mod.int.lm$residuals     0.982   0.664
\end{verbatim}

O teste entrega um valor \emph{p} \textgreater{} 0.05, indicando que não
é possível rejeitar \(H_{0}\) de normalidade dos resíduos.

Uma outra maneira comum de verificar essa suposição é criando um
\emph{gráfico Q-Q}. Se os resíduos forem normalmente distribuídos, os
pontos em um gráfico Q-Q ficarão em uma linha diagonal reta. Este
gráfico (Figura~\ref{fig-normresiduos}) pode ser contruído com a função
\texttt{ggqqplot()} do pacote \texttt{ggpubr}.

\begin{Shaded}
\begin{Highlighting}[]
\NormalTok{ggpubr}\SpecialCharTok{::}\FunctionTok{ggqqplot}\NormalTok{(mod.int.lm}\SpecialCharTok{$}\NormalTok{residuals,}
                 \AttributeTok{conf.int =} \ConstantTok{TRUE}\NormalTok{,}
                 \AttributeTok{shape =} \DecValTok{19}\NormalTok{,}
                 \AttributeTok{xlab =} \StringTok{"Quantis teóricos"}\NormalTok{,}
                 \AttributeTok{ylab =} \StringTok{"Resíduos"}\NormalTok{,}
                 \AttributeTok{color =} \StringTok{"dodgerblue4"}\NormalTok{)}
\end{Highlighting}
\end{Shaded}

\begin{figure}[H]

\centering{

\includegraphics[width=0.7\linewidth,height=0.7\textheight]{15-anova_files/figure-pdf/fig-normresiduos-1.pdf}

}

\caption{\label{fig-normresiduos}Normalidade dos resíduos - QQ plot.}

\end{figure}%

O gráfico QQ de normalidade, mostra que os resíduos seguem
aproximadamente uma linha reta, permitindo assumir a normalidade dos
mesmos.

\paragraph{Pesquisa de valores atípicos nos
resíduos}\label{pesquisa-de-valores-atuxedpicos-nos-resuxedduos}

Para a verificação da presença de valores atípicos entre os resíduos,
cria-se uma variável que será denominada de \texttt{residuos} (observe o
banco de dados com a função \texttt{str()} para ver o acréscimo dessa
variável):

\begin{Shaded}
\begin{Highlighting}[]
\NormalTok{dados}\SpecialCharTok{$}\NormalTok{residuos }\OtherTok{\textless{}{-}}\NormalTok{ mod.int.lm}\SpecialCharTok{$}\NormalTok{residuals}
\FunctionTok{str}\NormalTok{ (dados)}
\end{Highlighting}
\end{Shaded}

\begin{verbatim}
tibble [48 x 4] (S3: tbl_df/tbl/data.frame)
 $ sexo    : Factor w/ 2 levels "Feminino","Masculino": 1 1 1 1 1 1 1 1 1 1 ...
 $ alcool  : Factor w/ 3 levels "nenhum","3 latas",..: 1 1 1 1 1 1 1 1 2 2 ...
 $ escore  : num [1:48] 65 70 60 60 60 55 60 55 70 65 ...
 $ residuos: Named num [1:48] 4.375 9.375 -0.625 -0.625 -0.625 ...
  ..- attr(*, "names")= chr [1:48] "1" "2" "3" "4" ...
\end{verbatim}

Para identificar os \emph{outliers}, usa-se função
\texttt{identify\_outliers()} do pacote \texttt{rstatix}:

\begin{Shaded}
\begin{Highlighting}[]
\NormalTok{dados }\SpecialCharTok{\%\textgreater{}\%} 
\NormalTok{  dplyr}\SpecialCharTok{::}\FunctionTok{group\_by}\NormalTok{(sexo, alcool) }\SpecialCharTok{\%\textgreater{}\%} 
\NormalTok{  rstatix}\SpecialCharTok{::}\FunctionTok{identify\_outliers}\NormalTok{(residuos)}
\end{Highlighting}
\end{Shaded}

\begin{verbatim}
# A tibble: 2 x 6
  sexo     alcool  escore residuos is.outlier is.extreme
  <fct>    <fct>    <dbl>    <dbl> <lgl>      <lgl>     
1 Feminino nenhum      70     9.38 TRUE       TRUE      
2 Feminino 3 latas     50   -12.5  TRUE       FALSE     
\end{verbatim}

Observando os resultados com os dados brutos, verifica-se que eles são
iguais aos atuais, confirmando, que neste caso, tanto faz avaliar os
dados brutos como os resíduos.

\paragraph{Verificação da homogeneidade da variância nos
resíduos}\label{verificauxe7uxe3o-da-homogeneidade-da-variuxe2ncia-nos-resuxedduos}

A verificação da homogeneidade da variância entre os resíduos pode ser
feita com o \texttt{teste\ de\ Levene}, como feito com os dados brutos.

\begin{Shaded}
\begin{Highlighting}[]
\NormalTok{car}\SpecialCharTok{::}\FunctionTok{leveneTest}\NormalTok{ (residuos }\SpecialCharTok{\textasciitilde{}}\NormalTok{ sexo}\SpecialCharTok{*}\NormalTok{alcool, }
                 \AttributeTok{data =}\NormalTok{ dados, }
                 \AttributeTok{center =}\NormalTok{ mean)}
\end{Highlighting}
\end{Shaded}

\begin{verbatim}
Levene's Test for Homogeneity of Variance (center = mean)
      Df F value Pr(>F)
group  5  1.5268 0.2021
      42               
\end{verbatim}

Uma outra maneira de avaliar a homogeneidade da variância, é construir
um gráfico diagnóstico\footnote{Outros gráficos diagnósticos podem ser
  obtidos para analisar resíduos em um modelo de regressão (124)}
(Figura~\ref{fig-residuals})) do modelo com a função \texttt{plot()},
tipo 1, resíduos versus ajustes (\emph{Residuals vs Fitted}).

\begin{Shaded}
\begin{Highlighting}[]
\FunctionTok{plot}\NormalTok{(mod.int.lm, }\DecValTok{1}\NormalTok{)}
\end{Highlighting}
\end{Shaded}

\begin{figure}[H]

\centering{

\includegraphics[width=0.7\linewidth,height=0.7\textheight]{15-anova_files/figure-pdf/fig-residuals-1.pdf}

}

\caption{\label{fig-residuals}Resíduos versus ajuste}

\end{figure}%

Não há correlações óbvias entre resíduos e valores ajustados (a média de
cada grupo) no gráfico abaixo,onde a linha vermelha tracejada
(Figura~\ref{fig-residuals}) segue praticamente uma linha horizontal em
torno de zero, o que é bom.

Isso significa que a suposição de homocedasticidade (variância constante
dos resíduos) está satisfeita, e é possível confiar nos testes de
significância dos coeficientes do modelo \texttt{lm}.

\begin{tcolorbox}[enhanced jigsaw, bottomrule=.15mm, opacitybacktitle=0.6, colframe=quarto-callout-important-color-frame, arc=.35mm, coltitle=black, toptitle=1mm, colback=white, colbacktitle=quarto-callout-important-color!10!white, breakable, bottomtitle=1mm, rightrule=.15mm, titlerule=0mm, toprule=.15mm, opacityback=0, leftrule=.75mm, left=2mm, title=\textcolor{quarto-callout-important-color}{\faExclamation}\hspace{0.5em}{Dados brutos vs.~resíduos}]

Não existe um consenso. No exemplo, foram verificados os pressupostos
das duas maneiras e os resultados não diferem.

\end{tcolorbox}

\subsection{Execução do teste de ANOVA de dois
fatores}\label{execuuxe7uxe3o-do-teste-de-anova-de-dois-fatores}

Como os pressupostos podem ser aceitos, pode-se prossguir com a
implementação da ANOVA de duas vias.

A inclusão de um efeito de interação em uma ANOVA de duas vias não é
obrigatória. Entretanto, para evitar conclusões errôneas, recomenda-se
verificar primeiro se a interação é significativa ou não e, dependendo
dos resultados, incluí-la ou não.

Se a interação não for significativa, é seguro removê-la do modelo
final. Por outro lado, se a interação for significativa, ela deverá ser
incluída no modelo final que será usado para interpretar os resultados.

Portanto, deve-se começar com um modelo que inclui os dois efeitos
principais (ou seja, \texttt{sexo} e \texttt{alcool}) e a interação:

\begin{Shaded}
\begin{Highlighting}[]
\NormalTok{mod.aov }\OtherTok{\textless{}{-}} \FunctionTok{aov}\NormalTok{(}\AttributeTok{formula =}\NormalTok{ escore }\SpecialCharTok{\textasciitilde{}}\NormalTok{ alcool }\SpecialCharTok{*}\NormalTok{ sexo,}
               \AttributeTok{data =}\NormalTok{ dados)}
\FunctionTok{summary}\NormalTok{ (mod.aov)}
\end{Highlighting}
\end{Shaded}

\begin{verbatim}
            Df Sum Sq Mean Sq F value   Pr(>F)    
alcool       2   3332  1666.1  20.065 7.65e-07 ***
sexo         1    169   168.7   2.032    0.161    
alcool:sexo  2   1978   989.1  11.911 7.99e-05 ***
Residuals   42   3488    83.0                     
---
Signif. codes:  0 '***' 0.001 '**' 0.01 '*' 0.05 '.' 0.1 ' ' 1
\end{verbatim}

Semelhante a uma ANOVA de uma via, o princípio de uma ANOVA de duas vias
baseia-se na dispersão total dos dados e em sua decomposição em quatro
componentes:

\begin{enumerate}
\def\labelenumi{\arabic{enumi}.}
\tightlist
\item
  a parcela atribuível ao primeiro fator
\item
  a parcela atribuível ao segundo fator
\item
  a parcela atribuível à interação dos dois fatores
\item
  a parte não explicada ou residual.
\end{enumerate}

A \textbf{soma dos quadrados} (coluna \texttt{Sum\ Sq}) mostra esses
quatro componentes. A ANOVA de duas vias consiste em usar um teste
estatístico para determinar se cada componente de dispersão (atribuível
aos dois fatores estudados e à interação deles) é significativamente
maior do que o componente residual. Se esse for o caso, concluí-se que o
efeito considerado (\texttt{alcool}, \texttt{sexo} ou a interação) é
significativo.

Vê-se que a variável \texttt{alcool} explica uma grande parte da
variabilidade da memória. Ela é o fator mais importante para explicar
essa variabilidade. Os valore \emph{p} são exibidos na última coluna do
resultado acima (\texttt{Pr(\textgreater{}F)}). A partir desses valores,
conclui-se que, no nível de significância de 5\%:

\begin{itemize}
\tightlist
\item
  controlando para o \texttt{alcool}, o escore de memória não é
  significativamente diferente entre os dois sexos (\emph{p} = 0,161),
\item
  controlando para o sexo, o escore memória é significativamente
  diferente (\emph{p} \textless{} 0,0001) para pelo menos uma categoria
  de ingesta de álcool, e
\item
  a interação entre sexo e álcool (exibida na linha \texttt{alcool:sexo}
  no resultado) é significativa (\emph{p} \textless{} 0,0001). Isso quer
  dizer que o efeito do álcool depende do sexo --- ou seja, homens e
  mulheres reagem de forma diferente ao consumo de álcool em termos de
  memória.
\end{itemize}

Portanto, com base no efeito de interação significativo, se observa que
relação entre os escores de memória e álcool é dependente do sexo. Como
ela é significativa, deve-se mantê-la no modelo e interpretar os
resultados desse modelo.

Se, ao contrário, a interação não fosse significativa (ou seja, se o
valor \emph{p} \textgreater{} 0,05), esse efeito de interação do modelo
seria removido. Abaixo, segue o código de uma ANOVA de dois fatores sem
interação, chamada de \textbf{modelo aditivo}:

\begin{Shaded}
\begin{Highlighting}[]
\NormalTok{mod.aov2 }\OtherTok{\textless{}{-}} \FunctionTok{aov}\NormalTok{(}\AttributeTok{formula =}\NormalTok{ escore }\SpecialCharTok{\textasciitilde{}}\NormalTok{ alcool }\SpecialCharTok{+}\NormalTok{ sexo, }
                 \AttributeTok{data =}\NormalTok{ dados)}
\end{Highlighting}
\end{Shaded}

Na Seção~\ref{sec-preresiduos}, foi construído um modelo para analisar
os resíduos. Este modelo está baseado na semelhança dos modelos de
regressão linear (veja \textbf{?@sec-rls}) com o modelo da ANOVA.
Observa-se que o código é bem semelhante, usando a fórmula
\texttt{variável\ dependente\ \textasciitilde{}\ variável\ independentes},
o sinal \texttt{+} é usado para incluir variáveis independentes sem
interação e o sinal \texttt{*} quando há interação. Ou seja, a ANOVA,
como todas as ANOVAs, é na verdade um modelo linear. Observe que o
código a seguir, usando a função \texttt{lm()} e após \texttt{Anova()}
do pacote \texttt{car}, também funciona e retorna os mesmos resultados
\footnote{A função \texttt{Anova()} do pacote \texttt{car}, usada para
  testar efeitos principais e de interação em modelos lineares gerais,
  não deve ser confundida com a função \texttt{anova()}, da base do R,
  porque esta fornece resultados sequenciais que dependem da ordem em
  que as variáveis aparecem no modelo.} :

\begin{Shaded}
\begin{Highlighting}[]
\NormalTok{ mod.int.lm }\OtherTok{\textless{}{-}} \FunctionTok{lm}\NormalTok{(}\AttributeTok{formula =}\NormalTok{ escore }\SpecialCharTok{\textasciitilde{}}\NormalTok{ sexo }\SpecialCharTok{*}\NormalTok{ alcool,}
                   \AttributeTok{data =}\NormalTok{ dados)}
 \FunctionTok{Anova}\NormalTok{(mod.int.lm)}
\end{Highlighting}
\end{Shaded}

\begin{verbatim}
Anova Table (Type II tests)

Response: escore
            Sum Sq Df F value    Pr(>F)    
sexo         168.7  1  2.0323    0.1614    
alcool      3332.3  2 20.0654 7.649e-07 ***
sexo:alcool 1978.1  2 11.9113 7.987e-05 ***
Residuals   3487.5 42                      
---
Signif. codes:  0 '***' 0.001 '**' 0.01 '*' 0.05 '.' 0.1 ' ' 1
\end{verbatim}

Atente para o fato que a função \texttt{aov()} pressupõe um projeto
balanceado, o que significa tamanhos de amostra iguais dentro dos níveis
das variáveis de agrupamento independentes. Para verificar se os dados
estão balanceados, proceda como mostrado na Tabela~\ref{tbl-tab2x3}. Os
resultados mostram o mesmo número de indivíduos em todas as células,
portanto, não importa qual o tipo de ANOVA a ser usado. Os resultados
serão iguais. Além disso, \texttt{aov()} usa as somas de quadrados do
tipo I.

Para delineamentos não balanceados, ou seja, números desiguais de
indivíduos em cada subgrupo, os métodos recomendados são:

\begin{itemize}
\tightlist
\item
  a ANOVA do tipo II, quando não há interação significativa, que pode
  ser feita no R com \texttt{Anova(mod,\ type\ =\ “II”)} ou
  \texttt{Anova(mod,\ type\ =\ 2)}, em que \texttt{mod} é o nome do seu
  modelo salvo, e
\item
  a ANOVA do tipo III, quando há uma interação significativa, que pode
  ser feita no R com \texttt{Anova(mod,\ type\ =\ “III”)} ou
  \texttt{Anova(mod,\ type\ =\ 3)}.
\end{itemize}

Fundamentalmente, a diferença entre um método e outro é como o \emph{R}
calcula a soma dos quadrados ao calcular a ANOVA. Quando os dados são
balanceados, os três tipos dão o mesmo resultado \footnote{Para os
  interessados, pode-se obter maiores informações sobre os diferentes
  tipos de ANOVA em
  https://www.r-bloggers.com/2011/03/anova-\%E2\%80\%93-type-iiiiii-ss-explained/}.

\subsection{Testes post hoc}\label{testes-post-hoc-1}

Neste estágio, chegou-se ao ponto em que se constatou que o efeito
principal do sexo não é significativo e que o efeito principal do álcool
é significativo. Além disso, mais importante, existe uma interação entre
o álcool e o sexo, o efeito do álcool depende do sexo.

Não é possível saber exatamente qual categoria da variável
\texttt{alcool} é diferente da outra em termos de escore de memória.
Para saber isso, há que comparar cada categoria duas a duas graças aos
testes \texttt{post-hoc}, também conhecidos como comparações entre pares
(125) (126) (127). Há vários testes post-hoc, sendo os mais comuns o
Tukey HSD, que testa todos os pares possíveis de grupos. Será utilizada
a função \texttt{tukey\_hsd\ ()}, do pacote \texttt{rstatix} , usando
como argumento o modelo com interação, \texttt{mod.aov}:

\begin{Shaded}
\begin{Highlighting}[]
\NormalTok{pwc }\OtherTok{\textless{}{-}}\NormalTok{ rstatix}\SpecialCharTok{::}\FunctionTok{tukey\_hsd}\NormalTok{ (mod.aov)}
\NormalTok{pwc}
\end{Highlighting}
\end{Shaded}

\begin{verbatim}
# A tibble: 19 x 9
   term        group1      group2 null.value estimate conf.low conf.high   p.adj
 * <chr>       <chr>       <chr>       <dbl>    <dbl>    <dbl>     <dbl>   <dbl>
 1 alcool      nenhum      3 lat~          0    0.938    -6.89      8.76 9.54e-1
 2 alcool      nenhum      6 lat~          0  -17.2     -25.0      -9.36 1.05e-5
 3 alcool      3 latas     6 lat~          0  -18.1     -26.0     -10.3  4.05e-6
 4 sexo        Feminino    Mascu~          0   -3.75     -9.06      1.56 1.61e-1
 5 alcool:sexo nenhum:Fem~ 3 lat~          0    1.88    -11.7      15.5  9.98e-1
 6 alcool:sexo nenhum:Fem~ 6 lat~          0   -3.12    -16.7      10.5  9.83e-1
 7 alcool:sexo nenhum:Fem~ nenhu~          0    6.25     -7.35     19.9  7.43e-1
 8 alcool:sexo nenhum:Fem~ 3 lat~          0    6.25     -7.35     19.9  7.43e-1
 9 alcool:sexo nenhum:Fem~ 6 lat~          0  -25       -38.6     -11.4  3.06e-5
10 alcool:sexo 3 latas:Fe~ 6 lat~          0   -5       -18.6       8.60 8.8 e-1
11 alcool:sexo 3 latas:Fe~ nenhu~          0    4.37     -9.23     18.0  9.28e-1
12 alcool:sexo 3 latas:Fe~ 3 lat~          0    4.37     -9.23     18.0  9.28e-1
13 alcool:sexo 3 latas:Fe~ 6 lat~          0  -26.9     -40.5     -13.3  7.96e-6
14 alcool:sexo 6 latas:Fe~ nenhu~          0    9.37     -4.23     23.0  3.29e-1
15 alcool:sexo 6 latas:Fe~ 3 lat~          0    9.37     -4.23     23.0  3.29e-1
16 alcool:sexo 6 latas:Fe~ 6 lat~          0  -21.9     -35.5      -8.27 2.78e-4
17 alcool:sexo nenhum:Mas~ 3 lat~          0    0       -13.6      13.6  1   e+0
18 alcool:sexo nenhum:Mas~ 6 lat~          0  -31.2     -44.9     -17.6  3.37e-7
19 alcool:sexo 3 latas:Ma~ 6 lat~          0  -31.2     -44.9     -17.6  3.37e-7
# i 1 more variable: p.adj.signif <chr>
\end{verbatim}

A tabela apresenta os resultados das comparações múltiplas para os
fatores \texttt{alcool} e \texttt{sexo} e, o mais importante, para a
\textbf{interação entre eles} (\texttt{alcool:sexo}). A coluna
\texttt{p.adj} (valor \emph{p} ajustado) é a chave para a interpretação.
Quando \texttt{p.adj} for menor que 0.05 (ou indicado por \texttt{*},
\texttt{**}, \texttt{***}, \texttt{****}), a diferença entre os grupos é
estatisticamente significativa.

\subsubsection{Interpretação por
Fator}\label{interpretauxe7uxe3o-por-fator}

\textbf{1. \ul{Efeito Principal de Álcool}}

\begin{itemize}
\item
  \textbf{Linha 2}: \texttt{alcool} - \texttt{nenhum} vs
  \texttt{6\ latas}

  \begin{itemize}
  \item
    \texttt{p.adj} = 1.05×10−5 (\texttt{****})
  \item
    \textbf{Resultado}: Há uma \textbf{diferença altamente
    significativa} no escore de memória entre o grupo que não consumiu
    álcool e o grupo que consumiu 6 latas de cerveja. O valor estimado
    de -17.2 indica que, em média, o grupo que bebeu 6 latas teve um
    escore de memória 17.2 pontos menor.
  \end{itemize}
\item
  \textbf{Linha 3}: \texttt{alcool} - \texttt{3\ latas} vs
  \texttt{6\ latas}

  \begin{itemize}
  \item
    \texttt{p.adj} = 4.05×10−6 (\texttt{****})
  \item
    \textbf{Resultado}: Há uma \textbf{diferença altamente
    significativa} no escore de memória entre o grupo que consumiu 3
    latas e o grupo que consumiu 6 latas. O valor estimado de -18.1
    indica que, em média, o grupo que bebeu 6 latas teve um escore de
    memória 18.1 pontos menor.
  \end{itemize}
\item
  \textbf{Linha 1}: \texttt{alcool} - \texttt{nenhum} vs
  \texttt{3\ latas}

  \begin{itemize}
  \item
    \texttt{p.adj} = 0.954 (\texttt{ns})
  \item
    \textbf{Resultado}: \textbf{Não há diferença significativa} no
    escore de memória entre o grupo que não consumiu álcool e o grupo
    que consumiu 3 latas.
  \end{itemize}
\end{itemize}

\textbf{Conclusão sobre o álcool}: O consumo de 6 latas de cerveja teve
um impacto negativo e altamente significativo no escore de memória em
comparação com os grupos que não beberam e os que beberam 3 latas.

\textbf{2. \ul{Efeito Principal de Sexo}}

\begin{itemize}
\item
  \textbf{Linha 4}: \texttt{sexo} - \texttt{Feminino} vs
  \texttt{Masculino}

  \begin{itemize}
  \item
    \texttt{p.adj} = 0.161 (\texttt{ns})
  \item
    \textbf{Resultado}: \textbf{Não há diferença significativa} no
    escore de memória entre homens e mulheres quando analisados como
    grupos únicos (sem considerar o consumo de álcool).
  \end{itemize}
\end{itemize}

\textbf{3. \ul{Efeito de Interação (Álcool x Sexo)}}

Esta é a parte mais complexa e crucial, pois explora se o efeito do
álcool é diferente para homens e mulheres.

\begin{itemize}
\item
  \textbf{Mulheres (\texttt{Feminino})}:

  \begin{itemize}
  \item
    \textbf{Linha 5}: \texttt{nenhum:Feminino} vs
    \texttt{3\ latas:Feminino}

    \begin{itemize}
    \item
      \texttt{p.adj} = 9.98×10−1 (\texttt{ns})
    \item
      \textbf{Resultado}: \textbf{Não há uma diferença altamente
      significativa} no escore de memória em mulheres que não beberam
      álcool e as que beberam 3 latas.
    \end{itemize}
  \item
    \textbf{Linha 6}: \texttt{nenhum:Feminino} vs
    \texttt{6\ latas:Feminino}

    \begin{itemize}
    \item
      \texttt{p.adj} = 9.83×10−1 (\texttt{ns})
    \item
      \textbf{Resultado}: \textbf{Não há uma diferença altamente
      significativa} no escore de memória em mulheres que não beberam
      álcool e as que beberam 6 latas.
    \end{itemize}
  \item
    \textbf{Linha 10}: \texttt{3\ latas:Feminino} vs
    \texttt{6\ latas:Feminino}

    \begin{itemize}
    \item
      \texttt{p.adj} = 8.8×10−1 (\texttt{ns})
    \item
      \textbf{Resultado}: \textbf{Não uma diferença significativa} em
      mulheres que beberam 3 latas e as que beberam 6 latas.
    \end{itemize}
  \end{itemize}
\item
  \textbf{Homens (\texttt{Masculino})}:

  \begin{itemize}
  \item
    \textbf{Linha 17}: \texttt{nenhum:Masculino} vs
    \texttt{3\ latas:Masculino}

    \begin{itemize}
    \item
      \texttt{p.adj} = 1 (\texttt{ns})
    \item
      \textbf{Resultado}: \textbf{Não há uma diferença altamente
      significativa} no escore de memória em homens que não beberam
      álcool e os que beberam 3 latas.
    \end{itemize}
  \item
    \textbf{Linha 18}: \texttt{nenhum:Masculino} vs
    \texttt{6\ latas:Masculino}

    \begin{itemize}
    \item
      \texttt{p.adj} = 3.37×10−7 (\texttt{****})
    \item
      \textbf{Resultado}: Há uma \textbf{diferença altamente
      significativa} no escore de memória em homens entre os que não
      beberam e os que beberam 6 latas.
    \end{itemize}
  \item
    \textbf{Linha 19}: \texttt{3\ latas:Masculino} vs
    \texttt{6\ latas:Masculino}

    \begin{itemize}
    \item
      \texttt{p.adj} = 3.37×10−7 (\texttt{****})
    \item
      \textbf{Resultado}: Há uma \textbf{diferença altamente
      significativa} em homens entre os que beberam 3 latas e os que
      beberam 6 latas.
    \end{itemize}
  \end{itemize}
\end{itemize}

\textbf{Conclusão sobre a interação}: O impacto negativo de 6 latas de
cerveja é significativo tanto para homens quanto para mulheres,
indicando que o consumo de álcool afeta a memória de forma semelhante em
ambos os sexos neste nível.

\begin{itemize}
\item
  \textbf{Comparações entre sexos dentro do mesmo nível de álcool}:

  \begin{itemize}
  \item
    \textbf{Linha 7}: \texttt{nenhum:Feminino} vs
    \texttt{nenhum:Masculino}

    \begin{itemize}
    \item
      \texttt{p.adj} = 0.743 (\texttt{ns})
    \item
      \textbf{Resultado}: \textbf{Não há diferença significativa} entre
      homens e mulheres que não consumiram álcool.
    \end{itemize}
  \item
    \textbf{Linha 12}: \texttt{3\ latas:Feminino} vs
    \texttt{3\ latas:Masculino}

    \begin{itemize}
    \item
      \texttt{p.adj} = 0.928 (\texttt{ns})
    \item
      \textbf{Resultado}: \textbf{Não há diferença significativa} entre
      homens e mulheres que consumiram 3 latas.
    \end{itemize}
  \item
    \textbf{Linha 16}: \texttt{6\ latas:Feminino} vs
    \texttt{6\ latas:Masculino}

    \begin{itemize}
    \item
      \texttt{p.adj} = 2.78×10−4 (\texttt{***})
    \item
      \textbf{Resultado}: \textbf{Há uma diferença significativa} no
      escore de memória entre mulheres e homens no grupo que consumiu 6
      latas.
    \end{itemize}
  \end{itemize}
\end{itemize}

\subsubsection{Conclusão Final}\label{conclusuxe3o-final}

\begin{itemize}
\item
  O consumo de \textbf{6 latas de cerveja} tem um \textbf{impacto
  negativo e altamente significativo} no escore de memória em comparação
  com o consumo de 3 latas ou nenhum consumo.
\item
  Embora não haja diferença geral entre os sexos, a análise da interação
  revela um ponto essencial: o consumo de 6 latas afeta o sexo
  masculino. Isso sugere que, no nível mais alto de consumo, o impacto
  sobre o escore de memória é diferente para homens e mulheres.
\end{itemize}

\subsubsection{Representação
gráfica}\label{representauxe7uxe3o-gruxe1fica}

Quando se tem muitos grupos para fazer a comparação a situação torna-se
mais complexa e fica mais fácil interpretar, usando gráficos
(Figura~\ref{fig-anova2tukey}) que tornam a tarefa um pouco mais
amigável:

\begin{Shaded}
\begin{Highlighting}[]
\CommentTok{\# Preparar os dados}
\NormalTok{pwc }\OtherTok{\textless{}{-}}\NormalTok{ pwc }\SpecialCharTok{\%\textgreater{}\%}
  \FunctionTok{mutate}\NormalTok{(}
    \AttributeTok{comparacao =} \FunctionTok{paste}\NormalTok{(group1, }\StringTok{"vs"}\NormalTok{, group2),}
    \AttributeTok{signif\_color =} \FunctionTok{ifelse}\NormalTok{(p.adj.signif }\SpecialCharTok{\%in\%} \FunctionTok{c}\NormalTok{(}\StringTok{"***"}\NormalTok{, }\StringTok{"****"}\NormalTok{), p.adj.signif, }\StringTok{"ns"}\NormalTok{))}
\end{Highlighting}
\end{Shaded}

Inicialmente, os dados foram modificados com a função \texttt{mutate(0}
do pacote \texttt{dplyr}, de maneira a comparar os grupos 1 e 2, de
acordo com a significância estatística. A seguir foi construido o
gráfico de maneira que inicialmente aparecem as comparações que não
foram significativas pontos médios de cor cinza e, após, as
significativas, de cores amarela e verde-claro, respectivamente,
\emph{p} \textless{} 0.001 e \emph{p} \textless{} 0.0001.

\begin{Shaded}
\begin{Highlighting}[]
\CommentTok{\# Gráfico com pontos (diferença média) e barras de erro(IC)}
\FunctionTok{ggplot}\NormalTok{(pwc, }\FunctionTok{aes}\NormalTok{(}\AttributeTok{x =}\NormalTok{ estimate, }
                \AttributeTok{y =} \FunctionTok{reorder}\NormalTok{(comparacao, estimate))) }\SpecialCharTok{+}
  \FunctionTok{geom\_vline}\NormalTok{(}\AttributeTok{xintercept =} \DecValTok{0}\NormalTok{, }
             \AttributeTok{linetype =} \StringTok{"dashed"}\NormalTok{, }
             \AttributeTok{color =} \StringTok{"gray40"}\NormalTok{) }\SpecialCharTok{+}
  \FunctionTok{geom\_errorbarh}\NormalTok{(}\FunctionTok{aes}\NormalTok{(}\AttributeTok{xmin =}\NormalTok{ conf.low, }\AttributeTok{xmax =}\NormalTok{ conf.high), }\AttributeTok{height =} \FloatTok{0.3}\NormalTok{) }\SpecialCharTok{+}
  \FunctionTok{geom\_point}\NormalTok{(}\FunctionTok{aes}\NormalTok{(}\AttributeTok{color =}\NormalTok{ signif\_color), }\AttributeTok{size =} \DecValTok{3}\NormalTok{) }\SpecialCharTok{+}
  \FunctionTok{scale\_color\_manual}\NormalTok{(}
    \AttributeTok{values =} \FunctionTok{c}\NormalTok{(}\StringTok{"****"} \OtherTok{=} \StringTok{"gold"}\NormalTok{, }\StringTok{"***"} \OtherTok{=} \StringTok{"greenyellow"}\NormalTok{, }\StringTok{"ns"} \OtherTok{=} \StringTok{"gray60"}\NormalTok{),}
    \AttributeTok{name =} \StringTok{"Significância"}
\NormalTok{  ) }\SpecialCharTok{+}
  \FunctionTok{labs}\NormalTok{(}
    \AttributeTok{title =} \StringTok{"Diferenças entre Grupos {-} Tukey HSD"}\NormalTok{,}
    \AttributeTok{x =} \StringTok{"Diferença Média"}\NormalTok{,}
    \AttributeTok{y =} \StringTok{"Comparações"}\NormalTok{) }\SpecialCharTok{+}
  \FunctionTok{theme\_bw}\NormalTok{() }\SpecialCharTok{+}
  \FunctionTok{theme}\NormalTok{(}
    \AttributeTok{legend.position =} \StringTok{"top"}\NormalTok{,}
    \AttributeTok{axis.text.y =} \FunctionTok{element\_text}\NormalTok{(}\AttributeTok{size =} \DecValTok{10}\NormalTok{),}
    \AttributeTok{plot.title =} \FunctionTok{element\_text}\NormalTok{(}\AttributeTok{face =} \StringTok{"bold"}\NormalTok{, }\AttributeTok{size =} \DecValTok{14}\NormalTok{))}
\end{Highlighting}
\end{Shaded}

\begin{figure}[H]

\centering{

\includegraphics[width=0.9\linewidth,height=0.9\textheight]{15-anova_files/figure-pdf/fig-anova2tukey-1.pdf}

}

\caption{\label{fig-anova2tukey}Gráfico da diferença entre os grupos -
Teste de TRukey HSD}

\end{figure}%

\subsection{Relatando uma ANOVA de dois
fatores}\label{relatando-uma-anova-de-dois-fatores}

Pode-se relatar os resultados da ANOVA de dois fatores da seguinte
maneira:

\begin{enumerate}
\def\labelenumi{(\arabic{enumi})}
\item
  Uma ANOVA de dois fatores foi realizada para avaliar se a memória de
  homens e mulheres era afetada pelo consumo do álcool avliado em três
  níveis:

  \begin{itemize}
  \tightlist
  \item
    Não consumiram álcool
  \item
    Consumiram 3 latas de cerveja
  \item
    Consumiram 6 latas de cerveja
  \end{itemize}
\item
  Os dados são apresentados como média e desvio padrão, na
  Tabela~\ref{tbl-result}).
\end{enumerate}

\global\setlength{\Oldarrayrulewidth}{\arrayrulewidth}

\global\setlength{\Oldtabcolsep}{\tabcolsep}

\setlength{\tabcolsep}{2pt}

\renewcommand*{\arraystretch}{1.5}



\providecommand{\ascline}[3]{\noalign{\global\arrayrulewidth #1}\arrayrulecolor[HTML]{#2}\cline{#3}}

\begin{longtable}[c]{|p{1.30in}|p{1.30in}|p{1.30in}|p{1.30in}}

\caption{\label{tbl-result}Efeito do Álcool* sobre a Memória de acordo
com o sexo**}

\tabularnewline

\ascline{1.5pt}{666666}{1-4}

\multicolumn{1}{>{\centering}m{\dimexpr 1.3in+0\tabcolsep}}{\textcolor[HTML]{000000}{\fontsize{11}{11}\selectfont{\global\setmainfont{Arial}{\textbf{Sexo}}}}} & \multicolumn{1}{>{\centering}m{\dimexpr 1.3in+0\tabcolsep}}{\textcolor[HTML]{000000}{\fontsize{11}{11}\selectfont{\global\setmainfont{Arial}{\textbf{Nenhum}}}}} & \multicolumn{1}{>{\centering}m{\dimexpr 1.3in+0\tabcolsep}}{\textcolor[HTML]{000000}{\fontsize{11}{11}\selectfont{\global\setmainfont{Arial}{\textbf{Um\ litro}}}}} & \multicolumn{1}{>{\centering}m{\dimexpr 1.3in+0\tabcolsep}}{\textcolor[HTML]{000000}{\fontsize{11}{11}\selectfont{\global\setmainfont{Arial}{\textbf{Dois\ litros}}}}} \\

\ascline{1.5pt}{666666}{1-4}\endfirsthead 

\ascline{1.5pt}{666666}{1-4}

\multicolumn{1}{>{\centering}m{\dimexpr 1.3in+0\tabcolsep}}{\textcolor[HTML]{000000}{\fontsize{11}{11}\selectfont{\global\setmainfont{Arial}{\textbf{Sexo}}}}} & \multicolumn{1}{>{\centering}m{\dimexpr 1.3in+0\tabcolsep}}{\textcolor[HTML]{000000}{\fontsize{11}{11}\selectfont{\global\setmainfont{Arial}{\textbf{Nenhum}}}}} & \multicolumn{1}{>{\centering}m{\dimexpr 1.3in+0\tabcolsep}}{\textcolor[HTML]{000000}{\fontsize{11}{11}\selectfont{\global\setmainfont{Arial}{\textbf{Um\ litro}}}}} & \multicolumn{1}{>{\centering}m{\dimexpr 1.3in+0\tabcolsep}}{\textcolor[HTML]{000000}{\fontsize{11}{11}\selectfont{\global\setmainfont{Arial}{\textbf{Dois\ litros}}}}} \\

\ascline{1.5pt}{666666}{1-4}\endhead



\multicolumn{1}{>{\raggedright}m{\dimexpr 1.3in+0\tabcolsep}}{\textcolor[HTML]{000000}{\fontsize{11}{11}\selectfont{\global\setmainfont{Arial}{Feminino}}}} & \multicolumn{1}{>{\centering}m{\dimexpr 1.3in+0\tabcolsep}}{\textcolor[HTML]{000000}{\fontsize{11}{11}\selectfont{\global\setmainfont{Arial}{60,6\ (5,0)}}}} & \multicolumn{1}{>{\centering}m{\dimexpr 1.3in+0\tabcolsep}}{\textcolor[HTML]{000000}{\fontsize{11}{11}\selectfont{\global\setmainfont{Arial}{62,5\ (6,6)}}}} & \multicolumn{1}{>{\centering}m{\dimexpr 1.3in+0\tabcolsep}}{\textcolor[HTML]{000000}{\fontsize{11}{11}\selectfont{\global\setmainfont{Arial}{57,5\ (7,1)}}}} \\





\multicolumn{1}{>{\raggedright}m{\dimexpr 1.3in+0\tabcolsep}}{\textcolor[HTML]{000000}{\fontsize{11}{11}\selectfont{\global\setmainfont{Arial}{Masculino}}}} & \multicolumn{1}{>{\centering}m{\dimexpr 1.3in+0\tabcolsep}}{\textcolor[HTML]{000000}{\fontsize{11}{11}\selectfont{\global\setmainfont{Arial}{66,9\ (10,3)}}}} & \multicolumn{1}{>{\centering}m{\dimexpr 1.3in+0\tabcolsep}}{\textcolor[HTML]{000000}{\fontsize{11}{11}\selectfont{\global\setmainfont{Arial}{66,9\ (12,5)}}}} & \multicolumn{1}{>{\centering}m{\dimexpr 1.3in+0\tabcolsep}}{\textcolor[HTML]{000000}{\fontsize{11}{11}\selectfont{\global\setmainfont{Arial}{35,6\ (10,8)}}}} \\

\ascline{1.3pt}{000000}{1-4}



\multicolumn{1}{>{\raggedright}m{\dimexpr 1.3in+0\tabcolsep}}{\textcolor[HTML]{000000}{\fontsize{11}{11}\selectfont{\global\setmainfont{Arial}{Valor\ P}}}} & \multicolumn{1}{>{\centering}m{\dimexpr 1.3in+0\tabcolsep}}{\textcolor[HTML]{000000}{\fontsize{11}{11}\selectfont{\global\setmainfont{Arial}{0,145}}}} & \multicolumn{1}{>{\centering}m{\dimexpr 1.3in+0\tabcolsep}}{\textcolor[HTML]{000000}{\fontsize{11}{11}\selectfont{\global\setmainfont{Arial}{0,396}}}} & \multicolumn{1}{>{\centering}m{\dimexpr 1.3in+0\tabcolsep}}{\textcolor[HTML]{000000}{\fontsize{11}{11}\selectfont{\global\setmainfont{Arial}{0,0003}}}} \\

\ascline{1.5pt}{666666}{1-4}



\multicolumn{4}{>{\raggedright}m{\dimexpr 5.2in+6\tabcolsep}}{\textcolor[HTML]{000000}{\fontsize{9}{9}\selectfont{\global\setmainfont{Arial}{*Um\ litro\ de\ cerveja\ (4,5\%)\ =\ 5\ unidades\ de\ alcool}}}} \\





\multicolumn{4}{>{\raggedright}m{\dimexpr 5.2in+6\tabcolsep}}{\textcolor[HTML]{000000}{\fontsize{9}{9}\selectfont{\global\setmainfont{Arial}{**\ Escore\ médio\ (desvio\ padrão)}}}} \\




\end{longtable}

\arrayrulecolor[HTML]{000000}

\global\setlength{\arrayrulewidth}{\Oldarrayrulewidth}

\global\setlength{\tabcolsep}{\Oldtabcolsep}

\renewcommand*{\arraystretch}{1}

\begin{enumerate}
\def\labelenumi{(\arabic{enumi})}
\setcounter{enumi}{2}
\item
  O efeito principal do sexo na memória foi não significativo
  (\emph{F}(1,42) = 2,03, \emph{p} = 0,161).
\item
  Houve um efeito principal significativo de acordo com a quantidade de
  álcool consumida na memória dos participantes (\emph{F}(2,42) = 20,07,
  \emph{p} \textless0,001).
\item
  As análises posteriores (teste de Tukey, Figura~\ref{fig-anova2tukey})
  revelaram que a memória não foi afetada nas mulheres pelo consumo de
  álcool, mas o consumo de 6 latas de cerveja afetou a memória dos
  homens quando comparados os homens que não consumiram álcool ou que
  consumiram até 3 latas de cerveja.
\item
  Visualização dos resultados:
\end{enumerate}

Serão apresentados gráficos de barra de erro (Figura~\ref{fig-result1}),
com \texttt{ggbarplot()}, do pacote \texttt{ggpubr}, utilizando, para
cores tonalidades de cinza. Para adicionar teste estatístico, usou-se a
função \texttt{get\_test\_label()} e para o teste \emph{post hoc}, a
função \texttt{get\_pwc\_label()}, ambas do pacote \texttt{rstatix}.

\begin{Shaded}
\begin{Highlighting}[]
\CommentTok{\# Construção de um gráfico de barra de erro}
\NormalTok{be }\OtherTok{\textless{}{-}}\NormalTok{ ggpubr}\SpecialCharTok{::}\FunctionTok{ggbarplot}\NormalTok{(dados, }
                        \AttributeTok{x =} \StringTok{"alcool"}\NormalTok{, }\AttributeTok{y =} \StringTok{"escore"}\NormalTok{, }
                        \AttributeTok{add =} \StringTok{"mean\_ci"}\NormalTok{,}
                        \AttributeTok{error.plot =} \StringTok{"upper\_errorbar"}\NormalTok{,}
                        \AttributeTok{fill =} \StringTok{"sexo"}\NormalTok{, }
                        \AttributeTok{palette =} \StringTok{"nejm"}\NormalTok{,}
                        \AttributeTok{position =} \FunctionTok{position\_dodge}\NormalTok{(}\FloatTok{0.8}\NormalTok{)) }\SpecialCharTok{+}
  \FunctionTok{theme}\NormalTok{(}\AttributeTok{legend.key.size =} \FunctionTok{unit}\NormalTok{(}\FloatTok{0.3}\NormalTok{, }\StringTok{\textquotesingle{}cm\textquotesingle{}}\NormalTok{)) }\SpecialCharTok{+}
  \FunctionTok{theme}\NormalTok{(}\AttributeTok{legend.position =} \StringTok{"right"}\NormalTok{)}

\CommentTok{\# Comparações por pares (pairwise comparisons)}
\NormalTok{pwc }\OtherTok{\textless{}{-}}\NormalTok{ dados }\SpecialCharTok{\%\textgreater{}\%}
\NormalTok{   dplyr}\SpecialCharTok{::}\FunctionTok{group\_by}\NormalTok{(alcool) }\SpecialCharTok{\%\textgreater{}\%}
\NormalTok{   rstatix}\SpecialCharTok{::}\FunctionTok{tukey\_hsd}\NormalTok{(}\AttributeTok{formula =}\NormalTok{ escore }\SpecialCharTok{\textasciitilde{}}\NormalTok{ sexo)}

\CommentTok{\# Calcular e adicionar as posições x e y.}
\NormalTok{pwc }\OtherTok{\textless{}{-}}\NormalTok{ pwc }\SpecialCharTok{\%\textgreater{}\%}
  \FunctionTok{add\_xy\_position}\NormalTok{(}\AttributeTok{fun =} \StringTok{"mean\_ci"}\NormalTok{, }
                  \AttributeTok{x =} \StringTok{"alcool"}\NormalTok{, }
                  \AttributeTok{dodge =} \FloatTok{0.8}\NormalTok{) }

\CommentTok{\# Cálculo do teste estatístico com pacote rstatix}
\NormalTok{anova }\OtherTok{\textless{}{-}}  \FunctionTok{anova\_test}\NormalTok{(mod.aov)}

\CommentTok{\# Acrescentar o teste e o valor P ajustado ao gráfico}
\NormalTok{be }\SpecialCharTok{+} \FunctionTok{stat\_pvalue\_manual}\NormalTok{(pwc,  }
                        \AttributeTok{label =} \StringTok{"p.adj"}\NormalTok{, }
                        \AttributeTok{tip.length =} \FloatTok{0.01}\NormalTok{,}
                        \AttributeTok{y.position =} \DecValTok{85}\NormalTok{) }\SpecialCharTok{+}
  \FunctionTok{labs}\NormalTok{ (}\AttributeTok{x =} \StringTok{"Ingestão de álcool"}\NormalTok{,}
        \AttributeTok{y =} \StringTok{"Média escore de memória"}\NormalTok{,}
        \AttributeTok{fill =} \StringTok{""}\NormalTok{,}
        \AttributeTok{subtitle =}\NormalTok{ rstatix}\SpecialCharTok{::}\FunctionTok{get\_test\_label}\NormalTok{ (anova, }\AttributeTok{detailed =} \ConstantTok{TRUE}\NormalTok{),}
        \AttributeTok{caption =}\NormalTok{ rstatix}\SpecialCharTok{::}\FunctionTok{get\_pwc\_label}\NormalTok{(pwc))}
\end{Highlighting}
\end{Shaded}

\begin{figure}[H]

\centering{

\includegraphics[width=0.8\linewidth,height=0.8\textheight]{15-anova_files/figure-pdf/fig-result1-1.pdf}

}

\caption{\label{fig-result1}Efeito da quantidade ingerida de álcool na
memória de acordo com o sexo}

\end{figure}%

Uma opção, é apresentar os resultados como um gráfico de linhas
(Figura~\ref{fig-result2}), já mostrado anteriormente , usando a função
\texttt{ggline()} do pacote \texttt{ggpubr}, com as cores do periódico
\emph{New England Journal of Medicine}:

\begin{Shaded}
\begin{Highlighting}[]
\CommentTok{\# Construção de um gráfico linha}
\NormalTok{gl }\OtherTok{\textless{}{-}}\NormalTok{ ggpubr}\SpecialCharTok{::}\FunctionTok{ggline}\NormalTok{(dados, }
                     \AttributeTok{x =} \StringTok{"alcool"}\NormalTok{, }\AttributeTok{y =} \StringTok{"escore"}\NormalTok{, }
                     \AttributeTok{add =} \StringTok{"mean\_ci"}\NormalTok{,}
                     \AttributeTok{color =} \StringTok{"sexo"}\NormalTok{,}
                     \AttributeTok{linetype =} \StringTok{"dashed"}\NormalTok{, }
                     \AttributeTok{linewidth =}  \FloatTok{0.9}\NormalTok{,}
                     \AttributeTok{palette =} \StringTok{"nejm"}\NormalTok{,}
                     \AttributeTok{position =} \FunctionTok{position\_dodge}\NormalTok{(}\FloatTok{0.2}\NormalTok{)) }\SpecialCharTok{+}
  \FunctionTok{theme}\NormalTok{(}\AttributeTok{legend.key.size =} \FunctionTok{unit}\NormalTok{(}\FloatTok{0.3}\NormalTok{, }\StringTok{\textquotesingle{}cm\textquotesingle{}}\NormalTok{)) }\SpecialCharTok{+}
  \FunctionTok{theme}\NormalTok{(}\AttributeTok{legend.position =} \StringTok{"right"}\NormalTok{)}

\CommentTok{\# Comparações por pares (pairwise comparisons)}
\NormalTok{pwc }\OtherTok{\textless{}{-}}\NormalTok{ dados }\SpecialCharTok{\%\textgreater{}\%}
\NormalTok{  dplyr}\SpecialCharTok{::}\FunctionTok{group\_by}\NormalTok{(alcool) }\SpecialCharTok{\%\textgreater{}\%}
\NormalTok{  rstatix}\SpecialCharTok{::}\FunctionTok{tukey\_hsd}\NormalTok{(}\AttributeTok{formula =}\NormalTok{ escore }\SpecialCharTok{\textasciitilde{}}\NormalTok{ sexo)}

\CommentTok{\# Calcular e adicionar as posições x e y.}
\NormalTok{pwc }\OtherTok{\textless{}{-}}\NormalTok{ pwc }\SpecialCharTok{\%\textgreater{}\%}
  \FunctionTok{add\_xy\_position}\NormalTok{(}\AttributeTok{fun =} \StringTok{"mean\_ci"}\NormalTok{, }
                  \AttributeTok{x =} \StringTok{"alcool"}\NormalTok{, }
                  \AttributeTok{dodge =} \FloatTok{0.8}\NormalTok{)}

\CommentTok{\# Cálculo do teste estatístico com pacote rstatix}
\NormalTok{anova }\OtherTok{\textless{}{-}}  \FunctionTok{anova\_test}\NormalTok{(mod.aov)}

\CommentTok{\# Acrescentar o teste e o valor P ajustado ao gráfico}
\NormalTok{gl }\SpecialCharTok{+} \FunctionTok{stat\_pvalue\_manual}\NormalTok{(pwc,  }
                        \AttributeTok{label =} \StringTok{"p.adj"}\NormalTok{, }
                        \AttributeTok{tip.length =} \FloatTok{0.01}\NormalTok{,}
                        \AttributeTok{y.position =} \DecValTok{85}\NormalTok{) }\SpecialCharTok{+}
  \FunctionTok{labs}\NormalTok{ (}\AttributeTok{x =} \StringTok{"Ingestão de álcool"}\NormalTok{,}
        \AttributeTok{y =} \StringTok{"Média escore de memória"}\NormalTok{,}
        \AttributeTok{subtitle =}\NormalTok{ rstatix}\SpecialCharTok{::}\FunctionTok{get\_test\_label}\NormalTok{ (anova, }
                                            \AttributeTok{detailed =} \ConstantTok{TRUE}\NormalTok{),}
        \AttributeTok{caption =}\NormalTok{ rstatix}\SpecialCharTok{::}\FunctionTok{get\_pwc\_label}\NormalTok{(pwc))}
\end{Highlighting}
\end{Shaded}

\begin{figure}[H]

\centering{

\includegraphics[width=0.8\linewidth,height=0.8\textheight]{15-anova_files/figure-pdf/fig-result2-1.pdf}

}

\caption{\label{fig-result2}Efeito da quantidade ingerida de álcool na
memória de acordo com o sexo}

\end{figure}%

\bookmarksetup{startatroot}

\chapter{Summary}\label{summary}

In summary, this book has no content whatsoever.

\begin{Shaded}
\begin{Highlighting}[]
\DecValTok{1} \SpecialCharTok{+} \DecValTok{1}
\end{Highlighting}
\end{Shaded}

\begin{verbatim}
[1] 2
\end{verbatim}

\bookmarksetup{startatroot}

\chapter*{References}\label{references}
\addcontentsline{toc}{chapter}{References}

\markboth{References}{References}

\phantomsection\label{refs}
\begin{CSLReferences}{0}{1}
\bibitem[\citeproctext]{ref-armitage2008statistical}
\CSLLeftMargin{1. }%
\CSLRightInline{Armitage P, Berry G, Matthews JNS. Statistical methods
in medical research. John Wiley \& Sons; 2008. }

\bibitem[\citeproctext]{ref-massad2004metodos}
\CSLLeftMargin{2. }%
\CSLRightInline{Massad E, Silveira PSP, Menezes RX de, Ortega NRS.
Métodos quantitativos em medicina. Editora Manole Ltda; 2004. }

\bibitem[\citeproctext]{ref-kendall1960studies}
\CSLLeftMargin{3. }%
\CSLRightInline{Kendall MG. Studies in the history of probability and
statistics. Where shall the history of statistics begin? Biometrika.
1960;47(3/4):447--9. }

\bibitem[\citeproctext]{ref-ine2014censos}
\CSLLeftMargin{4. }%
\CSLRightInline{Breve História dos Censos {[}Internet{]}. Instituto
Nacional de Estatistica. Statistics Portugal; 2014. Disponível em:
\url{https://censos.ine.pt/xportal/xmain?xpid=CENSOS&amp;xpgid=censos_bhistoria}}

\bibitem[\citeproctext]{ref-salgado2011sir}
\CSLLeftMargin{5. }%
\CSLRightInline{Salgado-Neto G, Salgado A. Sir Francis Galton e os
extremos superiores da curva normal. Revista de Ciências Humanas.
2011;45(1):223--39. }

\bibitem[\citeproctext]{ref-stolley1995beginnings}
\CSLLeftMargin{6. }%
\CSLRightInline{Stolley PD, Lasky T. The Beginnings of Epidemiology. Em:
Investigating Disease Patterns. Scientific American Library; 2000. p.
23--49. }

\bibitem[\citeproctext]{ref-history2009}
\CSLLeftMargin{7. }%
\CSLRightInline{Editors History com. {Florence Nightingale}.
https://www.history.com/topics/womens-history/florence-nightingale-1; }

\bibitem[\citeproctext]{ref-moore_2000}
\CSLLeftMargin{8. }%
\CSLRightInline{Moore DS. Topics in Inferency. Em: The basic practice of
statistics. W.H. Freeman; 2000. p. 417. }

\bibitem[\citeproctext]{ref-salsburg2009}
\CSLLeftMargin{9. }%
\CSLRightInline{Salsburg D. Uma senhora toma chá... Em: Uma senhora toma
chá. Zahar; 2009. p. 17--23. }

\bibitem[\citeproctext]{ref-hald2007}
\CSLLeftMargin{10. }%
\CSLRightInline{Hald A. Biography of Fisher. Em: A History of Parametric
Statistics Inference from Bernoulli to Fisher,1713-1935. John Wiley \&
Sons; 2007. p. 159--63. }

\bibitem[\citeproctext]{ref-kruskal1980}
\CSLLeftMargin{11. }%
\CSLRightInline{Kruskal W. The Significance of Fisher: A Review of R.A.
Fisher: The Life of a Scientist. Journal of the American Statistical
Association {[}Internet{]}. 1980;75(372):1019--30. Disponível em:
\url{https://doi.org/10.1080/01621459.1980.10477590}}

\bibitem[\citeproctext]{ref-stolley1995lung}
\CSLLeftMargin{12. }%
\CSLRightInline{Stolley PD, Lasky T. Lung Cancer: New Methods of
Studying Disease. Em: Investigating Disease Patterns. Scientific
American Library; 2000. p. 51--79. }

\bibitem[\citeproctext]{ref-matthews2018douglas}
\CSLLeftMargin{13. }%
\CSLRightInline{Matthews R, Chalmers I, Rothwell P. Douglas G Altman:
statistician, researcher, and driving force behind global initiatives to
improve the reliability of health research. British Medical Journal
Publishing Group; 2018. }

\bibitem[\citeproctext]{ref-altman1994scandal}
\CSLLeftMargin{14. }%
\CSLRightInline{Altman DG. The scandal of poor medical research. Vol.
308, Bmj. British Medical Journal Publishing Group; 1994. p. 283--4. }

\bibitem[\citeproctext]{ref-rproject2022}
\CSLLeftMargin{15. }%
\CSLRightInline{R Core Team. The R Project for Statistical Computing
\textbar{} What is R? Disponível em:
https://www.r-project.org/about.html; 2022. }

\bibitem[\citeproctext]{ref-cranmirrors2022}
\CSLLeftMargin{16. }%
\CSLRightInline{R Core Team. The R Project for Statistical Computing
\textbar{} CRAN Mirrors. Disponível em:
https://cran.r-project.org/mirrors.html; 2022. }

\bibitem[\citeproctext]{ref-Whitney2020}
\CSLLeftMargin{17. }%
\CSLRightInline{Whitney L et al. R programming language continues to
grow in popularity {[}Internet{]}. TechRepublic. 2020. Disponível em:
\url{https://www.techrepublic.com/article/r-programming-language-continues-to-grow-in-popularity}}

\bibitem[\citeproctext]{ref-oliveira2022epidemiologia}
\CSLLeftMargin{18. }%
\CSLRightInline{Oliveira Filho PF de. Natureza dos Dados. Em:
Epidemiologia e Bioestatística--Fundamentos para a Leitura Crítica. 2ª
edição. Editora Rubio; 2022. p. 3--6. }

\bibitem[\citeproctext]{ref-kirkwood2003essential}
\CSLLeftMargin{19. }%
\CSLRightInline{Kirkwood BR, Sterne JA. Defining the Data. Em: Essential
Medical Statistics. Second Edition. Blackwell Science Company; 2003. p.
9--14. }

\bibitem[\citeproctext]{ref-sternbach2000glasgow}
\CSLLeftMargin{20. }%
\CSLRightInline{Sternbach GL. The Glasgow coma scale. The Journal of
emergency medicine. 2000;19(1):67--71. }

\bibitem[\citeproctext]{ref-american2006apgar}
\CSLLeftMargin{21. }%
\CSLRightInline{Pediatrics AA of, Obstetricians AC of. The apgar score.
Pediatrics. 2006;117(4):1444--7. }

\bibitem[\citeproctext]{ref-bowers2008scratch}
\CSLLeftMargin{22. }%
\CSLRightInline{Bowers D. First things first-the nature of data. Em:
Medical Statistics from Scratch. Second Edition. John Wiley; Sons; 2008.
p. 3--13. }

\bibitem[\citeproctext]{ref-mendes2012}
\CSLLeftMargin{23. }%
\CSLRightInline{Ribeiro Mendes F. O que é um trabalho científico. Em:
Iniciacão Cientifica. Autonomia Editora; 2012. p. 17--26. }

\bibitem[\citeproctext]{ref-hulley2015delineando}
\CSLLeftMargin{24. }%
\CSLRightInline{Hulley SB, Cummings SR, Browner WS, Grady DG, Newman TB.
Elaborando a questão de pesquisa e desenvolvendo o plano de estudo. Em:
Delineando a pesquisa clinica. Quarta Edição. Artmed Editora; 2015. p.
15--24. }

\bibitem[\citeproctext]{ref-mccombes2019sampling}
\CSLLeftMargin{25. }%
\CSLRightInline{McCombes S. Sampling Methods {[}Internet{]}.
https://www.scribbr.com/methodology/sampling-methods/. scribbr.com Team;
2019. Disponível em: \url{https://www.scribbr.com/}}

\bibitem[\citeproctext]{ref-callegari2003bioestatistica}
\CSLLeftMargin{26. }%
\CSLRightInline{Callegari-Jacques SM. Amostras. Em: Bioestatistica:
principios e aplicações. Artmed Editora; 2003. p. 146--7. }

\bibitem[\citeproctext]{ref-faul2007g}
\CSLLeftMargin{27. }%
\CSLRightInline{Faul F, Erdfelder E, Lang A-G, Buchner A. G* Power 3: A
flexible statistical power analysis program for the social, behavioral,
and biomedical sciences. Behavior research methods. 2007;39(2):175--91.
}

\bibitem[\citeproctext]{ref-cohen1988statistical}
\CSLLeftMargin{28. }%
\CSLRightInline{Cohen J. Statistical power analysis for the behavioral
sciences. Lawrence Erlbaum Associates; 1988. }

\bibitem[\citeproctext]{ref-grimes2002overview}
\CSLLeftMargin{29. }%
\CSLRightInline{Grimes DA, Schulz KF. An overview of clinical research:
the lay of the land. The lancet. 2002;359(9300):57--61. }

\bibitem[\citeproctext]{ref-fletcher2014epidemiologia}
\CSLLeftMargin{30. }%
\CSLRightInline{Fletcher RH, Fletcher SW, Fletcher GS. Prognóstico. Em:
Epidemiologia Clínica: Elementos Essenciais. Artmed Editora; 2014. p.
108--9. }

\bibitem[\citeproctext]{ref-grimes2002descriptive}
\CSLLeftMargin{31. }%
\CSLRightInline{Grimes DA, Schulz KF. Descriptive studies: what they can
and cannot do. The Lancet. 2002;359(9301):145--9. }

\bibitem[\citeproctext]{ref-fletcher2014risco}
\CSLLeftMargin{32. }%
\CSLRightInline{Fletcher RH, Fletcher SW, Fletcher GS. Risco: da doença
à exposição. Em: Epidemiologia Clínica: Elementos Essenciais. Artmed
Editora; 2014. p. 88. }

\bibitem[\citeproctext]{ref-grimes2005compared}
\CSLLeftMargin{33. }%
\CSLRightInline{Grimes DA, Schulz KF. Compared to what? Finding controls
for case-control studies. The Lancet. 2005;365(9468):1429--33. }

\bibitem[\citeproctext]{ref-ernster1994nested}
\CSLLeftMargin{34. }%
\CSLRightInline{Ernster VL. Nested case-control studies. Preventive
Medicine. 1994;23(5):587--90. }

\bibitem[\citeproctext]{ref-hulley2015aninhado}
\CSLLeftMargin{35. }%
\CSLRightInline{Newman TB, Browner WS, Cummings SR, Hulley SB.
Delineando estudos de caso-controle. Em: Delineando a pesquisa clinica.
Quarta Edição. Artmed Editora; 2015. p. 111. }

\bibitem[\citeproctext]{ref-grimes2002cohort}
\CSLLeftMargin{36. }%
\CSLRightInline{Grimes DA, Schulz KF. Cohort studies: marching towards
outcomes. The Lancet. 2002;359(9303):341--5. }

\bibitem[\citeproctext]{ref-fletcher2014coorte}
\CSLLeftMargin{37. }%
\CSLRightInline{Fletcher RH, Fletcher SW, Fletcher GS. Risco: da doença
à exposição. Em: Epidemiologia Clínica: Elementos Essenciais. Artmed
Editora; 2014. p. 68. }

\bibitem[\citeproctext]{ref-kannel1979diabetes}
\CSLLeftMargin{38. }%
\CSLRightInline{Kannel WB, McGee DL. Diabetes and cardiovascular risk
factors: the Framingham study. Circulation. 1979;59(1):8--13. }

\bibitem[\citeproctext]{ref-david2019gordis}
\CSLLeftMargin{39. }%
\CSLRightInline{Celentano DD, Szklo M. Cohort Studies. Em: Gordis
Epidemiology. 6th Edition. Elsevier; 2019. p. 179. }

\bibitem[\citeproctext]{ref-coutinho1998principios}
\CSLLeftMargin{40. }%
\CSLRightInline{Coutinho M. Principios de epidemiologia clínica aplicada
a cardiologia. Arquivos Brasileiros de Cardiologia. 1998;71:109--16. }

\bibitem[\citeproctext]{ref-cambridge2014systematic}
\CSLLeftMargin{41. }%
\CSLRightInline{McCambridge J, Witton J, Elbourne DR. Systematic review
of the Hawthorne effect: new concepts are needed to study research
participation effects. Journal of Clinical Epidemiology.
2014;67(3):267--77. }

\bibitem[\citeproctext]{ref-bland1994statistic}
\CSLLeftMargin{42. }%
\CSLRightInline{Bland JM, Altman DG. Statistic Notes: Regression towards
the mean. BMJ. 1994;308(6942):1499. }

\bibitem[\citeproctext]{ref-fletcher2014tratamento}
\CSLLeftMargin{43. }%
\CSLRightInline{Fletcher RH, Fletcher SW, Fletcher GS. Tratamento. Em:
Epidemiologia Clínica: Elementos Essenciais. Artmed Editora; 2014. p.
143. }

\bibitem[\citeproctext]{ref-kabisch2011randomized}
\CSLLeftMargin{44. }%
\CSLRightInline{Kabisch M, Ruckes C, Seibert-Grafe M, Blettner M.
Randomized controlled trials: part 17 of a series on evaluation of
scientific publications. Deutsches {Ä}rzteblatt International.
2011;108(39):663. }

\bibitem[\citeproctext]{ref-elander1995placebo}
\CSLLeftMargin{45. }%
\CSLRightInline{Elander G, Hermerén G. Placebo effect and randomized
clinical trials. Theoretical Medicine. 1995;16(2):171--82. }

\bibitem[\citeproctext]{ref-schulz2002blinding}
\CSLLeftMargin{46. }%
\CSLRightInline{Schulz KF, Grimes DA. Blinding in randomised trials:
hiding who got what. The Lancet. 2002;359(9307):696--700. }

\bibitem[\citeproctext]{ref-montori2001intention}
\CSLLeftMargin{47. }%
\CSLRightInline{Montori VM, Guyatt GH. Intention-to-treat principle.
CMAJ. 2001;165(10):1339--41. }

\bibitem[\citeproctext]{ref-christensen2007methodology}
\CSLLeftMargin{48. }%
\CSLRightInline{Christensen E. Methodology of superiority vs.
equivalence trials and non-inferiority trials. Journal of hepatology.
2007;46(5):947--54. }

\bibitem[\citeproctext]{ref-trial2020crossover}
\CSLLeftMargin{49. }%
\CSLRightInline{Health Improvement O for, Disparities. Crossover
randomised controlled trial: comparative studies {[}Internet{]}. Office
for Health Improvement and Disparities. UK Health improvement; 2020.
Disponível em:
\url{https://www.gov.uk/guidance/crossover-randomised-controlled-trial-comparative-studies}}

\bibitem[\citeproctext]{ref-steering1989final}
\CSLLeftMargin{50. }%
\CSLRightInline{Physicians' Health Study Research Group* SC of the.
Final report on the aspirin component of the ongoing Physicians' Health
Study. New England Journal of Medicine. 1989;321(3):129--35. }

\bibitem[\citeproctext]{ref-hennekens1996lack}
\CSLLeftMargin{51. }%
\CSLRightInline{Hennekens CH, Buring JE, et al. Lack of effect of
long-term supplementation with beta carotene on the incidence of
malignant neoplasms and cardiovascular disease. New England Journal of
Medicine. 1996;334(18):1145--9. }

\bibitem[\citeproctext]{ref-stanley2007design}
\CSLLeftMargin{52. }%
\CSLRightInline{Stanley K. Design of randomized controlled trials.
Circulation. 2007;115(9):1164--9. }

\bibitem[\citeproctext]{ref-blog_2024}
\CSLLeftMargin{53. }%
\CSLRightInline{Blog TJR. Positron vs rstudio -- is it time to switch?
{[}Internet{]}. R-bloggers. 2024. Disponível em:
\url{https://www.r-bloggers.com/2024/12/positron-vs-rstudio-is-it-time-to-switch/}}

\bibitem[\citeproctext]{ref-positron2025}
\CSLLeftMargin{54. }%
\CSLRightInline{Software P. Frequently asked questions {[}Internet{]}.
Positron. Posit Software, PBC; 2025. Disponível em:
\url{https://positron.posit.co/faqs.html}}

\bibitem[\citeproctext]{ref-chang2021cookbook}
\CSLLeftMargin{55. }%
\CSLRightInline{Chang W. Cookbook for R. Cookbook for R.
http://www.cookbook-r.com; 2021. }

\bibitem[\citeproctext]{ref-verzani2004using}
\CSLLeftMargin{56. }%
\CSLRightInline{Verzani J. Using R for introductory statistics. Chapman;
Hall/CRC; 2004. }

\bibitem[\citeproctext]{ref-tyler2018pacman}
\CSLLeftMargin{57. }%
\CSLRightInline{Rinker TW, Kurkiewicz D. pacman: Package Management for
R {[}Internet{]}. Buffalo, New York; 2018. Disponível em:
\url{http://github.com/trinker/pacman}}

\bibitem[\citeproctext]{ref-cursor2015livro}
\CSLLeftMargin{58. }%
\CSLRightInline{Damiani A, Milz B, Lente C, al et. Ciência de Dados em R
{[}Internet{]}. R6 Consultoria; 2015. Disponível em:
\url{https://livro.curso-r.com/index.html}}

\bibitem[\citeproctext]{ref-wickham2017factors}
\CSLLeftMargin{59. }%
\CSLRightInline{Wickham H, Grolemund G. 15 Factors\textbar R for data
science {[}Internet{]}. Welcome \textbar{} R for Data Science. O'Reilly;
2017. Disponível em: \url{https://r4ds.had.co.nz/factors.html}}

\bibitem[\citeproctext]{ref-zuur2009beginner}
\CSLLeftMargin{60. }%
\CSLRightInline{Zuur AF, Ieno EN, Meesters EH. Getting Data into R. Em:
A Beginner's Guide to R. Springer; 2009. p. 29--56. }

\bibitem[\citeproctext]{ref-ooms2022writexl}
\CSLLeftMargin{61. }%
\CSLRightInline{Ooms J. writexl: Export Data Frames to Excel 'xlsx'
Format {[}Internet{]}. 2022. Disponível em:
\url{https://CRAN.R-project.org/package=writexl}}

\bibitem[\citeproctext]{ref-write2022csv}
\CSLLeftMargin{62. }%
\CSLRightInline{Team RC. write.table: Data Output/CSV files
{[}Internet{]}. DataCamp; 2022. Disponível em:
\url{https://www.rdocumentation.org/packages/utils/versions/3.6.2/topics/write.table}}

\bibitem[\citeproctext]{ref-wickham2019tidyverse}
\CSLLeftMargin{63. }%
\CSLRightInline{Wickham H, Averick M, Bryan J, Chang W, et al. Welcome
to the Tidyverse. Journal of Open Source Software. 2019;4(43):1686. }

\bibitem[\citeproctext]{ref-wickham2022tidyr}
\CSLLeftMargin{64. }%
\CSLRightInline{Wickham H, Girlich M. tidyr: Tidy Messy Data
{[}Internet{]}. 2022. Disponível em:
\url{https://CRAN.R-project.org/package=tidyr}}

\bibitem[\citeproctext]{ref-wickham2014tidy}
\CSLLeftMargin{65. }%
\CSLRightInline{Wickham H. Tidy Data. Journal of Statistical Software.
2014;59(10):11--23. }

\bibitem[\citeproctext]{ref-bache2022magritt}
\CSLLeftMargin{66. }%
\CSLRightInline{Bache SM, Wickham H. magrittr: A Forward-Pipe Operator
for R. 2022. }

\bibitem[\citeproctext]{ref-wickham2015dplyr}
\CSLLeftMargin{67. }%
\CSLRightInline{Wickham H, François R, Henry L, Müller K, et al. dplyr:
A grammar of data manipulation. R package version 04. 2015;3:156. }

\bibitem[\citeproctext]{ref-madi2010prevalence}
\CSLLeftMargin{68. }%
\CSLRightInline{Madi JM, Souza R da S de, Araujo BF de, Oliveira Filho
PF, et al. Prevalence of toxoplasmosis, HIV, syphilis and rubella in a
population of puerperal women using Whatman 903{\textregistered} filter
paper. The Brazilian Journal of Infectious Diseases. 2010;14(1):24--9. }

\bibitem[\citeproctext]{ref-damiani2022forcats}
\CSLLeftMargin{69. }%
\CSLRightInline{Damiani A, Milz B, Lente C, Outros. O pacote forcats
{[}Internet{]}. Ciência de Dados em R. R6 Consultoria; 2022. Disponível
em:
\url{https://livro.curso-r.com/7-6-forcats.html\#o-que-s\%C3\%A3o-fatores}}

\bibitem[\citeproctext]{ref-grolemund2011dates}
\CSLLeftMargin{70. }%
\CSLRightInline{Grolemund G, Wickham H. Dates and Times Made Easy with
lubridate. Journal of Statistical Software {[}Internet{]}.
2011;40(3):1--25. Disponível em:
\url{https://www.jstatsoft.org/v40/i03/}}

\bibitem[\citeproctext]{ref-field2012df}
\CSLLeftMargin{71. }%
\CSLRightInline{Field A, Miles J, Field Z. Everithing you ever wanted to
know about statistics (well, sort of). Em: Discovering statistics using
R. Sage Publications, Ltd; 2012. p. 38. }

\bibitem[\citeproctext]{ref-arango2009tab}
\CSLLeftMargin{72. }%
\CSLRightInline{Arango HG. Organização dos dados em tabelas. Em:
Bioestatística: teórica e computacional. 3ª edição. Guanabara Koogan;
2009. p. 32--57. }

\bibitem[\citeproctext]{ref-geeks2025case_when}
\CSLLeftMargin{73. }%
\CSLRightInline{GeeksforGeeks. Case when statement in R Dplyr package
using case\_when() function {[}Internet{]}. GeeksforGeeks.
GeeksforGeeks; 2025. Disponível em:
\url{https://www.geeksforgeeks.org/r-language/case-when-statement-in-r-dplyr-package-using-case_when-function/}}

\bibitem[\citeproctext]{ref-schirmer2000manual}
\CSLLeftMargin{74. }%
\CSLRightInline{Schirmer J, Outros. Fatores de Risco reprodutivo. Em:
Assistência Pré-Natal: Manual Técnico {[}Internet{]}. 3a Edição.
Ministério da Saúde; 2000. p. 25--6. Disponível em:
\url{https://bvsms.saude.gov.br/bvs/publicacoes/cd04_11.pdf}}

\bibitem[\citeproctext]{ref-Daniel_Cross_2013}
\CSLLeftMargin{75. }%
\CSLRightInline{Daniel WW, Cross CL. Grouped data: The frequency
distribuition. Em: Biostatistics: A Foundation for Analysis in the
Health Sciences. Tenth Edition. Wiley; 2013. p. 22-\/-23. }

\bibitem[\citeproctext]{ref-arango2009classes}
\CSLLeftMargin{76. }%
\CSLRightInline{Arango HG. Números de classes e Intervalo de Classes.
Em: Bioestatística teórica e computacional. Terceira edição. Guanabara
Koogan, RJ; 2009. p. 35--40. }

\bibitem[\citeproctext]{ref-viana2013birth}
\CSLLeftMargin{77. }%
\CSLRightInline{Viana K de J, Taddei JA de AC, Cocetti M, Warkentin S.
Birth weight in Brazilian children under two years of age. Cadernos de
Sa{ú}de P{ú}blica. 2013;29:349--56. }

\bibitem[\citeproctext]{ref-iannone2020gt}
\CSLLeftMargin{78. }%
\CSLRightInline{Iannone R, Cheng J, Schloerke B, et al. Gt: Easily
create presentation-ready display tables. 2020; Disponível em:
\url{https://CRAN.R-project.org/package=gt}}

\bibitem[\citeproctext]{ref-gohel2025flextable}
\CSLLeftMargin{79. }%
\CSLRightInline{Gohel D, Skintzos P. flextable: Functions for Tabular
Reporting {[}Internet{]}. 2025. Disponível em:
\url{https://CRAN.R-project.org/package=flextable}}

\bibitem[\citeproctext]{ref-navarro2024flextable}
\CSLLeftMargin{80. }%
\CSLRightInline{Navarro D. Use of flextable {[}Internet{]}. Notes from a
data witch. 2024. Disponível em:
\url{https://blog.djnavarro.net/posts/2024-07-04_flextable/}}

\bibitem[\citeproctext]{ref-field2012graphs}
\CSLLeftMargin{81. }%
\CSLRightInline{Field A, Miles J, Field Z. Exploring data with graphs.
Em: Discovering statistics using R. Sage Publications, Ltd; 2012. p.
117. }

\bibitem[\citeproctext]{ref-wickham2016ggplot2}
\CSLLeftMargin{82. }%
\CSLRightInline{Wickham H. ggplot2: Elegant Graphics for Data Analysis
{[}Internet{]}. Springer-Verlag New York; 2016. Disponível em:
\url{https://ggplot2.tidyverse.org}}

\bibitem[\citeproctext]{ref-wickham2010layered}
\CSLLeftMargin{83. }%
\CSLRightInline{Wickham H. A layered grammar of graphics. Journal of
Computational and Graphical Statistics. 2010;19(1):3--28. }

\bibitem[\citeproctext]{ref-wickham2023ggplot2}
\CSLLeftMargin{84. }%
\CSLRightInline{Wickham H, Navarro D, Pedersen TL. ggplot2: Elegant
Graphics for Data Analysis (3e) {[}Internet{]}. 2023. Disponível em:
\url{http://www.new.pmean.com/ggplot2-book/}}

\bibitem[\citeproctext]{ref-holtz2025colorbrewer}
\CSLLeftMargin{85. }%
\CSLRightInline{Holtz Y. R Color Brewer's palettes {[}Internet{]}. --
the R Graph Gallery. 2025. Disponível em:
\url{https://r-graph-gallery.com/38-rcolorbrewers-palettes.html}}

\bibitem[\citeproctext]{ref-hvitfeldt2024paletteer}
\CSLLeftMargin{86. }%
\CSLRightInline{Hvitfeldt E. Use any color palette with paletteer
{[}Internet{]}. The R Graph Gallery. 2024. Disponível em:
\url{https://r-graph-gallery.com/package/paletteer.html}}

\bibitem[\citeproctext]{ref-debnath2015short}
\CSLLeftMargin{87. }%
\CSLRightInline{Debnath L, Basu K. A short history of probability theory
and its applications. International Journal of Mathematical Education in
Science and Technology. 2015;46(1):13--39. }

\bibitem[\citeproctext]{ref-menezes2004probabilidade}
\CSLLeftMargin{88. }%
\CSLRightInline{Menezes RX de. Introdução à Probabilidade. Em: Massad E,
Menezes RX de, Silveira PSP, Ortega NRS, editores. Métodos Quantitativos
em Medicina. Barueri, São Paulo: Editora Manole Ltda.; 2004. p. 151--87.
}

\bibitem[\citeproctext]{ref-pagano2000random}
\CSLLeftMargin{89. }%
\CSLRightInline{Pagano M, Kimberly G. Theoretical Probability
Distributions. Em: Principles of Biostatistics. Second Edition. CRC
Press; 2000. p. 162. }

\bibitem[\citeproctext]{ref-gonzalez2021normal}
\CSLLeftMargin{90. }%
\CSLRightInline{Gonzalez JCS. Normal distribution in R {[}Internet{]}. R
CODER. 2021. Disponível em: \url{https://r-coder.com/}}

\bibitem[\citeproctext]{ref-jain_2022norm}
\CSLLeftMargin{91. }%
\CSLRightInline{Jain S. A Guide to dnorm, pnorm, rnorm, and qnorm in R
{[}Internet{]}. GeeksforGeeks. 2022. Disponível em:
\url{https://www.geeksforgeeks.org/}}

\bibitem[\citeproctext]{ref-robertson2022bernoulli}
\CSLLeftMargin{92. }%
\CSLRightInline{Robertson E, O'Connor J. Jacob (Jacques) Bernoulli
{[}Internet{]}. Maths History. School of Mathematics; Statistics,
University of St Andrews; 2022. Disponível em:
\url{https://mathshistory.st-andrews.ac.uk/Biographies/Bernoulli_Jacob/}}

\bibitem[\citeproctext]{ref-fisher1993poisson}
\CSLLeftMargin{93. }%
\CSLRightInline{Fisher LD, Van Belle G. Poisson Random Variables. Em:
Biostatistics: A Methodology for the Health Sciences. New York, NY: John
Wiley \& Sons; 1993. p. 211--8. }

\bibitem[\citeproctext]{ref-peat2014descriptive}
\CSLLeftMargin{94. }%
\CSLRightInline{Peat J, Barton B. Descriptive statistics. Em: Medical
statistics : a guide to SPSS, data analysis, and critical appraisal. New
York, NY: John Wiley \& Sons; 2014. p. 24--51. }

\bibitem[\citeproctext]{ref-meyer2019package}
\CSLLeftMargin{95. }%
\CSLRightInline{Meyer D, Dimitriadou E, Hornik K, Weingessel A, Leisch
F, Chang C-C, et al. Package {«e1071»}. The R Journal. 2019;1--67. }

\bibitem[\citeproctext]{ref-joanes1998skewkurt}
\CSLLeftMargin{96. }%
\CSLRightInline{Joanes D, Gill C. Comparing Measures of Sample Skewness
and Kurtosis. Journal of the Royal Statistical Society.
1998;47(1):183--9. }

\bibitem[\citeproctext]{ref-george2020asymmetry}
\CSLLeftMargin{97. }%
\CSLRightInline{George D, Mallery P. Descriptive Statistics. Em: IBM
SPSS Statistics 26 Step by Step: A Simple Guide and Reference. New York,
NY: Taylor \& Francis Group; 2020. p. 114--20. }

\bibitem[\citeproctext]{ref-pagano2000sampling}
\CSLLeftMargin{98. }%
\CSLRightInline{Pagano M, Gavreau K. The Central Limit Theorem. Em:
Principles of Biostatistics. Second Edition. Pacific Grove, CA: Duxbury;
2000. p. 197--8. }

\bibitem[\citeproctext]{ref-motulsky2010ci}
\CSLLeftMargin{99. }%
\CSLRightInline{Motulsky H. The Theory of Confidence Intervals. Em:
Intuitive Biostatistics: A Nonmathematical Guide to Statistical
Thinking. Second Edition. New York, NY: Oxford University Press; 2010.
p. 96--102. }

\bibitem[\citeproctext]{ref-signorell2022desctools}
\CSLLeftMargin{100. }%
\CSLRightInline{Signorell A et al. DescTools: Tools for Descriptive
Statistics {[}Internet{]}. 2022. Disponível em:
\url{https://cran.r-project.org/package=DescTools}}

\bibitem[\citeproctext]{ref-kelen1988hypothesis}
\CSLLeftMargin{101. }%
\CSLRightInline{Kelen GD, Brown CB, Ashton J. Statistical reasoning in
clinical trials: hypothesis testing. Am J Emerg Med. 1988;1(1):52--61. }

\bibitem[\citeproctext]{ref-menezes2004hipoteses}
\CSLLeftMargin{102. }%
\CSLRightInline{Menezes RX de, Burattini MN. Testes de Hipótese e
intervalos de Confiança. Em: Massad E, Menezes RX de, Silveira PSP,
Ortega NRS, editores. Métodos Quantitativos em Medicina. Barueri, São
Paulo: Editora Manole Ltda.; 2004. p. 225--41. }

\bibitem[\citeproctext]{ref-guyatt1995basic}
\CSLLeftMargin{103. }%
\CSLRightInline{Guyatt G, Jaeschke R, Heddle N, et al. Basic statistics
for clinicians: 1. Hypothesis testing. CMAJ: Canadian Medical
Association Journal. 1995;152(1):27. }

\bibitem[\citeproctext]{ref-fletcher2014acaso}
\CSLLeftMargin{104. }%
\CSLRightInline{Fletcher RH, Fletcher SW, Fletcher GS. Acaso. Em:
Epidemiologia Clínica: Elementos Essenciais. Quinta Edição. Artmed
Editora; 2014. p. 108--9. }

\bibitem[\citeproctext]{ref-menezes2004testes}
\CSLLeftMargin{105. }%
\CSLRightInline{Menezes RX de, Burattini MN. Testes de Hipótese e
intervalos de Confiança. Em: Massad E, Menezes RX de, Silveira PSP,
Ortega NRS, editores. Métodos Quantitativos em Medicina. Barueri, São
Paulo: Editora Manole Ltda.; 2004. p. 225--41. }

\bibitem[\citeproctext]{ref-pagano2000t_test}
\CSLLeftMargin{106. }%
\CSLRightInline{Pagano M, Kimberly G. Comparison of Two Means. Em:
Principles of Biostatistics. Second Edition. CRC Press; 2000. p.
262--72. }

\bibitem[\citeproctext]{ref-zimmerman2004note}
\CSLLeftMargin{107. }%
\CSLRightInline{Zimmerman DW. A note on preliminary tests of equality of
variances. Br J Math Stat Psychol. 2004;57(1):173--81. }

\bibitem[\citeproctext]{ref-razali2011power}
\CSLLeftMargin{108. }%
\CSLRightInline{Razali NM, Wah YB, et al. Power comparisons of
shapiro-wilk, kolmogorov-smirnov, lilliefors and anderson-darling tests.
Journal of statistical modeling and analytics. 2011;2(1):21--33. }

\bibitem[\citeproctext]{ref-ghasemi2012normality}
\CSLLeftMargin{109. }%
\CSLRightInline{Ghasemi A, Zahediasl S. Normality tests for statistical
analysis: a guide for non-statisticians. International journal of
endocrinology and metabolism. 2012;10(2):486. }

\bibitem[\citeproctext]{ref-yap2011comparisons}
\CSLLeftMargin{110. }%
\CSLRightInline{Yap BW, Sim CH. Comparisons of various types of
normality tests. Journal of Statistical Computation and Simulation.
2011;81(12):2141--55. }

\bibitem[\citeproctext]{ref-fox2018car}
\CSLLeftMargin{111. }%
\CSLRightInline{Fox J, Weisberg S. An R Companion to Applied Regression
{[}Internet{]}. Third. Thousand Oaks {CA}: Sage; 2019. Disponível em:
\url{https://socialsciences.mcmaster.ca/jfox/Books/Companion/}}

\bibitem[\citeproctext]{ref-kassambara2022rstatix}
\CSLLeftMargin{112. }%
\CSLRightInline{Kassambara A. rstatix: Pipe-Friendly Framework for Basic
Statistical Tests {[}Internet{]}. 2022. Disponível em:
\url{https://CRAN.R-project.org/package=rstatix}}

\bibitem[\citeproctext]{ref-cohen1988power}
\CSLLeftMargin{113. }%
\CSLRightInline{Cohen J. Statistical power analysis for the behavioral
sciences. 2nd Edition. Routledge; 1988. }

\bibitem[\citeproctext]{ref-lindenau2012effect}
\CSLLeftMargin{114. }%
\CSLRightInline{Lindenau JD, Guimaraes LSP. Calculating the Effect Size
in SPSS. Revista HCPA {[}Internet{]}. 2012;32(3):363--81. Disponível em:
\url{https://seer.ufrgs.br/hcpa}}

\bibitem[\citeproctext]{ref-kassambara2022ggpubr}
\CSLLeftMargin{115. }%
\CSLRightInline{Kassambara A. ggpubr:'ggplot2' based publication ready
plots {[}R package ggpubr version 0.5.0{]} {[}Internet{]}. The
Comprehensive R Archive Network. Comprehensive R Archive Network (CRAN);
2022. Disponível em:
\url{https://cloud.r-project.org/web/packages/ggpubr/index.html}}

\bibitem[\citeproctext]{ref-field2012anova}
\CSLLeftMargin{116. }%
\CSLRightInline{Field A, Miles J, Field Z. Comparing several means:
ANOVA (GML 1). Em: Discovering Statistics Using R. Sage Publications,
Ltd; 2012. p. 399--400. }

\bibitem[\citeproctext]{ref-menezes2004anova}
\CSLLeftMargin{117. }%
\CSLRightInline{Menezes RX de. Análise de Variância. Em: Massad E,
Menezes RX de, Silveira PSP, Ortega NRS, editores. Métodos Quantitativos
em Medicina. Barueri, São Paulo: Editora Manole Ltda.; 2004. p.
297--300. }

\bibitem[\citeproctext]{ref-ferreira2011distf}
\CSLLeftMargin{118. }%
\CSLRightInline{Ferreira DF, Helms BP. Aproximação normal da
distribuição F. Rev Bras Biom. 2011;29(2):222--8. }

\bibitem[\citeproctext]{ref-peat2014anova}
\CSLLeftMargin{119. }%
\CSLRightInline{Peat J, Barton B. Continuous variables: analysis of
variance. Em: Medical statistics : a guide to SPSS, data analysis, and
critical appraisal. New York, NY: John Wiley \& Sons; 2014. p. 114. }

\bibitem[\citeproctext]{ref-dag2018onewaytests}
\CSLLeftMargin{120. }%
\CSLRightInline{Dag O, Dolgun A, Konar NM. Onewaytests: An R Package for
One-Way Tests in Independent Groups Designs. R Journal.
2018;10(1):175--99. }

\bibitem[\citeproctext]{ref-ben2020effectsize}
\CSLLeftMargin{121. }%
\CSLRightInline{Ben-Shachar MS, Lüdecke D, Makowski D. effectsize:
Estimation of effect size indices and standardized parameters. Journal
of Open Source Software. 2020;5(56):2815. }

\bibitem[\citeproctext]{ref-watson2021effectsize}
\CSLLeftMargin{122. }%
\CSLRightInline{Watson P. Rules of thumb on magnitudes of effect sizes
{[}Internet{]}. MRC Cognition and Brain Sciences Unit. Cambridge
University; 2021. Disponível em:
\url{https://imaging.mrc-cbu.cam.ac.uk/statswiki/FAQ/effectSize}}

\bibitem[\citeproctext]{ref-field2012factorial}
\CSLLeftMargin{123. }%
\CSLRightInline{Field A, Miles J, Field Z. Factorial ANOVA (GLM3). Em:
Discovering statistics using R. Sage Publications, Ltd; 2012. p. 513--4.
}

\bibitem[\citeproctext]{ref-patterson2015diagnostic}
\CSLLeftMargin{124. }%
\CSLRightInline{Patterson R, Coffman J, Goldstein-Greenwood J, Others.
Understanding Diagnostic Plots for Linear Regression Analysis
{[}Internet{]}. Research Data Services + Sciences. University of
Virginia Library; 2015. Disponível em:
\url{https://data.library.virginia.edu/diagnostic-plots/}}

\bibitem[\citeproctext]{ref-wickens2004anova}
\CSLLeftMargin{125. }%
\CSLRightInline{Wickens TD, Keppel G. Two-way factorial experiments. Em:
Design and analysis: A researcher's handbook. Pearson Prentice-Hall;
2004. p. 193--286. }

\bibitem[\citeproctext]{ref-maxwell2017factorial}
\CSLLeftMargin{126. }%
\CSLRightInline{Maxwell SE, Delaney HD, Kelley K. Two-way
Between-Subject Factorial Designs. Em: Designing experiments and
analyzing data: A model comparison perspective. Third Edition.
Routledge; 2017. p. 312--82. }

\bibitem[\citeproctext]{ref-lenth2018emmeans}
\CSLLeftMargin{127. }%
\CSLRightInline{Lenth R, Singmann H, Love J, Buerkner P, Herve M.
Emmeans: Estimated marginal means, aka least-squares means. R package
version. 2018;1(1):3. }

\end{CSLReferences}


\backmatter


\end{document}
